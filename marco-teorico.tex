	\section*{Presentación}

La psicología de Carl Gustav Jung nos sumerge en un fascinante viaje hacia las profundidades de la psique humana, explorando conceptos como el \textbf{inconsciente colectivo}, los \textbf{arquetipos}, la \textbf{individuación}, la \textbf{sincronicidad} y la \textbf{intuición}.

Jung nos invita a adentrarnos en el inconsciente colectivo, una capa universal de la psique compartida por toda la humanidad, donde residen los arquetipos, patrones de pensamiento y comportamiento que se manifiestan en mitos, sueños, símbolos y rituales.

A través del proceso de individuación, nos guía en la búsqueda de la totalidad y realización personal, integrando los aspectos conscientes e inconscientes para alcanzar un mayor equilibrio. En este camino enfrentamos la sombra y abrazamos lo reprimido, avanzando hacia una mayor integración y plenitud.

La sincronicidad nos revela coincidencias cargadas de significado, uniendo el mundo interior con los eventos exteriores. La intuición, como función psíquica, permite captar estas conexiones más allá de la lógica, abriéndonos a una comprensión más profunda del sentido de la vida.

\section*{Elementos centrales de la Psicología Analítica}

\begin{itemize}
	\item \textbf{Símbolo}: lenguaje del alma; vehículo de transformación e integración.
	\item \textbf{Self (Sí-mismo)}: núcleo organizador de la psique; totalidad del ser.
	\item \textbf{Función trascendente}: mecanismo que integra los opuestos internos.
	\item \textbf{Persona}: máscara social que puede alejar del verdadero yo.
	\item \textbf{Ánima/Ánimus}: contrapartes psíquicas femenina y masculina presentes en cada persona.
\end{itemize}

\vspace{0.5cm}
\begin{quote}
	\emph{“Aquello a lo que te resistes, persiste. Aquello que aceptas, te transforma.”} \\ — C. G. Jung
\end{quote}

\section*{Glosario básico junguiano}
\begin{description}[leftmargin=1.5cm]
	\item[Inconsciente colectivo] Capa universal de la psique que contiene símbolos y patrones comunes a la humanidad.
	\item[Arquetipo] Imagen primordial presente en el inconsciente colectivo: el Héroe, la Madre, el Sabio, etc.
	\item[Sombra] Aspectos rechazados o reprimidos del yo.
	\item[Self] Totalidad de la psique, integradora del consciente e inconsciente.
	\item[Persona] Máscara que mostramos al mundo.
	\item[Ánima/Ánimus] Elementos femeninos o masculinos internos, según el sexo biológico.
	\item[Individuación] Camino hacia la integración total del ser.
	\item[Función trascendente] Capacidad de la psique para encontrar soluciones integradoras a tensiones internas.
	\item[Sincronicidad] Coincidencias significativas que revelan un orden más profundo.
	\item[Intuición] Percepción directa de verdades internas o simbólicas.
\end{description}

\section*{Citas fundamentales de Carl Gustav Jung}

\subsection*{Inconsciente colectivo}

\begin{quote}
	\emph{“El inconsciente colectivo es una parte de la psique que no se origina en la experiencia personal y que no es una adquisición individual, sino que es innato.”} \\
	— C. G. Jung, \emph{El hombre y sus símbolos}, 1964, p. 58.
\end{quote}

\begin{quote}
	\emph{“El inconsciente colectivo contiene toda la herencia espiritual de la evolución de la humanidad, renacida en la estructura cerebral de cada individuo.”} \\
	— C. G. Jung, \emph{Arquetipos e inconsciente colectivo}, 1954, §342.
\end{quote}

\subsection*{Arquetipos}

\begin{quote}
	\emph{“Los arquetipos son sistemas de disposición a la acción y, al mismo tiempo, imágenes y emociones heredadas.”} \\
	— C. G. Jung, \emph{Arquetipos e inconsciente colectivo}, §54.
\end{quote}

\begin{quote}
	\emph{“El arquetipo no es una imagen determinada, sino una forma que puede adoptar innumerables representaciones.”} \\
	— C. G. Jung, \emph{Psicología y alquimia}, 1944, §155.
\end{quote}

\subsection*{La sombra}

\begin{quote}
	\emph{“Uno no se ilumina imaginando figuras de luz, sino haciendo consciente la oscuridad.”} \\
	— C. G. Jung, \emph{Aion: Estudios sobre el simbolismo del sí-mismo}, 1951, §335.
\end{quote}

\begin{quote}
	\emph{“La sombra es todo aquello que el sujeto se niega a reconocer de sí mismo y que, sin embargo, siempre se le impone por el otro lado, directa o indirectamente.”} \\
	— C. G. Jung, \emph{Encuentros con la sombra} (obra póstuma).
\end{quote}

\subsection*{Self e individuación}

\begin{quote}
	\emph{“El Self es tanto el centro como la circunferencia que abarca a la totalidad de la psique, consciente e inconsciente.”} \\
	— C. G. Jung, \emph{Aion}, §44.
\end{quote}

\begin{quote}
	\emph{“La individuación significa volverse un individuo indivisible, y en cuanto proceso psicológico, se trata de la autorrealización del Sí-mismo.”} \\
	— C. G. Jung, \emph{Símbolos de transformación}, 1912, §266.
\end{quote}

\subsection*{Función trascendente}

\begin{quote}
	\emph{“La función trascendente surge del conflicto entre los contenidos conscientes e inconscientes y de su confrontación.”} \\
	— C. G. Jung, \emph{La función trascendente}, 1916.
\end{quote}

\begin{quote}
	\emph{“No se trata de eliminar las tensiones internas, sino de mantenerlas vivas hasta que una forma simbólica las integre.”} \\
	— C. G. Jung, \emph{La psicología de la transferencia}, §136.
\end{quote}

\subsection*{Sincronicidad}

\begin{quote}
	\emph{“La sincronicidad significa la coincidencia temporal de dos o más sucesos relacionados entre sí de un modo no causal, cuyo sentido reside en su simultaneidad.”} \\
	— C. G. Jung, \emph{Sincronicidad: una interpretación acausal}, 1952, §827.
\end{quote}

\subsection*{Psique colectiva y pueblos}

\begin{quote}
	\emph{“Todo pueblo tiene su alma colectiva y todo lo que no puede hacer el individuo, lo puede su comunidad.”} \\
	— C. G. Jung, \emph{Civilización en transición}, OC, vol. 10.
\end{quote}

\begin{quote}
	\emph{“Los grandes cambios históricos y culturales no nacen del raciocinio, sino de la irrupción de lo inconsciente colectivo en la conciencia.”} \\
	— C. G. Jung, \emph{El problema del alma moderna}, OC, vol. 10.
\end{quote}

\begin{thebibliography}{99}
	
	\bibitem{jung1964}
	Jung, C. G. (1964). \emph{El hombre y sus símbolos}. Paidós.
	
	\bibitem{jung1954}
	Jung, C. G. (1954). \emph{Arquetipos e inconsciente colectivo}. En \emph{Obras completas}, vol. 9/1. Trotta.
	
	\bibitem{jung1944}
	Jung, C. G. (1944). \emph{Psicología y alquimia}. En \emph{Obras completas}, vol. 12. Trotta.
	
	\bibitem{jung1951}
	Jung, C. G. (1951). \emph{Aion: Estudios sobre el simbolismo del sí-mismo}. En \emph{Obras completas}, vol. 9/2. Trotta.
	
	\bibitem{jung1912}
	Jung, C. G. (1912). \emph{Símbolos de transformación}. En \emph{Obras completas}, vol. 5. Trotta.
	
	\bibitem{jung1916}
	Jung, C. G. (1916). \emph{La función trascendente}. En \emph{Obras completas}, vol. 8. Trotta.
	
	\bibitem{jung1952}
	Jung, C. G. (1952). \emph{Sincronicidad: una interpretación acausal}. En \emph{La realidad del alma}, vol. 8. Paidós.
	
	\bibitem{jung1957}
	Jung, C. G. (1957). \emph{El problema del alma moderna}. En \emph{Obras completas}, vol. 10. Trotta.
	
	\bibitem{jung1959}
	Jung, C. G. (1959). \emph{Civilización en transición}. En \emph{Obras completas}, vol. 10. Trotta.
	
	\bibitem{jungPost}
	Jung, C. G. (compilación póstuma). \emph{Encuentros con la sombra}. Kairos.
	
\end{thebibliography}
