\documentclass[a4paper,12pt]{article}

% ------------ PAQUETES ------------
\usepackage[spanish]{babel}
\usepackage[utf8]{inputenc}
\usepackage[T1]{fontenc}
\usepackage{setspace}
\usepackage{geometry}
\geometry{margin=2.5cm}
\usepackage{titlesec}
\usepackage{fontspec}
\usepackage{graphicx}
\usepackage{caption}
\usepackage{float}
\usepackage{longtable}
\usepackage{tocloft}
\usepackage{csquotes}
\usepackage{fontspec}
\usepackage{hyperref}
\usepackage{xurl}
\usepackage{xcolor}
\usepackage{tabularx}
\usepackage{ltablex}
\usepackage{titlesec}
\usepackage{setspace}
\keepXColumns

\definecolor{verdelimon}{rgb}{0.4, 1.0, 0.0}

% ------------ CONFIGURACIONES ------------
\setmainfont{Arial}
\setlength{\parindent}{0pt}
\setlength{\parskip}{6pt}
\setstretch{1.0}
\titleformat{\section}{\normalfont\Large\bfseries}{\thesection.}{1em}{}
\titleformat{\subsection}{\normalfont\large\bfseries}{\thesubsection.}{1em}{}

% ------------ INICIO DOCUMENTO ------------
\begin{document}
	
	% ------------ PORTADA ------------
\input{estatua_pizarro_caratula}
	% ------------ ÍNDICE ------------
	\tableofcontents
	
	\listoftables
	
	
	\listoffigures

	\newpage
	
	% ------------ SECCIONES DEL DOCUMENTO ------------
	
	\section{Introducción}
	
	La ciudad de Lima, capital del Perú, está llena de monumentos y símbolos que representan distintos momentos de su historia. Uno de los más polémicos es la estatua ecuestre de Francisco Pizarro, conquistador español y fundador de la ciudad en 1535. Esta estatua no solo representa una figura histórica, sino que también encarna los debates sobre la identidad nacional, la memoria colectiva y la tensión entre el legado colonial y las reivindicaciones indígenas.
	
	\section{Francisco Pizarro y la conquista del Perú}
	
	Francisco Pizarro lideró la expedición que culminó con la caída del Imperio Inca y la fundación de Lima el 18 de enero de 1535. Su figura ha sido tradicionalmente celebrada como la del fundador, pero también criticada como responsable de un proceso de colonización violento. Sin embargo, es fundamental entender que Pizarro no actuó solo: su victoria fue posible gracias al apoyo de diversos pueblos indígenas. Por tanto, es innegable su papel en la historia del Perú y como fundador de la ciudad de Lima.
	
	\section{El mestizaje como realidad fundacional}
	
	La figura de Francisco Pizarro no solo representa la llegada de los españoles a tierras peruanas, sino también el inicio de un complejo proceso de mestizaje, tanto cultural como biológico. Este mestizaje no fue una mera consecuencia accidental de la conquista, sino un fenómeno fundacional que marcaría profundamente la identidad del Perú.
	
	Pizarro tuvo como compañera a una mujer de sangre incaica: \textbf{Quispe Sisa}, también conocida como \textit{Inés Huaylas Yupanqui}, hija del inca Huayna Cápac y de una noble de la región de Huaylas. Esta unión simboliza el primer cruce de caminos entre dos mundos: el europeo y el indígena. Fruto de esta relación nació \textbf{Francisca Pizarro Yupanqui}, quien años después fue llevada a España y casada con su tío Hernando Pizarro, en lo que puede considerarse una estrategia de consolidación familiar y política.
	
	\begin{quote}
		“Quispe Sisa, llamada también Inés Huaylas Yupanqui, fue hija del Inca Huayna Cápac. Francisco Pizarro la tomó por compañera y con ella tuvo una hija, Francisca, quien se casaría luego con Hernando Pizarro en España.”
		\begin{flushright}
			— Juan José Vega, \textit{La guerra de los viracochas} (1971)
		\end{flushright}
	\end{quote}
	
	Esta primera unión entre conquistador e indígena noble no fue aislada: representa el inicio de un proceso que dio lugar al nacimiento del pueblo mestizo peruano. Ignorar esta realidad —como lo hace cierta corriente indigenista radical— es fragmentar nuestra memoria histórica.
	
	Más allá de la violencia del proceso de conquista, el mestizaje fue también un espacio de encuentro y creación. La cultura peruana actual, incluida su religiosidad, su lengua y su arte, es el resultado de ese cruce de caminos. La identidad nacional, por tanto, no puede explicarse sin reconocer este mestizaje originario, en el que la figura de Pizarro y su descendencia con Quispe Sisa ocupan un lugar simbólico importante.
	

	
	\section{Desplazamientos de la estatua de Francisco Pizarro}
	
	Diversas corrientes ideológicas han buscado —y aún buscan— minimizar el papel de España en la historia del Perú. Un reflejo evidente de ello ha sido el tratamiento político e ideológico dado al monumento de Francisco Pizarro, cuyo simbolismo ha generado debates a lo largo del siglo XX y XXI. Como consecuencia, distintas gestiones municipales ordenaron su retiro o reubicación, respondiendo a tensiones entre el legado hispánico y narrativas indigenistas. Tales decisiones han sido interpretadas por muchos como una injusticia histórica hacia el fundador de la ciudad de Lima.
	
	\subsection{Historia de los traslados}
	
	\begin{table}[htbp]
		\centering
		\caption{Movimientos de la estatua de Francisco Pizarro}
		\begin{tabularx}{\textwidth}{|c|X|X|}
			\hline
			\textbf{Año} & \textbf{Ubicación} & \textbf{Autoridad Responsable} \\
			\hline
			1935 & Plazuela frente a Palacio de Gobierno (Plaza Pizarro) & Alcalde Luis Gallo Porras \\
			\hline
			1952 & Reubicada frente a la Catedral & Alcalde Eduardo Dibós Dammert \\
			\hline
			2003 & Parque de la Muralla & Alcalde Luis Castañeda Lossio \\
			\hline
		\end{tabularx}
		\label{tab:movimientos-pizarro}
	\end{table}
	
	
	\clearpage
	
	\subsection{Alcaldes involucrados}
	
\begin{table}[htbp]
	\centering
	\caption{Alcaldes involucrados en decisiones sobre la estatua de Pizarro}
	\label{tab:alcaldes-pizarro}
	\begin{tabularx}{\textwidth}{|X|X|X|}
		\hline
		\textbf{Nombre del alcalde} & \textbf{Años de gestión} & \textbf{Decisión respecto a la estatua} \\
		\hline
		Luis Gallo Porras & 1934–1937 & Ordenó su instalación frente a la Catedral como homenaje fundacional \\
		\hline
		Eduardo Dibós Dammert & 1950–1952 & Dispuso su traslado a una plazuela contigua por presión de sectores religiosos \\
		\hline
		Luis Castañeda Lossio & 2003–2010 & Retiró la estatua del centro y la reubicó en el Parque de la Muralla \\
		\hline
		Rafael López Aliaga & 2023– Acualmente & Promovió su regreso al Centro Histórico (Pasaje Santa Rosa) \\
		\hline
	\end{tabularx}
\end{table}



	
	
\section{Corrientes ideológicas y debates históricos}

Durante los siglos XX y XXI, diversas corrientes ideológicas influyeron en el tratamiento simbólico de la estatua de Francisco Pizarro y en otros elementos del paisaje urbano limeño. Estas corrientes, con frecuencia ligadas a debates sobre identidad nacional, memoria histórica y laicismo, propiciaron cambios relevantes en el uso del espacio público:

\begin{itemize}
	\item El \textbf{indigenismo}, particularmente fuerte durante el gobierno de Juan Velasco Alvarado, promovió la eliminación de símbolos coloniales que se percibían como opresores.
	
	\item Las corrientes \textbf{anticlericales}, activas desde inicios del siglo XX, motivaron la demolición de templos históricos, como la Iglesia de Nuestra Señora de los Desamparados en 1938, con el argumento de modernizar el entorno de Palacio de Gobierno.
	
	\item Algunos \textbf{alcaldes populistas o conservadores} utilizaron el monumento a Pizarro como herramienta simbólica para reforzar distintas narrativas: desde la exaltación del mestizaje y la hispanidad hasta la revalorización de la memoria indígena.
\end{itemize}

\clearpage	
\section{Apoyo indígena en la conquista}

Es imposible pensar que los 180–300 soldados españoles conquistaron el Tahuantinsuyo sin aliados nativos. Los pueblos que colaboraron activamente fueron:

\begin{itemize}
	\item \textbf{Huancas}
	\item \textbf{Cañaris}
	\item \textbf{Chachapoyas}
	\item \textbf{Yungas}, \textbf{Huaylas}, \textbf{Tarmas}, \textbf{Chankas}
\end{itemize}

Estos pueblos aportaron miles de guerreros, conocimientos del terreno, logística y estrategia. Muchos de ellos estaban sometidos por los incas y vieron en los españoles una oportunidad de liberación. Por ello, la conquista fue un esfuerzo conjunto de españoles e indígenas. El indigenismo, como ideología, tiende a minimizar este proceso complejo y reducir la identidad peruana a una sola de sus raíces.

\begin{quote}
	“La conquista del Imperio incaico no fue solo obra de los españoles, sino también de miles de indígenas que se aliaron con ellos buscando liberarse del dominio inca. [...] La alianza entre españoles y pueblos como los cañaris, los huancas y los chachapoyas fue decisiva.”
	\begin{flushright}
		— Pablo Macera, \textit{Historia del Perú republicano} (1979, p. 21)
	\end{flushright}
\end{quote}

\clearpage
\subsection*{El papel de los aliados indígenas en el asedio de Lima}
\addcontentsline{toc}{subsection}{El papel de los aliados indígenas en el asedio de Lima}

Durante el gran asedio de Lima liderado por Manco Inca Yupanqui en agosto de 1536, se produjo uno de los episodios más tensos de la conquista. La ciudad, recientemente fundada por Francisco Pizarro, fue sitiada por miles de guerreros incas decididos a recuperar el control del valle del Rímac.

Pizarro logró resistir esta ofensiva gracias a varios factores, entre los cuales destaca el papel crucial de pueblos indígenas que decidieron aliarse con los españoles:

\begin{itemize}
	\item \textbf{Apoyo militar de pueblos aliados,} incluyendo caciques costeños que habían sido previamente subyugados por los incas y vieron en los españoles una oportunidad de liberación.
	
	\item \textbf{Tropas, guías y mensajeros indígenas,} que actuaron como exploradores y combatientes, facilitando el conocimiento del terreno y la defensa de la ciudad.
	
	\item \textbf{La rivalidad preexistente entre pueblos originarios,} particularmente entre los del litoral y los del Tahuantinsuyo, que se tradujo en un respaldo táctico a los españoles frente al poder incaico.
\end{itemize}

\textbf{En particular, los pueblos de la región yunga} —como Maranga, Surco, Lurigancho y otros del valle del Rímac— contribuyeron con recursos, combatientes y lealtad estratégica. Su colaboración fue fundamental en la defensa de la ciudad durante este asedio prolongado.

\begin{quote}
	“Los indios costeños, antiguos enemigos de los incas, se mantuvieron fieles a los españoles durante el asedio. Entre ellos, los de Maranga y Surco contribuyeron activamente en la defensa de Lima.”
	\begin{flushright}
		— Raúl Porras Barrenechea, \textit{Fuentes históricas peruanas} (1963)
	\end{flushright}
\end{quote}

\clearpage
\subsection*{San Cristóbal y la protección milagrosa de Lima}
\addcontentsline{toc}{subsection}{San Cristóbal y la protección milagrosa de Lima}

Durante el asedio de Lima en 1536, cuando las huestes de Manco Inca Yupanqui amenazaban con retomar el control del valle del Rímac, los españoles y sus aliados indígenas se encontraban en inferioridad numérica. En medio de esta crisis, la tradición piadosa limeña sostiene que los habitantes de la ciudad encomendaron su defensa a \textbf{San Cristóbal}, santo protector de los viajeros y de los pueblos en peligro.

Según las crónicas coloniales, la súplica a este santo fue escuchada y, de manera inesperada, los incas comenzaron a retirarse, levantando el asedio. Este hecho fue interpretado como una \textbf{intervención milagrosa}, y en señal de gratitud se erigió una gran cruz en la cima del cerro que, desde entonces, lleva el nombre de \textit{Cerro San Cristóbal}. Este lugar se convirtió con el tiempo en un símbolo de fe y protección para los limeños.

\begin{quote}
	“La ciudad de los Reyes fue librada milagrosamente del cerco de los indios rebeldes por intercesión de San Cristóbal, a quien los vecinos habían encomendado su defensa.”
	\begin{flushright}
		— Rubén Vargas Ugarte, \textit{Historia del culto de los santos en el Perú}, Vol. I
	\end{flushright}
\end{quote}

	
	
\section*{Conclusión}
\addcontentsline{toc}{section}{Conclusión}

La figura de Francisco Pizarro, como fundador de Lima, ha sido históricamente objeto de controversias, especialmente desde la perspectiva de corrientes ideológicas que buscan reducir su papel a una imposición extranjera. Sin embargo, este trabajo ha mostrado que tanto el símbolo —la estatua— como el personaje representan dimensiones fundamentales de la identidad limeña y nacional: el mestizaje cultural y la permanencia de la fe católica.

\textbf{Por un lado,} la historia demuestra que la conquista del Perú no fue un evento unidireccional. Diversos pueblos originarios, entre ellos los yungas, participaron activamente del proceso, motivados por la oportunidad de liberarse del dominio incaico. Reconocer esta realidad es hacer justicia a la agencia histórica de estos pueblos y a la complejidad del proceso que dio origen al Perú mestizo.

\textbf{Por otro lado,} la dimensión espiritual de la conquista y la fundación de Lima está fuertemente marcada por la fe católica. La tradición del milagro atribuido a San Cristóbal durante el asedio de 1536 es una muestra clara de cómo la religiosidad popular forma parte del inconsciente colectivo limeño. La fe no solo acompañó los orígenes de la ciudad, sino que ha resistido intentos de desarraigo ideológico y permanece como eje articulador de identidad.

En suma, la estatua de Pizarro, lejos de ser solo un monumento polémico, es un recordatorio de las tensiones y convergencias que dieron origen a nuestra nación. Es símbolo de historia, de mestizaje, de conflicto y de reconciliación. Y es también, inevitablemente, testimonio de una fe que ha moldeado el alma del Perú.

\begin{quote}
	“El indigenismo ha terminado construyendo una identidad excluyente, que muchas veces niega el mestizaje y reemplaza una visión colonialista por otra visión igualmente dogmática.”
	\begin{flushright}
		— Francisco Miró Quesada Rada, \textit{El Comercio}, columna “Integración social” (2019)
	\end{flushright}
\end{quote}

\section{Reflexión personal}

He escogido la estatua de Francisco Pizarro como símbolo para este trabajo porque, desde que era niño, sentí su presencia como parte del paisaje histórico de Lima. Aunque nunca la vi en su ubicación original frente a la Catedral, recuerdo claramente que estaba junto al Palacio de Gobierno, en una plazuela especialmente destinada para ella. Con el paso del tiempo, me sorprendió que un alcalde, que ni siquiera era limeño de nacimiento, decidiera retirarla cediendo a presiones ideológicas. Ese hecho marcó en mí una inquietud persistente sobre cómo ciertos acontecimientos en nuestra ciudad han ido desplazando símbolos profundamente ligados a la historia, la identidad y, especialmente, a la fe.

Este trabajo, más allá del análisis histórico, busca también reivindicar la dimensión espiritual de nuestra identidad nacional. A lo largo del ensayo presento hechos comprobables, pero hay un elemento que, pese a todos los ataques contra la fe católica, no ha podido ser arrancado del alma peruana: las tradiciones profundamente enraizadas, como la del Señor de los Milagros o la devoción a la Virgen del Rosario. Esta última fue en su momento reconocida como Patrona del Reino del Perú desde el convento de Santo Domingo, aquí en Lima.

Durante la gestión del alcalde Luis Castañeda Lossio, presencié también el contexto en que muchas decisiones se tomaban en una ciudad ya influida por el auge evangélico liberal de los años noventa. Este fenómeno, que viví desde dentro al haber sido pastor evangélico, transformó profundamente la sensibilidad religiosa de muchos limeños. Algunas de sus expresiones estaban impregnadas de un sentimiento antiespañol y anticatólico. En muchos casos, se intentaba suplantar la influencia cultural y espiritual de la Iglesia católica por una visión distinta, más cercana al protestantismo norteamericano.

Aunque el auge del movimiento evangélico se dio principalmente durante los años 90, especialmente bajo el gobierno de Alberto Fujimori, su influencia perduró en las décadas siguientes. Esta expansión religiosa no solo transformó el paisaje espiritual de Lima, sino que también preparó un terreno cultural que, en ciertos sectores, se expresó con un sentir liberal, anticatólico y antiespañol. En ese contexto ya influido, muchas decisiones posteriores, como la reubicación de símbolos históricos vinculados a la fe católica o al pasado hispánico, pueden entenderse no solo como acciones políticas puntuales, sino como parte de un cambio cultural más amplio que alteró la forma en que los limeños reinterpretan su historia e identidad.

Sin embargo, lo verdaderamente profundo es que esa herencia católica nunca se fue del todo. Nuestra sociedad mestiza ha incorporado esa fe como una amalgama que nos une más allá de credos personales. El auge actual de la gastronomía peruana, reconocida mundialmente por su fusión, es una metáfora viva de lo que somos: una síntesis cultural, étnica y espiritual.

Fui testigo de la reforma educativa impulsada por el general Juan Velasco Alvarado. En ella se exaltaban los valores andinos y el indigenismo. De niño, comprendía que esos elementos eran valiosos para muchos peruanos, pero sentía que no expresaban lo propio de quienes vivimos en la costa. Con el tiempo entendí que esa diversidad no nos divide: nos enriquece. Todos somos parte del Perú. Pero hay algo que atraviesa y unifica esa diversidad: la fe católica. Practicada o no, forma parte del inconsciente colectivo nacional. Tanto es así que la propia Constitución Política del Perú reconoce el papel histórico de la Iglesia Católica en la formación del país.

Finalmente, una referencia importante. Víctor Raúl Haya de la Torre, fundador del Partido Aprista Peruano, estudió en un colegio presbiteriano de Lima, donde fue influido por el pastor escocés John A. Mackay. Este escribió el libro \textit{El otro Cristo español}, en el que cuestionaba la visión católica del cristianismo promovida por España y proponía una alternativa protestante, más individualista y liberal. Estas ideas influyeron en sectores que, de manera directa o indirecta, buscaron debilitar nuestras raíces espirituales.

Este trabajo es, entonces, una pequeña contribución para reivindicar desde el fondo del inconsciente colectivo aquello que nunca se ha ido del todo: la herencia católica, hispánica y mestiza que forma parte esencial de la identidad peruana.

	
	\begin{thebibliography}{9}
		\bibitem{orrego}
		Orrego, J. L. (2011). \textit{Nota sobre iglesias desaparecidas en Lima}. Blog PUCP. Recuperado de: \url{https://blog.pucp.edu.pe/blog/juanluisorrego/2011/06/06/nota-sobre-iglesias-desaparecidas-en-lima/}
		
		\bibitem{wikipedia}
		Wikipedia (2025). \textit{Estatua ecuestre de Francisco Pizarro (Lima)}. Recuperado de: \url{https://es.wikipedia.org/wiki/Estatua_ecuestre_de_Francisco_Pizarro_(Lima)}
		
		\bibitem{confederacion}
		Confederación Hispánica. (2018). \textit{Los aliados de Francisco Pizarro}. Recuperado de: \url{https://confederacinhispanica.wordpress.com/2018/10/28/los-aliados-de-francisco-pizarro-gonzalez/}
		
		\bibitem{expreso}
		Expreso. (2025). \textit{La estatua de Pizarro vuelve al Centro Histórico}. Recuperado de: \url{https://www.expreso.com.pe/...}
		
		\bibitem{libertad}
		Libertad Digital. (2025). \textit{La estatua de Pizarro y el mestizaje peruano}. Recuperado de: \url{https://www.libertaddigital.com/...}
				
		\bibitem{limalaunica}
		Lima La Única. (2010, noviembre 3). \textit{El monumento a Francisco Pizarro}. Recuperado de \url{https://www.limalaunica.pe/2010/11/el-monumento-francisco-pizarro.html}
		
		\bibitem{macera1979}
		Macera, P. (1979). \textit{Historia del Perú republicano}. Lima: Instituto Nacional de Cultura.
		
		\bibitem{miroquesada2019}
		Miró Quesada Rada, F. (2019, julio 14). Integración social. \textit{El Comercio}. Recuperado de \url{https://elcomercio.pe/opinion/columnistas/integracion-social-sociedad-peru-indigenismo-e-indianismo-por-francisco-miro-quesada-rada-noticia/}
		
		\bibitem{vega1971}
		Vega, J. J. (1971). \textit{La guerra de los viracochas}. Lima: Editorial Milla Batres.
		
		\bibitem{porras1963}
		Porras Barrenechea, R. (1963). \textit{Fuentes históricas peruanas}. Lima: Universidad Nacional Mayor de San Marcos.
		
		\bibitem{vargasugarte}
		Vargas Ugarte, R. (1953). \textit{Historia del culto de los santos en el Perú} (Vol. I). Lima: Imprenta Santa María.
						
	
	\end{thebibliography}
	
	\appendix
	\addcontentsline{toc}{section}{Apéndices}
	
\section*{Ap\'endice: Fotos Hist\'oricas}	
\begin{figure}[H]
	\centering
	\includegraphics[scale=0.2]{Pizarro_statue_Lima}
	\caption[Estatua de Pizarro - Ubicación Original]{Estatua de Pizarro Emplazamiento Original}
	\label{fig:pizarrostatuelima}
\end{figure}

\begin{figure}[H]
	\centering
	\includegraphics[width=0.4\linewidth]{"pizarro2 lima"}
	\caption[Estatua de pizarro - 2do. Emplazamiento]{Estatua de Pizarro - 2do. Emplazamiento}
	\label{fig:pizarro2-lima}
\end{figure}

\begin{figure}[H]
	\centering
	\includegraphics[scale=0.8]{ultimo-traslado.jpg}
	\caption[Estatua de pizarro -Traslado]{Estatua de Pizarro Traslado}	
\end{figure}

\begin{figure}[H]
		\centering
	\includegraphics[scale=0.4]{Estatua_de_Pizarro,_Lima_(2025)_10}
	\caption[Pizarro - Emplazamiento actual]{Pizarro - Emplazamiento actual}
	\label{fig:estatuadepizarrolima202510}
\end{figure}
	
	
\end{document}
