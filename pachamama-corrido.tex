\documentclass[12pt]{article}
\usepackage[utf8]{inputenc}
\usepackage[spanish]{babel}
\usepackage{geometry}
\usepackage{parskip}
\usepackage{hyperref}
\usepackage{setspace}
\geometry{a4paper, margin=2.5cm}
\title{La Pachamama y la Teología de la Liberación Reciclada:\\Una Crítica desde la Ortodoxia Católica}
\author{Jorge L. Ayona Inglis}
\date{\today}

\begin{document}

\maketitle
\onehalfspacing

\noindent
La aparición del símbolo de la Pachamama en contextos eclesiales recientes, particularmente durante el Sínodo para la Amazonía en 2019, ha suscitado intensos debates dentro del catolicismo contemporáneo. Uno de los cuestionamientos más sólidos sostiene que dicha representación, lejos de ser un gesto de apertura cultural, encarna una herencia ideológica de la Teología de la Liberación reformulada y adaptada a los discursos ecológicos actuales. En este sentido, la figura de la Pachamama estaría siendo instrumentalizada como vehículo de una reinterpretación del cristianismo con fuerte carga ideológica, lo que algunos han identificado como una forma contemporánea de “marxismo cultural”.

Durante los años setenta y ochenta, la Teología de la Liberación surgió como una respuesta teológica al sufrimiento de los pueblos latinoamericanos, haciendo uso del análisis marxista como herramienta de interpretación socioeconómica. A pesar de sus intentos por vincularse con la opción preferencial por los pobres, varios de sus exponentes fueron advertidos por el Magisterio sobre los riesgos de reducir el Evangelio a una praxis meramente política. Documentos como la \textit{Instrucción sobre algunos aspectos de la Teología de la Liberación} (\textit{Libertatis Nuntius}, 1984) alertaron del peligro de sustituir la revelación cristiana por ideologías seculares, y de instrumentalizar la fe como medio de lucha revolucionaria \cite{libertatis}.

Con la caída del Muro de Berlín y el descrédito del marxismo clásico, muchos de estos teólogos cambiaron de enfoque, asimilando el discurso ecológico emergente. Así nació una “ecoteología” que enfatizaba la sacralidad de la tierra, el protagonismo de los pueblos originarios y el respeto por sus símbolos ancestrales. La figura de la Pachamama, tradicionalmente ligada a cosmovisiones andinas con componentes panteístas o animistas, se incorporó como símbolo de esa nueva teología, presentándose como una “madre tierra” a la que se debe veneración. Sin embargo, esta evolución ideológica no representó una superación de los errores anteriores, sino más bien una reconfiguración de los mismos en clave ambientalista.

La inclusión de la Pachamama en actos litúrgicos o en espacios eclesiales no es meramente una cuestión estética o cultural. Al tener raíces devocionales paganas, su representación en ambientes católicos plantea un serio problema de sincretismo y confusión doctrinal. Más grave aún es su asociación simbólica con la figura de la Virgen María, lo cual contradice no solo la iconografía mariana tradicional, sino la teología que sustenta la unicidad y pureza del culto mariano dentro del catolicismo. El \textit{Catecismo de la Iglesia Católica} (n. 971) recuerda que, aunque María es venerada con amor singular, esta devoción debe siempre derivarse de su relación única con Cristo y jamás mezclarse con prácticas ajenas al cristianismo \cite{catecismo}.

El Papa Juan Pablo II, en su encíclica \textit{Redemptoris Missio}, ya había advertido sobre los riesgos de una inculturación mal entendida. Si bien reconocía la importancia de integrar los valores de las culturas locales al Evangelio, insistía en que esta integración no podía realizarse a costa de la verdad revelada, ni mucho menos mediante la adopción acrítica de símbolos que contradicen la fe cristiana (n. 52) \cite{redemptoris}.

Algunos defensores de la utilización de la Pachamama en contextos eclesiales sostienen que se trata de una expresión legítima de la diversidad cultural, en sintonía con la “ecología integral” propuesta por el Papa Francisco en \textit{Laudato Si’} (nn. 11–13). Sin embargo, este argumento no toma en cuenta que, en muchas comunidades, la Pachamama sigue siendo objeto de culto religioso, con rituales que atribuyen poder a una divinidad que no es el Dios revelado en Jesucristo \cite{laudato}. Por tanto, su inclusión simbólica en ambientes litúrgicos no solo es teológicamente impropia, sino que puede inducir a error a los fieles.

En conclusión, la utilización de la Pachamama como símbolo en actos eclesiales no representa una inculturación auténtica, sino una peligrosa forma de sincretismo teológico. Más que un puente entre culturas, se trata de una señal de que persiste una corriente ideológica que busca reinterpretar el cristianismo a la luz de categorías seculares. La verdadera inculturación exige discernimiento, fidelidad a la Revelación y respeto por la fe del pueblo cristiano, no la adaptación superficial a modas ideológicas disfrazadas de apertura cultural.

\newpage
\begin{thebibliography}{9}

\bibitem{libertatis}
Congregación para la Doctrina de la Fe. 
\textit{Libertatis Nuntius: Instrucción sobre algunos aspectos de la Teología de la Liberación}. 
Vaticano, 1984. Disponible en: \url{https://www.vatican.va}

\bibitem{redemptoris}
Juan Pablo II. 
\textit{Redemptoris Missio: Carta encíclica sobre la misión de la Iglesia}. 
Vaticano, 1990.

\bibitem{laudato}
Papa Francisco. 
\textit{Laudato Si’: Encíclica sobre el cuidado de la casa común}. 
Vaticano, 2015.

\bibitem{catecismo}
Santa Sede. 
\textit{Catecismo de la Iglesia Católica}. 2ª ed. 
Ciudad del Vaticano, 1997.

\end{thebibliography}

\end{document}
