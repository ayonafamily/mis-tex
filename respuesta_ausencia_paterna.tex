\documentclass[a4paper,12pt]{article}
\usepackage[utf8]{inputenc}
\usepackage[spanish]{babel}
\usepackage{geometry}
\geometry{a4paper, margin=1in}
\usepackage{parskip}
\usepackage{titlesec}
\usepackage{hyperref}
\usepackage{polyglossia}
\setmainfont{Arial}
\usepackage{xurl}
\titleformat{\section}{\large\bfseries}{\thesection.}{1em}{}

\title{Reflexión sobre la Ausencia Paterna y la Implicación en la Educación}
\author{\textit Jorge Ayona \\ \href{https://orcid.org/0009-0006-6551-9681}{ORCID: 0009-0006-6551-9681}\\
\href{mailto:jorge.ayona@estudiante.ucsm.edu.pe}{jorge.ayona@estudiante.ucsm.edu.pe}}
\date{\today}

\begin{document}

\maketitle

\section*{Planteamiento del problema}

En el contexto del fenómeno, cada vez más frecuente, de la ausencia paterna —que supera en incidencia a la materna—, resulta pertinente reflexionar sobre la cantidad de casos en los que los niños se ven afectados por el abandono o la separación de sus padres debido a conflictos conyugales. En este marco, cabe preguntarse también qué estrategias podrían favorecer una mayor implicación de los padres varones en la educación de sus hijos y si dicha implicación debe ser promovida exclusivamente en el ámbito privado y familiar o si, por el contrario, instituciones como la escuela y la catequesis están llamadas a desempeñar un rol activo en este proceso.



\section*{Consideraciones}

Definitivamente, este no es un asunto meramente privado, ya que tiene un impacto social profundo y afecta al bien común. Como vimos en el curso, los resultados de encuestas realizadas en países como el Reino Unido e Israel destacan la importancia vital de la presencia del padre en la crianza de los hijos.

Es cierto que existen diversas razones por las que un padre puede estar ausente: motivos laborales, enfermedad, fallecimiento, entre otros. Sin embargo, aquí nos centramos en los casos de separación y abandono por conflictos familiares. En este contexto, es importante señalar que, a nivel gubernamental, los juzgados suelen favorecer la custodia materna, relegando al padre a un rol secundario.Esto puede ser especialmente problemático cuando algunas madres utilizan a los hijos como instrumento de venganza hacia el padre, incluso en ausencia de violencia por parte de este. También se han dado casos en los que la madre ejerce violencia psicológica o emocional sobre los hijos, impidiendo el vínculo con el padre simplemente por su condición de varón.

Estas situaciones requieren una intervención a nivel legislativo. Una propuesta viable sería establecer la \textbf{tenencia compartida}, para que los hijos no se vean privados ni de la figura paterna ni de la materna tras una separación.

Desde el ámbito educativo y pastoral, también es fundamental actuar. La catequesis, por ejemplo, no debería limitarse a la enseñanza doctrinal, sino que debe incluir un acompañamiento cercano a los niños que atraviesan estas situaciones. En el caso de mi madre, aunque no vivió una separación familiar, sí perdió a su madre a temprana edad. Fue gracias al apoyo y la acogida de las religiosas de su colegio que pudo salir adelante. Esto demuestra la importancia de una comunidad educativa y pastoral comprometida.

Además, aunque una pareja no se encuentre en peligro inmediato de separación, es cierto que muchas enfrentan dificultades cotidianas que afectan la vida familiar. En mi país, por ejemplo, uno de los problemas más comunes son los \textbf{horarios laborales excesivos}. Si bien en algunos casos esto responde al deseo de los hombres de progresar en su carrera profesional, también hay una presión social e institucional que empuja al varón a definirse únicamente como proveedor económico. Sin embargo, el padre no solo debe proveer dinero, sino también afecto, estabilidad emocional y presencia. Por eso, dedicar largas horas al trabajo a costa del tiempo con los hijos es algo que debe cuestionarse.

Desde la catequesis, se pueden promover \textbf{actividades conjuntas entre padres e hijos}, así como \textbf{visitas pastorales a los hogares}, no solo para acompañar a los niños, sino para conocer y apoyar a toda la familia. Este esfuerzo debe involucrar tanto a los catequistas como a los sacerdotes.

A nivel legislativo, sería importante impulsar leyes que regulen las jornadas laborales y reduzcan la presión que muchas empresas ejercen sobre sus empleados. También se puede comenzar por \textbf{crear conciencia social} a través de publicaciones en redes sociales, programas de televisión y medios de comunicación, donde la prensa católica tiene un papel fundamental.

Finalmente, sería muy valioso que en todas las di\'ocesis y parroquias se ofrecieran \textbf{programas para parejas}, aprovechando incluso los recursos digitales actuales para brindar \textbf{cursos y acompañamiento en línea}. La tecnología puede ser una gran aliada para fortalecer los vínculos familiares y prevenir crisis.

\newpage
\section*{Conclusi\'on}

Concluimos que la ausencia paterna por separación o abandono no es un asunto meramente privado, sino un problema social que impacta profundamente en el desarrollo emocional de los hijos y en el bien común. Se ha evidenciado la necesidad de reformas legislativas que promuevan la tenencia compartida y leyes laborales que permitan a los padres estar más presentes en la vida de sus hijos. Al mismo tiempo, la escuela, la catequesis y la comunidad eclesial deben asumir un rol activo, no solo en la formación doctrinal, sino también en el acompañamiento integral de los niños y sus familias. Actividades conjuntas, visitas pastorales y programas para parejas pueden contribuir a fortalecer los lazos familiares. Finalmente, los medios de comunicación y las nuevas tecnologías pueden ser aliados estratégicos para generar conciencia y ofrecer apoyo formativo. Todo esto apunta a una meta común: una educación en la que padre y madre sean, juntos, protagonistas activos y responsables del crecimiento humano y espiritual de sus hijos.

Al respecto, perm\'itanme terminar con palabras del Papa Francisco:

\begin{quote}
	La parroquia no es una estructura caduca; precisamente porque tiene una gran plasticidad, puede tomar formas muy diversas que requieren la docilidad y la creatividad misionera del pastor y de la comunidad. [...] Si logra reformarse y adaptarse continuamente, será “la Iglesia que vive en medio de las casas de sus hijos” y esto supone que realmente esté en contacto con los hogares y con la vida del pueblo, y no se convierta en una prolija estructura separada de la gente o en un grupo de selectos que se miran a sí mismos. [...] La parroquia es presencia eclesial en el territorio, ámbito para la escucha de la Palabra, del crecimiento de la vida cristiana, del diálogo, del anuncio, de la caridad generosa, de la adoración y de la celebración. [...] Debe estar en contacto con las familias y con la vida del pueblo. 
\end{quote}
\hfill \textit{Evangelii Gaudium}, n. 28.

\begin{quote}
	“Cada Iglesia particular, porción de la Iglesia católica bajo la guía de su obispo, está llamada a la conversión misionera. Ella es el sujeto primario de la evangelización, en cuanto manifestación concreta de la única Iglesia en un lugar del mundo, y en ella, realmente está presente y actúa la Iglesia de Cristo, una, santa, católica y apostólica. Es ella la que está llamada a hacer presente y operante la Iglesia en un ámbito concreto y con todas sus acciones eclesiales.”
\end{quote}
	
	\hfill \textit Evangelii Gaudium, n. 30.

\noindent

\newpage
\begin{thebibliography}{9}
	
	\bibitem{Efesios6}
	Biblia. \textit{Carta a los Efesios 6, 1-4}. 
	"Hijos, obedeced a vuestros padres en el Señor, porque esto es justo. Honra a tu padre y a tu madre... Y vosotros, padres, no exasperéis a vuestros hijos, sino educadlos con la disciplina y la instrucción del Señor."
		
	
	\bibitem{PapaFrancisco}
	Papa Francisco. \textit{Catequesis sobre los padres y la educación de los hijos}. Audiencia General, 20 de mayo de 2015. 
	Disponible en: \url{https://www.aciprensa.com/noticias/55587/catequesis-del-papa-francisco-sobre-los-padres-y-la-educacion-de-los-hijos}
	
	\bibitem{Gravissimum}
	Concilio Vaticano II. \textit{Gravissimum Educationis}, Declaración sobre la educación cristiana, 28 de octubre de 1965. 
	Disponible en: \url{https://www.vatican.va/archive/hist_councils/ii_vatican_council/documents/vat-ii_decl_19651028_gravissimum-educationis_sp.html}
	
	\bibitem{FamiliarisConsortio}
	San Juan Pablo II. \textit{Familiaris Consortio}, Exhortación Apostólica sobre la misión de la familia cristiana en el mundo actual, 22 de noviembre de 1981. 
	Disponible en: \url{https://www.vatican.va/content/john-paul-ii/es/apost_exhortations/documents/hf_jp-ii_exh_19811122_familiaris-consortio.html}
	
	\bibitem{EvangeliiGaudium}
	Papa Francisco. \textit{Evangelii Gaudium}. Exhortación Apostólica sobre el anuncio del Evangelio en el mundo actual, 24 de noviembre de 2013. 
	Disponible en: \url{https://www.vatican.va/content/francesco/es/apost_exhortations/documents/papa-francesco_esortazione-ap_20131124_evangelii-gaudium.html}
	
	
\end{thebibliography}

\end{document}
