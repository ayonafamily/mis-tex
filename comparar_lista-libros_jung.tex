\documentclass[12pt]{article}
\usepackage[utf8]{inputenc}
\usepackage[spanish]{babel}
\usepackage[T1]{fontenc}
\usepackage{geometry}
\geometry{a4paper, margin=2.5cm}
\usepackage{tabularx}
\usepackage[table]{xcolor}
\usepackage{titlesec}
\usepackage{lmodern}
\titleformat{\section}{\normalfont\Large\bfseries}{\thesection}{1em}{}

\title{Obras de Carl Gustav Jung y su utilidad pastoral}
\author{Jorge L. Ayona Inglis}
\date{}

\begin{document}

\maketitle

\section*{Comparación de textos fundamentales de Jung}

\rowcolors{2}{gray!10}{white}
\begin{tabularx}{\textwidth}{|>{\raggedright\arraybackslash}p{3.5cm}|>{\raggedright\arraybackslash}X|>{\raggedright\arraybackslash}X|>{\raggedright\arraybackslash}p{3cm}|}
\hline
\rowcolor{gray!30}
\textbf{Obra} & \textbf{Temática Principal} & \textbf{Utilidad Pastoral / Espiritual} & \textbf{Nivel de Lectura} \\
\hline
\textbf{El hombre y sus símbolos} & Introducción clara a los símbolos, los sueños y arquetipos (Sombra, Ánima, Sí Mismo). & Permite entender lo simbólico en la experiencia de fe y en la vida espiritual. Útil para acompañamiento interior. & Accesible \\
\hline
\textbf{Tipos psicológicos} & Clasificación de las funciones psíquicas: pensamiento, sentimiento, intuición y sensación. & Comprensión de los temperamentos en clave vocacional, educativa o de dirección espiritual. & Intermedio \\
\hline
\textbf{Psicología y religión} & Reflexión sobre la religión como proceso psíquico profundo y simbólico. & Explora el papel de la fe, el mito y lo numinoso. Aporta al diálogo fe–psicología. & Avanzado \\
\hline
\textbf{Símbolos de transformación} & Análisis profundo de cómo los símbolos afectan la psique. Transición entre pulsión y espíritu. & Aporta herramientas para leer lo simbólico en textos religiosos y experiencia mística. & Avanzado \\
\hline
\textbf{Recuerdos, sueños, pensamientos} & Autobiografía espiritual y psicológica. Diálogo entre ciencia, fe y alma. & Profundamente inspirador para integrar biografía y espiritualidad. Ideal para la lectura interior. & Medio \\
\hline
\end{tabularx}

\vspace{1cm}
\section*{Sugerencia de itinerario de lectura}
\begin{itemize}
  \item \textbf{Paso 1:} \emph{El hombre y sus símbolos}
  \item \textbf{Paso 2:} \emph{Recuerdos, sueños, pensamientos}
  \item \textbf{Paso 3:} \emph{Tipos psicológicos}
  \item \textbf{Paso 4:} \emph{Psicología y religión}
  \item \textbf{Paso 5:} \emph{Símbolos de transformación}
\end{itemize}

\section*{Observación final}
Estas obras permiten enriquecer la pastoral desde una comprensión más profunda del alma, del inconsciente y de los símbolos religiosos como caminos de transformación.

\end{document}
