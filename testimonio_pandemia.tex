\documentclass[12pt,a4paper]{article}
\usepackage[utf8]{inputenc}
\usepackage[spanish]{babel}
\usepackage{geometry}
\usepackage{parskip}
\usepackage{amsmath,amssymb}
\usepackage{csquotes}
\usepackage{titlesec}
\usepackage{fancyhdr}
\usepackage{setspace}
\usepackage{lmodern}
\usepackage{tabularx}
\usepackage{booktabs}
\usepackage{array}


\geometry{top=2.54cm,left=2.54cm,right=2.54cm,bottom=2.54cm}

\setstretch{1.2}

\titleformat{\section}[block]{\Large\bfseries\filcenter}{}{1em}{}

\title{Testimonio espiritual: \emph{Gracia, discernimiento y verdad}}
\author{Jorge L. Ayona Inglis}
\date{\today}

\pagestyle{fancy}
\fancyhf{}
\rhead{\thepage}
\lhead{Testimonio espiritual}

\begin{document}
	
	\maketitle
	
	\section*{Primera parte: Lo que entregué durante la pandemia}
	
	Durante la pandemia, como muchos, pasé por momentos de incertidumbre, miedo, aislamiento y búsqueda. En medio de ese vacío, comencé a explorar caminos que parecían dar respuestas prácticas e inmediatas: afirmaciones positivas, visualización creativa, principios del \emph{Kybalion}, decretos mentales.
	Y funcionaban. Al menos por un tiempo.
	
	Yo creía —y no era algo falso del todo— que la gracia de Dios podía manifestarse también a través de estos recursos. Intentaba unir lo espiritual con lo mental, lo que me enseñaba la fe con lo que ofrecían otras corrientes. En el fondo, buscaba protegerme, sostenerme, no derrumbarme.
	
	Pero con el tiempo, algo empezó a incomodarme. Sentía una contradicción entre afirmar con fuerza: \emph{"esto se hará porque yo lo decreto"}, y la humildad del \emph{"hágase tu voluntad"} de Jesús. Entre visualizar lo que yo deseaba, y orar con la confianza de que Dios sabe lo que me conviene. Entre intentar controlar, y abandonarme.
	
	Fue entonces cuando decidí detenerme. No por miedo, sino por amor. Por fidelidad. Me di cuenta de que si verdaderamente quería vivir según Dios, debía entregar también mis formas de "espiritualidad alternativa". No porque fueran malas en sí mismas, sino porque no quería quedarme con nada que me alejara —ni un milímetro— de su voluntad.
	
	Le dije al Señor:
	\begin{displayquote}
		Aunque estas cosas funcionen, aunque surjan de mi subconsciente, de ahora en adelante solo aceptaré lo que tú me des por gracia. No quiero más que tu voluntad. No quiero vivir por proyección, sino por comunión.
	\end{displayquote}
	
	Y allí comenzó algo nuevo. No necesité renegar de lo anterior, pero sí dejarlo a sus pies. Y descubrí una paz más profunda. Una libertad que no viene del control, sino de la confianza. Una fe más limpia. Un corazón más ligero.
	
	Hoy, sé que la verdadera transformación no está en decretar el futuro, sino en dejarse transformar por Aquel que ya nos ama desde la eternidad.
	
	\section*{Segunda parte: Lo invisible no siempre es divino}
	
	En el camino espiritual que recorrí durante la pandemia y después, no solo me enfrenté a ideas y prácticas externas, sino también a experiencias internas difíciles de explicar. Fenómenos parapsicológicos —percepciones, visiones, sincronicidades, intuiciones profundas— comenzaron a manifestarse. Algunas veces se sentían como \emph{conexiones}, como si algo superior me estuviera guiando.
	
	Pero pronto comencé a preguntarme: \emph{"¿De dónde viene todo esto? ¿Es Dios... o es algo más?"}
	
	Recordé las enseñanzas del padre Oscar González Quevedo, jesuita, quien clasificaba estos fenómenos en tres fuentes:
	\begin{itemize}
		\item El ser humano mismo, con capacidades aún no comprendidas.
		\item El enemigo espiritual, disfrazado de \emph{"ángel de luz"}.
		\item Dios mismo, en su acción misteriosa y libre.
	\end{itemize}
	
	Ese discernimiento me abrió los ojos. Comprendí que no todo lo invisible es divino. Que no toda percepción o poder interior es un don del cielo. Y más aún:
	\begin{displayquote}
		Aun lo que venga del ser humano —si no se basa en la gracia, si no se rinde a la voluntad de Dios— puede llevar directamente al orgullo.
	\end{displayquote}
	
	Y el orgullo espiritual es quizás el más sutil de todos. Porque nos hace creer que estamos cerca de la luz, cuando en realidad estamos caminando hacia nosotros mismos, no hacia el Señor.
	
	Comprendí que debía entregar también esas experiencias a Dios. No porque fueran malas en sí, sino porque no quería vivir una espiritualidad paralela, una fe construida sobre poderes o emociones, sino una fe sencilla, profunda, centrada en la Cruz y la Resurrección.
	
	Le dije al Señor:
	\begin{displayquote}
		No quiero vivir por lo que veo, sino por lo que tú me enseñas. No quiero confiar en lo que siento, sino en lo que tú me revelas. No quiero seguir señales, sino tu Palabra. Aun lo extraordinario, si no viene de ti, lo rechazo.
	\end{displayquote}
	
	Y encontré paz.
	
	Hoy sé que no necesito experiencias extrañas para creer. No necesito señales. Necesito a Cristo. Y todo lo que no me lleve a Él —aunque brille, aunque impresione, aunque me hable al oído— lo dejaré a un lado. Porque la gracia es suficiente. Y solo en ella está la verdad.

\section*{Discernimiento espiritual según el P. González Quevedo y la Iglesia}

El P. Oscar González Quevedo (jesuita) enseñaba que los fenómenos parapsicológicos pueden clasificarse en tres fuentes:
\begin{table}[h!]
	\centering
	\renewcommand{\arraystretch}{1.3}
	\begin{tabularx}{\textwidth}{|>{\raggedright\arraybackslash}X|
			>{\raggedright\arraybackslash}X|
			>{\raggedright\arraybackslash}X|}
		\hline
		\textbf{Origen del fenómeno} & \textbf{Características} & \textbf{Riesgo espiritual} \\
		\hline
		Humano (psíquico) &
		Telepatía, percepción extrasensorial, proyección mental, etc. &
		Orgullo, fascinación, autoidolatría \\
		\hline
		Preternatural (demoníaco) &
		Engaño, confusión, atracción por el poder o el conocimiento prohibido &
		Desviación, opresión espiritual \\
		\hline
		Sobrenatural (divino) &
		Fruto de la gracia, orientado a la santidad, acompañado de paz y humildad &
		Ninguno si se acoge en obediencia \\
		\hline
	\end{tabularx}
	\caption{Clasificación del P. González Quevedo sobre los fenómenos espirituales}
\end{table}

Este discernimiento es clarísimo:

\begin{displayquote}
	“Aun lo que sea humano… de no basarse en la gracia divina, conduce al orgullo.”
\end{displayquote}

Eso es exactamente lo que dicen San Agustín, Santo Tomás y los grandes místicos. La elevación sin conversión, la percepción sin humildad, el poder sin cruz, la sabiduría sin caridad, no vienen de Dios.

\section*{Jesús y la tentación del poder espiritual (Mt 4)}

En el desierto, el demonio le ofreció a Jesús:
\begin{itemize}
	\item Convertir piedras en pan (poder sobre la materia),
	\item Arrojarse desde el Templo y que lo salven los ángeles (espectáculo religioso),
	\item Dominar todos los reinos (poder absoluto).
\end{itemize}

Y Jesús respondió:

\begin{displayquote}
	“No tentarás al Señor tu Dios... al Señor tu Dios adorarás y a Él solo servirás.”
\end{displayquote}

Como Jesús, también elijamos la obediencia por encima de la fascinación. Elijamos el camino estrecho, el de la fe pura.

\section*{¿Qué hacer con lo vivido? Discernimiento práctico}

No tenemos que rechazar ni demonizar todo lo vivido. Pero sí necesitamos purificarlo con la gracia:
\begin{itemize}
	\item Orar por discernimiento constante.
	\item Renunciar a toda curiosidad espiritual sin propósito cristiano.
	\item Ofrecer todo don, visión, percepción a la voluntad de Dios.
	\item Someter todo impulso “espiritual” a la verdad del Evangelio y la comunidad de la Iglesia.
\end{itemize}

\begin{displayquote}
	“No todo espíritu viene de Dios… pónganlos a prueba” (1 Jn 4,1)
\end{displayquote}

\section*{Conclusión personal y espiritual}

Renunciar incluso a lo “sorprendente” para quedarnos con lo verdadero— es lo que hicieron los santos.

\vspace{0.5em}
\noindent
\textbf{No buscaste} la “experiencia”, sino la gracia.\\
\textbf{No la} “manifestación”, sino la obediencia.\\
\textbf{No el} “poder”, sino la humildad.

\vspace{0.5em}
\noindent
Esa es la verdadera luz.

	
\end{document}
