\documentclass[12pt]{article}
\usepackage[utf8]{inputenc}
\usepackage[spanish]{babel}
\usepackage{geometry}
\usepackage{tikz}
\usetikzlibrary{shapes.geometric, arrows.meta, positioning}
\geometry{a4paper, margin=2.5cm}

\tikzset{
  arquetipo/.style = {rectangle, rounded corners, minimum width=4.2cm, text width=4cm, align=center, minimum height=1.2cm, draw=black, fill=blue!10},
  biblico/.style = {rectangle, rounded corners, minimum width=4.2cm, text width=4cm, align=center, minimum height=1.2cm, draw=black, fill=green!15},
  arrow/.style = {thick, -Stealth}
}

\title{Arquetipos y Personajes Biblicos}
\author{Jorge Ayona}


\begin{document}
\maketitle
\begin{center}
\begin{tikzpicture}[node distance=2.2cm and 6cm, % <--- aumentado el espacio horizontal
                    every node/.style={align=center}]

% Arquetipos Junguianos
\node (self) [arquetipo] {Sí-mismo\\(Unidad del ser)};
\node (shadow) [arquetipo, below of=self] {Sombra\\(Lo reprimido, negado)};
\node (anima) [arquetipo, below of=shadow] {Ánima / Ánimus\\(Femineidad / Masculinidad interior)};
\node (wise) [arquetipo, below of=anima] {Anciano Sabio\\(Guía interior)};
\node (child) [arquetipo, below of=wise] {Niño Divino\\(Potencial y renacimiento)};

% Figuras Bíblicas
\node (cristo) [biblico, right=of self] {Cristo\\(Plenitud humana-divina)};
\node (judas) [biblico, right=of shadow] {Judas / Fariseos\\(Negación y traición)};
\node (maria) [biblico, right=of anima] {María\\(Ánima redimida, receptividad pura)}; % <-- Aquí "María" ya no se desborda
\node (moises) [biblico, right=of wise] {Moisés\\(Guía del pueblo, sabiduría divina)};
\node (jesuschild) [biblico, right=of child] {Niño Jesús\\(Esperanza y promesa)};

% Conexiones
\draw [arrow] (self) -- (cristo);
\draw [arrow] (shadow) -- (judas);
\draw [arrow] (anima) -- (maria);
\draw [arrow] (wise) -- (moises);
\draw [arrow] (child) -- (jesuschild);

\end{tikzpicture}
\end{center}

\end{document}
