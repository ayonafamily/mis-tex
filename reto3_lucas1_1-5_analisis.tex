
\documentclass[a4paper,12pt]{article}
\usepackage{polyglossia}
\setdefaultlanguage{spanish}
\setotherlanguage{greek}
\newfontfamily\greekfont{FreeSerif}
\usepackage{fontspec}
\setmainfont{Arial}
\usepackage[margin=2.5cm]{geometry}
\usepackage{xcolor}
\usepackage{tabularx}
\usepackage{array}
\usepackage{titlesec}
\usepackage{hyperref}
\usepackage{fancyhdr}
\usepackage{graphicx}
\usepackage{float}
\usepackage{booktabs}
\usepackage{colortbl}
\usepackage{caption}
\usepackage{lipsum}
\usepackage{titling}

% Definiciones para tablas
\definecolor{verdelimon}{RGB}{154,205,50}
\newcolumntype{Y}{>{\raggedright\arraybackslash}X}


\begin{document}
	
\begin{titlepage}
	\begin{center}
		\huge \textbf{Universidad Católica de Santa María}\\[1cm]
		\large Facultad de Ciencias Sociales\\[3mm]
		\textbf{Escuela de Teología}\\[3mm]
		
		\begin{figure}[h]
			\centering
			\includegraphics[scale=0.15]{ucsm.jpg}
		\end{figure}
		
		\vspace{3mm}
		\textcolor{cato}{\rule{\linewidth}{0.5mm}}
		
		\vspace{4mm}
		{\large\textbf{
			El Evangelio como fuerza de integración cultural\\
			Una lectura teológica del mestizaje en el Perú a partir del relato “La fiesta de los negros”
		}}\\
		
		\vspace{5mm}
		\large{Curso: Identidad y Peruanidad}\\[2mm]
		\large{Profesora: Ana Rosario Miaury Vilca}\\[2mm]
		\large{Alumno: Jorge Ayona}\\
		\href{https://orcid.org/0009-0006-6551-9681}{ORCID: 0009-0006-6551-9681}\\[4mm]
		
		\textcolor{cato}{\rule{\linewidth}{0.5mm}}\\[2mm]
		{\large \today}
	\end{center}
\end{titlepage}
	
	
\tableofcontents
\newpage

	
%%% Versiculo 1	
\section{Texto: \textgreek{Ἐπειδήπερ πολλοὶ ἐπεχείρησαν ἀνατάξασθαι διήγησιν περὶ τῶν πεπληροφορημένων ἐν ἡμῖν πραγμάτων}}

\arrayrulecolor{verdelimon}
\begin{tabularx}{\textwidth}{|Y|Y|Y|}
\rowcolor{verdelimon}
\textcolor{white}{\textbf{Función sintáctica}} & \textcolor{white}{\textbf{Texto griego}} & \textcolor{white}{\textbf{Morfología}} \\
\hline
Conjunción causal enfática & \textgreek{Ἐπειδήπερ} & Conjunción subordinante causal \\
\hline
Sujeto del verbo principal & \textgreek{πολλοὶ} & Adjetivo sustantivado, nominativo masculino plural, 1ª declinación \\
\hline
Verbo principal & \textgreek{ἐπεχείρησαν} & Aoristo indicativo activo, 3ª persona plural de \textgreek{ἐπιχειρέω} \\
\hline
Infinitivo como complemento & \textgreek{ἀνατάξασθαι} & Infinitivo aoristo medio de \textgreek{ἀνατάσσω} \\
\hline
Complemento directo & \textgreek{διήγησιν} & Sustantivo femenino singular acusativo, 3ª declinación \\
\hline
Preposición + genitivo & \textgreek{περὶ τῶν πεπληροφορημένων} & \textbf{περὶ} rige genitivo; participio perfecto pasivo masc./neutro genitivo plural \\
\hline
Circunstancial de lugar & \textgreek{ἐν ἡμῖν} & \textbf{ἐν} rige dativo; pronombre personal 1ª persona plural dativo \\
\hline
Complemento del participio & \textgreek{πραγμάτων} & Sustantivo neutro plural genitivo, 3ª declinación \\
\hline
\end{tabularx}
\arrayrulecolor{black}

\vspace{0.5cm}
\arrayrulecolor{verdelimon}
\begin{tabularx}{\textwidth}{|Y|}
\hline
\textbf{Traducción literal:} Puesto que muchos intentaron ordenar un relato sobre las cosas plenamente confirmadas entre nosotros. \\
\hline
\textbf{Traducción fluida:} Muchos intentaron escribir un relato ordenado sobre los hechos que se han cumplido entre nosotros. \\
\hline
\end{tabularx}
\arrayrulecolor{black}

%%% Versiculo 2

\section{Texto: \textgreek{ἔδοξε κἀμοὶ παρηκολουθηκότι ἄνωθεν πᾶσιν ἀκριβῶς καθεξῆς σοι γράψαι, κράτιστε Θεόφιλε}}

\arrayrulecolor{verdelimon}
\begin{tabularx}{\textwidth}{|Y|Y|Y|}
	\rowcolor{verdelimon}
	\textcolor{white}{\textbf{Función sintáctica}} & \textcolor{white}{\textbf{Texto griego}} & \textcolor{white}{\textbf{Morfología}} \\
	\hline
	Verbo principal impersonal & \textgreek{ἔδοξε} & Aoristo indicativo activo, 3ª persona singular de \textgreek{δοκέω}: “pareció bien” \\
	\hline
	Pronombre dativo + part. perf. & \textgreek{κἀμοὶ παρηκολουθηκότι} & \textgreek{κἀμοὶ} = καὶ ἐμοὶ (dat. de interés); participio perfecto activo masc. dat. sing. de \textgreek{παρακολουθέω} \\
	\hline
	Adverbio de origen & \textgreek{ἄνωθεν} & Adverbio: “desde el principio” \\
	\hline
	Pronombre indefinido & \textgreek{πᾶσιν} & Dativo plural de \textgreek{πᾶς}: “todas las cosas” \\
	\hline
	Adverbio de modo & \textgreek{ἀκριβῶς} & Adverbio: “con precisión” \\
	\hline
	Adverbio de orden & \textgreek{καθεξῆς} & Adverbio: “en orden, sucesivamente” \\
	\hline
	Pronombre dativo & \textgreek{σοι} & Dativo singular de \textgreek{σύ}: “a ti” \\
	\hline
	Verbo infinitivo & \textgreek{γράψαι} & Aoristo infinitivo activo de \textgreek{γράφω}: “escribir” \\
	\hline
	Vocativo + nombre propio & \textgreek{κράτιστε Θεόφιλε} & “Excelentísimo Teófilo” (vocativo de respeto) \\
	\hline
\end{tabularx}
\arrayrulecolor{black}

\vspace{0.5cm}

\arrayrulecolor{verdelimon}
\begin{tabularx}{\textwidth}{|Y|}
	\hline
	\textbf{Traducción literal:} Parecióme también a mí, que he investigado desde el principio todas las cosas cuidadosamente, escribírtelas ordenadamente, excelentísimo Teófilo. \\
	\hline
	\textbf{Traducción fluida:} A mí también me pareció bien, después de haber investigado todo desde el principio con diligencia, escribirlo ordenadamente para ti, excelentísimo Teófilo. \\
	\hline
\end{tabularx}
\arrayrulecolor{black}

%%% Versiculo 3

\section{Texto: \textgreek{καθὼς παρέδοσαν ἡμῖν οἱ ἀπ’ ἀρχῆς αὐτόπται καὶ ὑπηρέται γενόμενοι τοῦ λόγου}}

\arrayrulecolor{verdelimon}
\begin{tabularx}{\textwidth}{|Y|Y|Y|}
\rowcolor{verdelimon}
\textcolor{white}{\textbf{Función sintáctica}} & \textcolor{white}{\textbf{Texto griego}} & \textcolor{white}{\textbf{Morfología}} \\
\hline
Conjunción comparativa & \textgreek{καθὼς} & Conjunción que introduce comparación o explicación \\
\hline
Verbo principal & \textgreek{παρέδοσαν} & Aoristo indicativo activo 3ª plural de \textgreek{παραδίδωμι} \\
\hline
Complemento indirecto & \textgreek{ἡμῖν} & Pronombre personal 1ª persona plural dativo \\
\hline
Sujeto del verbo & \textgreek{οἱ...αὐτόπται καὶ ὑπηρέται} & Nominativo masculino plural; testigos oculares y servidores \\
\hline
Participio circunstancial & \textgreek{γενόμενοι} & Participio aoristo medio, nominativo masculino plural \\
\hline
Complemento genitivo & \textgreek{τοῦ λόγου} & Genitivo masculino singular; complemento del participio \\
\hline
\end{tabularx}
\arrayrulecolor{black}

\vspace{0.5cm}
\arrayrulecolor{verdelimon}
\begin{tabularx}{\textwidth}{|Y|}
\hline
\textbf{Traducción literal:} Tal como nos lo transmitieron quienes desde el principio fueron testigos oculares y servidores de la palabra. \\
\hline
\textbf{Traducción fluida:} Según lo que nos transmitieron quienes desde el principio fueron testigos directos y servidores de la Palabra. \\
\hline
\end{tabularx}
\arrayrulecolor{black}

% versiculo 4

\section{Texto: \textgreek{ἵνα ἐπιγνῷς περὶ ὧν κατηχήθης λόγων τὴν ἀσφάλειαν}}

\arrayrulecolor{verdelimon}
\begin{tabularx}{\textwidth}{|Y|Y|Y|}
	\rowcolor{verdelimon}
	\textcolor{white}{\textbf{Función sintáctica}} & \textcolor{white}{\textbf{Texto griego}} & \textcolor{white}{\textbf{Morfología}} \\
	\hline
	Conjunción final & \textgreek{ἵνα} & Introduce proposición final: “para que” + subjuntivo \\
	\hline
	Verbo principal en subjuntivo & \textgreek{ἐπιγνῷς} & Aoristo subjuntivo activo, 2ª persona singular de \textgreek{ἐπιγινώσκω} — “conozcas plenamente” \\
	\hline
	Preposición + pronombre relativo & \textgreek{περὶ ὧν} & \textbf{περὶ} rige genitivo: “acerca de las cosas que” \\
	\hline
	Verbo pasivo aoristo & \textgreek{κατηχήθης} & Aoristo pasivo indicativo, 2ª persona singular de \textgreek{κατηχέω} — “fuiste instruido” \\
	\hline
	Genitivo de relación (objeto de instrucción) & \textgreek{λόγων} & Genitivo plural de \textgreek{λόγος} — “palabras, enseñanzas” \\
	\hline
	Complemento directo & \textgreek{τὴν ἀσφάλειαν} & Acusativo femenino singular: “la certeza, seguridad” \\
	\hline
\end{tabularx}
\arrayrulecolor{black}

\vspace{0.5cm}

\arrayrulecolor{verdelimon}
\begin{tabularx}{\textwidth}{|Y|}
	\hline
	\textbf{Traducción literal:} Para que conozcas con certeza las cosas acerca de las cuales fuiste instruido. \\
	\hline
	\textbf{Traducción fluida:} Para que tengas plena certeza de las enseñanzas que has recibido. \\
	\hline
\end{tabularx}
\arrayrulecolor{black}


%%% versiculo 5
\section{Texto: \textgreek{Ἐγένετο ἐν ταῖς ἡμέραις Ἡρῴδου βασιλέως τῆς Ἰουδαίας ἱερεύς τις ὀνόματι Ζαχαρίας ἐξ ἐφημερίας Ἀβιά, καὶ γυνὴ αὐτῷ ἐκ τῶν θυγατέρων Ἀαρών, καὶ τὸ ὄνομα αὐτῆς Ἐλισάβετ}}

\arrayrulecolor{verdelimon}
\begin{tabularx}{\textwidth}{|Y|Y|Y|}
	\rowcolor{verdelimon}
	\textcolor{white}{\textbf{Función sintáctica}} & \textcolor{white}{\textbf{Texto griego}} & \textcolor{white}{\textbf{Morfología}} \\
	\hline
	Verbo principal & \textgreek{Ἐγένετο} & Aoristo indicativo medio, 3ª persona singular de \textgreek{γίγνομαι} \\
	\hline
	Circunstancial de tiempo & \textgreek{ἐν ταῖς ἡμέραις} & \textbf{ἐν} rige dativo; “en los días” \\
	\hline
	Complemento del dativo & \textgreek{Ἡρῴδου βασιλέως} & Genitivo masculino singular: “de Herodes el rey” \\
	\hline
	Sujeto indefinido & \textgreek{ἱερεύς τις} & Sustantivo + indefinido nominativo masculino singular: “un sacerdote” \\
	\hline
	Nombre propio en aposición & \textgreek{ὀνόματι Ζαχαρίας} & Dativo de denominación: “llamado Zacarías” \\
	\hline
	Origen sacerdotal & \textgreek{ἐξ ἐφημερίας Ἀβιά} & \textbf{ἐξ} rige genitivo: “de la clase de Abías” \\
	\hline
	Coordinación + sujeto & \textgreek{καὶ γυνὴ αὐτῷ} & Nominativo femenino singular + dativo personal: “y su mujer” \\
	\hline
	Origen familiar & \textgreek{ἐκ τῶν θυγατέρων Ἀαρών} & \textbf{ἐκ} rige genitivo: “de las hijas de Aarón” \\
	\hline
	Nombre propio & \textgreek{τὸ ὄνομα αὐτῆς Ἐλισάβετ} & Acusativo neutro + genitivo + nombre propio: “su nombre era Elisabet” \\
	\hline
\end{tabularx}
\arrayrulecolor{black}

\vspace{0.5cm}

\arrayrulecolor{verdelimon}
\begin{tabularx}{\textwidth}{|Y|}
	\hline
	\textbf{Traducción literal:} Hubo en los días de Herodes, rey de Judea, un sacerdote llamado Zacarías, de la clase de Abías; y su mujer, de las hijas de Aarón, se llamaba Elisabet. \\
	\hline
	\textbf{Traducción fluida:} En tiempos del rey Herodes de Judea, vivía un sacerdote llamado Zacarías, de la clase de Abías; su esposa, descendiente de Aarón, se llamaba Elisabet. \\
	\hline
\end{tabularx}
\arrayrulecolor{black}



\end{document}