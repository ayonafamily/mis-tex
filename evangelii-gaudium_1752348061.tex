\documentclass[12pt]{article}

% PAQUETES
\usepackage[utf8]{inputenc}
\usepackage[T1]{fontenc}
\usepackage[spanish]{babel}
\usepackage[a4paper, margin=2.54cm]{geometry}
\usepackage{setspace}
\usepackage{parskip}
\usepackage{helvet}
\usepackage{hyperref}

\renewcommand{\familydefault}{\sfdefault}
\usepackage{csquotes}
\usepackage[style=apa, backend=biber, language=spanish]{biblatex}
\usepackage[backend=biber,style=apa]{biblatex}
\addbibresource{eg.bib}  % archivo .bib con tus referencias

% DATOS DEL DOCUMENTO
\title{\textit{Evangelii Gaudium} e identidad cultural y religiosa del Perú}

\author{Jorge Ayona\\
	{\footnotesize \protect\href{https://orcid.org/0009-0006-6551-9681}{ORCID: 0009-0006-6551-9681}}\\
	\texttt{jorge.ayona@estudiante.ucsm.edu.pe}}

\date{29 de mayo del 2025}

\begin{document}
	\maketitle
	\newpage
	\onehalfspacing
	
	\section*{Introducción}
	
	En el presente trabajo, he considerado tanto el documento de Aparecida (2007)\nocite{CELAM2007} como la exhortación apostólica \textit{Evangelii Gaudium}\nocite{Francisco2013} reflejan los aportes decisivos del entonces arzobispo de Buenos Aires, Jorge Mario Bergoglio, después Papa Francisco, de feliz memoria. En ambos documentos se reconoce su compromiso con una Iglesia en salida, encarnada en los contextos reales de América Latina, y se expresa una opción preferencial por los pobres que no cae en ideologismos, sino que busca la reconciliación entre los pueblos, culturas y sectores sociales. Ambos documentos reflejan también un enfoque teológico y pastoral que busca encarnar el Evangelio en las culturas concretas de nuestros pueblos, desde una mirada integral, misionera y dialogante.
	
	En Aparecida, el futuro Papa tuvo un rol decisivo como redactor del texto final. Allí se vislumbra ya su insistencia en una Iglesia en salida, comprometida con los más pobres, abierta al diálogo con todos y llamada a formar discípulos misioneros en medio del mundo. Esta línea se desarrollará con mayor madurez y fuerza en \textit{Evangelii Gaudium}, donde Francisco, ya como Sucesor de Pedro, presenta un programa eclesial que nace de su experiencia pastoral en América Latina, pero con resonancia universal.
	
	Francisco, en coherencia con su formación ignaciana, vivió y promovió el principio de dar por cierta, en la medida de lo posible, la proposición del prójimo, tal como lo expresa San Ignacio de Loyola en los \textit{Ejercicios Espirituales}\nocite{Ignacio1548}: 
	\begin{displayquote}
		“se ha de presuponer que todo buen cristiano ha de ser más pronto a salvar la proposición del prójimo, que a condenarla [...]” (Ignacio de Loyola, 1548/2025, Anotación 22).
	\end{displayquote}
	Es decir, entenderlo desde su mejor intención. Esta actitud es clave para cualquier proceso de reconciliación, inculturación y diálogo en contextos tan diversos y complejos como el peruano.
	
	Desde esta perspectiva, el presente trabajo busca analizar algunos núcleos fundamentales de \textit{Evangelii Gaudium} —la inculturación de la fe (nn. 68–70), la inclusión social de los pobres (nn. 186–216) y el diálogo social para la construcción de la paz (nn. 238–258)— a la luz de la realidad peruana y en diálogo con las intuiciones del Documento de Aparecida. En particular, se subrayan dos aportes clave del texto de Aparecida: el llamado a ser discípulos misioneros y la pastoral de élites, como respuestas pastorales frente a los desafíos actuales de evangelización, sin caer en reduccionismos ideológicos.
	
	\section{La inculturación de la fe y la identidad cultural}
	
	En los números 68 al 70 de \textit{Evangelii Gaudium}, el Papa Francisco insiste en que la fe cristiana no puede vivirse al margen de la cultura. Muy por el contrario, debe penetrar y transformarla desde dentro, respetando sus símbolos, su sensibilidad y su sabiduría ancestral. Como afirma el Papa citando a Juan Pablo II, “la fe que no se hace cultura es una fe no plenamente acogida, no totalmente pensada, no fielmente vivida” (EG 68). Esta afirmación resuena poderosamente en el contexto latinoamericano, y en particular en el Perú, donde conviven raíces indígenas, mestizas y occidentales que conforman una identidad compleja y viva.
	
	Esta idea coincide con lo que desarrollé en un trabajo anterior acerca del historiador británico Christopher Dawson, quien sostiene que no existen “culturas cristianas” en sentido estricto, sino culturas que acogen los valores cristianos sin dejar de ser ellas mismas (Ayona, 2025a)\nocite{Ayona2025a}. Por tanto, la inculturación es un proceso de transformación interior, no de suplantación.
	
	Este planteamiento resulta clave para comprender que la evangelización no pretende destruir lo propio de los pueblos, sino que, como recoge \textit{Evangelii Gaudium}, el anuncio cristiano no anula las culturas, sino que las fecunda desde dentro, respetando su propio dinamismo (cf. EG 69), dando frutos nuevos sin uniformidad ni colonialismo espiritual. La visión de Dawson y la enseñanza del Papa Francisco coinciden en que la gracia no suprime la cultura, sino que la supone y la transforma, como también lo expresa EG 115.
	
	Este enfoque nos desafía a mirar con otros ojos la identidad nacional del Perú, una identidad profundamente marcada por la religiosidad popular, las celebraciones sincréticas y las expresiones de fe en la vida cotidiana de los pueblos andinos y amazónicos, las cuales no siempre han sido plenamente valoradas por la Iglesia institucional. La inculturación, entonces, no es solo un método pastoral, sino una exigencia teológica y misionera que afirma que Dios ya está obrando en el corazón de las culturas, y que la tarea de la Iglesia es reconocer, purificar y potenciar esa presencia.
	
	En este sentido, Francisco nos exhorta a no caer en el reduccionismo de “importar modelos pastorales” ajenos a nuestras realidades locales. La misión exige creatividad, escucha y una profunda valoración de la sabiduría popular. Cuando la fe cristiana es verdaderamente inculturada, se convierte en fermento de dignidad, justicia y reconciliación dentro de las culturas, sin negar sus raíces ni su identidad.
	
	Sin embargo, el proceso de inculturación también implica reconocer las debilidades y deformaciones presentes en las culturas populares, como el machismo, el alcoholismo, la violencia doméstica o las creencias supersticiosas que aún persisten en muchos contextos (Francisco, 2013, p. 69)\nocite{Francisco2013}. Es responsabilidad de la Iglesia acompañar este proceso de purificación y maduración cultural para que la fe cristiana se traduzca en auténticos signos de justicia y fraternidad. Así, la inculturación no es solo un enriquecimiento cultural, sino una transformación profunda que permite superar las limitaciones sociales y espirituales de las comunidades.
	
	Finalmente, el desafío de la inculturación también se refleja en la ruptura de la transmisión generacional de la fe, que ha afectado a muchos países latinoamericanos. El Papa Francisco advierte que la falta de espacios de diálogo familiar, la influencia de medios masivos de comunicación, el relativismo y el consumismo desenfrenado, junto con una pastoral insuficiente, han llevado al desencanto y la pérdida de identidad en muchos fieles (Francisco, 2013, pp. 69-70)\nocite{Francisco2013}. Por tanto, la inculturación de la fe debe ir acompañada de una renovación pastoral que promueva una experiencia mística y comprometida con la realidad social, fortaleciendo así la identidad cristiana desde sus raíces culturales.
	
	En este marco, el Perú presenta una riqueza particular: un mestizaje cultural que no debe entenderse como pérdida de lo indígena o lo europeo, sino como una oportunidad para la integración. Este mestizaje, que autores del “boom” latinoamericano han expresado a través de lo “real maravilloso”, puede constituir una vía de reconciliación con nosotros mismos. En lugar de vivir divididos, debemos reconocer que conviven diversos modos de vida, y que todos podemos aprender unos de otros. Esta es una justa exigencia para inculturar el Evangelio.
	
	\begin{displayquote}
		“Una cultura popular evangelizada contiene valores de fe y de solidaridad que pueden provocar el desarrollo de una sociedad más justa y creyente, y posee una sabiduría peculiar que hay que saber reconocer con una mirada agradecida” (EG 68).
	\end{displayquote}
	
	Esta sabiduría popular, lejos de ser un residuo folclórico, es un lugar teológico donde el Espíritu Santo ya actúa, y donde la Iglesia debe aprender a escuchar y acompañar.
	
	\section{La inclusión social de los pobres}
	
	El número 186 a 216 de \textit{Evangelii Gaudium} se dedica a una profunda reflexión sobre la realidad de la pobreza y la injusticia social. El Papa Francisco denuncia que la exclusión social y la desigualdad económica constituyen uno de los principales escándalos de nuestra época. En este sentido, coincide plenamente con las reflexiones contenidas en el Documento de Aparecida (CELAM, 2007)\nocite{CELAM2007}, que denuncian un modelo económico que margina a amplios sectores sociales y excluye a los pobres del acceso a bienes básicos.
	
	En la línea de la Doctrina Social de la Iglesia, Francisco insiste en que la opción preferencial por los pobres no debe limitarse a un discurso, sino que requiere una transformación estructural que permita el acceso a la educación, salud, trabajo digno y participación social. En sus palabras, “los pobres no son una categoría abstracta o una cuestión para políticos o sociólogos, sino personas concretas que sufren” (Francisco, 2013, p. 187)\nocite{Francisco2013}.
	
	Asimismo, el Papa condena el “pensamiento único” que reduce la realidad social a la economía de mercado, y critica la idolatría del dinero y el capitalismo financiero, que precarizan la vida humana y destruyen el planeta (Francisco, 2013, pp. 188-193)\nocite{Francisco2013}. Por ello, llama a repensar la economía y la política desde un enfoque ético y solidario, donde la dignidad humana sea el criterio fundamental.
	
	En el contexto peruano, esta llamada adquiere una fuerza especial, dado el enorme contraste entre regiones urbanas ricas y zonas rurales pobres, la persistencia de la pobreza extrema en comunidades indígenas y campesinas, y la exclusión social que limita las posibilidades de desarrollo integral. Este fenómeno no es solo económico, sino también cultural y religioso, pues implica la marginación de formas de vida y saberes ancestrales, y a menudo un desconocimiento de la riqueza espiritual que habita en las comunidades originarias.
	
	Para afrontar estos retos, Francisco propone una Iglesia cercana a los pobres, que acompañe y empodere, que promueva la justicia social y la solidaridad efectiva, y que sea capaz de denunciar las estructuras de pecado que perpetúan la exclusión (Francisco, 2013, pp. 196-202)\nocite{Francisco2013}.
	
	A este respecto, he señalado en otro trabajo que la inculturación de la fe no puede ser una mera adaptación cosmética, sino una transformación radical que permita a los pueblos expresar su identidad cultural y religiosa en la vida cotidiana, sin caer en un folklorismo vacío (Ayona, 2025b)\nocite{Ayona2025b}. Esto es esencial para que la Iglesia pueda ser verdaderamente una comunidad de amor y justicia que refleje el rostro de Cristo en los rostros de los excluidos.
	
	\section{El diálogo social para la construcción de la paz}
	
	Los números 238 a 258 de \textit{Evangelii Gaudium} subrayan la importancia del diálogo social y político para la construcción de la paz. Francisco advierte que la ausencia de diálogo, la violencia y la corrupción socavan la convivencia y destruyen el tejido social.
	
	Este llamado coincide con la visión pastoral expresada en Aparecida (CELAM, 2007)\nocite{CELAM2007} y con el magisterio social de la Iglesia, que promueve la cultura del encuentro y el respeto a la dignidad humana como fundamentos para la paz duradera.
	
	El Papa señala que la política, entendida como el arte del bien común, debe estar al servicio de la persona humana y no de intereses particulares. Requiere la participación activa y responsable de los ciudadanos, una ética pública clara y una defensa irrestricta de los derechos humanos (Francisco, 2013, pp. 239-244)\nocite{Francisco2013}.
	
	Además, enfatiza el rol de la educación en la formación de ciudadanos comprometidos con la justicia y la paz, y la necesidad de abrir espacios de diálogo entre las diversas culturas, religiones y sectores sociales, como forma de superar la violencia y construir un futuro común.
	
	Esta perspectiva nos invita a repensar los procesos políticos y sociales del Perú, con sus tensiones étnicas, regionales y económicas, a partir de una visión que privilegia la escucha, la justicia y el respeto mutuo.
	
	\section*{Conclusión}
	
	\textit{Evangelii Gaudium} representa un llamado profundo a la conversión pastoral y cultural de la Iglesia en América Latina, invitándonos a caminar en fidelidad creativa con el Evangelio y las culturas vivas de nuestros pueblos. En diálogo con el Documento de Aparecida, se confirma la opción por una Iglesia misionera, cercana a los pobres y abierta al diálogo.
	
	El aporte de San Ignacio de Loyola en la forma de interpretar al otro y el énfasis en la inculturación y la inclusión social son claves para entender la identidad cultural y religiosa del Perú en el presente.
	
	\printbibliography
	
\end{document}
