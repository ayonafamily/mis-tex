
\documentclass[12pt]{article}

% --------------------
% PAQUETES
% --------------------
\usepackage[utf8]{inputenc}
\usepackage[T1]{fontenc}
\usepackage[spanish]{babel}
\usepackage[a4paper, margin=2.54cm]{geometry} % Márgenes APA (1 pulgada = 2.54 cm)
\usepackage{setspace} % Interlineado
\usepackage{parskip}  % No sangría, espacio entre párrafos
\usepackage{helvet}   % Usa Helvetica (muy parecida a Arial)
\renewcommand{\familydefault}{\sfdefault} % Usa fuente sans serif por defecto
\usepackage{csquotes} % Para biblatex
\usepackage[style=apa, backend=biber, language=spanish]{biblatex}
\addbibresource{bibliografia.bib} % archivo de bibliografía

% --------------------
% DATOS DEL DOCUMENTO
% --------------------
\title{¿Qué es Teología?}
\author{Jorge Ayona\\\texttt{jorge.ayona@estudiante.ucsm.edu.pe}}
\date{23 de mayo del 2025}

% --------------------
% DOCUMENTO
% --------------------
\begin{document}
\maketitle
\onehalfspacing  % Interlineado 1.5 (APA)

La palabra teología viene del griego \textit{theós} (Dios) y \textit{lógos} (palabra, discurso, razón). Etimológicamente, significa “discurso sobre Dios”. La Real Academia Española define la teología como el “conjunto de conocimientos acerca de Dios y de sus atributos y perfecciones, que se deducen de la revelación y de la razón” (RAE, 2024, s.v. teología). [...]

\printbibliography

\end{document}
