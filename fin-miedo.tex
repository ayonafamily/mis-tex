\documentclass[12pt]{article}
\usepackage[utf8]{inputenc}
\usepackage[spanish]{babel}
\usepackage{geometry}
\usepackage{parskip}
\usepackage{times}
\usepackage{hyperref}

\geometry{a4paper, margin=2.5cm}
\setlength{\parindent}{0pt}

\title{\textbf{\LARGE\centering ¿Fin del mundo o fin del miedo?\\[0.5em] \large Discerniendo los signos de los tiempos con esperanza cristiana}}
\author{\textit{Jorge L. Ayona Inglis}}
\date{}

\begin{document}

\maketitle

\section*{Introducción}

\section*{Introducción}

El concepto de \textit{sionismo} resulta central para cualquier análisis teológico, político o escatológico relacionado con el Estado moderno de Israel. El sionismo es un movimiento político, nacionalista y cultural surgido a fines del siglo XIX, cuyo objetivo principal fue el restablecimiento de un hogar nacional para el pueblo judío en la Tierra de Israel (también llamada Sion, en referencia bíblica a Jerusalén). Fundado formalmente por Theodor Herzl, el sionismo respondió al auge del antisemitismo europeo proponiendo la creación de un Estado judío soberano como solución al llamado “problema judío”. Con el tiempo, el movimiento se diversificó en varias corrientes ideológicas —religiosas, laicas, socialistas y revisionistas—, pero todas coincidieron en la aspiración de reunir al pueblo judío disperso y garantizar su autodeterminación nacional. Tras la creación del Estado de Israel en 1948, el sionismo se transformó en una fuerza orientada a preservar y fortalecer dicho Estado y su legitimidad internacional \cite{laqueur}.

Esta definición permite abordar críticamente la relación entre el proyecto sionista y las nociones bíblicas de tierra, promesa y cumplimiento profético. ¿Es el Estado moderno de Israel el cumplimiento de las promesas hechas a Abraham y los profetas? ¿O se trata de un proyecto político desvinculado del plan escatológico de Dios? Estas preguntas guiarán la reflexión que se desarrolla en el presente ensayo.

Teniendo claro lo anterior, observo que en tiempos de conflicto internacional y agitación política, muchos cristianos vuelven su mirada hacia las profecías bíblicas en busca de señales del fin. La tensión entre Israel e Irán, la invasión de Ucrania por parte de Rusia y los ecos apocalípticos en redes sociales han llevado a muchos a hablar del Armagedón, de Gog y Magog, y del cumplimiento inminente de profecías antiguas. Pero, ¿es este el fin del mundo… o el fin del miedo?

Como alguien que fue formado en un contexto evangélico, en un seminario de línea dispensacionalista, entiendo bien esta urgencia. En los años ochenta, cuando cursaba estudios teológicos en el seminario de la Alianza Cristiana y Misionera en Lima, se nos enseñaba que los eventos políticos en Medio Oriente podían y debían leerse como cumplimiento literal de las profecías. Leí entonces el libro \emph{Jesus Is Coming} de William E. Blackstone —traducido al español como \emph{Cuando Él Venga}—, en el que encontré una afirmación que me sorprendió:

\begin{quote}
“El actual retorno de los judíos no es el cumplimiento de las profecías. Están regresando en incredulidad, y el verdadero cumplimiento requiere arrepentimiento y fe en el Mesías.” \\ \emph{—William E. Blackstone, \textnormal{\emph{Cuando Él Venga (Jesus Is Coming)}, ca. cap. 7–9}}
\end{quote}

Esta afirmación, viniendo de uno de los pioneros del sionismo cristiano, no solo desafió mis presupuestos, sino que abrió una puerta a una lectura más profunda de las Escrituras.

\section*{De la literalidad a la profundidad}
Durante años adherí a la interpretación dispensacionalista clásica: una lectura literal y cronológica de los acontecimientos del fin, con énfasis en Israel como reloj profético de Dios. Pero con el tiempo, y especialmente a la luz del \emph{Catecismo de la Iglesia Católica}, descubrí que la Escritura no puede reducirse a un único sentido.

\begin{quote}
“Gracias a la unidad del designio de Dios, no sólo el texto de la Escritura, sino también la realidad y los acontecimientos de los que habla pueden ser signos.” \\ \emph{(CIC, n. 117)}
\end{quote}

La Iglesia reconoce cuatro sentidos de la Escritura: el literal, el alegórico (o tipológico), el moral y el anagógico, que apunta a las realidades últimas. A la luz de esto, comprendí que profecías como la de Ezequiel 37, sobre el valle de los huesos secos, no deben interpretarse de forma meramente geopolítica. Algunos autores veían en esa visión una representación progresiva del renacimiento de Israel: primero físico, luego espiritual. Pero incluso si hubiera elementos proféticos allí, no debemos forzar la historia a calzar en un molde escatológico prefabricado. Menos aún cuando se pierde de vista la justicia, que es la clave de todo juicio divino.

\section*{Israel y la justicia profética}
Hoy muchos intentan justificar todas las acciones del Estado de Israel como si fueran automáticamente parte del plan divino. Pero la Biblia es clara: Dios no respalda el pecado de su pueblo, aunque lo haya elegido. Las palabras de los profetas son duras cuando Israel oprime, despoja, o practica la injusticia:

\begin{quote}
“¡Ay de los que juntan casa con casa, y añaden heredad a heredad hasta ocuparlo todo! ¿Habitaréis vosotros solos en medio de la tierra?” \\ \emph{(Isaías 5,8)}
\end{quote}

San Juan Crisóstomo comenta sobre estos abusos sociales en sus homilías sobre Isaías, afirmando que:

\begin{quote}
“El profeta condena no la posesión de casas, sino el deseo de acaparar todo sin justicia, despreciando al prójimo. Tal avaricia enciende la ira de Dios más que los sacrificios lo aplacan.”
\end{quote}

De forma similar, San Ambrosio enseña:

\begin{quote}
“El que toma la heredad del pobre, el que ensancha sus campos a costa del indefenso, se hace enemigo de Dios, aunque rece con los labios.” \\ \emph{(Comentario al Salmo 118, sermón 20)}
\end{quote}

Desde esa perspectiva, la situación actual de Israel no puede leerse fuera del marco moral. El sufrimiento del pueblo palestino, el despojo de tierras y la falta de justicia, la crueldad y opresión no son ajenos al juicio profético. En ese sentido, más que cumplir una profecía, Israel podría estar repitiendo los errores que llevaron al exilio y al castigo en el Antiguo Testamento, y también haciendo a otro pueblo lo que los nazis les hacían.

\section*{Perspectivas judías críticas del sionismo}

Si bien una parte importante del mundo cristiano evangélico considera el restablecimiento del Estado de Israel en 1948 como cumplimiento profético, es importante señalar que dentro del mismo judaísmo existen posturas críticas respecto a dicha interpretación.

Grupos judíos ortodoxos, como Neturei Karta y la Edah HaChareidis, rechazan abiertamente la legitimidad teológica del Estado de Israel, sosteniendo que su establecimiento constituye una transgresión de la voluntad divina. Estos movimientos sostienen que el retorno del pueblo judío a la Tierra Prometida debe ocurrir exclusivamente por intervención mesiánica, y no por medio de acciones humanas ni de procesos políticos o militares.

Basándose en una lectura del Talmud (Ketubot 111a), argumentan que los judíos están sujetos a tres juramentos: no subir en masa a la Tierra Santa, no rebelarse contra las naciones y no provocar un odio excesivo de los pueblos. Desde esta perspectiva, cualquier intento de restablecer la soberanía nacional sin la venida del Mesías constituye una forma de rebelión contra Dios.

Esta postura subraya la diversidad de interpretaciones dentro del mismo judaísmo, y plantea un desafío a las lecturas proféticas que identifican automáticamente el Israel moderno con el cumplimiento escatológico de las promesas bíblicas (Ravitzky, 1996).



\section*{El verdadero sentido del fin}
La esperanza cristiana no se basa en identificar fechas ni mapas. El fin de los tiempos no es una excusa para el miedo, sino una llamada a la vigilancia amorosa:

\begin{quote}
“No os toca a vosotros conocer los tiempos o las sazones que el Padre puso en su sola potestad.” \\ \emph{(Hechos 1,7)}
\end{quote}

\begin{quote}
“Cuando estas cosas comiencen a suceder, erguíos y levantad vuestra cabeza, porque se acerca vuestra redención.” \\ \emph{(Lucas 21,28)}
\end{quote}

San Agustín, comentando el Apocalipsis, dice:

\begin{quote}
“No es sabio afirmar con seguridad lo que está velado; conviene más la espera paciente que la presunción.” \\ \emph{(De civitate Dei, XX, 19)}
\end{quote}

El fin del mundo no será una victoria del caos, sino de la justicia, la verdad y el amor de Dios. No estamos llamados a temer, sino a confiar. El verdadero fin que debemos desear es el fin del miedo, el fin de la injusticia, el fin del pecado que nos aparta de Dios.

\section*{Conclusión}
Hoy, a la luz de la Escritura y de la enseñanza de la Iglesia, no leo los titulares como predicciones proféticas, sino como signos que nos llaman a una conversión más profunda. Más importante que saber si estamos en Gog y Magog es preguntarnos: ¿Estoy preparado para recibir al Señor hoy?

El fin del mundo llegará… pero lo que más necesita terminar es el miedo, la injusticia y la ceguera espiritual. Que, con serenidad, esperanza y discernimiento, seamos capaces de leer los signos de los tiempos no con angustia, sino con confianza filial en el Dios que es Señor de la historia.

\section*{Posdata pastoral (23 de junio 2025):}
A la hora de cerrar este texto, se reporta un creciente éxodo de ciudadanos israelíes fuera de su territorio. Esto, que parece contradecir las visiones escatológicas triunfalistas, puede ser leído no como un fracaso del plan de Dios, sino como un llamado urgente a revisar nuestros criterios de interpretación y a renovar nuestra esperanza no en naciones, sino en el Reino que no pasa.

\newpage

\section*{Apéndice: Exégesis de 2 Pedro 1,20--21}

\textbf{Texto bíblico:}

\begin{quote}
“Pero ante todo sabed que ninguna profecía de la Escritura es de interpretación privada; porque jamás la profecía fue proferida por voluntad humana, sino que, movidos por el Espíritu Santo, hablaron los hombres de parte de Dios.” \\ \emph{(2 Pedro 1,20–21)}
\end{quote}

\textbf{Comentario exegético:}

Este pasaje subraya dos cosas: el origen divino de la profecía y la necesidad de interpretarla no según criterios privados o arbitrarios, sino guiados por el Espíritu Santo y en comunión con la Tradición viva de la Iglesia. El término griego \emph{idías epilýseōs} sugiere una interpretación no sujeta a iniciativa individual, sino a discernimiento comunitario e inspirado.

Según el Catecismo de la Iglesia Católica (nn. 113–119), la Escritura debe ser leída:

\begin{itemize}
  \item En unidad con toda la Escritura (analogia fidei),
  \item Dentro de la Tradición viva de la Iglesia,
  \item Con la asistencia del Espíritu Santo.
\end{itemize}

Así se evita que las profecías sean interpretadas de modo ideológico, sensacionalista o supersticioso. La verdadera lectura bíblica es aquella que conduce a la verdad, la esperanza y la conversión del corazón.

\begin{thebibliography}{9}

\bibitem{biblia}
Sociedades Bíblicas Unidas. (1995). \textit{La Biblia: Versión Reina-Valera Actualizada}. United Bible Societies.

\bibitem{blackstone}
Blackstone, W. E. (1908). \textit{Jesús viene} (trad. Casa Bautista de Publicaciones, 1970). Chicago: Fleming H. Revell Company.

\bibitem{agustin}
San Agustín. (2007). \textit{La ciudad de Dios} (De civitate Dei) (trad. María José Fagúndez). Madrid: Biblioteca de Autores Cristianos.

\bibitem{ambrosio}
San Ambrosio. (2006). \textit{Sobre el Espíritu Santo} (De Spiritu Sancto) (trad. Instituto Patrístico Augustinianum). Madrid: Biblioteca de Autores Cristianos.

\bibitem{ravitzky}
Ravitzky, A. (1996). \textit{Messianism, Zionism, and Jewish Religious Radicalism} (M. Swirsky \& J. Chipman, Trans.). Chicago: University of Chicago Press.

\bibitem{laqueur}
Laqueur, W. (2003). \textit{A History of Zionism: From the French Revolution to the Establishment of the State of Israel} (2nd ed.). New York: Schocken Books.


\end{thebibliography}


\end{document}
