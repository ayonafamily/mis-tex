\documentclass[a4paper12pt]{article}

    \usepackage[breakable]{tcolorbox}
    \usepackage{parskip} % Stop auto-indenting (to mimic markdown behaviour)
    
    \usepackage{tabularx}
    \usepackage[spanish]{babel}

    % Basic figure setup, for now with no caption control since it's done
    % automatically by Pandoc (which extracts ![](path) syntax from Markdown).
    \usepackage{graphicx}
    % Keep aspect ratio if custom image width or height is specified
    \setkeys{Gin}{keepaspectratio}
    % Maintain compatibility with old templates. Remove in nbconvert 6.0
    \let\Oldincludegraphics\includegraphics
    % Ensure that by default, figures have no caption (until we provide a
    % proper Figure object with a Caption API and a way to capture that
    % in the conversion process - todo).
    \usepackage{caption}
    \DeclareCaptionFormat{nocaption}{}
    \captionsetup{format=nocaption,aboveskip=0pt,belowskip=0pt}

    \usepackage{float}
    \floatplacement{figure}{H} % forces figures to be placed at the correct location
    \usepackage{xcolor} % Allow colors to be defined
    \usepackage{enumerate} % Needed for markdown enumerations to work
    \usepackage{geometry} % Used to adjust the document margins
    \usepackage{amsmath} % Equations
    \usepackage{amssymb} % Equations
    \usepackage{textcomp} % defines textquotesingle
    % Hack from http://tex.stackexchange.com/a/47451/13684:
    \AtBeginDocument{%
        \def\PYZsq{\textquotesingle}% Upright quotes in Pygmentized code
    }
    \usepackage{upquote} % Upright quotes for verbatim code
    \usepackage{eurosym} % defines \euro

    \usepackage{iftex}
    \ifPDFTeX
        \usepackage[T1]{fontenc}
        \IfFileExists{alphabeta.sty}{
              \usepackage{alphabeta}
          }{
              \usepackage[mathletters]{ucs}
              \usepackage[utf8x]{inputenc}
          }
    \else
        \usepackage{fontspec}
        \usepackage{unicode-math}
    \fi

    \usepackage{fancyvrb} % verbatim replacement that allows latex
    \usepackage{grffile} % extends the file name processing of package graphics
                         % to support a larger range
    \makeatletter % fix for old versions of grffile with XeLaTeX
    \@ifpackagelater{grffile}{2019/11/01}
    {
      % Do nothing on new versions
    }
    {
      \def\Gread@@xetex#1{%
        \IfFileExists{"\Gin@base".bb}%
        {\Gread@eps{\Gin@base.bb}}%
        {\Gread@@xetex@aux#1}%
      }
    }
    \makeatother
    \usepackage[Export]{adjustbox} % Used to constrain images to a maximum size
    \adjustboxset{max size={0.9\linewidth}{0.9\paperheight}}

    % The hyperref package gives us a pdf with properly built
    % internal navigation ('pdf bookmarks' for the table of contents,
    % internal cross-reference links, web links for URLs, etc.)
    \usepackage{hyperref}
    % The default LaTeX title has an obnoxious amount of whitespace. By default,
    % titling removes some of it. It also provides customization options.
    \usepackage{titling}
    \usepackage{longtable} % longtable support required by pandoc >1.10
    \usepackage{booktabs}  % table support for pandoc > 1.12.2
    \usepackage{array}     % table support for pandoc >= 2.11.3
    \usepackage{calc}      % table minipage width calculation for pandoc >= 2.11.1
    \usepackage[inline]{enumitem} % IRkernel/repr support (it uses the enumerate* environment)
    \usepackage[normalem]{ulem} % ulem is needed to support strikethroughs (\sout)
                                % normalem makes italics be italics, not underlines
    \usepackage{soul}      % strikethrough (\st) support for pandoc >= 3.0.0
    \usepackage{mathrsfs}
    \usepackage{hyperref}
    \usepackage{xurl}

    
    % Colors for the hyperref package
    \definecolor{urlcolor}{rgb}{0,.145,.698}
    \definecolor{linkcolor}{rgb}{.71,0.21,0.01}
    \definecolor{citecolor}{rgb}{.12,.54,.11}

    % ANSI colors
    \definecolor{ansi-black}{HTML}{3E424D}
    \definecolor{ansi-black-intense}{HTML}{282C36}
    \definecolor{ansi-red}{HTML}{E75C58}
    \definecolor{ansi-red-intense}{HTML}{B22B31}
    \definecolor{ansi-green}{HTML}{00A250}
    \definecolor{ansi-green-intense}{HTML}{007427}
    \definecolor{ansi-yellow}{HTML}{DDB62B}
    \definecolor{ansi-yellow-intense}{HTML}{B27D12}
    \definecolor{ansi-blue}{HTML}{208FFB}
    \definecolor{ansi-blue-intense}{HTML}{0065CA}
    \definecolor{ansi-magenta}{HTML}{D160C4}
    \definecolor{ansi-magenta-intense}{HTML}{A03196}
    \definecolor{ansi-cyan}{HTML}{60C6C8}
    \definecolor{ansi-cyan-intense}{HTML}{258F8F}
    \definecolor{ansi-white}{HTML}{C5C1B4}
    \definecolor{ansi-white-intense}{HTML}{A1A6B2}
    \definecolor{ansi-default-inverse-fg}{HTML}{FFFFFF}
    \definecolor{ansi-default-inverse-bg}{HTML}{000000}

    % common color for the border for error outputs.
    \definecolor{outerrorbackground}{HTML}{FFDFDF}

    % commands and environments needed by pandoc snippets
    % extracted from the output of `pandoc -s`
    \providecommand{\tightlist}{%
      \setlength{\itemsep}{0pt}\setlength{\parskip}{0pt}}
    \DefineVerbatimEnvironment{Highlighting}{Verbatim}{commandchars=\\\{\}}
    % Add ',fontsize=\small' for more characters per line
    \newenvironment{Shaded}{}{}
    \newcommand{\KeywordTok}[1]{\textcolor[rgb]{0.00,0.44,0.13}{\textbf{{#1}}}}
    \newcommand{\DataTypeTok}[1]{\textcolor[rgb]{0.56,0.13,0.00}{{#1}}}
    \newcommand{\DecValTok}[1]{\textcolor[rgb]{0.25,0.63,0.44}{{#1}}}
    \newcommand{\BaseNTok}[1]{\textcolor[rgb]{0.25,0.63,0.44}{{#1}}}
    \newcommand{\FloatTok}[1]{\textcolor[rgb]{0.25,0.63,0.44}{{#1}}}
    \newcommand{\CharTok}[1]{\textcolor[rgb]{0.25,0.44,0.63}{{#1}}}
    \newcommand{\StringTok}[1]{\textcolor[rgb]{0.25,0.44,0.63}{{#1}}}
    \newcommand{\CommentTok}[1]{\textcolor[rgb]{0.38,0.63,0.69}{\textit{{#1}}}}
    \newcommand{\OtherTok}[1]{\textcolor[rgb]{0.00,0.44,0.13}{{#1}}}
    \newcommand{\AlertTok}[1]{\textcolor[rgb]{1.00,0.00,0.00}{\textbf{{#1}}}}
    \newcommand{\FunctionTok}[1]{\textcolor[rgb]{0.02,0.16,0.49}{{#1}}}
    \newcommand{\RegionMarkerTok}[1]{{#1}}
    \newcommand{\ErrorTok}[1]{\textcolor[rgb]{1.00,0.00,0.00}{\textbf{{#1}}}}
    \newcommand{\NormalTok}[1]{{#1}}

    % Additional commands for more recent versions of Pandoc
    \newcommand{\ConstantTok}[1]{\textcolor[rgb]{0.53,0.00,0.00}{{#1}}}
    \newcommand{\SpecialCharTok}[1]{\textcolor[rgb]{0.25,0.44,0.63}{{#1}}}
    \newcommand{\VerbatimStringTok}[1]{\textcolor[rgb]{0.25,0.44,0.63}{{#1}}}
    \newcommand{\SpecialStringTok}[1]{\textcolor[rgb]{0.73,0.40,0.53}{{#1}}}
    \newcommand{\ImportTok}[1]{{#1}}
    \newcommand{\DocumentationTok}[1]{\textcolor[rgb]{0.73,0.13,0.13}{\textit{{#1}}}}
    \newcommand{\AnnotationTok}[1]{\textcolor[rgb]{0.38,0.63,0.69}{\textbf{\textit{{#1}}}}}
    \newcommand{\CommentVarTok}[1]{\textcolor[rgb]{0.38,0.63,0.69}{\textbf{\textit{{#1}}}}}
    \newcommand{\VariableTok}[1]{\textcolor[rgb]{0.10,0.09,0.49}{{#1}}}
    \newcommand{\ControlFlowTok}[1]{\textcolor[rgb]{0.00,0.44,0.13}{\textbf{{#1}}}}
    \newcommand{\OperatorTok}[1]{\textcolor[rgb]{0.40,0.40,0.40}{{#1}}}
    \newcommand{\BuiltInTok}[1]{{#1}}
    \newcommand{\ExtensionTok}[1]{{#1}}
    \newcommand{\PreprocessorTok}[1]{\textcolor[rgb]{0.74,0.48,0.00}{{#1}}}
    \newcommand{\AttributeTok}[1]{\textcolor[rgb]{0.49,0.56,0.16}{{#1}}}
    \newcommand{\InformationTok}[1]{\textcolor[rgb]{0.38,0.63,0.69}{\textbf{\textit{{#1}}}}}
    \newcommand{\WarningTok}[1]{\textcolor[rgb]{0.38,0.63,0.69}{\textbf{\textit{{#1}}}}}
    \makeatletter
    \newsavebox\pandoc@box
    \newcommand*\pandocbounded[1]{%
      \sbox\pandoc@box{#1}%
      % scaling factors for width and height
      \Gscale@div\@tempa\textheight{\dimexpr\ht\pandoc@box+\dp\pandoc@box\relax}%
      \Gscale@div\@tempb\linewidth{\wd\pandoc@box}%
      % select the smaller of both
      \ifdim\@tempb\p@<\@tempa\p@
        \let\@tempa\@tempb
      \fi
      % scaling accordingly (\@tempa < 1)
      \ifdim\@tempa\p@<\p@
        \scalebox{\@tempa}{\usebox\pandoc@box}%
      % scaling not needed, use as it is
      \else
        \usebox{\pandoc@box}%
      \fi
    }
    \makeatother

    % Define a nice break command that doesn't care if a line doesn't already
    % exist.
    \def\br{\hspace*{\fill} \\* }
    % Math Jax compatibility definitions
    \def\gt{>}
    \def\lt{<}
    \let\Oldtex\TeX
    \let\Oldlatex\LaTeX
    \renewcommand{\TeX}{\textrm{\Oldtex}}
    \renewcommand{\LaTeX}{\textrm{\Oldlatex}}
    % Document parameters
    % Document title
    
\usepackage{fontspec}
\setmainfont{Arial}
   
    \title{\vspace{5cm}Entornos Privados De Aprendizaje \\ (PLE) \\ Apuntes de Clase \vspace{2cm}}
   
    \author{Jorge Ayona \\}
   
    \date{\today}
  
    
    
    
% Pygments definitions
\makeatletter
\def\PY@reset{\let\PY@it=\relax \let\PY@bf=\relax%
    \let\PY@ul=\relax \let\PY@tc=\relax%
    \let\PY@bc=\relax \let\PY@ff=\relax}
\def\PY@tok#1{\csname PY@tok@#1\endcsname}
\def\PY@toks#1+{\ifx\relax#1\empty\else%
    \PY@tok{#1}\expandafter\PY@toks\fi}
\def\PY@do#1{\PY@bc{\PY@tc{\PY@ul{%
    \PY@it{\PY@bf{\PY@ff{#1}}}}}}}
\def\PY#1#2{\PY@reset\PY@toks#1+\relax+\PY@do{#2}}

\@namedef{PY@tok@w}{\def\PY@tc##1{\textcolor[rgb]{0.73,0.73,0.73}{##1}}}
\@namedef{PY@tok@c}{\let\PY@it=\textit\def\PY@tc##1{\textcolor[rgb]{0.24,0.48,0.48}{##1}}}
\@namedef{PY@tok@cp}{\def\PY@tc##1{\textcolor[rgb]{0.61,0.40,0.00}{##1}}}
\@namedef{PY@tok@k}{\let\PY@bf=\textbf\def\PY@tc##1{\textcolor[rgb]{0.00,0.50,0.00}{##1}}}
\@namedef{PY@tok@kp}{\def\PY@tc##1{\textcolor[rgb]{0.00,0.50,0.00}{##1}}}
\@namedef{PY@tok@kt}{\def\PY@tc##1{\textcolor[rgb]{0.69,0.00,0.25}{##1}}}
\@namedef{PY@tok@o}{\def\PY@tc##1{\textcolor[rgb]{0.40,0.40,0.40}{##1}}}
\@namedef{PY@tok@ow}{\let\PY@bf=\textbf\def\PY@tc##1{\textcolor[rgb]{0.67,0.13,1.00}{##1}}}
\@namedef{PY@tok@nb}{\def\PY@tc##1{\textcolor[rgb]{0.00,0.50,0.00}{##1}}}
\@namedef{PY@tok@nf}{\def\PY@tc##1{\textcolor[rgb]{0.00,0.00,1.00}{##1}}}
\@namedef{PY@tok@nc}{\let\PY@bf=\textbf\def\PY@tc##1{\textcolor[rgb]{0.00,0.00,1.00}{##1}}}
\@namedef{PY@tok@nn}{\let\PY@bf=\textbf\def\PY@tc##1{\textcolor[rgb]{0.00,0.00,1.00}{##1}}}
\@namedef{PY@tok@ne}{\let\PY@bf=\textbf\def\PY@tc##1{\textcolor[rgb]{0.80,0.25,0.22}{##1}}}
\@namedef{PY@tok@nv}{\def\PY@tc##1{\textcolor[rgb]{0.10,0.09,0.49}{##1}}}
\@namedef{PY@tok@no}{\def\PY@tc##1{\textcolor[rgb]{0.53,0.00,0.00}{##1}}}
\@namedef{PY@tok@nl}{\def\PY@tc##1{\textcolor[rgb]{0.46,0.46,0.00}{##1}}}
\@namedef{PY@tok@ni}{\let\PY@bf=\textbf\def\PY@tc##1{\textcolor[rgb]{0.44,0.44,0.44}{##1}}}
\@namedef{PY@tok@na}{\def\PY@tc##1{\textcolor[rgb]{0.41,0.47,0.13}{##1}}}
\@namedef{PY@tok@nt}{\let\PY@bf=\textbf\def\PY@tc##1{\textcolor[rgb]{0.00,0.50,0.00}{##1}}}
\@namedef{PY@tok@nd}{\def\PY@tc##1{\textcolor[rgb]{0.67,0.13,1.00}{##1}}}
\@namedef{PY@tok@s}{\def\PY@tc##1{\textcolor[rgb]{0.73,0.13,0.13}{##1}}}
\@namedef{PY@tok@sd}{\let\PY@it=\textit\def\PY@tc##1{\textcolor[rgb]{0.73,0.13,0.13}{##1}}}
\@namedef{PY@tok@si}{\let\PY@bf=\textbf\def\PY@tc##1{\textcolor[rgb]{0.64,0.35,0.47}{##1}}}
\@namedef{PY@tok@se}{\let\PY@bf=\textbf\def\PY@tc##1{\textcolor[rgb]{0.67,0.36,0.12}{##1}}}
\@namedef{PY@tok@sr}{\def\PY@tc##1{\textcolor[rgb]{0.64,0.35,0.47}{##1}}}
\@namedef{PY@tok@ss}{\def\PY@tc##1{\textcolor[rgb]{0.10,0.09,0.49}{##1}}}
\@namedef{PY@tok@sx}{\def\PY@tc##1{\textcolor[rgb]{0.00,0.50,0.00}{##1}}}
\@namedef{PY@tok@m}{\def\PY@tc##1{\textcolor[rgb]{0.40,0.40,0.40}{##1}}}
\@namedef{PY@tok@gh}{\let\PY@bf=\textbf\def\PY@tc##1{\textcolor[rgb]{0.00,0.00,0.50}{##1}}}
\@namedef{PY@tok@gu}{\let\PY@bf=\textbf\def\PY@tc##1{\textcolor[rgb]{0.50,0.00,0.50}{##1}}}
\@namedef{PY@tok@gd}{\def\PY@tc##1{\textcolor[rgb]{0.63,0.00,0.00}{##1}}}
\@namedef{PY@tok@gi}{\def\PY@tc##1{\textcolor[rgb]{0.00,0.52,0.00}{##1}}}
\@namedef{PY@tok@gr}{\def\PY@tc##1{\textcolor[rgb]{0.89,0.00,0.00}{##1}}}
\@namedef{PY@tok@ge}{\let\PY@it=\textit}
\@namedef{PY@tok@gs}{\let\PY@bf=\textbf}
\@namedef{PY@tok@ges}{\let\PY@bf=\textbf\let\PY@it=\textit}
\@namedef{PY@tok@gp}{\let\PY@bf=\textbf\def\PY@tc##1{\textcolor[rgb]{0.00,0.00,0.50}{##1}}}
\@namedef{PY@tok@go}{\def\PY@tc##1{\textcolor[rgb]{0.44,0.44,0.44}{##1}}}
\@namedef{PY@tok@gt}{\def\PY@tc##1{\textcolor[rgb]{0.00,0.27,0.87}{##1}}}
\@namedef{PY@tok@err}{\def\PY@bc##1{{\setlength{\fboxsep}{\string -\fboxrule}\fcolorbox[rgb]{1.00,0.00,0.00}{1,1,1}{\strut ##1}}}}
\@namedef{PY@tok@kc}{\let\PY@bf=\textbf\def\PY@tc##1{\textcolor[rgb]{0.00,0.50,0.00}{##1}}}
\@namedef{PY@tok@kd}{\let\PY@bf=\textbf\def\PY@tc##1{\textcolor[rgb]{0.00,0.50,0.00}{##1}}}
\@namedef{PY@tok@kn}{\let\PY@bf=\textbf\def\PY@tc##1{\textcolor[rgb]{0.00,0.50,0.00}{##1}}}
\@namedef{PY@tok@kr}{\let\PY@bf=\textbf\def\PY@tc##1{\textcolor[rgb]{0.00,0.50,0.00}{##1}}}
\@namedef{PY@tok@bp}{\def\PY@tc##1{\textcolor[rgb]{0.00,0.50,0.00}{##1}}}
\@namedef{PY@tok@fm}{\def\PY@tc##1{\textcolor[rgb]{0.00,0.00,1.00}{##1}}}
\@namedef{PY@tok@vc}{\def\PY@tc##1{\textcolor[rgb]{0.10,0.09,0.49}{##1}}}
\@namedef{PY@tok@vg}{\def\PY@tc##1{\textcolor[rgb]{0.10,0.09,0.49}{##1}}}
\@namedef{PY@tok@vi}{\def\PY@tc##1{\textcolor[rgb]{0.10,0.09,0.49}{##1}}}
\@namedef{PY@tok@vm}{\def\PY@tc##1{\textcolor[rgb]{0.10,0.09,0.49}{##1}}}
\@namedef{PY@tok@sa}{\def\PY@tc##1{\textcolor[rgb]{0.73,0.13,0.13}{##1}}}
\@namedef{PY@tok@sb}{\def\PY@tc##1{\textcolor[rgb]{0.73,0.13,0.13}{##1}}}
\@namedef{PY@tok@sc}{\def\PY@tc##1{\textcolor[rgb]{0.73,0.13,0.13}{##1}}}
\@namedef{PY@tok@dl}{\def\PY@tc##1{\textcolor[rgb]{0.73,0.13,0.13}{##1}}}
\@namedef{PY@tok@s2}{\def\PY@tc##1{\textcolor[rgb]{0.73,0.13,0.13}{##1}}}
\@namedef{PY@tok@sh}{\def\PY@tc##1{\textcolor[rgb]{0.73,0.13,0.13}{##1}}}
\@namedef{PY@tok@s1}{\def\PY@tc##1{\textcolor[rgb]{0.73,0.13,0.13}{##1}}}
\@namedef{PY@tok@mb}{\def\PY@tc##1{\textcolor[rgb]{0.40,0.40,0.40}{##1}}}
\@namedef{PY@tok@mf}{\def\PY@tc##1{\textcolor[rgb]{0.40,0.40,0.40}{##1}}}
\@namedef{PY@tok@mh}{\def\PY@tc##1{\textcolor[rgb]{0.40,0.40,0.40}{##1}}}
\@namedef{PY@tok@mi}{\def\PY@tc##1{\textcolor[rgb]{0.40,0.40,0.40}{##1}}}
\@namedef{PY@tok@il}{\def\PY@tc##1{\textcolor[rgb]{0.40,0.40,0.40}{##1}}}
\@namedef{PY@tok@mo}{\def\PY@tc##1{\textcolor[rgb]{0.40,0.40,0.40}{##1}}}
\@namedef{PY@tok@ch}{\let\PY@it=\textit\def\PY@tc##1{\textcolor[rgb]{0.24,0.48,0.48}{##1}}}
\@namedef{PY@tok@cm}{\let\PY@it=\textit\def\PY@tc##1{\textcolor[rgb]{0.24,0.48,0.48}{##1}}}
\@namedef{PY@tok@cpf}{\let\PY@it=\textit\def\PY@tc##1{\textcolor[rgb]{0.24,0.48,0.48}{##1}}}
\@namedef{PY@tok@c1}{\let\PY@it=\textit\def\PY@tc##1{\textcolor[rgb]{0.24,0.48,0.48}{##1}}}
\@namedef{PY@tok@cs}{\let\PY@it=\textit\def\PY@tc##1{\textcolor[rgb]{0.24,0.48,0.48}{##1}}}

\def\PYZbs{\char`\\}
\def\PYZus{\char`\_}
\def\PYZob{\char`\{}
\def\PYZcb{\char`\}}
\def\PYZca{\char`\^}
\def\PYZam{\char`\&}
\def\PYZlt{\char`\<}
\def\PYZgt{\char`\>}
\def\PYZsh{\char`\#}
\def\PYZpc{\char`\%}
\def\PYZdl{\char`\$}
\def\PYZhy{\char`\-}
\def\PYZsq{\char`\'}
\def\PYZdq{\char`\"}
\def\PYZti{\char`\~}
% for compatibility with earlier versions
\def\PYZat{@}
\def\PYZlb{[}
\def\PYZrb{]}
\makeatother


    % For linebreaks inside Verbatim environment from package fancyvrb.
    \makeatletter
        \newbox\Wrappedcontinuationbox
        \newbox\Wrappedvisiblespacebox
        \newcommand*\Wrappedvisiblespace {\textcolor{red}{\textvisiblespace}}
        \newcommand*\Wrappedcontinuationsymbol {\textcolor{red}{\llap{\tiny$\m@th\hookrightarrow$}}}
        \newcommand*\Wrappedcontinuationindent {3ex }
        \newcommand*\Wrappedafterbreak {\kern\Wrappedcontinuationindent\copy\Wrappedcontinuationbox}
        % Take advantage of the already applied Pygments mark-up to insert
        % potential linebreaks for TeX processing.
        %        {, <, #, %, $, ' and ": go to next line.
        %        _, }, ^, &, >, - and ~: stay at end of broken line.
        % Use of \textquotesingle for straight quote.
        \newcommand*\Wrappedbreaksatspecials {%
            \def\PYGZus{\discretionary{\char`\_}{\Wrappedafterbreak}{\char`\_}}%
            \def\PYGZob{\discretionary{}{\Wrappedafterbreak\char`\{}{\char`\{}}%
            \def\PYGZcb{\discretionary{\char`\}}{\Wrappedafterbreak}{\char`\}}}%
            \def\PYGZca{\discretionary{\char`\^}{\Wrappedafterbreak}{\char`\^}}%
            \def\PYGZam{\discretionary{\char`\&}{\Wrappedafterbreak}{\char`\&}}%
            \def\PYGZlt{\discretionary{}{\Wrappedafterbreak\char`\<}{\char`\<}}%
            \def\PYGZgt{\discretionary{\char`\>}{\Wrappedafterbreak}{\char`\>}}%
            \def\PYGZsh{\discretionary{}{\Wrappedafterbreak\char`\#}{\char`\#}}%
            \def\PYGZpc{\discretionary{}{\Wrappedafterbreak\char`\%}{\char`\%}}%
            \def\PYGZdl{\discretionary{}{\Wrappedafterbreak\char`\$}{\char`\$}}%
            \def\PYGZhy{\discretionary{\char`\-}{\Wrappedafterbreak}{\char`\-}}%
            \def\PYGZsq{\discretionary{}{\Wrappedafterbreak\textquotesingle}{\textquotesingle}}%
            \def\PYGZdq{\discretionary{}{\Wrappedafterbreak\char`\"}{\char`\"}}%
            \def\PYGZti{\discretionary{\char`\~}{\Wrappedafterbreak}{\char`\~}}%
        }
        % Some characters . , ; ? ! / are not pygmentized.
        % This macro makes them "active" and they will insert potential linebreaks
        \newcommand*\Wrappedbreaksatpunct {%
            \lccode`\~`\.\lowercase{\def~}{\discretionary{\hbox{\char`\.}}{\Wrappedafterbreak}{\hbox{\char`\.}}}%
            \lccode`\~`\,\lowercase{\def~}{\discretionary{\hbox{\char`\,}}{\Wrappedafterbreak}{\hbox{\char`\,}}}%
            \lccode`\~`\;\lowercase{\def~}{\discretionary{\hbox{\char`\;}}{\Wrappedafterbreak}{\hbox{\char`\;}}}%
            \lccode`\~`\:\lowercase{\def~}{\discretionary{\hbox{\char`\:}}{\Wrappedafterbreak}{\hbox{\char`\:}}}%
            \lccode`\~`\?\lowercase{\def~}{\discretionary{\hbox{\char`\?}}{\Wrappedafterbreak}{\hbox{\char`\?}}}%
            \lccode`\~`\!\lowercase{\def~}{\discretionary{\hbox{\char`\!}}{\Wrappedafterbreak}{\hbox{\char`\!}}}%
            \lccode`\~`\/\lowercase{\def~}{\discretionary{\hbox{\char`\/}}{\Wrappedafterbreak}{\hbox{\char`\/}}}%
            \catcode`\.\active
            \catcode`\,\active
            \catcode`\;\active
            \catcode`\:\active
            \catcode`\?\active
            \catcode`\!\active
            \catcode`\/\active
            \lccode`\~`\~
        }
    \makeatother

    \let\OriginalVerbatim=\Verbatim
    \makeatletter
    \renewcommand{\Verbatim}[1][1]{%
        %\parskip\z@skip
        \sbox\Wrappedcontinuationbox {\Wrappedcontinuationsymbol}%
        \sbox\Wrappedvisiblespacebox {\FV@SetupFont\Wrappedvisiblespace}%
        \def\FancyVerbFormatLine ##1{\hsize\linewidth
            \vtop{\raggedright\hyphenpenalty\z@\exhyphenpenalty\z@
                \doublehyphendemerits\z@\finalhyphendemerits\z@
                \strut ##1\strut}%
        }%
        % If the linebreak is at a space, the latter will be displayed as visible
        % space at end of first line, and a continuation symbol starts next line.
        % Stretch/shrink are however usually zero for typewriter font.
        \def\FV@Space {%
            \nobreak\hskip\z@ plus\fontdimen3\font minus\fontdimen4\font
            \discretionary{\copy\Wrappedvisiblespacebox}{\Wrappedafterbreak}
            {\kern\fontdimen2\font}%
        }%

        % Allow breaks at special characters using \PYG... macros.
        \Wrappedbreaksatspecials
        % Breaks at punctuation characters . , ; ? ! and / need catcode=\active
        \OriginalVerbatim[#1,codes*=\Wrappedbreaksatpunct]%
    }
    \makeatother

    % Exact colors from NB
    \definecolor{incolor}{HTML}{303F9F}
    \definecolor{outcolor}{HTML}{D84315}
    \definecolor{cellborder}{HTML}{CFCFCF}
    \definecolor{cellbackground}{HTML}{F7F7F7}

    % prompt
    \makeatletter
    \newcommand{\boxspacing}{\kern\kvtcb@left@rule\kern\kvtcb@boxsep}
    \makeatother
    \newcommand{\prompt}[4]{
        {\ttfamily\llap{{\color{#2}[#3]:\hspace{3pt}#4}}\vspace{-\baselineskip}}
    }
    

    
    % Prevent overflowing lines due to hard-to-break entities
    \sloppy
    % Setup hyperref package
    \hypersetup{
      breaklinks=true,  % so long urls are correctly broken across lines
      colorlinks=true,
      urlcolor=urlcolor,
      linkcolor=linkcolor,
      citecolor=citecolor,
      }
    % Slightly bigger margins than the latex defaults
    
    \geometry{verbose,tmargin=1in,bmargin=1in,lmargin=1in,rmargin=1in}
    
    

\begin{document}
	
	   
    
    \maketitle
    
    
\newpage
    
    \hypertarget{componentes-de-un-ple}{%
\section{Componentes de un PLE}\label{componentes-de-un-ple}}

El Entorno Personal de Aprendizaje, como vemos, es un término que posee
una determinada estructura de la cual se pueden desprender varias
partes. Castañeda y Adell (2010) identifican los tres primeros
componentes que se indican a continuación; el cuarto componente
relacionado con la producción se agrega desde la perspectiva del
proyecto EPA! porque se considera relevante para la verificación y
visibilización de los aprendizajes logrados por los estudiantes:

\begin{figure}
\centering
\includegraphics{Conceptualizacio.jpg}
\caption{Tabla Conceptual}
\end{figure}

    Castañeda, L. y Adell, J. (2011): El desarrollo profesional de los
docentes en entornos personales de aprendizaje (PLE). En Roig Vila, R. y
Laneve, C. (Eds.) La práctica educativa en la Sociedad de la
Información: Innovación a través de la investigación / La pratica
educativa nella Società dell'informazione: L'innovazione attraverso la
ricerca. Alcoy: Marfil. 83-95, identifican los tres primeros componentes
que se indican a continuación; el cuarto componente relacionado con la
producción se agrega desde la perspectiva del proyecto EPA! porque se
considera relevante para la verificación y visibilización de los
aprendizajes logrados por los estudiantes:

    \begin{itemize}
\tightlist
\item
  Herramientas: Los instrumentos que cada persona elige para mantener
  una comunicación con los demás en la Red personal de Aprendizaje.
\item
  Red personal de aprendizaje: Conjunto de personas con las que se
  mantiene contacto y entorno virtual en donde se comparte la
  información.-
\item
  Recursos: Se compone de toda la información que hay en Internet que
  puede ser valiosa para la propia formación. Se suelen generar por los
  miembros de las redes personales de aprendizaje con los que se ponen
  en práctica las diversas herramientas.
\item
  Producción: Comprende toda actividad creativa que contiene información
  digitalizada, y que responde a las necesidades de los diferentes
  sectores económicos.
\end{itemize}

    PLE implica, entonces, buscar, seleccionar, decidir, valorar y, en suma,
construir y reconstruir la propia red de recursos, los flujos de
información y el aprendizaje.

    \begin{figure}
\centering
\includegraphics{2.jpg}
\caption{Componentes del PLE}
\end{figure}

    \hypertarget{explorando-la-ia}{%
\section{Explorando la IA}\label{explorando-la-ia}}

    \hypertarget{ver-muxe1s-inteligencia-artificial}{%
\subsection{Ver más Inteligencia
Artificial}\label{ver-muxe1s-inteligencia-artificial}}

La inteligencia artificial generativa se basa en modelos avanzados de
aprendizaje automático que pueden crear contenido nuevo, como imágenes,
texto, música o videos, a partir de datos existentes. 1. OpenAI GPT
(ChatGPT)

o Descripción: Herramienta de procesamiento de lenguaje natural que
permite interactuar con modelos generativos como GPT.

o Enlace: ChatGPT - OpenAI

\begin{enumerate}
\def\labelenumi{\arabic{enumi}.}
\setcounter{enumi}{1}
\tightlist
\item
  Hugging Face
\end{enumerate}

o Descripción: Plataforma para explorar, usar y crear modelos de
lenguaje generativo como GPT, BERT, etc.

o Enlace: Hugging Face

\begin{enumerate}
\def\labelenumi{\arabic{enumi}.}
\setcounter{enumi}{2}
\tightlist
\item
  Google Colab
\end{enumerate}

o Descripción: Entorno de programación en línea que permite ejecutar
código Python directamente en el navegador. Ideal para explorar y
entrenar modelos generativos.

o Enlace: Google Colab

    \hypertarget{los-mejores-modelos-de-lenguaje-grandes-llm-para-2025-y-cuxf3mo-elegir-el-adecuado-para-tu-sitio}{%
\section{Los mejores modelos de lenguaje grandes (LLM) para 2025 y cómo
elegir el adecuado para tu
sitio}\label{los-mejores-modelos-de-lenguaje-grandes-llm-para-2025-y-cuxf3mo-elegir-el-adecuado-para-tu-sitio}}

Los grandes modelos de lenguaje (LLM) están emergiendo como elementos
transformadores en el campo del desarrollo web. Están haciendo que la
creación, mantenimiento y monetización de sitios web sea más accesible
para aquellos sin habilidades técnicas.

La facilidad con la que la Inteligencia Artificial (IA) puede ayudar a
los principiantes a realizar tareas complejas ha establecido a los LLMs
como herramientas esenciales para los dueños de sitios web. Sin embargo,
elegir el mejor modelo de lenguaje grande es clave.

Para simplificar este proceso, nuestro equipo de expertos ha elaborado
esta lista de grandes modelos de lenguaje, facilitándote la elección del
modelo de IA perfecto para las necesidades de tu sitio web.

Estos modelos de base pueden procesar eficazmente el feedback humano, lo
que los hace ideales para la creación de sitios web impulsados por IA.

    Tabla de Contenidos ¿Qué son los grandes modelos de lenguaje? ¿Cómo
funcionan los grandes modelos de lenguaje? 8 principales modelos de
lenguaje grandes 1. GPT 3.5 2. GPT-4 3. Gemini 4. LlaMa 5. Falcon 6.
Cohere 7. PaLM 8. Claude v1 \#\# Cómo elegir el mejor LLM para tu sitio
web \#\#\# Preguntas Frecuentes sobre LLM - ¿Cómo puedo beneficiarme al
usar LLM para la creación de sitios web? - ¿Pueden los modelos de
lenguaje como GPT-3.5 y GPT-4 ayudar a monetizar mi sitio web? - ¿Cómo
se entrenan los grandes modelos de lenguaje para ser tan poderosos? -
¿Cuáles son los modelos de lenguaje grandes más populares? - ¿Los
modelos de lenguaje grandes comprenden lo que están diciendo?

    \hypertarget{quuxe9-son-los-grandes-modelos-de-lenguaje}{%
\subsubsection{¿Qué son los grandes modelos de
lenguaje?}\label{quuxe9-son-los-grandes-modelos-de-lenguaje}}

Los modelos de lenguaje grandes son sistemas avanzados de IA que son
capaces de entender y generar lenguaje humano. Están construidos
utilizando arquitecturas complejas de redes neuronales, como los modelos
transformer, inspirados en el cerebro humano.

Estos modelos están entrenados con grandes cantidades de datos, lo que
les permite comprender el contexto y producir resultados coherentes
basados en texto, ya sea respondiendo a una pregunta o creando una
narrativa.

En términos simples, un modelo de lenguaje grande es una IA generativa
altamente avanzada que está diseñada para entender y generar lenguaje
humano.

Esta innovación está transformando cómo nos comunicamos con las
computadoras y la tecnología.

    \hypertarget{cuxf3mo-funcionan-los-grandes-modelos-de-lenguaje}{%
\subsubsection{¿Cómo funcionan los grandes modelos de
lenguaje?}\label{cuxf3mo-funcionan-los-grandes-modelos-de-lenguaje}}

Los modelos de lenguaje grandes funcionan consumiendo grandes cantidades
de información en forma de texto escrito, como libros, artículos y otros
datos de Internet. Cuanto más datos de alta calidad procesan estos
modelos de aprendizaje profundo, se vuelven mejores en comprender y
utilizar el lenguaje humano.

Echemos un vistazo más de cerca a los conceptos básicos detrás de cómo
funcionan:

\hypertarget{arquitectura}{%
\paragraph{Arquitectura}\label{arquitectura}}

La arquitectura del modelo transformer es la innovación central detrás
de los grandes modelos de lenguaje. Esta técnica de deep learning
utiliza el mecanismo de atención para ponderar la importancia de
diferentes palabras en una secuencia, permitiendo que el LLM maneje
dependencias a largo plazo entre palabras.

\hypertarget{mecanismo-de-atenciuxf3n}{%
\paragraph{Mecanismo de Atención}\label{mecanismo-de-atenciuxf3n}}

Uno de los componentes clave de la arquitectura transformer es el
mecanismo de atención, que permite al modelo centrarse en diferentes
partes del texto de entrada original al generar la salida.

Esto le permite capturar relaciones entre palabras, independientemente
de su distancia entre sí en el texto.

\hypertarget{datos-de-entrenamiento}{%
\paragraph{Datos de Entrenamiento}\label{datos-de-entrenamiento}}

Los LLMs se entrenan en enormes conjuntos de datos que contienen partes
de Internet. Esto les permite aprender no solo gramática y hechos, sino
también estilo, retórica, razonamiento e incluso cierta cantidad de
sentido común.

\hypertarget{tokens}{%
\paragraph{Tokens}\label{tokens}}

El texto se descompone en fragmentos llamados tokens, que pueden ser tan
cortos como un carácter o tan largos como una palabra. El modelo procesa
estos tokens en lotes, comprendiendo y generando lenguaje.

\hypertarget{proceso-de-entrenamiento}{%
\paragraph{Proceso de Entrenamiento}\label{proceso-de-entrenamiento}}

Pre-entrenamiento: Los LLMs primero pasan por un aprendizaje no
supervisado en enormes corpus de texto. Predicen la próxima palabra en
una secuencia, aprendiendo patrones de lenguaje, hechos e incluso
algunas habilidades de razonamiento. Reajuste: después del entrenamiento
previo, los modelos se ajustan en tareas específicas (por ejemplo,
traducción, resumen) con datos etiquetados. Este proceso de ajuste de
instrucciones personaliza el modelo para que funcione mejor en esas
tareas. Enfoque por Capas

La arquitectura transformer tiene múltiples capas, cada una compuesta
por mecanismos de atención y redes neuronales recurrentes. A medida que
la información pasa por estas capas, se vuelve cada vez más abstracta,
permitiendo que el modelo genere texto coherente y contextualmente
relevante.

\hypertarget{capacidad-generativa}{%
\paragraph{Capacidad Generativa}\label{capacidad-generativa}}

Los modelos de lenguaje grandes son generativos, lo que significa que
pueden producir texto basado en las entradas del usuario de una manera
coherente. Los patrones aprendidos del mecanismo de atención le dan a un
modelo de lenguaje grande su capacidad generativa.

\hypertarget{interactividad}{%
\paragraph{Interactividad}\label{interactividad}}

Los modelos de lenguaje grandes pueden interactuar con los usuarios en
tiempo real a través de un modelo de chatbot para generar texto basado
en prompts, responder preguntas e incluso imitar ciertos estilos de
escritura.

\hypertarget{limitaciones}{%
\subsubsection{Limitaciones:}\label{limitaciones}}

\begin{itemize}
\item
  Los LLMs realmente no \textbf{entienden} el texto. Reconocen patrones
  de sus datos de entrenamiento.
\item
  Son sensibles a la secuencia de entrada y podrían dar respuestas
  diferentes para preguntas ligeramente variadas.
\item
  No tienen la capacidad de razonar o pensar críticamente de la misma
  manera que los humanos. Basan sus respuestas en patrones observados
  durante el entrenamiento.
\end{itemize}

    \hypertarget{principales-modelos-de-lenguaje-grandes}{%
\section{8 principales modelos de lenguaje
grandes}\label{principales-modelos-de-lenguaje-grandes}}

Ahora, echemos un vistazo a los mejores modelos de lenguaje de 2025.
Cada modelo ofrece capacidades únicas que redefinen la creación de
sitios web, la monetización y los enfoques de marketing.

\begin{enumerate}
\def\labelenumi{\arabic{enumi}.}
\tightlist
\item
  GPT 3.5 Imagen del sitio de ChatGPT. El Transformer Pre-entrenado
  Generativo (GPT) 3.5, desarrollado por OpenAI, es un modelo de
  lenguaje de última generación que ha llevado el procesamiento del
  lenguaje natural (NLP) a nuevos niveles.
\end{enumerate}

Con su refinada arquitectura transformer, las redes neuronales de GPT
3.5 son capaces de entender y generar texto similar al humano, lo que
las hace excepcionalmente versátiles en diversas aplicaciones. Puede
construir frases, párrafos e incluso artículos completos con un estilo
que refleja la composición humana.

Sus inmensos datos de entrenamiento, que abarcan vastas porciones de la
web, lo equipan con diversos estilos lingüísticos y una amplia gama de
conocimientos.

\hypertarget{mejores-casos-de-uso}{%
\subparagraph{Mejores casos de uso:}\label{mejores-casos-de-uso}}

\begin{itemize}
\item
  Creación de sitios web
\item
  Creación de contenido: GPT 3.5 se destaca en producir contenido
  generado por IA para sitios web, desde publicaciones de blog y
  preguntas frecuentes hasta copias de páginas de destino personalizadas
  para tu público objetivo. Adapta hábilmente su tono y voz para a
  diversas demografías de sitios web. Optimización SEO: cuando se trata
  de optimizar el contenido web con modelos de lenguaje, GPT 3.5 es el
  más destacado. Se puede utilizar junto con herramientas de SEO de IA
  para escribir contenido que sea amigable para el lector y que esté
  optimizado para los motores de búsqueda.
\item
  Monetización:
\item
  Copy de anuncios: el éxito de los anuncios en línea a menudo se
  vincula con el copy. GPT 3.5 puede generar copys de anuncios
  persuasivos y atractivos que pueden aumentar los clics y las
  conversiones. Análisis del comportamiento del usuario: GPT 3.5 es
  principalmente un LLM de generación de texto, pero puede integrarse
  con herramientas analíticas para obtener información y ayudarte a
  deducir patrones de comportamiento del usuario.
\item
  Marketing:
\end{itemize}

Publicaciones atractivas para redes sociales: GPT 3.5 puede ayudarte a
crear publicaciones en redes sociales que capten la atención, lo que
lleva a tasas de engagement más altas. Campañas de correo electrónico:
las campañas de correo electrónico personalizadas tienen una tasa de
éxito más alta. GPT 3.5 puede automatizar la generación de contenido de
correo electrónico, adaptando cada correo a las preferencias,
comportamientos e historial de compras de cada cliente.

\begin{enumerate}
\def\labelenumi{\arabic{enumi}.}
\setcounter{enumi}{1}
\tightlist
\item
  GPT-4 Imagen del sitio web de GPT-4. GPT-4, la última iteración de la
  IA generativa de OpenAI, cuenta con mejoras drásticas sobre las
  capacidades de procesamiento de lenguaje natural de GPT 3.5.
\end{enumerate}

Al comparar el rendimiento de GPT-3.5 frente a GPT-4, es fácil ver que
GPT-4 no es simplemente una mejora lineal en el procesamiento del
lenguaje natural.

Supuestamente entrenado en un billón de parámetros, se considera el
modelo de lenguaje más grande del mercado. La diferencia es bastante
evidente. De los dos modelos GPT, GPT-4 no solo comprende y genera mejor
texto, sino que tiene la capacidad de procesar imágenes y videos, lo que
lo hace más versátil.

¡Importante! Sin embargo, vale la pena señalar que, aunque GPT-4 integra
tanto el procesamiento de datos visuales como textuales con respecto a
la entrada, solo puede generar respuestas en formato de texto.

Mejores casos de uso:

\begin{itemize}
\item
  Creación de sitios web
\item
  Creación de contenido dinámico: GPT-4 puede generar contenido de alta
  calidad y contextualmente relevante, desde artículos hasta entradas de
  blog, basado en los prompts del usuario y sus datos de entrenamiento.
  Su competencia en la traducción multilingüe permite atender sin
  esfuerzo a una audiencia global a través de contenido localizado.
\item
  Prompts de diseño:el modelo multimodal puede sugerir imágenes
  relevantes o temas visuales con el contenido que genera. Esto
  simplifica las decisiones de diseño para los desarrolladores de sitios
  web. Contenido interactivo:GPT-4 puede impulsar secciones interactivas
  de preguntas y respuestas, secciones dinámicas de preguntas frecuentes
  y chatbots de IA en sitios web para involucrar a los visitantes y
  proporcionar respuestas en tiempo real.
\item
  Monetización
\item
  Publicidad dirigida: las habilidades de GPT-4 para combinar texto
  atractivo con imágenes relevantes pueden ayudarte a crear campañas
  publicitarias cautivadoras que atraigan a los usuarios de manera
  efectiva.
\item
  Experiencias de usuario personalizadas: a través de sus vastos datos
  de entrenamiento y comprensión de las señales de texto y visuales,
  GPT-4 puede proporcionar una experiencia web altamente personalizada,
  ajustando el contenido que genera en función de los comportamientos y
  preferencias individuales del usuario.
\item
  Marketing
\item
  Colaboraciones con influencers: GPT-4 puede cambiar las reglas del
  juego para las colaboraciones de influencers. Su capacidad para crear
  contenido que se alinea tanto con la marca del influencer como con el
  negocio colaborador asegura que las campañas sean efectivas,
  auténticas y resuenen con las audiencias deseadas.
\item
  Marketing de video: GPT-4 simplifica el proceso de marketing de video
  produciendo guiones convincentes y sugiriendo elementos visuales
  efectivos. Su capacidad para crear narrativas e integrar mensajes
  clave ayuda a que el video capte la atención del espectador y logre
  sus objetivos de marketing.
\end{itemize}

Lectura sugerida

Consulta nuestra guía sobre cómo crear una página web con ChatGPT

\begin{enumerate}
\def\labelenumi{\arabic{enumi}.}
\setcounter{enumi}{2}
\tightlist
\item
  Gemini Imagen del sitio web de Gemini Gemini es un nuevo chatbot de
  LLM desarrollado por Google AI. Está entrenado en un enorme conjunto
  de datos de texto y código. Esto lo hace capaz de producir texto,
  traducir varios idiomas, crear código, generar contenido variado y
  proporcionar respuestas informativas a las preguntas.
\end{enumerate}

Gemini, uno de los principales modelos de lenguaje multimodal de gran
tamaño, también puede acceder a datos del mundo real a través de Google
Search. Esto le permite comprender y abordar un espectro más amplio de
prompts y consultas.

\hypertarget{mejores-casos-de-uso-2}{%
\subparagraph{Mejores casos de uso:}\label{mejores-casos-de-uso-2}}

\begin{itemize}
\item
  Creación de sitios web
\item
  Gráficos de alta calidad: Gemini puede generar gráficos de alta
  calidad que son relevantes para el contenido del sitio web. Estos
  gráficos se pueden utilizar para crear encabezados llamativos, botones
  de llamada a la acción y otros elementos que harán que el sitio web
  sea visualmente más atractivo.
\item
  Creación de diseños efectivos: Gemini puede analizar el contenido de
  un sitio web y los patrones de tráfico para crear un diseño que sea
  fácil de navegar. Esto puede ayudar a mejorar la experiencia del
  usuario en el sitio web y aumentar las conversiones.
\item
  Monetización
\item
  Mejora de las apariencias: usar Gemini para el diseño web puede
  optimizar el proceso creativo, permitiendo a los desarrolladores
  generar diseños responsivos e interfaces de usuario intuitivas con
  perspectivas impulsadas por la IA. Gemini también puede sugerir
  cambios de diseño adaptados al público objetivo del sitio web que
  hagan más que tomen medidas mientras navegan por tu sitio.
\item
  Marketing
\item
  Copy publicitario impulsado por IA: Gemini puede generar copy
  publicitario y materiales promocionales impulsados por IA que se
  adapten al contenido del sitio web y a la audiencia objetivo, lo que
  ayuda a aumentar la conciencia de marca, impulsar el tráfico y generar
  clientes potenciales.
\item
  Diseños efectivos: Gemini puede crear diseños efectivos para anuncios
  y materiales promocionales que sean fáciles de leer y entender. Esto
  puede ayudar a garantizar que el mensaje del anuncio sea claro y
  conciso. \#\#\#\#\#\# Lectura sugerida Consulta nuestra guía sobre
  cómo crear un sitio web con Gemini
\end{itemize}

\begin{enumerate}
\def\labelenumi{\arabic{enumi}.}
\setcounter{enumi}{3}
\tightlist
\item
  LlaMa Sitio web de LlaMa. LlaMA es un nuevo modelo de lenguaje grande
  de código abierto desarrollado por Meta AI que aún está en desarrollo.
  Está diseñado para ser un LLM versátil y potente que se puede utilizar
  para diversas tareas, incluyendo la resolución de consultas, la
  comprensión del lenguaje natural y la comprensión de lectura.
\end{enumerate}

LlaMA es el resultado del enfoque especializado de Meta en modelos de
aprendizaje de idiomas para aplicaciones educativas. Las habilidades del
LLM pueden convertirlo en un asistente de inteligencia artificial ideal
para plataformas de tecnología educativa.

\hypertarget{mejores-casos-de-uso-3}{%
\subparagraph{Mejores casos de uso:}\label{mejores-casos-de-uso-3}}

\begin{itemize}
\item
  Sitios web
\item
  Experiencia de aprendizaje personalizada: la integración de LlaMA en
  plataformas de aprendizaje de idiomas y otros sitios web de tecnología
  educativa puede ayudar a ofrecer una experiencia de tutoría
  personalizada con ejercicios interactivos. Mejor interactividad: LlaMA
  también podría usarse para generar ejercicios interactivos que ayuden
  a los estudiantes a practicar sus habilidades de gramática,
  vocabulario y comprensión. El LLM también puede ampliar estas ofertas
  para ayudar a enseñar a los estudiantes lenguajes de programación.
\item
  Monetización
\item
  Suscripción y contenido premium: los sitios web educativos pueden
  monetizar su currícula con LlaMA utilizando modelos de suscripción y
  planes de contenido premium que brinden acceso a los usuarios a
  tutorías personalizadas de LlaMA.
\item
  Marketing
\item
  Creación de contenido atractivo: LlaMa se puede utilizar para crear
  resúmenes de lecciones atractivos y contenido interactivo para
  promocionar plataformas de aprendizaje de idiomas en las redes
  sociales. Puede integrarse con la herramienta Make-A-Video de Meta
  para hacer videos cortos sobre las últimas lecciones. Su naturaleza de
  código abierto también permite una fácil integración con otras
  herramientas de IA para redes sociales para ayudar a tu marca a
  construir presencia en redes sociales.
\end{itemize}

\begin{enumerate}
\def\labelenumi{\arabic{enumi}.}
\setcounter{enumi}{4}
\tightlist
\item
  Falcon Imagen del sitio de Falcon LLM. Falcon es un modelo de lenguaje
  de código abierto desarrollado por el Instituto de Innovación
  Tecnológica. Recientemente superó a LlaMa en la tabla de clasificación
  de Open LLM de Hugging Face como el mejor modelo de lenguaje.
\end{enumerate}

Falcon es un modelo autorregresivo que se entrena con un conjunto de
datos de mayor calidad, incluida una enorme combinación de texto y
código, que cubre muchos idiomas y dialectos. También utiliza una
arquitectura más avanzada, que procesa los datos de manera más eficiente
y hace mejores predicciones.

Por lo tanto, este nuevo modelo preentrenado ha utilizado menos
parámetros para aprender (40 mil millones) que los mejores modelos de
PNL.

\hypertarget{mejores-casos-de-uso-4}{%
\subparagraph{Mejores casos de uso:}\label{mejores-casos-de-uso-4}}

\begin{itemize}
\item
  Creación de sitios web
\item
  Sitios web multilingües: usar Falcon para sitios web multilingües
  garantiza una traducción y localización sin problemas, mejorando la
  experiencia del usuario. Este modelo de deepl learning puede ser una
  herramienta valiosa para los negocios que quieren llegar a una
  audiencia global. Mejora de la comunicación en el negocio: Las
  capacidades de análisis de sentimientos de Falcon también pueden
  utilizarse para mejorar la comunicación intercultural. Al entender los
  matices de diferentes idiomas y culturas, Falcon puede ayudar a los
  negocios a comunicarse de manera efectiva con clientes y socios en
  todo el mundo.
\item
  Monetización
\item
  Mercados de nicho: el soporte multilingüe del LLM puede ayudarte a
  hacer tu sitio web disponible en mercados de nicho en idiomas locales,
  permitiéndote acceder a una nueva fuente de ingresos. Venta de espacio
  publicitario: puedes vender espacio publicitario en tu sitio web
  multilingüe a negocios que quieran llegar a una audiencia global.
\item
  Marketing
\end{itemize}

Creación de materiales de marketing localizados: puedes usar Falcon para
crear materiales de marketing localizados, como folletos, páginas de
destino y publicaciones en redes sociales adaptados a audiencias
específicas. - Marketing personalizado: las capacidades de traducción de
Falcon se pueden aprovechar para crear materiales de marketing
personalizados en función de preferencias de idioma e intereses
particulares.

\begin{enumerate}
\def\labelenumi{\arabic{enumi}.}
\setcounter{enumi}{5}
\tightlist
\item
  Cohere Imagen del sitio web de Cohere. Cohere es un modelo de lenguaje
  grande desarrollado por una startup canadiense con el mismo nombre.
  Este LLM de código abierto está entrenada en base a un conjunto de
  datos diverso e inclusivo, lo que lo convierte en un experto en
  manejar numerosos idiomas y acentos.
\end{enumerate}

Además, los modelos de Cohere están entrenados en base a un corpus de
texto grande y diverso, lo que los hace más efectivos para manejar una
amplia gama de tareas.

\hypertarget{mejores-casos-de-uso-5}{%
\subparagraph{Mejores casos de uso}\label{mejores-casos-de-uso-5}}

\begin{itemize}
\tightlist
\item
  Creación de sitios web
\end{itemize}

Colaboración efectiva en equipo: utilizar Cohere para la colaboración en
equipo agiliza los procesos de desarrollo web. Este LLM proporciona
herramientas web para la coordinación en tiempo real, el control de
versiones y la comunicación de proyectos. Siendo de código abierto y
basado en la nube, garantiza una fácil integración y amplia
accesibilidad para todos los equipos. Optimización de la creación de
contenido: Cohere se puede utilizar para optimizar el proceso de
desarrollo de contenido generando texto, traduciendo y escribiendo
diferentes tipos de contenido creativo. Esto puede ahorrar a los equipos
de desarrollo web una cantidad significativa de tiempo y esfuerzo. -
Monetización

Acceso a la página web de pago: puedes utilizar la herramienta de
procesamiento de pagos de Cohere para ofrecer diferentes niveles de
acceso a los visitantes, como un plan básico gratuito y un plan premium
por una tarifa mensual. - Servicios de suscripción: también puedes
monetizar servicios o características adicionales por un costo extra.
Esto podría incluir características como herramientas de colaboración
avanzadas, más espacio de almacenamiento o acceso a una gama más amplia
de recursos. - Marketing

\begin{itemize}
\tightlist
\item
  Contenido creativo: con Cohere, los equipos de marketing pueden crear
  contenido creativo para copy de anuncios, publicaciones en redes
  sociales y campañas de correo electrónico, mejorando el impacto de sus
  estrategias promocionales.
\item
  Personalización de contenido: el contenido puede adaptarse a distintas
  audiencias utilizando las capacidades multilingües, multi-acento y de
  análisis de sentimientos de Cohere, aumentando la relevancia y
  efectividad de cada iniciativa de marketing.
\item
  Seguimiento de la efectividad de la campaña: la API de Cohere se puede
  utilizar para integrarse con otras herramientas de marketing de IA
  para rastrear la efectividad de tus campañas de marketing. Puede
  procesar los datos de la campaña para proporcionar información más
  útil.
\end{itemize}

\begin{enumerate}
\def\labelenumi{\arabic{enumi}.}
\setcounter{enumi}{6}
\tightlist
\item
  PaLM Imagen del LLM PaLM 2. PaLM es un modelo de lenguaje grande
  desarrollado por Google AI. Este LLM se está convirtiendo en uno de
  los modelos de lenguaje de IA más poderosos, ya que tiene acceso al
  vasto conjunto de datos de Google para su entrenamiento.
\end{enumerate}

Representa un avance en el aprendizaje automático y la IA responsable.
PaLM actualmente está en desarrollo, pero ya puede entender lenguaje,
generar respuestas naturales a preguntas y ofrecer traducción
automática, generación de código, resumen y otras capacidades creativas.

PaLM también está diseñado teniendo en cuenta la privacidad y la
seguridad de los datos. Es capaz de cifrar datos y protegerlos del
acceso no autorizado. Esto lo hace ideal para proyectos sensibles, como
la construcción de sitios web de comercio electrónico seguros y
plataformas que manejan información de usuario sensible.

\hypertarget{mejores-casos-de-uso-6}{%
\subparagraph{Mejores casos de uso:}\label{mejores-casos-de-uso-6}}

\begin{itemize}
\tightlist
\item
  Creación de sitios web
\end{itemize}

Sitios de comercio electrónico: PaLM es ideal para construir sitios web
y plataformas de comercio electrónico seguros que manejan información
sensible del usuario. Este modelo de lenguaje grande puede cifrar
números de tarjetas de crédito y otros datos sensibles, así como
monitorear el tráfico del sitio web en busca de actividad sospechosa. -
Personalización de experiencias de usuario: PaLM se puede utilizar para
personalizar las experiencias de usuario en sitios web. Puede recomendar
productos a los usuarios en base a sus intereses. - Diseños creativos:
los diseñadores web pueden apoyarse en PaLM para generar diseños más
creativos para sitios web visualmente atractivos y fáciles de usar. -
Monetización:

\begin{itemize}
\tightlist
\item
  Protección de datos y privacidad: tu sitio web puede destacar que está
  utilizando PaLM para la privacidad y protección de datos. Esto puede
  ayudar a construir confianza con los usuarios y animarlos a compartir
  su información personal.
\item
  de soluciones de protección de datos y privacidad: PaLM puede ser
  utilizado para desarrollar y vender soluciones de protección de datos
  y privacidad para negocios. Estas soluciones pueden ayudar a los
  negocios a proteger sus datos de acceso no autorizado.
\item
  Marketing de la seguridad: destacar la seguridad de los sitios web
  impulsados por PaLM puede ser una estrategia de marketing clave para
  los negocios, enfatizando el cifrado y la protección contra el acceso
  no autorizado para fomentar la confianza del cliente.
\item
  Marketing:
\end{itemize}

Asociación con organizaciones de protección de datos y privacidad: al
forjar asociaciones con organizaciones de protección de datos y
privacidad, los negocios pueden reforzar la credibilidad de sus sitios,
mostrando su compromiso con la seguridad y el cumplimiento regulatorio.
Estudios de caso: elaborar estudios de caso que subrayen las ventajas de
emplear PaLM para experiencias de sitios web seguras y personalizadas
puede servir como materiales de marketing potentes para negocios y
clientes potenciales.

\begin{enumerate}
\def\labelenumi{\arabic{enumi}.}
\setcounter{enumi}{7}
\tightlist
\item
  Claude v1 Página de Claude. Claude v1 es un modelo grande de lenguaje
  desarrollado por la startup de IA americana Anthropic. Es un asistente
  de IA versátil diseñado específicamente para simplificar la creación,
  administración y optimización de sitios web.
\end{enumerate}

Con sus avanzadas capacidades de lenguaje natural, Claude v1 facilita a
cualquiera la construcción, ejecución y crecimiento de un sitio web sin
necesidad de habilidades técnicas avanzadas.

Claude utiliza una arquitectura más avanzada que otros LLMs, lo que le
permite procesar información de manera más eficiente y hacer mejores
predicciones.

\hypertarget{mejores-casos-de-uso-7}{%
\subparagraph{Mejores casos de uso:}\label{mejores-casos-de-uso-7}}

\begin{itemize}
\item
  Creación de sitios web
\item
  Gestión automatizada: Claude v1 simplifica la gestión de sitios web al
  automatizar tareas tediosas, permitiendo a los dueños de sitios
  concentrarse en estrategias de nivel superior y en la creación de
  contenido de marketing.
\item
  Creación de contenido: puede generar de forma autónoma artículos
  frescos basados en temas clave, responder a las consultas de los
  clientes utilizando sus avanzadas capacidades de conversación y
  proporcionar análisis en tiempo real sin necesidad de revisar
  manualmente los paneles de control.
\item
  SEO: Claude v1 puede manejar la optimización técnica para ofrecer
  mejoras de SEO y mejoras de velocidad del sitio en segundo plano.
  Recomendará e implementará cambios para mejorar el rendimiento del
  sitio.
\item
  Monetización
\item
  Engagement del cliente: Claude v1 puede transformar la monetización
  del sitio maximizando la participación del cliente. Al analizar los
  comportamientos de los visitantes, el modelo de IA puede entregar
  contenido personalizado, optimizar sugerencias de productos para
  plataformas de comercio electrónico y seleccionar artículos que
  resuenen con cada visitante.
\item
  Personalización de anuncios: Claude v1 también puede seleccionar
  anuncios adaptados a las demografías y comportamientos de los
  visitantes para optimizar los ingresos por publicidad. Sus capacidades
  de personalización pueden ayudar a mejorar la retención de clientes,
  amplificando los ingresos de las ventas, membresías y publicidad.
\item
  Marketing
\item
  Optimización de campañas: el modelo base no solo puede identificar
  segmentos de audiencia ideales, sino también optimizar automáticamente
  las campañas para un rendimiento máximo. En términos de SEO, también
  puede crear contenido alineado con los términos de búsqueda
  principales.
\item
  Email marketing: también puedes automatizar campañas de marketing por
  correo electrónico utilizando la capacidad de Claude para segmentar
  automáticamente los contactos y desplegar mensajes de correo
  electrónico en función del comportamiento, mejorando la participación
  del usuario. Mejora de páginas de destino: Claude v1 puede crear y
  perfeccionar páginas de destino de forma autónoma utilizando pruebas
  A/B para obtener mejores conversiones.
\end{itemize}

Lectura sugerida * 26 herramientas para programar con inteligencia
artificial*

\hypertarget{cuxf3mo-elegir-el-mejor-llm-para-tu-sitio-web}{%
\subsubsection{Cómo elegir el mejor LLM para tu sitio
web}\label{cuxf3mo-elegir-el-mejor-llm-para-tu-sitio-web}}

Para optimizar tu sitio web, es crucial seleccionar el modelo de
lenguaje grande adecuado. Así es cómo:

\begin{itemize}
\tightlist
\item
  Integración con hosting
\end{itemize}

El rendimiento y éxito de los sitios web con grandes modelos de lenguaje
están fundamentalmente vinculados a la infraestructura subyacente. Los
servicios de hosting de Hostinger están específicamente optimizados para
sitios web impulsados por IA con necesidades computacionales exigentes.

Hostinger también ofrece un conjunto de características de IA,
incluyendo el creador de páginas web de Hostinger, el creador de
logotipos y el escritor, que hacen que el proceso de creación de sitios
web sea optimizado y amigable para principiantes.

    \hypertarget{rendimiento-y-capacidades}{%
\subsection{Rendimiento y capacidades}\label{rendimiento-y-capacidades}}



\begin{longtable}[]{@{}
  >{\raggedright\arraybackslash}p{(\columnwidth - 10\tabcolsep) * \real{0.0779}}
  >{\raggedright\arraybackslash}p{(\columnwidth - 10\tabcolsep) * \real{0.2273}}
  >{\raggedright\arraybackslash}p{(\columnwidth - 10\tabcolsep) * \real{0.1688}}
  >{\raggedright\arraybackslash}p{(\columnwidth - 10\tabcolsep) * \real{0.1429}}
  >{\raggedright\arraybackslash}p{(\columnwidth - 10\tabcolsep) * \real{0.2403}}
  >{\raggedright\arraybackslash}p{(\columnwidth - 10\tabcolsep) * \real{0.1429}}@{}}
\toprule\noalign{}
\begin{minipage}[b]{\linewidth}\raggedright
LLM
\end{minipage} & \begin{minipage}[b]{\linewidth}\raggedright
Procesamiento de Lenguaje Natural
\end{minipage} & \begin{minipage}[b]{\linewidth}\raggedright
Generación de contenido
\end{minipage} & \begin{minipage}[b]{\linewidth}\raggedright
Soporte multilingüe
\end{minipage} & \begin{minipage}[b]{\linewidth}\raggedright
Facilita la colaboración en equipo
\end{minipage} & \begin{minipage}[b]{\linewidth}\raggedright
Privacidad de datos
\end{minipage} \\
\midrule\noalign{}
\endhead
\bottomrule\noalign{}
\endlastfoot
GPT -- 3.5 & Bueno & Muy Bueno & Bueno & A través de API & Regular \\
GPT -- 4 & Excelente & Excelente & Excelente & A través de API &
Regular \\
Gemini & Excelente & Excelente & Excelente & A través de API & Bueno \\
LlaMa & Muy Bueno & Muy Bueno & Excelente & Directamente & Regular \\
Falcon & Muy Bueno & Excelente & Excelente & Directamente & Bueno \\
Cohere & Excelente & Excelente & Muy Bueno & Directamente & Bueno \\
PaLM & Excelente & Bueno & Muy Bueno & A través de API & Excelente \\
Claude v1 & Excelente & Excelente & Muy Bueno & A través de API & \\
\end{longtable}

    \hypertarget{costo-y-asequibilidad}{%
\subsection{Costo y asequibilidad}\label{costo-y-asequibilidad}}

Ahora profundicemos en las consideraciones de costo y asequibilidad para
tu LLM:

\begin{itemize}
\tightlist
\item
  GPT-3.5: desde \$0,002/1000 tokens , equivalente a aproximadamente 750
  palabras
\item
  GPT-4: desde \$0,03/1000 tokens
\item
  Gemini: gratis
\item
  Llama: gratis
\item
  Falcon: gratis
\item
  Cohere: desde \$0,4/1 millón de tokens
\item
  PaLM: vista previa pública gratuita, las versiones pagas se anunciarán
  más cerca de la disponibilidad general.
\item
  Claude v1: a partir de \$1,63/millón de tokens para Prompt y
  \$5,51/millón de tokens para Completion
\end{itemize}

\hypertarget{conclusiuxf3n}{%
\subsection{Conclusión}\label{conclusiuxf3n}}

Tener el mejor modelo de lenguaje grande a tu disposición es esencial
para garantizar un funcionamiento efectivo en tu sitio. Dado que algunos
de los LLMs discutidos aún están en desarrollo, este artículo también te
guió a través de cómo se entrenan los grandes modelos de lenguaje.

Este conocimiento te ayudará a tomar una decisión más informada al
introducir modelos de lenguaje en tu proceso de desarrollo de sitios
web.

Estas son nuestras recomendaciones sobre los mejores LLMs para tu sitio
web:

Sitios web pequeños: como los blogs, pueden funcionar bien con un LLM
como GPT-3.5, que puede generar contenido de manera asequible. También
se puede utilizar para una tarea específica, como responder preguntas y
traducir idiomas. Sitios web medianos: pueden beneficiarse de LLMs más
avanzados, como GPT-4 o Gemini. Son más potentes que GPT-3.5 y se pueden
utilizar para tareas más complejas. Sitios web grandes: pueden ser más
útiles los LLMs de código abierto, como LlaMA, Falcon o Cohere. Estos
pueden facilitar la personalización y automatización de la experiencia
del sitio web para mejorar la comodidad del visitante. Finalmente, el
mejor LLM para tu sitio web dependerá de tu presupuesto, tus necesidades
y el tipo de tu sitio web. Si estás atrapado entre dos LLM, siempre
puedes probar cada uno individualmente y elegir el que mejor se adapte a
ti.

Si conoces otros LLMs que sean capaces de competir con los grandes
jugadores mencionados anteriormente, cuéntanos en la sección de
comentarios a continuación.

    \hypertarget{preguntas-frecuentes-sobre-llm}{%
\subsection{Preguntas Frecuentes sobre
LLM}\label{preguntas-frecuentes-sobre-llm}}

Abordemos algunas consultas frecuentes sobre modelos de lenguaje
grandes.

\begin{itemize}
\tightlist
\item
  ¿Cómo puedo beneficiarme al usar LLM para la creación de sitios web?
  Los modelos de lenguaje grandes (LLMs) ofrecen una amplia gama de
  beneficios para la creación de sitios web. Pueden generar contenido de
  alta calidad, incluyendo publicaciones de blog, artículos y
  descripciones de productos. Además, pueden atender las consultas de
  los visitantes, mejorando la experiencia del usuario y potencialmente
  aumentando el tráfico del sitio.
\end{itemize}

Para un alcance global, los LLM facilitan la traducción de sitios web a
varios idiomas. Además, pueden mejorar la participación del usuario
personalizando las experiencias del sitio web, sugiriendo productos o
contenido adaptado a los intereses de los visitantes.

\begin{itemize}
\tightlist
\item
  ¿Pueden los modelos de lenguaje como GPT-3.5 y GPT-4 ayudar a
  monetizar mi sitio web?
\end{itemize}

Modelos de lenguaje grandes como GPT-3.5 y GPT-4 son revolucionarios
cuando se trata de la monetización de sitios web con IA. Pueden producir
contenido convincente, atrayendo visitantes y prolongando su
participación en el sitio. Además, permiten la creación de anuncios
precisos con mayor potencial de clics. También aumentan las ventas
personalizando la experiencia del usuario para cada visitante.

\begin{itemize}
\tightlist
\item
  ¿Cómo se entrenan los grandes modelos de lenguaje para ser tan
  poderosos?
\end{itemize}

Los modelos de lenguaje grandes se entrenan utilizando enormes conjuntos
de datos que contienen miles de millones de palabras a través de un
proceso llamado modelado de lenguaje. Este proceso permite a los LLMs
aprender las relaciones estadísticas entre palabras y frases. Los
inmensos volúmenes de datos y el refinamiento iterativo del
entrenamiento los hacen expertos en entender patrones de lenguaje
intrincados para generar texto coherente y gramaticalmente correcto.

-¿Cuáles son los modelos de lenguaje grandes más populares?

Algunos de los principales grandes modelos de lenguaje incluyen GPT-3.5,
GPT-4, Gemini, Cohere, PaLM y Claude v1. Estos LLMs destacan en tareas
como la generación de texto, la traducción de idiomas, la creación de
contenido creativo, la respuesta a consultas y la generación de código.

\begin{itemize}
\tightlist
\item
  ¿Los modelos de lenguaje grandes comprenden lo que están diciendo?
\end{itemize}

Aunque los modelos de lenguaje grandes pueden generar textos relevantes
y gramaticalmente correctos, realmente no ``entienden'' el contenido
como lo hacen los humanos. Su fortaleza radica en reconocer patrones de
grandes conjuntos de datos, no en la comprensión genuina.

    % Add a bibliography block to the postdoc
    
    
    
\end{document}
