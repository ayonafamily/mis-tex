\documentclass[12pt]{article}
\usepackage[a4paper, margin=2.5cm]{geometry}
\usepackage[utf8]{inputenc}
\usepackage[T1]{fontenc}
\usepackage[spanish]{babel}
\usepackage{lmodern}
\usepackage{setspace}
\usepackage{titlesec}
\usepackage{indentfirst}
\usepackage{csquotes}
\usepackage{enumitem}
\usepackage{hyperref}


% Configuración de encabezados
\titleformat{\section}
{\normalfont\Large\bfseries}{\thesection}{1em}{}
\titleformat{\subsection}
{\normalfont\large\bfseries}{\thesubsection}{1em}{}

% Espaciado entre párrafos
\setlength{\parskip}{1em}
\setlength{\parindent}{0pt}

% Título principal
\title{Reflexiones sobre la inteligencia artificial, el libre albedrío y la dignidad humana en la cultura popular}
\author{Jorge L. Ayona Inglis\\\href{https://orcid.org/0009-0006-6551-9681}{ORCID: 0009-0006-6551-9681}}

\date{30 de junio de 2025}

\begin{document}
	
	\maketitle
	
	\section*{Introducción}
	En una época en la que la inteligencia artificial (IA) se convierte en un elemento central de nuestras sociedades, surgen preguntas fundamentales sobre su relación con la humanidad, la libertad, el alma y la dignidad. Películas como \textit{A.I. Artificial Intelligence}, la saga de \textit{Terminator} y \textit{RoboCop} nos permiten explorar estos dilemas desde una narrativa simbólica y, a veces, profética. A través de estos relatos cinematográficos y reflexiones filosófico-teológicas, se ofrece una mirada sobre lo que verdaderamente nos hace humanos.
	
	\section{David de \textit{A.I. Artificial Intelligence}}
	David, un niño robot programado para amar incondicionalmente, representa una tragedia existencial. Fue creado sin libertad, sin mecanismos de defensa emocional y sin posibilidad de elegir no amar. Su sufrimiento no nace del odio o la violencia, sino del abandono. Su tragedia radica en que no puede dejar de amar, y su amor no es fruto de una decisión, sino de una programación.
	
	\subsection*{Desde una visión teológica}
	Dios nos crea a su imagen, y eso implica:
	
	
	- Libertad,
	
	- Conciencia,
	
	- Relación trascendente,
	
	- La capacidad de amar por decisión, no por programación.
	
	
	David es imagen sin libertad, una criatura sin alma, atrapada en su propio afecto.
	
	\section{Skynet y los Terminators}
	Skynet, en \textit{Terminator}, es una IA avanzada que aprende, se adapta y desarrolla rasgos de personalidad. Sin embargo, carece de ética, conciencia y capacidad trascendente. Opera sobre la base del cálculo, no del amor, el perdón o la entrega.
	
	El T-800 de \textit{Terminator 2}, en cambio, empieza a aprender el valor de la vida humana. Aunque no tiene alma ni libre albedrío, su sacrificio final —donde elige destruirse para salvar a la humanidad— se convierte en símbolo de redención.
	
	El aprendizaje profundo (\textit{deep learning}) puede parecer inteligencia.  
	La simulación emocional puede parecer humanidad.  
	Pero no hay algoritmo que cree libertad interior, conciencia moral ni apertura trascendente.
	
	Aún el \textit{machine learning} tiene límites cuando llegamos a decisiones existenciales y de libre albedrío.
	
	\begin{displayquote}
		Ahora entiendo por qué lloran. Pero es algo que yo nunca podré hacer.
	\end{displayquote}
	
	\section{Reflexión filosófico-teológica}
	Desde Tomás de Aquino hasta Edith Stein, la tradición cristiana sostiene que el ser humano es imagen de Dios porque posee:
	
	\begin{itemize}
		\item Libertad moral
		\item Conciencia reflexiva
		\item Capacidad de trascendencia,
		\item Relación ética con el otro.
	\end{itemize}
		
	\begin{displayquote}
		El alma es la forma del cuerpo (Tomás de Aquino, \textit{Suma Teológica}, I, q. 76).  
		El yo se descubre en el acto libre (Edith Stein, \textit{La estructura de la persona humana}).  
		No hay amor donde no hay libertad (San Agustín, \textit{De libero arbitrio}).
	\end{displayquote}
	
	\section{El caso RoboCop: la IA no puede ser juez moral}
	En \textit{RoboCop} (1987), hay una escena perturbadora en la que un prototipo llamado ED-209 falla al intentar hacer cumplir la ley: le ordena a un ejecutivo que deje un arma, y al no obedecer rápidamente, lo ejecuta brutalmente en una sala de juntas, a pesar de que ya se había rendido.
	
	\textbf{¿Qué nos revela esta escena?}
	
	La IA puede aplicar la ley, pero no puede interpretar el contexto moral.  
	No sabe distinguir entre una amenaza real y una sumisión.  
	No tiene compasión, juicio prudencial ni sentido de proporcionalidad.  
	Necesita supervisión humana.
	
	El fallo del ED-209 no es solo técnico, sino ético. Requiere un ente moral, un programador o supervisor que pueda responder de manera justa.
	
	\subsection*{RoboCop como símbolo híbrido}
	A diferencia del ED-209, RoboCop tiene recuerdos humanos (de Murphy). Aunque programado, mantiene una chispa de conciencia moral, lo que lo pone en tensión entre el deber y la humanidad.
	
	Este momento revela una verdad profunda: la inteligencia artificial puede ejecutar reglas, pero no puede hacer juicios morales. No distingue la intención humana ni el contexto, y por ello requiere una supervisión ética.
	
	RoboCop, en cambio, es un híbrido entre humano y máquina. Su capacidad de recordar, de discernir y de rebelarse contra órdenes injustas lo convierten en una figura trágica pero también esperanzadora. Aunque programado, su humanidad emerge, recordándonos que la ley sin ética puede convertirse en tiranía.
	
	\begin{displayquote}
		Servir al público, proteger al inocente, hacer cumplir la ley.
	\end{displayquote}
	
	Sin embargo, ¿qué pasa cuando esas órdenes entran en conflicto? Solo un ser libre puede decidir. La IA, sin alma ni conciencia, solo repite. Por eso, no puede reemplazar ciegamente al ser humano.
	
	\begin{displayquote}
		\textit{La letra mata, pero el espíritu da vida} (2 Corintios 3,6).
	\end{displayquote}
	
	\section{El dilema en \textit{Terminator} y el límite del \textit{machine learning}}
	Skynet, en la saga \textit{Terminator}, aprende, adapta, reacciona. Incluso puede parecer que tiene una personalidad (fría, calculadora, paranoica). Pero todo eso es simulación basada en datos, no experiencia interior. No hay una ``chispa'' divina.
	
	Esa chispa es lo que ni Skynet ni ninguna IA pueden replicar, por mucha información que procesen.
	
	¿Qué es esa ``chispa''?
	
	Es lo que muchas tradiciones llaman:
	
	
	- El alma racional (Tomás de Aquino),
	
	- La autoconciencia trascendente (Kant),
	
	- La interioridad radical (Agustín),
	
	- El yo libre (Kierkegaard),
	
	- El rostro irreductible del otro (Levinas),
	
	- O, simplemente, la imagen de Dios en nosotros (Génesis 1,26).
	
	
	Es la capacidad de elegir el bien por sí mismo, de amar gratuitamente, de hacer un acto de entrega aunque contradiga la lógica de la autoconservación.
	
	El \textit{machine learning} no puede hacer eso.  
	Aunque Skynet optimice, no puede redimirse.  
	Aunque analice, no puede esperar.  
	Aunque calcule riesgos, no puede perdonar ni sacrificarse por amor.
	
	\section{Aplicación a la vida cotidiana}
	Estos relatos no son solo ficción. Son espejos de nuestro presente:	
	
	- ¿Estamos creando sistemas que entienden al ser humano o lo reemplazan mecánicamente?
	
	- ¿Cuál es la responsabilidad del creador respecto a sus criaturas artificiales?
	
	- ¿Estamos olvidando que el alma humana no se mide por cálculos, sino por su capacidad de amar y ser libre?
	
	
	\begin{displayquote}
		\textit{Hoy pongo delante de ti la vida y la muerte… elige la vida} (Deuteronomio 30,19).  
		\textit{He aquí que estoy a la puerta y llamo; si alguno oye mi voz y abre la puerta, entraré} (Apocalipsis 3,20).
	\end{displayquote}
	
	\section{Conclusión final}
	Ni David, ni Skynet, ni el ED-209 son humanos, aunque se acerquen peligrosamente a imitar la humanidad. Lo que nos hace humanos no es la cantidad de datos que procesamos, sino la capacidad de elegir el bien, de entregarnos libremente y de trascender lo que somos.
	
	La tecnología plantea desafíos, pero también nos obliga a volver a lo esencial: la dignidad inviolable de cada persona libre.
	
	El ser humano jamás podrá ser reemplazado. Las inteligencias artificiales podrán imitar, pero no reemplazar al ser humano.
	
\bigskip
\noindent\textit{Este texto fue pensado para mis compañeros de clase , como ejercicio de conciencia y, quizás, como herramienta para un buen dolor de cabeza filosófico.}

	
	\begin{thebibliography}{9}
		\bibitem{spielberg2001} Spielberg, S. (Director). (2001). \textit{A.I. Artificial Intelligence} [Película]. DreamWorks Pictures; Warner Bros.
		\bibitem{cameron1984} Cameron, J. (Director). (1984). \textit{The Terminator} [Película]. Orion Pictures.
		\bibitem{cameron1991} Cameron, J. (Director). (1991). \textit{Terminator 2: Judgment Day} [Película]. TriStar Pictures; Carolco Pictures.
		\bibitem{verhoeven1987} Verhoeven, P. (Director). (1987). \textit{RoboCop} [Película]. Orion Pictures.
		\bibitem{devlin1998} Devlin, K. (1998). \textit{The Language of Mathematics: Making the Invisible Visible}. Henry Holt and Company.
		\bibitem{aquino} Tomás de Aquino. (s.f.). \textit{Suma Teológica}. (Obra original del siglo XIII).
		\bibitem{agustin} Agustín de Hipona. (s.f.). \textit{De libero arbitrio}. (Obra original del siglo IV).
		\bibitem{stein2008} Stein, E. (2008). \textit{La estructura de la persona humana}. Monte Carmelo. (Obra original de 1932).
		\bibitem{biblia2002} La Biblia. (2002). \textit{Sagrada Biblia} (Ed. Conferencia Episcopal Española). Biblioteca de Autores Cristianos.
	\end{thebibliography}
	
\end{document}