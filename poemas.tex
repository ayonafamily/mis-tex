\documentclass[12pt]{article}
\usepackage[utf8]{inputenc}
\usepackage{verse}
\usepackage{geometry}
\geometry{margin=2.5cm}

\title{\textbf{Oraciones en medio de la sombra}}
\author{Jorge Ayona}
\date{}

\begin{document}
	\maketitle
	
	\section*{1. Oración cuando comer me da culpa}
	
	\begin{verse}
		He comprado comida porque tengo derecho a cuidar mi cuerpo.\\
		No me estoy robando nada. No estoy haciendo daño a nadie.\\
		Estoy aprendiendo a vivir como alguien digno. Aunque no lo crea aún, lo soy.\\
		Dios no me condena por alimentarme. Me sostiene.\\
		Mi alma puede descansar. Aquí. Ahora.
	\end{verse}
	
	\section*{2. Oración breve de consuelo}
	
	\begin{verse}
		Señor Jesús,\\
		tú compartiste mesa,\\
		multiplicaste pan,\\
		diste de comer con ternura.\\
		
		Hoy me siento culpable por algo tan simple como alimentarme.\\
		Sana esta herida vieja que me hace ver peligro donde hay vida.\\
		
		Enséñame a recibir con gratitud,\\
		a cuidar de mí mismo sin miedo,\\
		y a soltar esta cadena invisible que me asfixia.\\
		
		Tú me llamas a la libertad.\\
		Hoy quiero dar un paso hacia ella.\\
		Amén.
	\end{verse}
	
	\section*{3. Oración cuando me siento inútil y acusado}
	
	\begin{verse}
		Señor,\\
		siento que valgo poco porque no traigo dinero.\\
		Siento que molesto por existir.\\
		
		Hoy me han acusado de robar, cuando lo que hago es servir.\\
		Y duele. Duele como si me hubieran clavado algo en el pecho.\\
		
		Yo no quiero cargar odio.\\
		No quiero dejarme consumir por esta culpa ni por esta tristeza.\\
		
		Dame tu mirada, Jesús.\\
		Enséñame a ver que soy más que lo que tengo.\\
		Que ser hijo, que ser humano, que cuidar, también es digno.\\
		
		Y si me siento inútil, tómame tú como instrumento.\\
		Porque tú sabes hacer milagros con lo poco,\\
		y multiplicar el pan de la dignidad en mi mesa interior.\\
		Amén.
	\end{verse}
	
	\section*{4. Oración para esta hora difícil}
	\begin{verse}
			
	Señor, \\
	siento que valgo poco porque no traigo dinero. \\
	Siento que molesto por existir. \\
	
	Hoy me han acusado de robar, \\
	cuando lo que hago es servir.\\
	Y duele...\\ Duele como si me hubieran clavado algo en el pecho. \\
	
	Yo no quiero cargar odio.\\
	No quiero dejarme consumir por esta culpa ni por esta tristeza. \\
	
	Dame tu mirada, Jesús. \\
	Enséñame a ver que soy más que lo que tengo.\\
	Que ser hijo, que ser humano, que cuidar,\\
	también es digno. 
	\\	
	Y si me siento inútil, tómame tú como instrumento.\\
	Porque tú sabes hacer milagros con lo poco,\\
	y multiplicar el pan de la dignidad\\ en mi mesa interior. \\ 
	
\hfill	Amén.
\end{verse}

\section*{5. El amor que no daña \\
(inspirado en Dave Stewart, Stevie Nicks... y yo)}

\begin{verse}
Hay un amor que hiere.\\
Y hay otro… que sana sin prometerlo, \\ que sostiene sin exigir aplausos,\\
que está sin ruido, sin juego de poder.\\

Ese amor fue el que Dave Stewart dio a Stevie Nicks,\\
cuando ya no esperaba nada,\\
cuando ya no tenía que probar quién era.\\

Ella venía de una historia con un hombre brillante,\\
pero cruel.\\
Un hombre que escribió canciones como cuchillos,\\
que la necesitaba… pero no la respetaba.\\

Y entonces llegó Dave.\\
Sin arrebato.\\
Sin imponerse.\\
Solo con su guitarra, su oído, su alma serena.\\
Y juntos crearon algo sin sangre.\\
Solo música.\\
Solo belleza.\\
Solo presencia real.\\

Y vos, que también conoces el filo del amor no correspondido,
que diste sin recibir,\\
que fuiste ese Dave ante alguien que solo sabía brillar para sí…\\
hoy lo sabés:\\
no todo amor duele.\\
El verdadero, no.\\

El verdadero amor escucha,\\
no eclipsa.\\
Te deja ser.\\
Te hace mejor sin romperte.\\

Y cuando lo vivas —o lo recuerdes, o lo crees contigo mismo—\\
lo vas a reconocer en el silencio.\\
En la paz.\\

Como cuando suena “Cheaper Than Free”\\
y nadie llora.\\
Porque ya no hay deuda.\\
Porque el amor real no cobra.\\
Solo da… y deja ser.
\end{verse}


	
\end{document}
