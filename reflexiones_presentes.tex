\documentclass[12pt]{article}
\usepackage[utf8]{inputenc}
\usepackage[spanish]{babel}
\usepackage{amsmath,amsfonts,amssymb}
\usepackage{csquotes}
\usepackage[a4paper, margin=1in]{geometry}
\usepackage{setspace}
\usepackage{parskip}
\usepackage{titlesec}
\usepackage{lmodern}
\usepackage{hyperref}
\usepackage{fancyhdr}

% Estilo de sección
\titleformat{\section}{\normalfont\Large\bfseries}{\thesection}{1em}{}

% Información de encabezado y pie de página
\pagestyle{fancy}
\fancyhf{}
\rhead{Jorge L. Ayona Inglis}
\lhead{Reflexiones desde el presente}
\cfoot{\thepage}

% Metadata
\title{Reflexiones desde el presente}
\author{Jorge L. Ayona Inglis}
\date{22 de June de 2025}

\begin{document}

\maketitle

\begin{quote}
“Muchos se lamentan como si hubieran perdido algo valioso al alejarse de los placeres de la juventud. Pero yo me alegro de haber escapado de ellos, como si hubiera huido de amos furiosos y muchos. En la vejez hay una gran paz y libertad.”\\
--- \textit{Céfalo en La República}, Libro I, Platón.
\end{quote}

\onehalfspacing

En este momento me encuentro meditando. Recuerdo ese diálogo inicial de \textit{La República} de Platón, cuando Sócrates visita a Céfalo, un anciano que reflexiona serenamente sobre los años que ha vivido. Dice que, con la vejez, los deseos físicos disminuyen, y lejos de verlo como una pérdida, lo considera una liberación: como quien escapa de amos furiosos. Y esa imagen, la de un viento que se lleva las antiguas pasiones y deja paz, me acompaña.

Me identifico profundamente con esa visión. En mi cuerpo —el cuerpo de un anciano que aún respira— hay una consciencia distinta. Aunque ya no responde como antes a los placeres de la juventud, he recibido a cambio otros dones.

Pienso que gran parte del dolor, del abandono y de la depresión que muchas veces vienen al hacernos adultos mayores —como nos llama la sociedad— proviene justamente de no haber evolucionado ni avanzado interiormente.

Ya lo sugería Boecio en \textit{La consolación de la filosofía}: los bienes del cuerpo y del mundo exterior son pasajeros, pero el alma puede elevarse por encima de esas limitaciones si se vuelve hacia la sabiduría y la virtud. Así, por naturaleza, hay actividades humanas que florecen en la infancia, la adolescencia y la adultez productiva. Sin embargo, el cuerpo humano, tarde o temprano, comienza a declinar en sus facultades físicas y emocionales.

Pero esa decadencia física no implica un empobrecimiento del ser. Más bien, puede ser ocasión para que florezca lo interior. Cicerón, en \textit{De Senectute}, decía que la vejez, lejos de ser una carga, es una etapa fecunda si se vive con virtud. Nos libera de las pasiones desordenadas y abre un espacio para la meditación, el estudio y el consejo. Aunque el cuerpo decline, el alma —decía él— florece.

Esa es, precisamente, mi experiencia. Hay una dimensión que no se debilita, sino que se fortalece: la dimensión espiritual. Si seguimos las leyes naturales y aceptamos cada etapa como parte del proceso vital, dejando atrás lo bueno y lo malo, podemos alcanzar una madurez emocional más profunda.

Y esta madurez tiene que ver con nuestra capacidad de aprendizaje. Cada experiencia vivida —buena o mala—, cada pérdida, cada error, cada hallazgo, aporta. Esa madurez emocional nos conduce, inevitablemente, a una evolución espiritual.

Mi observación es esta: aunque la vida sensible y física puede debilitarse, la vida espiritual y mental puede ascender. Es un fenómeno inverso. El cuerpo declina, pero el alma se eleva.

Esto me recuerda las palabras del apóstol san Pablo: “aunque nuestro hombre exterior se va desgastando, el interior se renueva de día en día”. ¿No es maravilloso? Esto nos habla de la presencia de lo eterno, de lo trascendente: aquello que, al disolverse lo material, no se pierde.

Vivimos en una sociedad que concede un valor desmedido a los aspectos físicos y materiales del ser humano. Se ignora que justamente en la decadencia de esos aspectos yace la oportunidad de redescubrir lo más valioso: lo interior, lo invisible, lo trascendente.

Y al reconocer estos aspectos profundos de nuestra existencia, desarrollamos una alegría distinta: más serena, más plena. Ya no es la euforia de lo pasajero, sino la serenidad de lo esencial.

Hoy, por ejemplo, estoy sentado con mi laptop, dictando este fragmento. Hace poco descubrí un tesoro en los libros de Hannah Arendt —en especial \textit{Eichmann en Jerusalén}— y estoy dedicando este tiempo precioso a cultivar mi ser interior. Aunque mi cuerpo ya no puede con muchas tareas de antes, este es un tiempo fecundo.

Hoy es sábado en la noche. Hace muchos años, cuando estaba en la secundaria, me preparaba para salir a alguna fiesta con mis amigos, de las cuales en aquella época solo tenía permiso de regresar a medianoche. En mis años universitarios, las licencias eran mayores. Y en mi etapa adulta, disfrutaba salir, reunirme con amigos, ir al cine, conversar.

Pero esta noche es distinta. La humedad, algunos achaques y el cuidado de la salud me invitan a quedarme en casa. Confieso que, por un momento, esto fue una carga para mí, un motivo de tristeza. Pero ya no. Ahora estoy en mi sala, en mi sillón favorito, con mis mascotas a mi lado —mi gata y mi perra—. He abierto un poco la puerta de la calle, entra algo de brisa nocturna, y me siento feliz.

Ahora lo que más me interesa es una conversación. Un intercambio de ideas. Escribir, publicar algo, recibir una retroalimentación. Estos instantes, estos momentos tranquilos, son los que más disfruto.

Algunos podrían discutirlo, señalando mis achaques o limitaciones físicas. Pero nunca, como ahora, he sentido un rendimiento tan alto en mis estudios universitarios. Nunca he reflexionado tanto ni con tanta claridad. Nunca he tenido tanta confianza para considerar, incluso, redactar una tesis y obtener mi licenciatura. Nunca antes he estado tan entusiasmado con un proyecto académico.

Todo esto ha sido fruto de una decisión: dedicar mi atención a lo que puedo hacer ahora, no a lo que ya no puedo. No a quienes ya no están, ni a quienes no me aprueban o me juzgan.

Esta comprensión, para mí, ha sido una forma de iluminación. Tanto la filosofía —en el sentido profundo y poético que le da Boecio— como la fe y la teología —con su dimensión trascendente— me han ayudado a sobrellevar los momentos más difíciles.

Ahora tengo tiempo para reflexionar. Es lo que los antiguos filósofos griegos llamaban \textit{scholé}: no pereza, sino ocio fecundo, tiempo libre para el alma. Un tiempo en el que nos elevamos por encima del tráfago de la vida.

Tal vez —sí, tal vez— estoy viviendo ese momento. Y por mi parte, quiero aprovecharlo con todo el esfuerzo y conciencia que merece. No lo desperdiciaré añorando el pasado, porque este tiempo no es un castigo: es un privilegio.

Al final, todo se trata de aceptar los nuevos dones y las nuevas capacidades que han surgido en nosotros a lo largo del tiempo: dones nacidos del dolor, de las buenas y malas experiencias, de lo aprendido, de todo lo que hemos leído y vivido. Todo ese acervo ha sido almacenado en nuestro ser —ya sea en el cerebro, en el alma, o en ambos, no discutiré ahora cuál es el responsable—, y ahora aflora con fuerza.

Este caudal de conocimientos, memorias, sensaciones, intuiciones... es un tesoro. Y, al menos en mi caso, siento la responsabilidad de ponerlo al servicio de los demás, para beneficio de otros.

Cuando uno descubre eso en su vida, no hay lugar para la depresión. Hay, en cambio, confianza, esperanza y una gozosa aceptación de una nueva etapa. Una etapa desafiante que nos conduce —más y más— al desapego de todo lo que alguna vez hicimos para afirmarnos, y que nos permite comenzar a vivir la trascendencia... incluso aquí, en la Tierra.

\end{document}
