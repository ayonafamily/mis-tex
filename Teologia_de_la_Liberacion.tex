\documentclass[12pt]{article}
\usepackage[utf8]{inputenc}
\usepackage[spanish]{babel}
\usepackage{geometry}
\usepackage{lmodern}
\usepackage{parskip}
\usepackage{hyperref}
\usepackage{graphicx}
\usepackage{titling}
\usepackage{tabularx}
\usepackage{array}
\usepackage{booktabs}

\geometry{a4paper, margin=2.5cm}

\begin{document}

\begin{titlepage}
    \begin{center}
        \vspace*{1cm}

        % Logo institucional (cambiar el nombre del archivo si es necesario)
        \includegraphics[width=0.3\textwidth]{logo-ucsm.png} \\[1cm]

        \textbf{\Large Universidad Católica de Santa María} \\
        Facultad de Ciencias Sociales y Humanidades \\
        Escuela Profesional de Teología \\

        \vfill

        \textbf{\LARGE La Teología de la Liberación: Análisis Crítico desde la Perspectiva Católica} \\[0.5cm]

        \textbf{Ensayo académico} \\[0.5cm]

        \vfill

        \begin{flushleft}
        \textbf{Autor:} Jorge Ayona Inglis \\
        \textbf{Código de estudiante:} 2025001248 \\
        \textbf{Correo institucional:} jorge.ayona@estudiante.ucsm.edu.pe \\
        \textbf{ORCID:} \texttt{https://orcid.org/0009-0006-6551-9681} \\
        %\textbf{Docente del curso:} Luisa Bertina Abarca \\
        %\textbf{Curso:} Introducción Al Nuevo Testamento \\
        \textbf{Fecha:} \today \\
        \end{flushleft}

        \vfill

    \end{center}
\end{titlepage}

\section*{Introducción}

Desde su aparición en el convulso contexto latinoamericano de las décadas de 1960 y 1970, la \textbf{Teología de la Liberación (TdL)} ha generado intensos debates dentro y fuera de la Iglesia. Surgida como una respuesta pastoral y teológica al sufrimiento de los pobres y oprimidos, buscó articular el Evangelio con la lucha por la justicia social, proponiendo una comprensión de la fe profundamente encarnada en la realidad histórica.

Este estudio ofrece un análisis teológico-académico de la TdL desde una perspectiva católica, considerando tanto su valor pastoral como sus límites doctrinales. Se abordan sus fundamentos teológicos, su desarrollo histórico, sus principales representantes y los efectos que ha tenido en la vida eclesial. Asimismo, se examina críticamente su método, especialmente su vínculo con el marxismo, y se ofrece una valoración bíblica, teológica y magisterial. Finalmente, se analiza su permanencia en el presente, particularmente en expresiones como la teología ecológica, la inculturación y las corrientes feministas o decoloniales.

El propósito es ofrecer un discernimiento equilibrado y fiel a la \textbf{Doctrina Social de la Iglesia}, al \textbf{Magisterio reciente} y al \textbf{Evangelio de Jesucristo}, único fundamento de una auténtica liberación cristiana.

\section*{Puntos clave de la Teología de la Liberación}

\begin{enumerate}
    \item \textbf{Opción preferencial por los pobres}: proclama que Dios toma partido por los oprimidos, y que la Iglesia debe reflejar esta opción en su vida y misión.
    \item \textbf{Liberación integral}: incluye también la liberación política, económica y social del ser humano.
    \item \textbf{Análisis de la realidad}: utiliza herramientas de análisis estructural —frecuentemente de inspiración marxista— para comprender la raíz de la injusticia.
    \item \textbf{Teología desde la praxis}: surge desde la experiencia y lucha concreta de los pobres.
    \item \textbf{Cristología liberadora}: Cristo es presentado como liberador de toda opresión.
\end{enumerate}

\section*{Historia y evolución}

La TdL nace en el marco del \textit{Concilio Vaticano II} y se despliega con fuerza a partir de la \textit{Conferencia de Medellín (1968)}. Surgió como respuesta a una coyuntura marcada por desigualdades estructurales, represión política y pobreza generalizada. En los años 70 y 80 conoció su auge, pero también atrajo la atención crítica del Magisterio. Desde los años 90, entró en una etapa de revisión y transformación.

\section*{Principales exponentes}

\begin{itemize}
    \item \textbf{Gustavo Gutiérrez (Perú)}: autor de \textit{Teología de la liberación: Perspectivas} (1971), considerado el padre del movimiento.
    \item \textbf{Leonardo Boff (Brasil)}: promotor de una TdL ecológica y defensor de los derechos humanos.
    \item \textbf{Jon Sobrino (El Salvador)}: centrado en una cristología desde los pobres.
    \item \textbf{Camilo Torres (Colombia)}: sacerdote que optó por la lucha armada.
\end{itemize}

\section*{Efectos positivos y negativos}

\textbf{Positivos}:
\begin{itemize}
    \item Revalorización del compromiso social de la Iglesia.
    \item Empoderamiento de comunidades cristianas de base.
    \item Lectura bíblica desde los pobres.
\end{itemize}

\textbf{Negativos}:
\begin{itemize}
    \item Reducción de la fe cristiana a ideología política.
    \item Tensiones doctrinales por el uso del marxismo.
    \item Fragmentación eclesial.
\end{itemize}

\section*{V. Valoración bíblica, teológica y magisterial}

\textbf{Bíblico:} fundamento en los profetas (Amós, Isaías), la predicación de Jesús (Lc 4,18) y las primeras comunidades (Hch 2,44–45).\\
\textbf{Teológico:} ha enriquecido la teología del Reino y la moral social, pero a veces subordinó la revelación a esquemas ideológicos.\\
\textbf{Magisterial:} documentos clave: Instrucciones de la CDF en 1984 y 1986. Reconoce sus méritos, pero rechaza su uso acrítico del marxismo.

%%%% Explicar más
\section*{VI. Crítica al método marxista de análisis teológico de la TdL}

\begin{enumerate}
    \item Reduccionismo materialista. Esto es en sí un defecto del marxismo: reducir la experiencia humana con toda su complejidad a la economía.
    \item Concepción conflictiva de la historia. De acuerdo a la tesis marxista: La historia de la humanidad es la historia de la lucha de clases.
    \item Riesgo de ideologización del Evangelio.
    \item Negación del sufrimiento redentor.
\end{enumerate}

\section*{VII. Discernimiento católico}

La TdL, cuando está purificada de ideologías incompatibles con la fe, representa un valioso aporte pastoral. El Papa Francisco impulsa una visión de ``Iglesia pobre y para los pobres'' que retoma sus intuiciones legítimas, centrada en Cristo y la Tradición.

\section*{Discernimiento actual}

\textbf{Aspectos positivos:}
\begin{itemize}
    \item Inclusión de nuevos excluidos de la reflexión teológica y la práctica pastoral(pueblos, mujeres, planeta). En lo político, laIglesia arrastra la relación con el poder político desde la disputa de las investiduras: la alianza con el poder político.
    \item Lectura contextual del Evangelio. No es interpretar al evangelio con la praxis, sino la praxis con el evangelio.
\end{itemize}

\textbf{Peligros:}
\begin{itemize}
    \item Riesgo de sincretismo religioso.
    \item Activismo sin base teológica.
    \item Olvido de la vida sacramental.
\end{itemize}
%%%%%%%%%%%%%%%%%%%%%%%%%%%%%%%%%%%%%%%%%%%%%%%%%%%%%%%%%%%%%%%%%%%%%%%%%%%%%%%%%%%%%%%%%%%%%%%%%%%%%%
\section*{Variantes y evolución de la Teología de la Liberación}

Desde sus orígenes en las décadas de 1960 y 1970, la Teología de la Liberación (TdL) ha experimentado una evolución significativa, dando lugar a diversas corrientes que adaptan sus principios fundamentales a nuevas realidades sociales, culturales y ecológicas. Estas variantes responden al dinamismo de la vida de la Iglesia en contextos plurales y reflejan una voluntad constante de encarnar el Evangelio en los signos de los tiempos.
%%%%%%%%%%%%%5
\subsection{Teología de la Liberación Clásica}
Es la formulación original, centrada en la opción preferencial por los pobres, el análisis estructural de la injusticia social (con influencias marxistas), y una praxis pastoral y política comprometida. Esta versión tuvo gran impacto en América Latina en las décadas de 1970 y 1980, especialmente tras Medellín (1968) y Puebla (1979), y fue desarrollada por autores como Gustavo Gutiérrez, Leonardo Boff y Jon Sobrino.

\subsection{Teología de la Liberación Ecológica}
Surgida en las últimas décadas, esta corriente amplía la noción de ``pobres'' incluyendo a la naturaleza como víctima de la explotación. Propone una ecología integral, como lo señala el Papa Francisco en \textit{Laudato Si'}, que articula el cuidado del medio ambiente con la justicia social. Leonardo Boff ha sido una figura clave en esta vertiente, que invita a reconocer la interconexión entre la crisis ambiental y la exclusión de los pobres.

\subsection{Teología Indígena e Inculturada}
Se trata de una TdL inculturada en las cosmovisiones de los pueblos originarios. Busca dialogar con sus símbolos, espiritualidades y sabidurías ancestrales, partiendo del principio de que el Evangelio puede encarnarse legítimamente en diversas culturas. Esta corriente fue impulsada por el Sínodo para la Amazonía (2019) y ha generado controversias, especialmente por el uso de símbolos como la ``Pachamama'', que algunos interpretan como sincretismo y otros como inculturación pastoral.

\subsection{Teología Feminista de la Liberación}
Inspirada en la TdL, esta corriente incorpora la perspectiva de género y la experiencia histórica de las mujeres en contextos de opresión patriarcal. Denuncia la exclusión de las mujeres en la Iglesia y la sociedad, y propone una lectura liberadora de la Biblia desde la experiencia femenina. Aunque sus expresiones son diversas, comparten la convicción de que la justicia de Dios incluye la igualdad y la dignidad de las mujeres.

\subsection{Teologías Contextuales y Decoloniales}
Más recientemente, han surgido corrientes que asumen la crítica a los modelos de pensamiento eurocéntricos y coloniales. Proponen una teología desde las periferias, que rescata las voces silenciadas de comunidades racializadas, empobrecidas o desplazadas. Estas teologías se inspiran en la TdL, pero incorporan categorías propias del pensamiento decolonial, poscolonial y de movimientos sociales contemporáneos.

\subsection{Teología de la Liberación Popular o Espiritual}
Algunos sectores han buscado integrar la espiritualidad tradicional con el compromiso social. Esta variante pone énfasis en la mística, la oración, la liturgia y la vida sacramental como fuentes de resistencia y liberación. Supone una superación del activismo ideológico, sin abandonar la dimensión profética del Evangelio.

Estas variantes muestran la importanciade abrirse a nuevas fronteras sin perder el núcleo evangélico: el Dios que escucha el clamor de los pobres (cf. Ex 3,7) y actúa en la historia para liberar al ser humano del pecado, la opresión y la muerte. El desafío actual consiste en discernir, a la luz del Magisterio y de la Tradición viva de la Iglesia, cuáles de estas corrientes permanecen fieles al Evangelio y cuáles requieren correcciones teológicas o pastorales.

\vspace{1cm}
\subsection{La Teología del Pueblo: una forma alternativa}

Dentro del amplio horizonte de las teologías latinoamericanas surgidas en el siglo XX, la Teología del Pueblo representa una variante distintiva y complementaria a la Teología de la Liberación. Aunque comparten preocupaciones comunes ---como la opción preferencial por los pobres, la necesidad de una Iglesia encarnada en la historia y la denuncia de la injusticia---, difieren en sus métodos, fundamentos y enfoque pastoral.

\subsubsection{Origen y representantes}
La Teología del Pueblo se desarrolló en Argentina a partir de la década de 1960, en paralelo con la Teología de la Liberación, pero sin adoptar su análisis marxista. Sus principales exponentes son Lucio Gera, Rafael Tello, Alberto Methol Ferré, y, en el plano pastoral, Jorge Mario Bergoglio, futuro Papa Francisco.

Estos teólogos pusieron el acento en la cultura, la religiosidad popular y la historia concreta de los pueblos latinoamericanos como espacio privilegiado de acción de la gracia. A diferencia de la visión estructuralista de la TdL clásica, que enfatiza el conflicto de clases, la Teología del Pueblo propone una lectura cultural y espiritual del pueblo como sujeto creyente.

\subsubsection{Características fundamentales}
\begin{itemize}
  \item \textbf{Cristología popular:} Jesús es el Emmanuel, Dios-con-nosotros, que camina con su pueblo. No se enfatiza tanto la lucha contra las estructuras como la presencia de Cristo en la fe del pueblo sencillo.
  \item \textbf{Religiosidad popular:} Se reconoce como una forma legítima de vivir y expresar la fe, con una sabiduría propia, capaz de resistir la opresión y alimentar la esperanza.
  \item \textbf{Evangelización desde la cultura:} La inculturación es clave: el Evangelio se anuncia desde dentro del lenguaje, símbolos y prácticas del pueblo.
  \item \textbf{Iglesia madre y pastora:} En lugar de una Iglesia revolucionaria o militante, se propone una Iglesia que acompaña, escucha, consuela y anima a su pueblo.
  \item \textbf{Desconfianza del elitismo ideológico:} Frente a ciertas posturas ideologizadas o iluministas de la TdL clásica, la Teología del Pueblo defiende la fe vivida del pueblo sencillo como fuente de teología.
\end{itemize}

\subsubsection{Influencia magisterial y eclesial}
La Teología del Pueblo ha influido notablemente en el pensamiento pastoral del Papa Francisco. En sus documentos magisteriales ---como \textit{Evangelii Gaudium}, \textit{Querida Amazonia} y \textit{Fratelli Tutti}--- se percibe su inspiración: la centralidad del pueblo fiel, la importancia de la cultura y la religiosidad popular, y la denuncia de una economía que mata.
%%%%%%%%%%%%%%%%%%%%%%%%%%%%%%%%%%%%%%%%%%%%%%%%%%%%%%%%%%%

\subsection*{Crítica a la Teología del Pueblo}

Si bien la Teología del Pueblo ha sido presentada como una alternativa pastoral equilibrada, profundamente enraizada en la realidad latinoamericana, es necesario ejercer un discernimiento crítico sobre sus fundamentos, supuestos y efectos.

En primer lugar, la idea de que esta corriente representa una “vía intermedia” entre el espiritualismo desencarnado y el activismo ideologizado es más una formulación retórica que una realidad comprobable. En muchos casos, la Teología del Pueblo no ha sido más que una reformulación de la Teología de la Liberación (TdL), despojada de su lenguaje marxista explícito pero anclada en los mismos presupuestos: una antropología sociológica, una cristología horizontal, y una eclesiología funcionalista. El “pueblo” se convierte en una categoría casi teológica, en detrimento de la centralidad de Cristo y de la Revelación como fuente de la teología.

Cuando se afirma que la fe “se deja evangelizar por el pueblo”, se invierte el mandato evangélico: es la Iglesia, desde Cristo, quien evangeliza al pueblo, respetando su cultura pero sin someterse a ella. Esta inversión puede llevar a una teología antropocéntrica, relativista, que corre el riesgo de sustituir la verdad revelada por los procesos sociales o la experiencia histórica. En ese sentido, la inculturación genuina se ve reemplazada por una adaptación cultural sin discernimiento, con peligros de sincretismo.

La afirmación de que “Dios se revela en el caminar del pueblo” también requiere precisión teológica. Si bien es legítimo hablar de la presencia de Dios en la historia, la Revelación, en sentido estricto, no ocurre en las luchas sociales o procesos culturales, sino en la persona de Jesucristo, Verbo encarnado. Confundir historia con revelación lleva a una deformación del concepto teológico fundamental de Revelación, tal como lo enseña el Magisterio de la Iglesia.

Además, es necesario advertir que esta corriente ha servido frecuentemente como vehículo de correctismo político dentro de la Iglesia. Ha favorecido posturas ambiguas en lo doctrinal, ha debilitado la reverencia litúrgica, y ha promovido un discurso sentimental, centrado más en la “escucha” que en el anuncio de la verdad. Esto ha contribuido a la pérdida de identidad católica en algunos ámbitos pastorales y a la rehabilitación de actores ideologizados del pasado, como Gustavo Gutiérrez, cuya teología fue objetada por la Santa Sede en los años 80.

En definitiva, aunque la Teología del Pueblo ha buscado presentar una pastoral contextualizada, en muchos casos no ha sido sino una versión adaptada de la Teología de la Liberación, más apta para el clima cultural del siglo XXI. Sus frutos —ambigüedad doctrinal, liturgias improvisadas, debilitamiento de la formación sacramental— nos invitan a discernir si realmente esta corriente es fiel al depósito de la fe, o si constituye una expresión más del proceso de secularización interna en la Iglesia.

Ya desde su definición fundamental, la Teología de la Liberación comete un error de origen al definir la teología como “reflexión crítica de la praxis”. Esta concepción invierte el orden propio de la teología, que no parte de la acción humana, sino de Dios mismo: su Palabra revelada, su acción salvífica, su autocomunicación en Cristo. La historia puede ser objeto de reflexión teológica, pero nunca su fuente ni su criterio hermenéutico. En el momento en que se invierte esa relación, la teología deja de ser teología para convertirse en ideología.

La teología, para ser verdaderamente católica, debe tener su origen en la Revelación y su fin en el conocimiento de Dios y la comunión con Él. Solo desde ahí puede discernir los signos de los tiempos y ofrecer una lectura verdaderamente liberadora de la realidad, que no reduzca la fe a sociología ni el Evangelio a activismo político.


\section*{Refutación teológica de la Teología de la Liberación}

La Teología de la Liberación ha surgido como una respuesta dolorida ante situaciones reales de injusticia, pobreza e indignidad humana que claman al cielo. Nadie puede negar la urgencia de que la Iglesia haga suyo el clamor de los pobres y responda con el testimonio del Evangelio. Sin embargo, no toda respuesta es conforme al misterio de Cristo ni a la misión confiada por Él a su Iglesia.

El Papa Juan Pablo II y el Cardenal Ratzinger (luego Benedicto XVI) criticaron explícitamente los abusos de la Teología de la Liberación en la Instrucción Libertatis Nuntius (1984), por su uso indebido del análisis marxista y su reducción del Evangelio a lucha de clases. ``Como guardián de la fe, debo señalar que, en varias de sus formulaciones, la llamada Teología de la Liberación incurre en errores graves, no tanto por su intención pastoral, que puede ser legítima, sino por el marco ideológico que adopta y las consecuencias doctrinales que ello acarrea''.

\subsection*{Reducción del Evangelio a un proyecto sociopolítico}
La fe cristiana no es una ideología ni un programa de reforma social. Cuando la teología toma como punto de partida categorías marxistas —como la lucha de clases, la dialéctica de opresores y oprimidos, o la praxis revolucionaria—, corre el riesgo de vaciar el contenido salvífico del Evangelio, reduciéndolo a una herramienta de cambio estructural. Pero la liberación más profunda es la que Cristo nos ha traído: la liberación del pecado y de la muerte, que ninguna revolución humana puede otorgar.

\subsection*{Distorsión de la verdad sobre el hombre y la historia}
El materialismo marxista niega la dimensión trascendente del ser humano. Interpreta la historia como una mera sucesión de conflictos entre clases sociales, lo cual lleva, en algunos casos, a justificar la violencia en nombre de la liberación. Esto es incompatible con la visión cristiana, que afirma la dignidad de toda persona humana, incluso la del opresor, y reconoce en la historia un lugar de gracia, no solo de conflicto.

\subsection*{Subordinación de la Revelación a la praxis}
Algunos teólogos de la liberación sostienen que la verdad teológica nace únicamente de la praxis histórica. Esto conduce a una teología que no recibe la verdad revelada, sino que la produce, disolviendo así el carácter objetivo y trascendente de la Palabra de Dios. Pero la fe cristiana nace de la escucha: ``la fe viene de la predicación, y la predicación, por la palabra de Cristo'' (Rom 10,17). No somos los autores de la verdad; la recibimos como don.

\subsection*{Pérdida de la dimensión eclesial y sacramental}
El cristianismo no se agota en el compromiso por la justicia. Está intrínsecamente vinculado a la Iglesia como sacramento de salvación, a los sacramentos que comunican la gracia, y a la unidad con el Sucesor de Pedro. La Teología de la Liberación, al priorizar la acción política sobre la vida litúrgica y eclesial, corre el riesgo de romper la comunión eclesial y oscurecer la dimensión sobrenatural de la Iglesia.

\subsection*{Negación del valor redentor del sufrimiento}
El sufrimiento, en la visión marxista, es un mal a eliminar a toda costa. En cambio, para el cristiano, el sufrimiento, aunque nunca buscado, puede ser asumido en unión con Cristo crucificado y tener un valor redentor. Esta es una verdad central del cristianismo, que no puede ser ignorada sin traicionar el misterio pascual.

La auténtica liberación cristiana no es la que nace de una praxis ideológica, sino la que brota de la cruz y la resurrección del Señor.

La Iglesia debe trabajar incansablemente por la justicia, pero no puede hacerlo renunciando a su identidad más profunda: la de ser portadora del Evangelio, sacramento de salvación y signo de la esperanza escatológica.

Como escribió San Pablo: ``Si anunciamos a Cristo solamente para esta vida, somos los más desdichados de todos los hombres'' (1 Cor 15,19).

Por tanto, exhorto a todos los teólogos a no confundir la justa indignación con el discernimiento, ni el clamor del pobre con una ideología que niega a Dios. Solo en Jesucristo, verdadero Dios y verdadero hombre, y en comunión con su Iglesia, se encuentra la verdadera y plena liberación del hombre.
\section*{Conclusión}

La TdL ha recordado a la Iglesia su misión profética en favor de los pobres. Ha renovado la pastoral y la espiritualidad en muchos contextos. Pero su asociación con el marxismo ha exigido discernimiento y corrección doctrinal. Hoy sobrevive en formas renovadas y fieles al Magisterio. El desafío es integrar sus mejores intuiciones dentro de una teología cristocéntrica, sacramental, escatológica y en comunión con toda la Iglesia.

\textit{Solo desde Cristo liberador, muerto y resucitado, se comprende y realiza la verdadera liberación: aquella que restaura la dignidad del hombre como hijo de Dios y anticipa en la historia los signos del Reino.}

\appendix
\section*{Sección testimonial: Reflexión crítica desde la experiencia vivida}

Este apéndice recoge la experiencia de un testigo ocular de la evolución de la Teología de la Liberación (TdL) en América Latina durante las décadas de 1970 y 1980. Se trata de una mirada desde la historia vivida, no meramente académica, que permite observar de primera mano los efectos pastorales y eclesiales de dicha corriente.

Si bien se reconoce el mérito de la Teología de la Liberación al denunciar estructuras de injusticia y al proclamar una opción preferencial por los pobres, también se evidencian ciertos riesgos pastorales derivados de su adopción, especialmente en contextos donde el análisis marxista fue asumido de manera acrítica. Esta influencia introdujo presupuestos incompatibles con la fe católica, tales como el materialismo ateo, la negación de la trascendencia y la primacía del conflicto de clases como motor de la historia.

Casos como el del sacerdote Camilo Torres en Colombia, vinculado a la lucha armada, o el del sacerdote Ernesto Cardenal en la revolución sandinista de Nicaragua, muestran cómo esta corriente puede derivar, en su versión más radicalizada, en un uso político del Evangelio. Posteriormente, estos procesos terminaron generando incluso persecución contra la misma Iglesia católica, como se observa en el contexto actual de Nicaragua.

Otro aspecto preocupante ha sido el desplazamiento de la dimensión espiritual y devocional en ciertas propuestas pastorales influenciadas por la TdL. En algunos ambientes se despreciaron expresiones tradicionales de la fe como las novenas, los sacramentales o las procesiones, tachándolas de alienantes. Sin embargo, para muchos fieles, estas prácticas han sido fuente auténtica de fe, consuelo y resistencia ante la adversidad.

Ya desde su formulación inicial, al definir la teología como ``reflexión crítica de la praxis a la luz de la fe'', la TdL incurre en una inversión del orden teológico. La teología, por su naturaleza, parte de Dios, de su revelación y del depósito de la fe custodiado por la Iglesia, y se orienta hacia Él. La historia y la praxis pueden ser objeto de reflexión teológica, pero no son su fuente ni su principio constitutivo. Cuando se sustituye la Revelación por el análisis sociológico o económico como punto de partida, la teología se convierte en ideología.

También se han observado desviaciones graves en algunas expresiones recientes de esta corriente, especialmente aquellas vinculadas a lo indígena o a la ecoteología, donde símbolos como la “Pachamama” han generado confusión. La exaltación de lo ancestral, cuando se realiza sin un discernimiento adecuado, puede oscurecer la centralidad de Cristo y derivar en sincretismo o panteísmo.

Una experiencia concreta ilustra este desplazamiento teológico: en un retiro organizado en 1978 por un grupo de jesuitas en un barrio popular de Lima, se propuso el ayuno como práctica de justicia social citando Isaías 58. Sin embargo, todo el enfoque del retiro se centró en la injusticia estructural y no en la conversión personal ni en la relación con Dios. Esta orientación pastoral, que pone entre paréntesis lo trascendente, representa un claro desequilibrio.

Asimismo, se ha notado en algunos sectores eclesiales, incluso en autoridades eclesiásticas, una actitud de silencio ante actos de blasfemia o profanación —como en el caso de una imagen ofensiva hacia la Virgen María— por temor a romper con lo “políticamente correcto”. Tal actitud revela una pérdida del sentido profético y del deber de custodia del depósito de la fe.

En definitiva, la Teología de la Liberación ha tenido un impacto profundo, pero también ambivalente, en la Iglesia latinoamericana. Su lectura desde la experiencia histórica y personal refuerza la necesidad de un discernimiento teológico serio y fiel al Magisterio. La Doctrina Social de la Iglesia ya ofrece una vía sólida para abordar los desafíos de la justicia, sin caer en reduccionismos ideológicos ni comprometer el núcleo de la fe católica.

En aquellos años, dentro del ámbito universitario, circulaba una copla popular que expresaba un anhelo de justicia social, pero también evidenciaba una peligrosa inversión del resentimiento:

\begin{quote}
\textit{¿Cuándo querrá el Dios del cielo\\
que la tortilla se vuelva,\\
que los pobres coman pan\\
y los ricos coman... (excremento).}
\end{quote}

Esta expresión, si bien comprensible en un contexto de sufrimiento social, plantea serias interrogantes desde la ética cristiana. ¿Acaso es católico desear que los pobres se conviertan en nuevos opresores? ¿Es pastoralmente aceptable fomentar una visión de la sociedad centrada en el conflicto irreconciliable entre clases, según la lógica de la lucha marxista?

La doctrina social de la Iglesia promueve una colaboración solidaria entre todos los sectores sociales, no una inversión de odios. La opción preferencial por los pobres no implica desprecio hacia los ricos, sino llamado a la conversión de todos. El Evangelio llegó incluso “hasta el palacio del César” (cf. Flp 4,22), mostrando que la fe no excluye a ningún grupo humano.

Asimismo, la participación de sacerdotes como Camilo Torres en acciones guerrilleras, o el involucramiento de clérigos en procesos revolucionarios de inspiración marxista —como fue el caso de Ernesto Cardenal en Nicaragua—, representa una ruptura con la vocación sacerdotal y con la enseñanza constante del Magisterio sobre la no violencia. El Evangelio es Buena Noticia para todos y no se impone por la fuerza de las armas ni se reduce a una ideología política.

Recuerdo, además, haber asistido a un retiro en Lima, dirigido por miembros de la Compañía de Jesús, en el cual el ayuno era propuesto no como medio de conversión y oración, sino como forma de protesta social. Se citaba Isaías 58 como justificación, pero todo se orientaba hacia una praxis política, con escaso o nulo contenido espiritual. Esta instrumentalización del ayuno cristiano, orientándolo a una lucha ideológica, vaciaba su sentido teológico y lo subordinaba a una finalidad ajena al Evangelio.

Por todo esto, me permito subrayar que ya desde su definición de teología como “reflexión crítica de la praxis”, la TdL incurre en un contrasentido. La teología auténtica parte de Dios y de su Revelación, no de la praxis humana. La historia puede ser objeto de reflexión teológica, pero no constituye su fuente ni su principio. De otro modo, la teología se convierte en ideología.

El testimonio histórico y pastoral permite constatar que las desviaciones ideológicas de la TdL han producido frutos amargos: sincretismo, confusión doctrinal, politización del clero y, en algunos casos, omisión de deberes pastorales frente a graves profanaciones o blasfemias, por priorizar lo “políticamente correcto”.

La Iglesia no necesita adoptar categorías ajenas a su fe para comprometerse con la justicia. Ya posee una rica Doctrina Social, que hunde sus raíces en el Evangelio y en la Tradición, y que permite iluminar las realidades sociales sin renunciar a la identidad católica. La verdadera liberación sólo puede provenir de Cristo, no del conflicto de clases ni de ideologías materiales. La caridad pastoral exige anunciar a todos —ricos y pobres, opresores y oprimidos— el Evangelio que salva y libera desde dentro.

Desde mi experiencia personal como católico latinoamericano que vivió de cerca las décadas de 1970 y 1980, deseo ofrecer una reflexión crítica —tanto teológica como pastoral— acerca de los frutos que ha producido la llamada Teología de la Liberación y, más recientemente, la llamada “Teología del Pueblo”. En el discernimiento católico, “por sus frutos los conoceréis” (Mt 7,16), y desde esta máxima evangélica me permito expresar algunas preocupaciones.

A lo largo de los años he podido constatar que muchas de las corrientes surgidas bajo el nombre de teologías liberadoras, aunque con una intención pastoral legítima de acercarse a los pobres, han generado en no pocos casos ambigüedad doctrinal, descuido litúrgico, irreverencia en los signos sagrados y una marcada improvisación en la celebración eucarística. Estos fenómenos no pueden considerarse secundarios, pues afectan la vida espiritual del pueblo fiel. Poner al ser humano como punto de partida de la teología —en lugar de Dios y su revelación—, con el pretexto de atraerlo desde una dimensión sociopolítica, ha desembocado en un vaciamiento espiritual y en una pérdida del sentido trascendente.

En tiempos recientes, esta tendencia ha sido reforzada bajo formas aparentemente renovadas, como el ecofeminismo, la teología indígena, o la llamada “teología decolonial”. Todas ellas, aunque presentadas como adaptaciones pastorales a los nuevos signos de los tiempos, parecen compartir los mismos presupuestos que la Teología de la Liberación clásica: una antropología inmanentista, una sospecha hacia la Tradición, y una instrumentalización del Evangelio con fines ideológicos.

La llamada Teología del Pueblo, presentada como alternativa más moderada, en muchos casos no ha sido más que una continuidad maquillada de la TdL. Su énfasis en la cultura popular y en la religiosidad del pueblo ha derivado, en la práctica, en un correctismo político y en un eclesialismo sin profecía. En nombre de una pastoral de cercanía, se han tolerado desviaciones doctrinales graves y se ha dado cabida a expresiones ajenas a la fe católica, como lo fue la entronización simbólica de la Pachamama durante el Sínodo para la Amazonía. Este hecho, más que una inculturación legítima, ha sido percibido como una forma de sincretismo e incluso de profanación, incompatible con el primer mandamiento y con la liturgia católica.

Es igualmente preocupante la rehabilitación eclesial de figuras históricamente asociadas al marxismo teológico, como Gustavo Gutiérrez. El hecho de que instituciones como la Pontificia Universidad Católica del Perú —tradicional bastión de formación católica— hayan sido cooptadas en su orientación ideológica por sectores de la nueva izquierda (denominados popularmente como “caviares”), evidencia un desplazamiento doctrinal que afecta gravemente la misión evangelizadora de la Iglesia. Lo mismo puede decirse de diversos nombramientos episcopales motivados más por afinidad ideológica que por fidelidad doctrinal.

Todo esto lleva a una constatación que interpela: si la teología deja de partir de la Revelación divina y se convierte en “reflexión crítica de la praxis”, como postula la TdL, deja de ser teología y se transforma en ideología. Ni la historia, ni la economía, ni las relaciones de poder son fuentes de la teología: pueden ser su objeto, pero no su principio ni su fin. La teología parte de Dios, tiende a Dios, y se nutre de la Escritura, la Tradición y el Magisterio.

Es alentador observar que en medio de esta crisis, muchos jóvenes católicos —hoy más conscientes del hambre de trascendencia— buscan una fe auténtica, sacramental, bien fundamentada y centrada en Jesucristo. Ellos están llamados a renovar la Iglesia no con ideologías del pasado, sino con una teología fiel, esperanzadora y profundamente espiritual. Que esta generación no sea estéril como aquellas que se anquilosaron en los discursos de los años 70, sino que abrace una renovación verdadera desde el corazón del Evangelio.

\section*{Comparación entre la Teología de la Liberación y la Doctrina Social de la Iglesia (DSI)}

\renewcommand{\arraystretch}{1.5}
\begin{tabularx}{\textwidth}{|>{\bfseries}m{4cm}|X|X|}
\hline
\textbf{Criterio} & \textbf{Teología de la Liberación} & \textbf{Doctrina Social de la Iglesia (DSI)} \\
\hline
Origen histórico & Surge en América Latina en los años 60-70, especialmente tras Medellín (1968) y el Vaticano II. & Se desarrolla desde \textit{Rerum Novarum} (1891) pero se intensifica tras el Vaticano II. \\
\hline
Inspiración teológica & Opción preferencial por los pobres como eje teológico. & Dignidad humana como principio fundamental. \\
\hline
Enfoque central & Liberación integral del oprimido: social, política, económica y espiritual. & Promoción del bien común, la justicia y la solidaridad. \\
\hline
Método teológico & Ver – Juzgar – Actuar; se apoya en las ciencias sociales y algunos análisis marxistas. Reflexion critica sobre la praxis, no el dogma.& Reflexión desde la Revelación y el Magisterio, abierta al discernimiento social. \\
\hline
Papel de la Iglesia & Iglesia comprometida con la transformación desde las bases sociales. & Iglesia como guía ética, sin aliarse a ideologías políticas. \\
\hline
Relación con el marxismo & Algunos teólogos adoptan herramientas de análisis marxista sin compartir su ideología. & Rechazo explícito del marxismo por su materialismo y ateísmo. \\
\hline
Papel de los pobres & Sujetos activos de la historia y de la fe. & Reafirma la opción preferencial por los pobres. \\
\hline
Magisterio eclesial & Cuestionada por la Congregación para la Doctrina de la Fe en los años 80 (ej. \textit{Libertatis Nuntius}). & Desarrollada y promovida por encíclicas como \textit{Centesimus Annus}, \textit{Caritas in Veritate}, \textit{Fratelli tutti}. \\
\hline
Recepción eclesial & Inicialmente polémica; en la actualidad, varios de sus temas son retomados por el Papa Francisco. & Oficialmente asumida como línea de acción pastoral y doctrina social. \\
\hline
Figuras representativas & Gustavo Gutiérrez, Leonardo Boff, Jon Sobrino. & Juan XXIII, Juan Pablo II, Benedicto XVI, Francisco. \\
\hline
Actualidad & Influye en pastorales de base, movimientos sociales y eclesiales. & En evolución constante, con énfasis actual en ecología, economía ética y fraternidad. \\
\hline
\end{tabularx}

\includegraphics[width=\linewidth]{ Teologia_Liberacion_vs_Pueblo.pdf}

\begin{thebibliography}{99}

\bibitem{gutierrez1971}
Gustavo Gutiérrez, 
\textit{Teología de la liberación: Perspectivas}. 
Salamanca: Ediciones Sígueme, 1971.

\bibitem{cdf1984}
Congregación para la Doctrina de la Fe, 
\textit{Instrucción sobre algunos aspectos de la “Teología de la liberación”}. 
Roma, 1984.

\bibitem{cdf1986}
Congregación para la Doctrina de la Fe, 
\textit{Instrucción sobre la libertad cristiana y la liberación}. 
Roma, 1986.

\bibitem{juanpabloii1987}
San Juan Pablo II, 
\textit{Encíclica Sollicitudo Rei Socialis}. 
Roma, 1987.

\bibitem{benedictoxvi2009}
Benedicto XVI, 
\textit{Encíclica Caritas in Veritate}. 
Roma, 2009.

\bibitem{francisco2015}
Papa Francisco, 
\textit{Encíclica Laudato Si’ sobre el cuidado de la casa común}. 
Ciudad del Vaticano, 2015.

\bibitem{francisco2013}
Papa Francisco, 
\textit{Exhortación apostólica Evangelii Gaudium}. 
Ciudad del Vaticano, 2013.

\bibitem{francisco2020}
Papa Francisco, 
\textit{Encíclica Fratelli Tutti}. 
Ciudad del Vaticano, 2020.

\bibitem{boff1995}
Leonardo Boff, 
\textit{Ecología: Grito de la Tierra, Grito de los Pobres}. 
Madrid: Trotta, 1995.

\bibitem{sobrino1991}
Jon Sobrino, 
\textit{Jesucristo liberador: Lectura histórico-teológica de Jesús de Nazaret}. 
Madrid: Trotta, 1991.

\bibitem{metholf2012}
Alberto Methol Ferré, 
\textit{Iglesia y mundo en la historia}. 
Buenos Aires: Ediciones Ciudad Nueva, 2012.

\bibitem{gera1970}
Lucio Gera, 
\textit{La religiosidad del pueblo: una teología pastoral}. 
Buenos Aires, 1970.

\bibitem{tello2007}
Rafael Tello, 
\textit{La fe del pueblo}. 
Buenos Aires: Ágape Libros, 2007.

\end{thebibliography}



\end{document}
