\documentclass[a4paper,12pt]{article}
\usepackage[utf8]{inputenc}
\usepackage[spanish]{babel}
\usepackage{fontenc}
\usepackage[margin=2.5cm]{geometry}
\usepackage{graphicx}
\usepackage{hyperref}
\usepackage{fontspec} % Necesario para usar fuentes del sistema
\setmainfont{Arial}
\usepackage{xcolor}
\usepackage{parskip}
\usepackage{fancyhdr}
\pagestyle{fancy}
\fancyhf{}
\rhead{Jorge L. Ayona Inglis}
\lhead{Identidad y Peruanidad}
\rfoot{\thepage}

% Define color institucional
\definecolor{verdelimon}{RGB}{128,0,32}

\begin{document}
\begin{titlepage}
	
	\begin{center}
		\huge \textbf{Universidad Católica de Santa María}\\[1cm]
		
		\large Facultad de Ciencias Sociales\\[3mm]
		\textbf{Escuela de Teología}\\[3mm]
		
		\begin{figure}[h]
			\centering
			\includegraphics[scale=0.8]{logo-ucsm.png} % Asegúrate de tener este archivo en tu carpeta
		\end{figure}		
		\vspace{3mm}
		
		\textcolor{verdelimon} {\rule{\linewidth}{0.5mm}}	
		
		\vspace{4mm}
		{\Large\textbf{Caracteristicas de la Cultura  Mestiza en el Perù}}\\[3mm]
		{\LARGE\textbf{El Señor De Los Milagros}}\\[5mm]
		
		\large{Curso: Identidad y Peruanidad}\\[2mm]
		\large{Profesor(a): Ana Rosario Miaury Vilca}\\[2mm]
		\large{Alumno: Jorge L. Ayona Inglis}\\[2mm]
		
		\large{\textbf{ORCID:} \href{https://orcid.org/0009-0006-6551-9681}{\texttt{0009-0006-6551-9681}}}\\[3mm]
		
		\textcolor{verdelimon}{\rule{\linewidth}{0.5mm}}	
		\vspace{0.5cm}
		
		{\large \today}
		
	\end{center}
	
\end{titlepage}

		
		\tableofcontents
		\newpage
		
		\section{Introducción}
		La cultura peruana es profundamente mestiza, fruto de una larga historia de integración, resistencia y sincretismo entre los pueblos originarios, los conquistadores españoles y la población afrodescendiente. Una de las expresiones más ricas y emblemáticas de esta cultura mestiza es el culto al Señor de los Milagros, una devoción católica nacida en Lima en el siglo XVII, que hoy en día sigue convocando a millones de fieles cada mes de octubre.
		
		Esta manifestación religiosa no solo representa una práctica devocional, sino también un símbolo de la identidad colectiva de los peruanos. Su origen, historia, iconografía y rituales reflejan con claridad cómo las diferentes culturas que dieron forma al Perú encontraron en la fe un espacio común.
		
		Como limeño, esta devoción ha marcado mi vida desde muy pequeño. Incluso en los colegios usábamos corbatas o escapularios alusivos. y mis compañeras sus hábitos. Los recuerdos que tengo me han seguido por toda mi vida.
		
		\section{Origen y sincretismo del Señor de los Milagros}
		La imagen del Señor de los Milagros, también conocida como el Cristo de Pachacamilla, fue pintada por un esclavo angoleño en una pared de adobe en el barrio de Pachacamilla (Lima), en el siglo XVII. A pesar de varios terremotos, esta imagen resistió intacta. En especial, el terremoto de 1655 —que destruyó buena parte de la ciudad— no logró derribar el muro donde estaba pintado el Cristo. Este hecho fue interpretado como un milagro y dio inicio a su veneración pública.
		
		El término Pachacamilla, por cierto, tiene una raíz en común con “Pachacámac”, nombre del dios prehispánico asociado al movimiento de la tierra (“el que sacude la tierra”). La permanencia de la imagen del Cristo durante un terremoto fue percibida, simbólicamente, como la victoria del Dios de los españoles (Cristo) sobre los dioses precolombinos, en especial sobre Pachacámac. Esta percepción marcó el inicio de un proceso de sustitución simbólica que está en el corazón del sincretismo religioso.
		
		\subsubsection{La Virgen de la Nube: el arquetipo de la Gran Madre en la procesión}
		\addcontentsline{toc}{subsubsection}{La Virgen de la Nube: el arquetipo de la Gran Madre en la procesión}
		
		Acompañando al Cristo morado de Pachacamilla, en la parte posterior del anda procesional, se encuentra la imagen de la Virgen de la Nube. Esta advocación mariana, originaria de la región andina del norte, simboliza el amparo materno que sigue al dolor redentor. 
		
		Desde una lectura simbólica inspirada en la psicología de Carl Jung, la Virgen representa el arquetipo de la \textbf{Gran Madre}: consuelo, protección y guía silenciosa. Su posición “a las espaldas” del Cristo crucificado no es casual: representa la \textbf{dimensión femenina y protectora del mestizaje espiritual peruano}, una síntesis entre el sufrimiento humano y la esperanza celestial.
		
		Asimismo, la nube —elemento etéreo— conecta con antiguas cosmovisiones indígenas, donde los fenómenos naturales eran expresión del mundo espiritual. La Virgen de la Nube se convierte, entonces, en una \textbf{síntesis viva de lo indígena, lo hispano y lo universal}, reforzando la unidad profunda de las tres raíces del alma peruana.
				
		\section{Elementos culturales presentes en el culto}
		El culto al Señor de los Milagros refleja una integración muy clara de tres matrices culturales:
		
		\begin{itemize}
			\item \textbf{Indígena:} La interpretación de los signos naturales (como los sismos) como mensajes divinos es una herencia andina. Además, el uso del color morado, asociado a la penitencia, se integra con elementos de cosmovisión prehispánica vinculados al ciclo de la vida y la muerte.
			
			\item \textbf{Africana:} La imagen fue pintada por un esclavo africano. Además, la estructura de la hermandad del Señor de los Milagros se organizó inicialmente con afrodescendientes. El ritmo de los cantos y la procesión evocan también expresiones culturales negras.
			
			\item \textbf{Hispana:} La liturgia, los rezos, los hábitos de los cargadores y la celebración eucarística provienen de la tradición católica española. También la iconografía del Cristo crucificado sigue los cánones del arte religioso europeo.
		\end{itemize}
		
		\section{Características de la cultura mestiza evidenciadas}
		
		\subsection{1. Sincretismo religioso}
		El culto al Señor de los Milagros es una expresión clara del sincretismo religioso. En ella se fusionan símbolos y prácticas de distintas culturas en una nueva forma religiosa compartida. Esta apelación al mestizaje muestra cómo el catolicismo, como su nombre indica, es universal y apela a los arquetipos y sentidos más profundos del alma humana. Es por ello que ha podido arraigarse en contextos culturales tan diversos como los de Perú.
		
		\subsection{2. Hibridación estética}
		La imagen del Cristo morado, los colores del anda, la música procesional, los turrones, el incienso y los altares urbanos que decoran las calles durante octubre constituyen una fusión de estilos y tradiciones. Es un ejemplo de cómo la estética mestiza combina elementos religiosos con el arte popular y callejero.
		
		\subsection{3. Flexibilidad identitaria}
		El culto al Señor de los Milagros convoca a personas de todas las clases sociales, edades, géneros y razas. Su identidad mestiza permite que cada grupo se sienta representado y acogido, sin perder la unidad espiritual. La devoción es tan fuerte que incluso ha sido llevada por los migrantes peruanos a distintas partes del mundo.
		
		\section{Reflexión final}
		
		La devoción al Señor de los Milagros no es solo una expresión religiosa, sino una manifestación profunda de la identidad peruana mestiza. En ella confluyen tres grandes raíces: la indígena, la africana y la hispana. Esta fusión no anula ninguna de ellas, sino que las integra en una vivencia común que atraviesa generaciones, territorios y clases sociales.
		
		El carácter procesional del culto, su lenguaje simbólico, su capacidad de convocatoria masiva y su permanencia a lo largo de los siglos revelan que esta devoción se ha convertido en un espacio de memoria y de pertenencia.
		
		\vspace{3mm}
		
		\noindent
		El culto al Señor de los Milagros se sigue manteniendo hasta el día de hoy. Han pasado siglos, y la ciudad de Lima lo considera su protector, especialmente frente a los temblores y terremotos. Aunque han sucedido varios sismos a lo largo de la historia, después del terremoto que dio origen a esta devoción, no se ha registrado ninguno con igual impacto catastrófico en la ciudad. Algunos han interpretado este hecho como una señal de que el antiguo dios Pachacámac —cuyo nombre significa “el que mueve la tierra”— fue contenido por la entrada del Cristo en estas tierras.
		
		Cada mes de octubre, miles de fieles salen a las calles en una procesión que refleja el alma mestiza de la ciudad. Esta devoción congrega también expresiones culturales como los turrones de Doña Pepa, los anticuchos, los picarones y los emolientes, herencia de raíces africanas, andinas y españolas. Esta riqueza de sabores, colores y gestos configura un mestizaje vivo, donde lo ancestral y lo nuevo se encuentran cada año en una fiesta de fe y comunión popular.
		
		\begin{center}
			\textit{Así, el Señor de los Milagros no solo camina por las calles de Lima,\\
				sino también por la historia profunda del Perú mestizo.}
		\end{center}
		
	
		\section*{Bibliografía}
		\addcontentsline{toc}{section}{Bibliografía}
		
		\begin{itemize}
			\item Dawson, Alexander. \textit{Latinoamérica desde 1800}. Cengage Learning, 2013.
			\item Vargas Ugarte, Rubén. \textit{Historia del culto de los santos en el Perú}. Lima: Imprenta del Congreso, 1953.
			\item De la Puente, José. “Cristo de Pachacamilla y el mestizaje peruano.” En: \textit{Estudios sobre religión e identidad}, PUCP, 2005.
			\item Portal católico del Perú: \url{https://www.arzobispadodelima.org/}
		\end{itemize}
		
\end{document}