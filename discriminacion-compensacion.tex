\documentclass[12pt]{article}
\usepackage[utf8]{inputenc}
\usepackage[spanish]{babel}
\usepackage{geometry}
\usepackage{parskip}
\usepackage{hyperref}
\usepackage{setspace}
\geometry{a4paper, margin=2.5cm}
\title{\textbf{\large\centering}\textbf{\Large ¿Quién Define la Vulnerabilidad?}\ Críticas éticas al Sistema de Cuotas y su Sesgo Ideológico}
\author{Jorge L. Ayona Inglis}
\date{\today}

\begin{document}

\maketitle
\onehalfspacing

\begin{abstract}
Este ensayo cuestiona la legitimidad y la equidad del sistema de cuotas como instrumento de justicia social. Si bien las cuotas raciales, de género u orientación sexual nacieron con la intención de corregir desventajas históricas, hoy generan nuevas formas de exclusión al privilegiar a grupos ideológicamente visibles, mientras otros, como los adultos mayores o neurodivergentes, quedan marginados sin base objetiva clara. Se propone una revisión ética y racional de los criterios que definen la vulnerabilidad y el acceso a medidas compensatorias.
\end{abstract}

\section*{Introducción}
En el debate contemporáneo sobre la justicia social, las llamadas "acciones afirmativas" o "cuotas" se han posicionado como instrumentos para igualar oportunidades entre grupos que han sufrido discriminación histórica. Sin embargo, cada vez más voces cuestionan la base ética y los criterios arbitrarios mediante los cuales se decide quién merece un cupo especial y quién no. \textit{"Yo, por ejemplo, soy discriminado por edad, pero no entro en el sistema de cuotas"}, afirma con razón un adulto mayor excluido del mercado laboral. \textit{"Parece que solo los grupos que pueden ejercer presión social obtienen beneficios."}

\section*{Discriminación por visibilidad}
La experiencia muestra que las cuotas benefician mayormente a grupos con presencia en el discurso público o con capacidad de movilización. Mientras tanto, otras formas de vulnerabilidad —como la edad avanzada, la discapacidad invisible, o ciertas condiciones socioeconómicas— no generan el mismo eco mediático y, por ende, no reciben reconocimiento institucional. Esto produce una \textbf{discriminación por omisión}: se ayuda no al que más lo necesita, sino al que más se nota.

\section*{La exclusión digital de los adultos mayores}
Un caso paradigmático de esta discriminación por omisión es la exclusión digital que afecta a muchos adultos mayores. En una sociedad cada vez más tecnologizada, el acceso a servicios esenciales como la banca, la salud, la administración pública o incluso la comunicación social exige habilidades digitales. La falta de capacitación, recursos y acompañamiento específico para este grupo genera una brecha real, que rara vez es compensada con políticas públicas o cuotas. La vulnerabilidad digital de los mayores no suele tener visibilidad mediática, ni activismo organizado, y por lo tanto, queda fuera de los esquemas de protección preferente. Esta omisión también es discriminación, aunque más sutil y silenciosa.

\section*{La paradoja de la justicia segmentada}
El argumento habitual a favor de las cuotas es que corrigen desigualdades estructurales. Pero, \textbf{\textit{¿qué sucede cuando una desventaja real no encaja en la narrativa dominante?}} El adulto mayor, el desempleado crónico sin afiliación ideológica, o el estudiante pobre pero blanco, no califican. \textbf{Esto plantea una paradoja: se combate la discriminación creando nuevas formas de ella}. Como bien ironiza la llamada "canción de la tortilla", \textit{al dar vuelta la opresión, no siempre se logra justicia, sino inversión del poder}.

\section*{El principio de igualdad radical}
Desde una perspectiva \textit{iusnaturalista} o cristiana, cada persona posee una dignidad intrínseca que no depende de su grupo de pertenencia. Esto contradice el modelo de cuotas identitarias, que fragmenta a la sociedad en subgrupos competitivos. La \textbf{igualdad ante la ley} debería proteger a todos los discriminados, incluso si su grupo es pequeño o no tiene visibilidad. Negarlo es repetir el mismo error que se busca corregir: basar los derechos en el poder y no en la justicia.

\section*{Conclusión}
El sistema de cuotas, aunque nacido de una intención correctiva, ha sido capturado por una lógica ideológica que favorece a quienes encajan en la narrativa del momento. Urge revisar los criterios de vulnerabilidad desde una \textbf{ética de la equidad universal}, que no excluya al invisible ni al solitario. Solo así se construirá una justicia verdaderamente incluyente.

\begin{thebibliography}{9}

\bibitem{rawls}
Rawls, John.
\textit{Teoría de la Justicia}. Fondo de Cultura Económica, 2006.

\bibitem{dworkin}
Dworkin, Ronald.
\textit{La igualdad como valor}. Barcelona: Ariel, 2000.

\bibitem{ratzinger}
Ratzinger, Joseph.
\textit{Informe sobre la fe}. Madrid: Palabra, 1985.

\bibitem{onu}
ONU.
\textit{Declaración sobre los Derechos de las Personas Mayores}. Naciones Unidas, 2012.

\bibitem{wokecrit}
Williams, Thomas.
\textit{Woke Racism}. Crown Publishing, 2021.

\end{thebibliography}

\end{document}
