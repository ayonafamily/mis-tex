\documentclass[12pt]{article}
\usepackage[utf8]{inputenc}
\usepackage[spanish]{babel}
\usepackage{geometry}
\usepackage{tabularx}
\usepackage{array}
\usepackage{booktabs}
\geometry{a4paper, margin=2.5cm}

\title{Comparación: Empirismo, Kant, Jung y la Visión Cristiana}
\author{Jorge Ayona}
\date{\today}

\begin{document}
\maketitle

\begin{tabularx}{\textwidth}{>{\raggedright\arraybackslash}p{3.3cm} 
>{\raggedright\arraybackslash}X 
>{\raggedright\arraybackslash}X 
>{\raggedright\arraybackslash}X 
>{\raggedright\arraybackslash}X}
\toprule
\textbf{Tema} & \textbf{Empirismo (Locke, Hume)} & \textbf{Kant} & \textbf{Jung} & \textbf{Fe cristiana} \\
\midrule

\textbf{Origen del conocimiento} & 
Todo conocimiento proviene de la experiencia (sensaciones y percepciones). La mente es una tabula rasa. & 
El conocimiento necesita experiencia, pero también estructuras a priori de la razón. & 
La experiencia se organiza según arquetipos heredados del inconsciente colectivo. & 
El alma ha sido creada por Dios con una inteligencia espiritual orientada a la verdad y el bien. \\

\textbf{Naturaleza de la mente al nacer} & 
Vacía, sin ideas innatas. Se llena por experiencia. & 
Dotada de estructuras innatas: espacio, tiempo, categorías. & 
Contiene imágenes arquetípicas anteriores a lo personal. & 
Creada por Dios con una vocación única, y una ley moral inscrita. (Rom 2,15) \\

\textbf>Forma de interpretar la realidad & 
Asociaciones de ideas basadas en la experiencia repetida. & 
La mente organiza el caos de los datos sensibles con categorías. & 
La psique proyecta arquetipos sobre los hechos y experiencias. & 
La razón y la fe permiten leer el mundo como creación de Dios. (Sab 13,5) \\

\textbf{Identidad del sujeto} & 
No hay yo sustancial: solo una serie de percepciones (Hume). & 
El sujeto trascendental es el que unifica las percepciones. & 
El Self (sí-mismo) es la totalidad de la psique consciente e inconsciente. & 
El ser humano es imagen de Dios (Gen 1,27), con alma inmortal. \\

\textbf{Finalidad de la vida} & 
No se plantea una finalidad trascendente; enfoque pragmático. & 
Conocer los límites de la razón y vivir moralmente. & 
Llegar a la individuación: plenitud interior, integración. & 
Unirse a Dios por gracia: santidad, redención, vida eterna. (Jn 17,3) \\
\bottomrule
\end{tabularx}

\end{document}
