\documentclass[12pt]{article}
\usepackage[utf8]{inputenc}
\usepackage[spanish]{babel}
\usepackage{amsmath}
\usepackage{geometry}
\usepackage{setspace}
\usepackage{hyperref}
\usepackage{xurl}
\usepackage{fontspec} % Para Arial
\setmainfont{Arial}

\geometry{margin=1in}
\setstretch{1.5}

\title{Reflexión Jurídica y Ética sobre el Juicio a Pedro Castillo: ¿Rebelión o Leguleyada?}
\author{Jorge Ayona \\ \href{https://orcid.org/0009-0006-6551-9681 }{ORCID: 0009-0006-6551-9681}}
\date{\today}

\begin{document}
	
	\maketitle
	
	\begin{abstract}
		Este ensayo reflexiona sobre el juicio al expresidente del Perú, Pedro Castillo, a partir de una experiencia personal de preocupación por el Estado de Derecho. Se analiza el delito de rebelión desde una perspectiva jurídica, ética y comparativa, considerando el contexto peruano y experiencias internacionales. Se concluye que tanto el Derecho positivo como la ética democrática exigen una sanción proporcional a la gravedad del intento de subversión institucional.
	\end{abstract}
	
	\section*{Introducción}
	
	El 7 de diciembre de 2022, el entonces presidente del Perú, Pedro Castillo, dirigió un mensaje a la nación en el que anunció la disolución del Congreso, la intervención del Poder Judicial y la detención de la fiscal de la Nación. Estos actos, que no llegaron a concretarse, fueron interpretados por diversos sectores como un intento de golpe de Estado. A partir de este hecho, se abrió un proceso judicial por el presunto delito de rebelión.
	
	\section*{Preambulo}
	
	\subsection*{Nota Personal}
	En el año 2002, al llegar a Trujillo con el propósito de fundar una comunidad cristiana, acompañé a un amigo llamado Sandro —quien vendía libros jurídicos— en sus recorridos. En una de esas visitas, una abogada llamada Fanny nos recibió y comenzó a hacerle preguntas sobre uno de los textos. Por curiosidad, yo ya lo había hojeado y, sin ser abogado, le indiqué cómo podía usar una concordancia entre artículos de dos libros distintos. Sorprendida, la doctora le preguntó a Sandro si yo era abogado.
	
	Aquella experiencia me marcó: no como una simple anécdota, sino como un recordatorio de que el pensamiento jurídico puede y debe estar al alcance de todos los ciudadanos comprometidos con la justicia y la verdad. Desde entonces, el Derecho dejó de ser algo ajeno para mí, y se volvió parte de mi búsqueda intelectual y ética.
	
	Este ensayo nace también de ese impulso.
	
	\section*{Aclaraci\'on}
	No pretendo ser jurista ni abogado, pero esta reflexión incluye necesariamente una perspectiva jurídica. Hablo desde mi formación como teólogo y filósofo, lo que me permite abordar el tema con una mirada más amplia e interdisciplinaria. Asumo plenamente cualquier imprecisión que pudiera surgir. No obstante, esta opinión ha sido elaborada a partir de una revisión documentada y razonada.
	
	Como ciudadano preocupado por el Estado de Derecho, me inquieta profundamente la forma en que se está conduciendo el juicio. Según la defensa de Castillo, como no se alzó en armas, no habría delito. Sin embargo, siendo jefe supremo de las Fuerzas Armadas, sus palabras y órdenes tienen un peso institucional que trasciende una simple proclama. El hecho de que el golpe no se haya concretado no elimina la intención ni la gravedad del acto. Me preocupa que el Derecho positivo se base en tecnicismos que impidan sancionar a quien realmente quebranta el orden democrático.
	
	\section*{Análisis jurídico del delito de rebelión}
	
	El Código Penal peruano, en su artículo 346, establece:
	
	\begin{quote}
		“El que se alza en armas para variar la forma de gobierno, deponer al gobierno legalmente constituido o suprimir o modificar el régimen constitucional, será reprimido con pena privativa de libertad no menor de diez ni mayor de veinte años” \cite{codigo_penal}.
	\end{quote}
	
	La jurisprudencia ha interpretado que este delito puede configurarse incluso si no se concreta el objetivo final, tratándose de un delito de peligro o de consumación anticipada. En el caso de Castillo, su condición de jefe supremo de las Fuerzas Armadas le otorgaba un poder institucional que puede considerarse como un alzamiento simbólico o institucional, aunque no se haya producido violencia física \cite{jurisprudencia_rebelion}.
	
	\section*{Dimensión ética}
	
	Desde una perspectiva ética, el intento de subvertir el orden constitucional es un acto gravísimo, independientemente de su éxito. El poder presidencial no es simbólico; sus órdenes tienen consecuencias reales. Permitir que un presidente intente un golpe y no sea sancionado por tecnicismos debilita la democracia y envía un mensaje de impunidad.
	
	\section*{Comparación internacional}
	
	En países como España, Alemania y Chile, los delitos contra el orden constitucional también se sancionan aunque no se consumen plenamente, siempre que haya intención, medios y actos preparatorios. Por ejemplo, el intento de golpe de Estado en España en 1981 fue sancionado aunque no se concretó, debido a la gravedad del acto y su amenaza al orden democrático \cite{golpe_espana}.
	
	\section*{Conclusión}
	
	El juicio a Pedro Castillo representa una prueba para el sistema judicial peruano. Tanto el Derecho como la ética coinciden en que se trató de un intento de quebrar el orden constitucional. La justicia no debe limitarse a la literalidad de la ley, sino también considerar su espíritu y la protección de la democracia.
	
	\begin{thebibliography}{9}
		
		\bibitem{codigo_penal}
		Congreso de la República del Perú. (2023). \textit{Código Penal del Perú}. Artículo 346. Recuperado de \url{https://www.pj.gob.pe}
		
		\bibitem{jurisprudencia_rebelion}
		Corte Suprema de Justicia del Perú. (2022). \textit{Recurso de Apelación N.º 248-2022/SUPREMA}. Sala Penal Permanente. Ponente: César San Martín Castro. Recuperado de \url{https://www.pj.gob.pe/wps/wcm/connect/434831004997113e82ccf69026c349a4/APELACI%C3%93N+248-2022+SUPREMA.pdf?MOD=AJPERES}
		
		\bibitem{golpe_espana}
		Preston, P. (2012). \textit{El final de la guerra: El golpe de Estado del 23-F y la consolidación de la democracia en España}. Madrid: Editorial Debate.
		
	\end{thebibliography}
	
\end{document}
