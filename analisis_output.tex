\documentclass[a4paper,12pt]{article}
\usepackage{fontspec}
\setmainfont{Arial}
\usepackage{polyglossia}
\setmainlanguage{spanish}
\setotherlanguage[variant=ancient]{greek}
\newfontfamily\greekfont[Script=Greek]{Linux Libertine O}
\usepackage{geometry}
\geometry{
  top=1in,
  bottom=1in,
  left=1in,
  right=1in
}
\usepackage[table]{xcolor}  % <-- Importante para colorear líneas de tabla
\usepackage{array}
\usepackage{tabularx}
\geometry{margin=1in}
\definecolor{verdelimon}{RGB}{154,205,50}
\newcolumntype{Y}{>{\raggedright\arraybackslash}X}
\arrayrulecolor{verdelimon}

\begin{document}
\section{Texto: \textgreek{ἐξορκίζω σε κατὰ τοῦ θεοῦ τοῦ ζῶντος}}

\begin{tabularx}{\textwidth}{|Y|Y|Y|}
\rowcolor{verdelimon}
\hline
\textcolor{white}{\textbf{Función sintáctica}} & \textcolor{white}{\textbf{Texto griego}} & \textcolor{white}{\textbf{Morfología}} \\
\hline
\hline
Verbo principal & \textgreek{ἐξορκίζω} & Verbo \\
\hline
Complemento directo & \textgreek{σε} & Pronombre personal \\
\hline
Preposición + régimen & \textgreek{κατὰ} & Preposición que rige el genitivo. Indica invocación de autoridad: “por” \\
\hline
Término de la preposición & \textgreek{τοῦ θεοῦ} & Artículo + sustantivo \\
\hline
Epíteto calificativo & \textgreek{τοῦ ζῶντος} & Participio presente activo \\
\hline \end{tabularx}

\vspace{1cm}
\arrayrulecolor{verdelimon}
\begin{tabularx}{\textwidth}{|Y|}
\hline
\textbf{Traducción literal:} Te conjuro por el Dios viviente. \\
\hline
\textbf{Traducción en español fluido:} Te ordeno solemnemente en el nombre del Dios vivo. \\
\hline
\end{tabularx}
\arrayrulecolor{black}
\end{document}
