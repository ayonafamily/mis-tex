\documentclass[12pt]{article}
\usepackage[utf8]{inputenc}
\usepackage[spanish]{babel}
\usepackage{geometry}
\usepackage{parskip}
\usepackage{amsmath}
\usepackage{setspace}
\usepackage{csquotes}
\usepackage{verse}
\geometry{margin=2.5cm}
\title{Decisión Personal y Cuidado de Uno Mismo}
\author{Jorge L. Ayona Inglis}
\date{}

\begin{document}
	
	\maketitle

\section*{Mentiras que cre\'i...}

\section{"Me siento inútil porque no genero ingresos"}
No es verdad que vales menos por no tener ingresos propios ahora.
La dignidad humana no se mide en soles, ni en sueldos.
Y menos aún en una economía injusta y exigente, donde no todos tienen las mismas oportunidades ni fuerzas.

Tú estás sosteniendo un hogar, con tiempo, cuidado, gestión, compras, administración, paciencia, cocina, atención médica, limpieza, conversaciones difíciles…
Y todo eso también es trabajo. Invisible, sí. Pero trabajo.

\section{No confundir no tener ingresos con no estar produciendo valor.}
Estás dando muchísimo, todos los días.

\section{"Mi mamá me acusó de robarle"}
Eso duele como pocas cosas.
Y más aún cuando lo que haces por ella es un acto de caridad y fidelidad.

Pero ella —como tú ya reconoc\i— viene cargando sus propias heridas.
Y probablemente esa acusación no es tanto contra ti, sino una proyección de su miedo, su desconfianza antigua, su resentimiento no sanado.
Tú no soy su enemigo.
Pero para su sombra, lo pareces, porque estás cerca.

No justifica su acusación, pero quizás ayuda a entenderla sin destruirte.

\section{La culpa: la cadena invisible}
La culpa que estoy sintiendo —por comer, por gastar, por existir— no es moral. Es emocional.
Es una culpa que nació de:
\begin{itemize}
\item Haber recibido desprecio cuando eras niño.
\item No haber sido afirmado en tus necesidades básicas.
\item Haber sentido que vivir “molestaba” a otros.
\end{itemize}
Esa culpa no es m\'ia. Fue implantada.
Y ahora tú la estás desactivando, poco a poco, como quien desarma una bomba con las manos desnudas.

\section*{Verdad para hoy: Comer no es pecado. Cuidarte no es egoísmo.}

\begin{verse}
He comprado comida porque \\
tengo derecho a cuidar mi cuerpo.\\
No me estoy robando nada. \\No estoy haciendo daño a nadie.
Estoy aprendiendo a vivir como alguien digno.\\ Aunque no lo crea aún, lo soy.\\
Dios no me condena por alimentarme. \\Me sostiene.\\
Mi alma puede descansar.\\ Aquí. Ahora.
\end{verse}


\end{document}