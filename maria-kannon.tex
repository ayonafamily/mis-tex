\documentclass[12pt,a4paper]{article}
\usepackage[utf8]{inputenc}
\usepackage{amsmath,amssymb}
\usepackage{geometry}
\geometry{margin=1in}
\usepackage{parskip}
\usepackage{hyperref}
\usepackage{xurl}
\usepackage{csquotes}
\usepackage{polyglossia}
\setmainlanguage{spanish}
\setmainfont{Arial}




\title{\textbf{Mar\'ia-Kannon y el eco de Apocalipsis 12: La respuesta de Dios a la persecuci\'on en Jap\'on}}
\author{Jorge Ayona \\ \href{https://orcid.org/0009-0006-6551-9681 }{ORCID: 0009-0006-6551-9681}}
\date{2 de Julio 2025}

\begin{document}
	
	% Cambiar "Referencias" por "Bibliografía"
	\addto\captionsspanish{\renewcommand{\refname}{Bibliografía}}
	
	\maketitle
	
	\section*{Introducción}
	
	En el corazón de la historia de la Iglesia en Japón brilla una página silenciosa y escondida: la de los \emph{Kakure Kirishitan}, los cristianos ocultos que, durante más de dos siglos, mantuvieron viva su fe en la clandestinidad, despojados de templos, sacramentos y ministros ordenados. Esta historia no solo es testimonio de fidelidad, sino también una manifestación concreta del misterio revelado en el capítulo 12 del Apocalipsis.
	
	\begin{quote}
		\textit{``Apareció en el cielo una gran señal: una mujer vestida del sol, con la luna bajo sus pies y una corona de doce estrellas sobre su cabeza. Estaba encinta, y gritaba con los dolores del parto y con el tormento de dar a luz''} (Ap 12,1--2).
	\end{quote}
	
	La Mujer, signo de la Virgen María y de la Iglesia, es perseguida por el dragón, figura del Mal. Pero Dios la lleva al desierto, \textit{``a un lugar preparado por Dios, para ser alimentada allí durante mil doscientos sesenta días''} (Ap 12,6).
	
	\section*{Una fe protegida en el silencio}
	
	Cuando la Iglesia fue perseguida en Japón, Dios también la llevó simbólicamente al desierto: al silencio, a la sombra, a la noche. Sin medios externos, la fe fue alimentada por la oración, la tradición oral y, sobre todo, por un signo vivo y encubierto: la figura de \emph{María-Kannon}.
	
	Kannon, la \textit{bodhisattva} de la compasión, profundamente venerada en el budismo japonés, fue vista por los cristianos perseguidos como una imagen de la Virgen María. Algunas estatuas escondían rosarios o crucifijos; otras eran simplemente veneradas como la Madre de Dios bajo otro rostro. Esta adaptación no fue un simple sincretismo, sino una expresión viva del \emph{Logos spermatikos} —la ``semilla del Verbo''— de la que hablaba San Justino Mártir (\emph{Apología I}, 46): la presencia del Verbo de Dios sembrada en los corazones de todas las culturas.
	
	\section*{Teología del desierto y del consuelo}
	
	Como en el Apocalipsis, Dios no evitó la persecución, pero protegió a los suyos con un lenguaje simbólico y maternal. María estuvo presente: no como Reina esplendorosa, sino como Kannon silenciosa. En este sentido, puede decirse que \emph{María-Kannon} fue la respuesta de Dios a la persecución: una presencia escondida, una ternura disfrazada, una compasión fiel.
	
	Este misterio se alinea con el magisterio de la Iglesia sobre la inculturación. San Juan Pablo II enseñó:
	
	\begin{quote}
		\textit{``La inculturación es la encarnación del Evangelio en las culturas autóctonas y, al mismo tiempo, la introducción de estas culturas en la vida de la Iglesia''} (\emph{Redemptoris Missio}, 52).
	\end{quote}
	
	\section*{El reencuentro con la luz}
	
	Cuando los misioneros regresaron a Japón en el siglo XIX, encontraron comunidades que, a pesar de no haber tenido sacerdotes por generaciones, habían conservado el bautismo, la oración, la devoción a María y una viva conciencia de pertenecer a la Iglesia de Cristo. Fue como descubrir una llama que había ardido en lo secreto, custodiada por corazones humildes y fieles.
	
	\section*{Conclusión}
	
	\emph{María-Kannon} es hoy un signo profético. Nos recuerda que la compasión, la fidelidad y la ternura de Dios no se apagan en la persecución, y que la Madre de Dios nunca abandona a sus hijos. Incluso en la noche más oscura, Dios tiene un lugar preparado donde la fe puede resistir, florecer y dar fruto en el silencio.
	
	Como enseña el Catecismo de la Iglesia Católica:
	
	\begin{quote}
		\textit{``Por su total adhesión a la voluntad del Padre, a la obra redentora de su Hijo, María está unida a Él con un vínculo estrecho e indisoluble. Esta unión alcanza su culminación al pie de la cruz''} (CIC, 964).
	\end{quote}
	
	Y como dice el Señor:
	
	\begin{quote}
		\textit{``Bienaventurados los que lloran, porque ellos serán consolados''} (Mt 5,4).
	\end{quote}
	
	En la historia de los cristianos ocultos, esta bienaventuranza se hizo carne, y la Virgen, bajo la forma de Kannon, fue el consuelo silencioso de un pueblo fiel escondido en la palma de la mano de Dios.
	
	\newpage
	
	\begin{thebibliography}{9}
		
		\bibitem{Justino}
		San Justino Mártir. \emph{Primera Apología}, cap. 46.  
		Disponible en: \url{https://www.documentacatholicaomnia.eu/04z/z_0100-0165__Iustinus_Martyr__Apologia_I__GR.pdf }  
		(Consulta recomendada en traducciones modernas autorizadas al español o latín).
		
		\bibitem{RedemptorisMissio}
		San Juan Pablo II. \emph{Redemptoris Missio}. Carta Encíclica sobre la permanente validez del mandato misionero, 7 de diciembre de 1990.  
		Disponible en: \url{https://www.vatican.va/content/john-paul-ii/es/encyclicals/documents/hf_jp-ii_enc_07121990_redemptoris-missio.html }
		
		\bibitem{CIC}
		Congregación para la Doctrina de la Fe. \emph{Catecismo de la Iglesia Católica}, 11 de octubre de 1992.  
		Disponible en: \url{https://www.vatican.va/archive/catechism_sp/index_sp.html }
		
		\bibitem{Apocalipsis12}
		Biblia. \emph{Apocalipsis 12, 1–6}.  
		“La mujer vestida del sol…”  
		Disponible en: \url{https://www.vatican.va/archive/ESL0506/__P10.HTM }
		
		\bibitem{Mateo5}
		Biblia. \emph{Evangelio según San Mateo 5,4}.  
		“Bienaventurados los que lloran…”  
		Disponible en: \url{https://www.vatican.va/archive/ESL0506/__P6E.HTM }
		
	\end{thebibliography}
	
\end{document}