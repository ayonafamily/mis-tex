\documentclass{article}%
\usepackage[T1]{fontenc}%
\usepackage[utf8]{inputenc}%
\usepackage{textcomp}%
\usepackage{lastpage}%
\usepackage{fontspec}%
\setmainfont{Cormorant Garamond}%
\usepackage[spanish]{babel}%
\usepackage{geometry}%
\geometry{margin=2.54cm}%
\usepackage{xcolor}%
\usepackage{sectsty}%
\usepackage{setspace}%
\setstretch{1.5}%
\definecolor{seccion}{RGB}{70, 50, 100}%
\definecolor{cita}{RGB}{100, 100, 100}%
\sectionfont{\color{seccion}}%
\usepackage{csquotes}%
\renewenvironment{quote}{\begin{quotation}\itshape\color{cita}}{\end{quotation}}%
\title{Reflexión nocturna: Saber y sombra}%
\author{Jorge Ayona}%
\date{\today}%
%
\begin{document}%
\normalsize%
\maketitle%
Hola Jorge, gracias por compartir algo tan profundo justo antes de dormir.%
\newline%
Lo que mencionas es importante, porque toca una herida muy común: la de sentir que el conocimiento propio es motivo de ataque o de envidia.%
\newline%
Vamos a mirarlo desde varios ángulos para ayudarte a integrar esta experiencia con más paz.%
\newline%
\newline%
\section{Primero, sobre la frase “de nada vale saber si no saber qué hacer”}%
\label{sec:Primero,sobrelafrasedenadavalesabersinosaberquhacer}%
Esa frase puede ser válida si se dice con humildad. Pero cuando se usa como indirecta, tiene carga pasivo{-}agresiva. No se critica el saber, sino cómo otros se sienten frente al que sabe.%
\newline

%
\section{¿Es tu sombra o es proyección de los demás?}%
\label{sec:Estusombraoesproyeccindelosdems?}%
La respuesta podría ser: ambas cosas.%
\newline%
\newline%
Desde tu sombra: %
Tal vez hay una parte tuya que teme sobresalir, por heridas pasadas.%
\newline%
Desde los otros: %
Algunas personas se sienten amenazadas cuando alguien sabe más que ellas.

%
\section{El ataque a la humildad}%
\label{sec:Elataquealahumildad}%
La verdadera humildad no es callar la sabiduría, sino ponerla al servicio con amor y paciencia.%
\newline

%
\section{¿Qué puedes hacer con todo esto?}%
\label{sec:Qupuedeshacercontodoesto?}%

    \definecolor{myblue}{RGB}{50, 100, 180}

        \setlength{
-sep}{6pt}
        \renewcommand\labelitemi{\textcolor{myblue}{$\blacksquare$}}

- Reconocer tu valor intelectual sin culpa.

- Aceptar que algunos no lo van a tolerar.

- Elegir cuándo compartir y cuándo callar.

- Seguir desarrollando tu autoestima.

    

%
\section{Para dormir en paz}%
\label{sec:Paradormirenpaz}%
\begin{quote}%
Mi saber no es una amenaza. Es un don. Y lo usaré con discernimiento y sin esconderme.%
\end{quote}%
Buenas noches, Jorge.

%
\end{document}