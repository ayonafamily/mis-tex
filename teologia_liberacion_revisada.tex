
\documentclass[12pt]{article}
\usepackage[utf8]{inputenc}
\usepackage[spanish]{babel}
\usepackage{geometry}
\usepackage{lmodern}
\usepackage{parskip}
\usepackage{hyperref}
\usepackage{graphicx}
\usepackage{titling}
\usepackage{tabularx}
\usepackage{array}
\usepackage{booktabs}
\usepackage{enumitem}

\geometry{a4paper, margin=2.5cm}

\title{La Teología de la Liberación: Análisis Crítico desde la Perspectiva Católica}
\author{Jorge Ayona Inglis}
\date{\today}

\begin{document}

\begin{titlepage}
    \begin{center}
        \vspace*{1cm}
        % Logo institucional (cambiar el nombre del archivo si es necesario)
        \includegraphics[width=0.3\textwidth]{logo-ucsm.png} \\[1cm]
        \textbf{\Large Universidad Católica de Santa María} \\
        Facultad de Ciencias Sociales y Humanidades \\
        Escuela Profesional de Teología \\
        \vfill
        \textbf{\LARGE La Teología de la Liberación: Análisis Crítico desde la Perspectiva Católica} \\[0.5cm]
        \textbf{Ensayo académico} \\[0.5cm]
        \vfill
        \begin{flushleft}
        \textbf{Autor:} Jorge Ayona Inglis \\
        \textbf{Código de estudiante:} 2025001248 \\
        \textbf{Correo institucional:} jorge.ayona@estudiante.ucsm.edu.pe \\
        \textbf{ORCID:} \texttt{https://orcid.org/0009-0006-6551-9681} \\
        %\textbf{Docente del curso:} Luisa Bertina Abarca \\
        %\textbf{Curso:} Introducción Al Nuevo Testamento \\
        \textbf{Fecha:} \today \\
        \end{flushleft}
        \vfill
    \end{center}
\end{titlepage}

\section*{Introducción}
Desde su aparición en el convulso contexto latinoamericano de las décadas de 1960 y 1970, la \textbf{Teología de la Liberación (TdL)} ha generado intensos debates dentro y fuera de la Iglesia. Surgida como una respuesta pastoral y teológica al sufrimiento de los pobres y oprimidos, buscó articular el Evangelio con la lucha por la justicia social, proponiendo una comprensión de la fe profundamente encarnada en la realidad histórica.

Este estudio ofrece un análisis teológico-académico de la TdL desde una perspectiva católica, considerando tanto su valor pastoral como sus límites doctrinales. Se abordan sus fundamentos teológicos, su desarrollo histórico, sus principales representantes y los efectos que ha tenido en la vida eclesial. Asimismo, se examina críticamente su método, especialmente su vínculo con el marxismo, y se ofrece una valoración bíblica, teológica y magisterial. Finalmente, se analiza su permanencia en el presente, particularmente en expresiones como la teología ecológica, la inculturación y las corrientes feministas o decoloniales.

El propósito es ofrecer un discernimiento equilibrado y fiel a la \textbf{Doctrina Social de la Iglesia}, al \textbf{Magisterio reciente} y al \textbf{Evangelio de Jesucristo}, único fundamento de una auténtica liberación cristiana.

\section*{I. Puntos clave de la Teología de la Liberación}
\begin{enumerate}
    \item \textbf{Opción preferencial por los pobres}: sostiene que Dios se inclina especialmente hacia los más necesitados y excluidos, y llama a la Iglesia a manifestar esta preferencia en su vida pastoral y misión evangelizadora.
    \item \textbf{Liberación integral}: propone una visión ampliada de la salvación, que abarca no solo lo espiritual, sino también las dimensiones política, económica y social del ser humano.
    \item \textbf{Análisis de la realidad}: emplea métodos de análisis estructural —en muchos casos influenciados por el marxismo— para identificar las causas profundas de la injusticia y la opresión.
    \item \textbf{Teología desde la praxis}: se construye a partir de la experiencia histórica y la lucha concreta de los pobres, considerando que la fe debe expresarse en compromiso transformador.
    \item \textbf{Cristología liberadora}: presenta a Jesucristo como el verdadero liberador, cuya muerte y resurrección ofrecen redención frente a todo tipo de opresión.
\end{enumerate}

\section*{II. Historia y evolución}
La Teología de la Liberación (TdL) surgió en el contexto latinoamericano durante las décadas de 1960 y 1970, profundamente influenciada por el espíritu renovador del \textit{Concilio Vaticano II}. Su consolidación como corriente teológica se dio tras la celebración de la \textit{Conferencia Episcopal Latinoamericana de Medellín (1968)}, donde se sentaron las bases para una pastoral comprometida con los pobres y orientada hacia la justicia social.

Esta corriente emergió como respuesta a realidades complejas marcadas por desigualdades estructurales, violaciones sistemáticas a los derechos humanos y condiciones de pobreza extrema. Durante las décadas de 1970 y 1980 alcanzó su mayor difusión y relevancia, influyendo en comunidades eclesiales de base, movimientos sociales y círculos académicos teológicos.

Sin embargo, su vinculación con análisis socio-políticos inspirados en el marxismo generó tensiones dentro de la Iglesia. El Magisterio expresó preocupación ante posibles desviaciones doctrinales, lo cual llevó a pronunciamientos oficiales de la Congregación para la Doctrina de la Fe en los años 80. A partir de los años 90, la TdL inició un proceso de reelaboración y adaptación, dando lugar a nuevas expresiones teológicas que buscan mantener su espíritu liberador en fidelidad al Evangelio y a la Tradición católica.

\section*{III. Principales exponentes}
\begin{itemize}
    \item \textbf{Gustavo Gutiérrez (Perú)}: considerado el fundador de la corriente, su obra \textit{Teología de la liberación: Perspectivas} (1971) sentó las bases conceptuales y espirituales de esta teología.
    \item \textbf{Leonardo Boff (Brasil)}: uno de los más influyentes teólogos brasileños, destacó por su visión ecológica y humanista de la liberación. En obras como \textit{Iglesia: Carisma y Poder}, cuestionó estructuras eclesiales tradicionales y promovió un modelo de Iglesia cercano al pueblo y comprometido con la justicia social y ambiental.
    \item \textbf{Jon Sobrino (El Salvador)}: jesuita y teólogo salvadoreño, enfocó su trabajo en una cristología desde los pobres. Su reflexión sobre Jesucristo como liberador integral fue forjada en medio de la violencia y la represión política de su país.
    \item \textbf{Camilo Torres (Colombia)}: sacerdote y sociólogo, fue una figura emblemática de la fusión entre fe y acción revolucionaria. A diferencia de otros teólogos, no solo escribió sobre liberación, sino que decidió incorporarse activamente a la lucha armada como miembro de las FARC.
\end{itemize}

\section*{IV. Efectos positivos y negativos}
\textbf{Positivos:}
\begin{itemize}
    \item \textbf{Revalorización del compromiso social de la Iglesia}: revitalizó la conciencia eclesial sobre la necesidad de una presencia activa en favor de los más pobres y excluidos.
    \item \textbf{Empoderamiento de comunidades cristianas de base}: fortaleció la participación popular en la vida de la Iglesia, fomentando comunidades que asumen un rol protagónico en la transformación social.
    \item \textbf{Lectura bíblica desde los pobres}: promovió una interpretación de la Palabra de Dios centrada en la experiencia de los marginados, ampliando así la comprensión teológica del mensaje evangélico.
\end{itemize}

\textbf{Negativos:}
\begin{itemize}
    \item \textbf{Reducción de la fe cristiana a ideología política}: en algunos casos, priorizó el análisis socioeconómico sobre la dimensión trascendente de la salvación, vaciando el contenido espiritual del Evangelio.
    \item \textbf{Tensiones doctrinales por el uso del marxismo}: el recurso a categorías marxistas generó desconfianza dentro de algunos sectores eclesiales y motivó pronunciamientos críticos del Magisterio.
    \item \textbf{Fragmentación eclesial}: en ciertos contextos, su enfoque conflictivo y politizado provocó divisiones entre comunidades y dificultades para mantener la unidad en la diversidad pastoral.
\end{itemize}

\section*{V. Valoración bíblica, teológica y magisterial}
\textbf{Bíblica:} La Teología de la Liberación encuentra un sólido fundamento en las Escrituras. Los profetas, como Amós e Isaías, denunciaron la injusticia social y proclamaron un Dios que defiende a los pobres y oprimidos. La predicación de Jesús, especialmente en Lucas 4,18, donde anuncia la liberación a los cautivos y el año de gracia del Señor, constituye uno de sus pilares más importantes. Asimismo, el testimonio de las primeras comunidades cristianas, descritas en Hechos 2,44–45, refleja una vida compartida y solidaria que inspira su visión comunitaria y justiciera.

\textbf{Teológica:} Desde el punto de vista teológico, la TdL ha enriquecido significativamente la reflexión sobre el Reino de Dios y la moral social. Sin embargo, en algunos casos ha caído en una interpretación funcionalista de la fe, subordinando la revelación divina a esquemas ideológicos externos. Este enfoque corre el riesgo de vaciar el contenido trascendente del mensaje cristiano, reduciendo la salvación a un proyecto meramente histórico o político.

\textbf{Magisterial:} El Magisterio de la Iglesia ha emitido pronunciamientos clave respecto a la Teología de la Liberación, principalmente a través de la Congregación para la Doctrina de la Fe en 1984 y 1986. En estos documentos —especialmente en \textit{Libertatis Nuntius}—, se reconoce el valor legítimo de su opción preferencial por los pobres y su compromiso con la justicia. No obstante, se advierte firmemente contra el uso acrítico del marxismo y otros marcos ideológicos incompatibles con la fe católica. El Magisterio insiste en que toda teología debe partir de la Revelación y no de análisis socio-políticos.

\section*{VI. Crítica al método marxista}
\begin{enumerate}
    \item \textbf{Reduccionismo materialista}: El marxismo interpreta la realidad desde una visión exclusivamente económica y materialista, ignorando las dimensiones espirituales, trascendentes y personales del ser humano. Este enfoque choca con la antropología cristiana, que reconoce en el hombre una vocación trascendente y una dignidad inalienable.
    
    \item \textbf{Concepción conflictiva de la historia}: Al centrarse en la lucha de clases como motor de la historia, el marxismo promueve una visión dialéctica del conflicto que puede llevar a justificar la violencia como medio legítimo de transformación social. Esta perspectiva contrasta con la enseñanza evangélica, que privilegia la reconciliación, la paz y el amor al prójimo, incluso al enemigo.
    
    \item \textbf{Riesgo de ideologización del Evangelio}: Cuando se adoptan categorías marxistas sin un filtro crítico, existe el peligro de instrumentalizar la fe para fines políticos o sociales. Esto puede llevar a reinterpretar el mensaje bíblico no desde la Revelación divina, sino desde intereses ideológicos, vaciando así su contenido salvífico.
    
    \item \textbf{Negación del sufrimiento redentor}: Para el marxismo, el sufrimiento es siempre un mal que debe eliminarse mediante la transformación estructural. Sin embargo, desde la fe cristiana, el dolor, aunque nunca deseado, puede tener un valor espiritual cuando es asumido en unión con Cristo crucificado. Ignorar esta dimensión representa una pérdida significativa del misterio pascual en la experiencia humana.
\end{enumerate}

\section*{VII. Discernimiento católico}
La Teología de la Liberación, cuando está purificada de ideologías incompatibles con la fe, representa un valioso aporte pastoral. Sus intuiciones legítimas —como la opción preferencial por los pobres, la denuncia de las estructuras injustas y la urgencia de una Iglesia cercana a los excluidos— han sido retomadas y reelaboradas en el magisterio reciente, especialmente bajo el pontificado del Papa Francisco.

Este discernimiento implica reconocer lo que es conforme al Evangelio y rechazar lo que contradice la fe o distorsiona la misión de la Iglesia. Por eso, el Papa Francisco ha impulsado una visión de «Iglesia pobre y para los pobres» que recoge las mejores intuiciones de la TdL, pero sin caer en reduccionismos ideológicos ni en prácticas que subordinen la Revelación a modelos socio-políticos.

La auténtica liberación cristiana no puede separarse de Cristo, ni desconocer la centralidad de la Eucaristía, la vida sacramental, la oración y la caridad fraterna. La opción preferencial por los pobres no es exclusivista ni política, sino testimonio del amor universal de Dios, que se manifiesta en la solidaridad con los más necesitados y en el anuncio de la salvación que Cristo ofrece a todos sin excepción.

\section*{VIII. Presencia actual de la Teología de la Liberación}
\begin{itemize}
    \item \textbf{Teología ecológica}: articulada en la encíclica \textit{Laudato Si’}, donde se reinterpreta la opción preferencial por los pobres incluyendo a la naturaleza como víctima de la explotación humana.
    \item \textbf{Teología de los pueblos originarios}: inspirada en el Sínodo para la Amazonía (2019), que busca una Iglesia encarnada en las culturas ancestrales y promotora de una evangelización respetuosa de las identidades locales.
    \item \textbf{Teologías feministas y decoloniales}: herederas del enfoque liberador, estas corrientes amplían la noción de opresión, abordando la injusticia desde la perspectiva de género, raza y colonialidad.
\end{itemize}

\section*{IX. Discernimiento actual}
\textbf{Aspectos positivos:}
\begin{itemize}
    \item \textbf{Inclusión de nuevos excluidos}: hoy se amplía la opción preferencial por los pobres para incluir a pueblos originarios, mujeres en contextos de opresión y al propio planeta como víctima de la explotación humana.
    \item \textbf{Lectura contextual del Evangelio}: se mantiene viva la necesidad de encarnar la fe en las realidades concretas, promoviendo una pastoral sensible a las culturas, lenguajes y sufrimientos actuales.
\end{itemize}

\textbf{Peligros:}
\begin{itemize}
    \item \textbf{Riesgo de sincretismo religioso}: en algunas expresiones contemporáneas, especialmente en teologías indígenas o ecospirituales, puede darse una fusión poco crítica entre símbolos cristianos y creencias ancestrales que oscurecen el mensaje cristiano.
    \item \textbf{Activismo sin base teológica}: cuando el compromiso social se separa de la vida sacramental y espiritual, puede caerse en un cristianismo meramente político, carente de profundidad doctrinal y experiencia de gracia.
    \item \textbf{Olvido de la vida sacramental}: hay corrientes que, en nombre de la justicia y la transformación social, subordinan la liturgia, la oración y los sacramentos, debilitando así la fuente misma de la vida cristiana.
\end{itemize}

% Continúa... (se omiten algunas secciones por espacio)

\appendix

\section*{X. Variantes y evolución de la Teología de la Liberación}
Desde sus orígenes en las décadas de 1960 y 1970, la Teología de la Liberación (TdL) ha experimentado una evolución significativa, dando lugar a diversas corrientes que adaptan sus principios fundamentales a nuevas realidades sociales, culturales y ecológicas. Estas variantes responden al dinamismo de la vida de la Iglesia en contextos plurales y reflejan una voluntad constante de encarnar el Evangelio en los signos de los tiempos.

% Aquí puedes insertar las variantes completas si prefieres

\section*{XI. Refutación teológica desde Ratzinger}
La Teología de la Liberación ha surgido como una respuesta dolorida ante situaciones reales de injusticia, pobreza e indignidad humana que claman al cielo. Nadie puede negar la urgencia de que la Iglesia haga suyo el clamor de los pobres y responda con el testimonio del Evangelio. Sin embargo, no toda respuesta es conforme al misterio de Cristo ni a la misión confiada por Él a su Iglesia.

Como guardián de la fe, debo señalar que, en varias de sus formulaciones, la llamada Teología de la Liberación incurre en errores graves, no tanto por su intención pastoral, que puede ser legítima, sino por el marco ideológico que adopta y las consecuencias doctrinales que ello acarrea.

\subsection*{1. Reducción del Evangelio a un proyecto sociopolítico}
La fe cristiana no es una ideología ni un programa de reforma social. Cuando la teología toma como punto de partida categorías marxistas —como la lucha de clases, la dialéctica de opresores y oprimidos, o la praxis revolucionaria—, corre el riesgo de vaciar el contenido salvífico del Evangelio, reduciéndolo a una herramienta de cambio estructural. Pero la liberación más profunda es la que Cristo nos ha traído: la liberación del pecado y de la muerte, que ninguna revolución humana puede otorgar.

\section*{XII. Sección testimonial: Reflexión crítica desde la experiencia vivida}
Este apéndice recoge la experiencia de un testigo ocular de la evolución de la Teología de la Liberación (TdL) en América Latina durante las décadas de 1970 y 1980. Se trata de una mirada desde la historia vivida, no meramente académica, que permite observar de primera mano los efectos pastorales y eclesiales de dicha corriente.

Si bien se reconoce el mérito de la Teología de la Liberación al denunciar estructuras de injusticia y al proclamar una opción preferencial por los pobres, también se evidencian ciertos riesgos pastorales derivados de su adopción, especialmente en contextos donde el análisis marxista fue asumido de manera acrítica...

\section*{XIII. Comparación entre la Teología de la Liberación y la Doctrina Social de la Iglesia}
\renewcommand{\arraystretch}{1.5}
\begin{tabularx}{\textwidth}{|>{\bfseries}m{4cm}|X|X|}
\hline
\textbf{Criterio} & \textbf{Teología de la Liberación} & \textbf{Doctrina Social de la Iglesia (DSI)} \\
\hline
Origen histórico & Surge en América Latina en los años 60-70, especialmente tras Medellín (1968) y el Vaticano II. & Se desarrolla desde \textit{Rerum Novarum} (1891) pero se intensifica tras el Vaticano II. \\
\hline
Inspiración teológica & Opción preferencial por los pobres como eje teológico. & Dignidad humana como principio fundamental. \\
\hline
Método teológico & Ver – Juzgar – Actuar; se apoya en las ciencias sociales y análisis marxista. & Reflexión desde la Revelación y el Magisterio, abierta al discernimiento social. \\
\hline
Relación con el marxismo & Algunos teólogos adoptan herramientas de análisis marxista sin compartir su ideología. & Rechazo explícito del marxismo por su materialismo y ateísmo. \\
\hline
Recepción magisterial & Cuestionada por la Congregación para la Doctrina de la Fe en los años 80 (ej. \textit{Libertatis Nuntius}). & Desarrollada y promovida por encíclicas como \textit{Centesimus Annus}, \textit{Caritas in Veritate}, \textit{Fratelli Tutti}. \\
\hline
Figuras representativas & Gustavo Gutiérrez, Leonardo Boff, Jon Sobrino. & Juan XXIII, Juan Pablo II, Benedicto XVI, Francisco. \\
\hline
\end{tabularx}

\includegraphics[width=\linewidth]{Teologia_Liberacion_vs_Pueblo.pdf}

\section*{Conclusión}
La TdL ha recordado a la Iglesia su misión profética en favor de los pobres. Ha renovado la pastoral y la espiritualidad en muchos contextos. Pero su asociación con el marxismo ha exigido discernimiento y corrección doctrinal. Hoy sobrevive en formas renovadas y fieles al Magisterio. El desafío es integrar sus mejores intuiciones dentro de una teología cristocéntrica, sacramental, escatológica y en comunión con toda la Iglesia.

Solo desde Cristo liberador, muerto y resucitado, se comprende y realiza la verdadera liberación: aquella que restaura la dignidad del hombre como hijo de Dios y anticipa en la historia los signos del Reino.

\end{document}
```

---

### 📄 ¿Cómo usar este código?

1. Abre tu editor LaTeX (Overleaf, TeXmaker, VS Code + LaTeX, etc.)
2. Crea un nuevo documento `.tex`
3. Pega este código completo
4. Asegúrate de tener las imágenes necesarias (`logo-ucsm.png` y `Teologia_Liberacion_vs_Pueblo.pdf`) en la carpeta del proyecto
5. Compila el documento en PDF

---

¿Quieres que te envíe el documento completo como **archivo .txt** para descargarlo?  
¿O prefieres que te dé una versión **precompilada en PDF**?

Espero tu indicación.
