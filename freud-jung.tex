\documentclass[12pt]{article}
\usepackage[utf8]{inputenc}
\usepackage[spanish]{babel}
\usepackage{geometry}
\usepackage{booktabs}
\usepackage{tabularx}
\usepackage{parskip}
\geometry{a4paper, margin=2.5cm}

\title{Comparación: La Sombra de Jung y el Inconsciente de Freud}
\author{Jorge Ayona}
\date{\today}

\begin{document}
\maketitle

\section*{Introducción}

A primera vista, la Sombra de Carl G. Jung puede parecer equivalente al inconsciente descrito por Sigmund Freud. Sin embargo, aunque hay puntos en común, ambos conceptos difieren profundamente en su origen, función y finalidad. A continuación, se ofrece una comparación clara entre ambos.

\section*{Tabla comparativa}

\begin{tabularx}{\textwidth}{>{\raggedright\arraybackslash}p{3cm} 
>{\raggedright\arraybackslash}X 
>{\raggedright\arraybackslash}X}
\toprule
\textbf{Tema} & \textbf{Freud: Inconsciente} & \textbf{Jung: Sombra} \\
\midrule

Naturaleza & 
Contenido reprimido de carácter pulsional: deseos sexuales, agresivos, traumas. & 
Aspectos negados o reprimidos del yo: tanto impulsos oscuros como potenciales no desarrollados. \\

Origen & 
Formado por experiencias infantiles reprimidas. & 
Resultado de la represión cultural, moral y social, y también del descuido del alma. \\

Contenido & 
Deseos prohibidos, recuerdos dolorosos, pulsiones inaceptables para el yo. & 
Emociones, traumas, deseos, pero también intuiciones, vitalidad, creatividad negada. \\

Función & 
Mantiene ocultos los conflictos para evitar angustia; provoca síntomas si no se libera. & 
Debe ser confrontada e integrada para alcanzar la individuación y la autenticidad. \\

Método de trabajo & 
Asociación libre, interpretación de sueños, transferencia terapéutica. & 
Análisis de sueños, proyecciones, símbolos, diálogo interior, integración de opuestos. \\

Finalidad & 
Liberación de tensiones inconscientes. Adaptación al entorno social. & 
Integración de la totalidad psíquica. Desarrollo del Sí-mismo y plenitud del ser. \\

Perspectiva espiritual & 
No considerada; Freud era materialista y crítico de la religión. & 
Compatible con lo espiritual. La sombra puede ser redimida, iluminada y reconciliada. \\

\bottomrule
\end{tabularx}

\section*{Cita clave de Jung}

\begin{quote}
“Uno no se ilumina imaginando figuras de luz, sino haciendo consciente su oscuridad.”  
\hfill – Carl Gustav Jung
\end{quote}

\section*{Conclusión}

La sombra junguiana no es solo un rincón oscuro a evitar: es un umbral hacia el autoconocimiento. Mientras que el inconsciente de Freud se asocia a lo reprimido y peligroso, la sombra en Jung también contiene lo que hemos olvidado de nuestro verdadero yo. Integrarla no es un acto psicológico solamente, sino un camino hacia la verdad interior.

\end{document}
