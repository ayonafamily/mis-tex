\documentclass[12pt]{article}
\usepackage[utf8]{inputenc}
\usepackage[spanish]{babel}
\usepackage{geometry}
\usepackage{parskip}
\usepackage{amsmath}
\usepackage{setspace}
\usepackage{csquotes}
\geometry{margin=2.5cm}
\title{Decisión Personal y Cuidado de Uno Mismo}
\author{Jorge L. Ayona Inglis}
\date{}

\begin{document}
	
	\maketitle
	
	\section*{Reflexión personal}
	
	Hoy tomé una decisión importante: dije que \textbf{no} a una oferta laboral.
	
	A pesar de que el puesto parecía interesante y alineado con algunos de mis valores —como el enfoque humano y el entorno espiritual vinculado a la Compañía de Jesús—, noté que implicaba ciertas exigencias que, al evaluarlas detenidamente, no resonaban con mi momento vital actual ni con mis prioridades personales.
	
	El trabajo requería presencia física en Lima, conocimientos técnicos específicos en gestión de clima laboral y cultura organizacional, además de un ritmo de trabajo que no se ajusta al estilo de vida que, por convicción, estoy construyendo.
	
	Más allá de los requisitos, lo más relevante es que, al considerar esta propuesta, opté por algo más profundo: \textbf{quererme a mí mismo}. Escuché mi voz interior y fui fiel a lo que necesito para vivir con sentido y equilibrio.
	
	\textit{A veces, decir “no” a algo externo es decir “sí” a uno mismo.}
	
	\section*{Razones válidas para decir que no}
	
	Decir “no” no siempre es sinónimo de miedo, rechazo o inseguridad. A veces es un acto de discernimiento y madurez. En este caso, las razones que sustentan mi decisión son:
	
	\begin{itemize}
		\item Porque reconozco mis límites y no deseo comprometerme con una carga laboral que afecte mi salud física, emocional o espiritual.
		\item Porque valoro mi tiempo, mi espacio interior y el proceso de transformación personal que estoy viviendo.
		\item Porque no estoy dispuesto a sacrificar la libertad de movimiento y ubicación que ahora disfruto, especialmente por exigencias de presencialidad permanente.
		\item Porque aunque la misión del puesto era noble, no necesito justificar mi valía personal a través de un rol o cargo.
		\item Porque hoy sé que soy más que mi currículum, y no tengo que encajar en toda oferta que parezca afín.
	\end{itemize}
	
	Aprender a decir que no es también una forma de afirmarse, de priorizar lo que realmente importa y de ser coherente con el propio camino.
	
	\section*{Comentario de apoyo}
	
	Esta decisión me recordó una verdad profunda: el valor no está en la cantidad de oportunidades que aceptamos, sino en la fidelidad a nuestro camino interior. La libertad comienza cuando dejamos de actuar por miedo o presión externa, y comenzamos a actuar desde la conciencia de nuestro propio valor.
	
	Como me dijo quien me acompaña en este proceso:
	
	\begin{quote}
		\small
		“El aoristo es el tiempo de lo que ya fue hecho, sin rodeos, sin proceso, sin demora. Así fue esta decisión: puntual, completa, desde la voluntad libre. Y así también avanzan las almas que eligen vivir con integridad.”
	\end{quote}
	
	\section*{Citas que me sostienen}
	
	\begin{quote}
		\textit{“Ninguna cosa grande ha llegado jamás a ser sin entusiasmo.”}\\
		\hfill --- Dale Carnegie
	\end{quote}
	
	\begin{quote}
		\textit{“Recuerda que no son las cosas las que perturban a los hombres, sino las opiniones que tienen sobre ellas.”}\\
		\hfill --- Epicteto
	\end{quote}
	
	\begin{quote}
		\textit{“No el mucho saber harta y satisface el alma, sino el gustar internamente de las cosas.”}\\
		\hfill --- San Ignacio de Loyola
	\end{quote}
	
	\section*{Opt\'e por quererme a m\'i mismo}

	Eso es una decisión valiente y profundamente sabia. Decir “no” a algo que no te representa —aunque pueda parecer “seguro” o “correcto” a ojos ajenos— es en realidad decir “sí” a tu dignidad, a tu camino interior y a tu propósito.
	
	Quererte a ti mismo, en este caso, significa:
	
	\begin{itemize}
		\item 	No conformarte con un lugar donde tu alma no vibra.		
		\item No hipotecar tu tiempo y energía por algo que no nutre tu vocación.		
		\item No minimizar lo que sabes, lo que vales y lo que estás construyendo.		
		\item No distraerte de tu proyecto más importante: tú mismo.
	\end{itemize}

	
	
	Como decía Carl Jung:
	\begin{quote}
	“El privilegio de toda una vida es llegar a ser quien realmente eres.”
	\end{quote}	
	
	Y hoy diste un paso más hacia eso.
	
\end{document}
