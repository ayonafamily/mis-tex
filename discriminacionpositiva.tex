\documentclass[12pt]{article}
\usepackage[utf8]{inputenc}
\usepackage[spanish]{babel}
\usepackage{geometry}
\usepackage{parskip}
\usepackage{hyperref}
\usepackage{setspace}
\geometry{a4paper, margin=2.5cm}
\title{La Opción Preferencial por los Pobres y la Discriminación Positiva: ¿Convergencia o Contradicción?}
\author{Jorge L. Ayona Inglis}
\date{\today}

\begin{document}

\maketitle
\onehalfspacing

En la actualidad, ciertos paralelismos entre la "opción preferencial por los pobres" en la doctrina social de la Iglesia y la "discriminación positiva" promovida por corrientes culturales contemporáneas, especialmente el pensamiento denominado "woke", invitan a un análisis comparativo. A primera vista, ambas propuestas parecen compartir una preocupación común por los marginados y excluidos; sin embargo, sus fundamentos, objetivos y consecuencias presentan diferencias sustanciales que es necesario dilucidar.

La opción preferencial por los pobres, formulada en el marco del Concilio Vaticano II y desarrollada en las conferencias episcopales latinoamericanas (Medellín, Puebla, Aparecida), parte de la Revelación cristiana y de la imitación de Cristo, quien se identificó con los pobres, los enfermos y los excluidos. No se trata de una opción ideológica, sino teológica y pastoral: nace del amor cristiano que reconoce en cada persona una dignidad sagrada. El Compendio de la Doctrina Social de la Iglesia subraya que esta opción no es exclusiva ni excluyente, sino una prioridad en la acción evangelizadora y social de la Iglesia (n. 182)\cite{compendio}.

Por su parte, la discriminación positiva o acción afirmativa consiste en políticas públicas o medidas institucionales que otorgan privilegios o compensaciones especiales a colectivos históricamente desfavorecidos (por raza, género, orientación, clase, etc.). Aunque surgió con fines correctivos, en muchos contextos ha derivado en dinámicas de confrontación, victimismo estructural y polarización social. En lugar de promover la igualdad desde la dignidad común, la lógica woke tiende a dividir a la sociedad en opresores y oprimidos, generando resentimiento y nuevos privilegios fundados en la pertenencia identitaria\cite{woke}.

La opción preferencial por los pobres apunta a la reconciliación, al servicio y a la inclusión desde la caridad. La discriminación positiva, en cambio, tiende a fomentar antagonismos grupales y a reemplazar la justicia por cuotas y correcciones forzadas. Además, mientras la primera se basa en una antropología trascendente, la segunda responde a una lógica materialista, influida por el neomarxismo y el pensamiento posmoderno.

Una diferencia adicional es el alcance espiritual. En el Evangelio, los pobres no son solo los carentes de bienes materiales, sino aquellos abiertos a Dios: los "pobres de espíritu" (Mt 5,3), los humildes, los que reconocen su necesidad de salvación. En este sentido, la pobreza evangélica es una actitud del alma, no una categoría sociológica\cite{evangelio}.

En síntesis, aunque ambas propuestas pueden coincidir en su sensibilidad hacia el sufrimiento humano, sus caminos son distintos. La opción preferencial por los pobres es una exigencia del Evangelio; la discriminación positiva es una construcción ideológica con riesgos evidentes de injusticia inversa. Confundir ambas es diluir el testimonio cristiano y someterlo a lógicas que no provienen del Reino de Dios, sino de intereses mundanos. La Iglesia está llamada a discernir con claridad, para no trocar la caridad por ideología ni la justicia por revancha.

\begin{thebibliography}{9}

\bibitem{compendio}
Pontificio Consejo Justicia y Paz.
\textit{Compendio de la Doctrina Social de la Iglesia}.
Libreria Editrice Vaticana, 2004.

\bibitem{woke}
Pluckrose, Helen y Lindsay, James.
\textit{Cynical Theories: How Activist Scholarship Made Everything about Race, Gender, and Identity}.
Pitchstone Publishing, 2020.

\bibitem{evangelio}
Biblia.
\textit{Evangelio según San Mateo 5,3}.
Versión oficial de la Sagrada Escritura. Conferencia Episcopal Española.

\end{thebibliography}

\end{document}
