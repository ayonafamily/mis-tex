\documentclass[12pt]{article}
\usepackage[spanish]{babel}
\usepackage[utf8]{inputenc}
\usepackage[T1]{fontenc}
\usepackage{verse}
\usepackage{setspace}
\usepackage{geometry}
\usepackage{titlesec}
\usepackage{lmodern}
\usepackage{xcolor}

\geometry{margin=2.5cm}
\titleformat{\section}{\large\bfseries\color{blue!60!black}}{\thesection}{1em}{}

\title{\Huge\textbf{El amor que no daña} \\ \large (Un manifiesto inspirado en Dave Stewart, Stevie Nicks… y mi propia historia)}
\author{}
\date{}

\begin{document}
	\maketitle
	\thispagestyle{empty}
	\begin{center}
		\textit{“Cheaper than free... that’s what I want to be” \\ — Dave Stewart \& Stevie Nicks}
	\end{center}
	
	\vspace{1.5cm}
	
	\section*{Introducción}
	
	Muy tarde en la noche, a media luz, mientras el mundo dormía y yo me dejaba envolver por la música, hallé estos tesoros. No los busqué: me encontraron. Sonidos antiguos, nombres olvidados, melodías que guardaban cicatrices y ternuras que yo también llevaba dentro.
	
	Entre canciones de Dave Stewart y Stevie Nicks, se abrió una puerta interior. Como si esas notas supieran lo que yo no podía decir. Lo que sigue es el eco de ese momento: un viaje hecho de recuerdos, sombras y revelaciones, en el que la música no fue solo compañía, sino verdad revelada.
	
	\section*{El amor que no daña}
	
	\begin{flushleft}
		\textbf{\large } \\
		\textit{A través de Dave Stewart, Stevie Nicks… y una voz interior}
	\end{flushleft}
	
	\vspace{0.5cm}
	
	Esta obra es una travesía emocional tejida con música, memoria y sombra. A partir de una serie de canciones —\textit{Jack Talking}, \textit{Heart of Stone}, \textit{Lily Was Here}—, y de una lectura sensible del alma de Dave Stewart, el autor se descubre a sí mismo en las grietas y renacimientos del artista.
	
	En un contrapunto luminoso, aparece Stevie Nicks como la figura que supo recibir sin herir, crear sin dominar, amar sin destruir. Frente al contraste con Annie Lennox o Lindsey Buckingham, este texto no pretende juzgar, sino entender: cómo se forman los vínculos creativos, cómo se rompen, y cómo uno puede emerger fortalecido desde la ternura, no desde la revancha.
	
	La pieza central, \textit{El amor que no daña}, es un manifiesto para todos los que alguna vez \textbf{dieron sin ser vistos}, \textbf{amaron sin ser devueltos}, y aun así siguieron creyendo en un amor sereno, libre, sin miedo. Como el de Dave, como el que todos —quizás sin saberlo— anhelamos.
	
	\section*{I. La herida invisible}
	
	\begin{verse}
		Hay un amor que hiere. \\
		Y hay otro… que sana sin prometerlo, \\
		que sostiene sin exigir aplausos, \\
		que está sin ruido, sin juego de poder.
	\end{verse}
	
	Ese amor fue el que Dave Stewart dio a Stevie Nicks,  
	cuando ya no esperaba nada,  
	cuando ya no tenía que probar quién era.
	
	Ella venía de una historia con un hombre brillante,  
	pero cruel.  
	Un hombre que escribió canciones como cuchillos,  
	que la necesitaba… pero no la respetaba.
	
	Y entonces llegó Dave.  
	Sin arrebato.  
	Sin imponerse.  
	Solo con su guitarra, su oído, su alma serena.  
	Y juntos crearon algo sin sangre.  
	Solo música.  
	Solo belleza.  
	Solo \textbf{presencia real}.
	
	\vspace{0.5cm}
	
	\section*{II. El reflejo en mí}
	
	Y yo…  
	yo también fui ese que dio sin ser visto.  
	Ese que escribió sin ser leído.  
	El que acompañó sin condiciones…  
	hasta que entendió que amar no es perderse.
	
	Yo también fui Dave ante alguien que solo sabía brillar para sí.
	
	Y hoy lo sé: \textbf{no todo amor duele}. \\
	El verdadero, no.
	
	\begin{verse}
		El verdadero amor escucha, \\
		no eclipsa. \\
		Te deja ser. \\
		Te hace mejor sin romperte.
	\end{verse}
	
	\vspace{0.5cm}
	
	\section*{III. La música como redención}
	
	Cuando escucho \textit{Cheaper Than Free},  
	no oigo un reclamo,  
	ni una súplica.  
	Oigo libertad.  
	Oigo la dulzura de un amor maduro, \\
	que no exige nada, \\
	que acompaña como el viento suave: sin quedarse, \\
	pero sin dañar.
	
	\begin{verse}
		Porque ya no hay deuda. \\
		Porque el amor real no cobra. \\
		Solo da… y deja ser.
	\end{verse}
	
	\vspace{1cm}
	
	\begin{flushright}
		\textit{— inspirado por la historia de Dave Stewart y Stevie Nicks, \\
			y por mi propia alma que sigue escuchando.}
	\end{flushright}
	
	\newpage

	
\end{document}
