\documentclass[a4paper,12pt]{article}
\usepackage[spanish]{babel}
\usepackage[utf8]{inputenc}
\usepackage[T1]{fontenc}
\usepackage{csquotes}
\usepackage{geometry}
\usepackage{setspace}
\usepackage{parskip}
\usepackage{hyperref}
\geometry{margin=2.5cm}
\onehalfspacing
\usepackage{hyperref}
\usepackage{graphicx}
\usepackage{color}
\definecolor{cato}{rgb}{0.4, 1.0, 0.0}
\usepackage{fontspec}
\setmainfont{Arial}



\begin{document}
	
\begin{titlepage}
	\begin{center}
		\huge \textbf{Universidad Católica de Santa María}\\[1cm]
		\large Facultad de Ciencias Sociales\\[3mm]
		\textbf{Escuela de Teología}\\[3mm]
		
		\begin{figure}[h]
			\centering
			\includegraphics[scale=0.15]{ucsm.jpg}
		\end{figure}
		
		\vspace{3mm}
		\textcolor{cato}{\rule{\linewidth}{0.5mm}}
		
		\vspace{4mm}
		{\large\textbf{
			El Evangelio como fuerza de integración cultural\\
			Una lectura teológica del mestizaje en el Perú a partir del relato “La fiesta de los negros”
		}}\\
		
		\vspace{5mm}
		\large{Curso: Identidad y Peruanidad}\\[2mm]
		\large{Profesora: Ana Rosario Miaury Vilca}\\[2mm]
		\large{Alumno: Jorge Ayona}\\
		\href{https://orcid.org/0009-0006-6551-9681}{ORCID: 0009-0006-6551-9681}\\[4mm]
		
		\textcolor{cato}{\rule{\linewidth}{0.5mm}}\\[2mm]
		{\large \today}
	\end{center}
\end{titlepage}

\tableofcontents
\newpage
	
	\abstract
	\addcontentsline{toc}{section}{Res\'umen}
	Este trabajo reflexiona sobre cómo el Evangelio ha actuado como principio de reconciliación y transformación en el encuentro entre culturas diversas durante el periodo virreinal del Perú, tomando como punto de partida la leyenda del Callao y la devoción al Señor de los Milagros. Se propone que el Evangelio no pertenece a una cultura específica, sino que las trasciende, las purifica e integra, generando un mestizaje espiritual donde los defectos de todas las partes son transformados por la gracia. Se incorporan citas bíblicas y reflexiones teológicas que apoyan esta visión, reconociendo también las tensiones históricas del proceso.
	
	\textit{Palabras clave:} mestizaje, Evangelio, inculturación, Perú virreinal, fe católica
	
	
	\section*{Introducción}
	\addcontentsline{toc}{section}{Introducción}
	La historia de la evangelización en América Latina está marcada por tensiones, encuentros y contradicciones entre culturas. En el caso peruano, estas dinámicas se reflejan en relatos como la leyenda del Callao, donde se describe un castigo divino sobre prácticas consideradas inmorales, y en contraste, en la figura del Señor de los Milagros, una expresión de fe profunda surgida desde el corazón mismo de la población afrodescendiente. Esta investigación busca explorar cómo el Evangelio actuó como principio transformador en este contexto, no como imposición cultural, sino como fuerza de reconciliación, conforme al mensaje cristiano universal. Es un ejemplo claro de nuestro mestizaje y el papel de la religi\'on.
	
	\section{Presentaci\'on del Tema \\ La fiesta de los Negros y la idea de castigo moral}
	Según una leyenda popular, el antiguo Puerto del Callao fue parcialmente destruido por el mar como castigo divino por prácticas religiosas y bailes considerados inmorales por la mentalidad cristiana de la época. 
	
	\bigskip
	\noindent\textit{Sin embargo, en lo que yo no estoy de acuerdo es en atribuir solo a los afrodescendientes diversos pecados.}
	
	\section{Interpretaci\'on Cr\'itica}Esta narrativa recuerda al episodio bíblico de Sodoma y Gomorra (cf. Gn 19,24-25), donde la corrupción moral colectiva atrae el juicio de Dios. Es probable que los narradores de esta leyenda estuvieran influenciados por esos referentes bíblicos, viendo el castigo natural como signo de la desaprobación divina. Véase el relato “La fiesta de los negros” (Arguedas \& Izquierdo Ríos, 1947, pp. XX–YY) incluido en el Apéndice.
	
	\subsection{El Señor de los Milagros: respuesta de fe desde los márgenes}
	Frente a esa imagen de castigo, emerge otra completamente distinta: la devoción al Señor de los Milagros, cuya imagen fue pintada por un esclavo de origen angoleño en un muro de adobe. Tras el terremoto de 1746, el mismo del relato anterior,la imagen quedó milagrosamente intacta, lo cual fue interpretado como señal del favor de Dios. Esta respuesta de fe no vino desde el centro del poder, sino desde un esclavo, lo cual recuerda las palabras de San Pablo: “Dios eligió lo necio del mundo para avergonzar a los sabios” (1 Co 1,27). \textbf{Es decir, se muestran 2 respuestas de integrantes del mismo grupo \'etnico a la fe.
	}
	\subsection{San Martín de Porres: símbolo de integración cristiana}
	Otro ejemplo de esta transformación lo representa San Martín de Porres, mulato, hijo ilegítimo, mestizo y marginado por su condición racial, pero elevado por su caridad y humildad. Es ejemplo de cómo la gracia no hace acepción de personas (cf. Rm 2,11) y cómo el Evangelio transforma a los excluidos en santos. Él encarnó un verdadero mestizaje espiritual, donde las heridas sociales fueron sanadas en Cristo.
	
	\section{La fe como fuerza unificadora}
	El Evangelio no pertenece a una cultura determinada. Como afirma San Pablo: “Ya no hay judío ni griego, esclavo ni libre, varón ni mujer, porque todos ustedes son uno en Cristo Jesús” (Ga 3,28). En los comienzos del cristianismo, incluso los apóstoles debieron afrontar el prejuicio racial y cultural respecto a si los gentiles podían ser salvados (cf. Hch 10,34-35; Hch 15,7-11). El mismo dilema puede aplicarse al Perú virreinal, donde la fe cristiana, bien asumida, actuó como principio de integración: tanto el africano esclavizado como el criollo fueron desafiados a dejarse transformar por el mensaje de Cristo.
		
	\section{El mestizaje como fruto de la acción del Espíritu}
	Este proceso no fue perfecto ni exento de pecado. Hubo abusos, prejuicios y estructuras de poder. Pero como en toda historia de salvación, Dios actúa también en medio del barro. El Evangelio toma a los hombres tal como son y los transforma lentamente, purificando tanto las costumbres africanas como los prejuicios hispánicos. Como afirma el Concilio Vaticano II: “El Evangelio de Cristo renueva constantemente la vida y la cultura del hombre caído” (Gaudium et Spes, 58).
	
	\section*{Conclusión}
	El relato del Callao y la historia del Señor de los Milagros no deben entenderse en oposición simplista, sino como expresiones diversas de un mismo proceso: la entrada del Evangelio en medio de la historia humana concreta. A través de hombres y mujeres de distintas razas, el mensaje de Cristo ha ido transformando corazones, creando un nuevo mestizaje espiritual. Lo que une no es la sangre ni la cultura, sino la fe. Como escribió Christopher Dawson, “la religión es la forma más profunda de unidad entre los hombres, porque une las raíces de la cultura y del espíritu”.
	Esto cobra particular importancia en resaltar la corriente mestiza de la peruanidad, desde la \'optica de la religi\'on. Frente a un desastre natural, todas las razas sufrieron. Y el Se\'nor de los Milagros, aunque iniciada su devoci\'on por los afroperuanos,une a todas las sangres, no solo en una proesi\'on multitudinaria, sino en gastronom\'ia relacionada con su culto en Lima: ...anticuchos, picarones, turrón de doña Pepa, entre otros.
	
	
	\section*{Epílogo: cómo juzgar el pasado desde la fe}
	\addcontentsline{toc}{section}{Epílogo: cómo juzgar el pasado desde la fe}
	
	Al examinar relatos antiguos como el de la leyenda del desastre del Callao o los procesos de evangelización en el Perú virreinal, es necesario recordar que no podemos juzgar épocas pasadas desde los parámetros actuales. Esto sería caer en un anacronismo moral. Como enseña Santo Tomás de Aquino:
	
	\begin{quote}
		“Para que un acto sea moralmente bueno, se requiere que lo sea en cuanto a su objeto, su fin y sus circunstancias.” (\textit{S. Th.}, I–II, q. 18, a. 4)
	\end{quote}
	
	 Por lo tanto, para entender fenómenos como las Cruzadas, la esclavitud o la evangelización colonial, es indispensable situarse en los determinantes históricos, políticos, sociales y espirituales de la época.
	
	La Escritura también refleja este proceso en el surgimiento del cristianismo. Al inicio, incluso los propios apóstoles dudaban de si los gentiles podían acceder plenamente a la salvación. La conversión del centurión Cornelio (Hch 10) y el Concilio de Jerusalén (Hch 15) marcan momentos cruciales donde la Iglesia aprende que “Dios no hace acepción de personas” (Hch 10,34) y que “hemos sido salvados por la gracia del Señor Jesús, del mismo modo que ellos” (Hch 15,11).
	
	Esta evolución muestra que incluso en los primeros tiempos cristianos existieron resistencias culturales y prejuicios que debieron ser superados. Lo mismo puede afirmarse de los procesos de mestizaje espiritual en América: fueron imperfectos, pero no exentos de gracia. La historia, como lugar teológico, debe ser leída con discernimiento, evitando tanto la condena simplista como la glorificación ingenua. El Evangelio actúa en la historia no destruyendo las culturas, sino transformándolas desde dentro.
	
\appendix
\section*{Apéndice: “La fiesta de los negros”}
\addcontentsline{toc}{section}{Ap\'endice: “La fiesta de los negros”} 

\begin{quote}
	
	Cuenta la historia que hace muchos años el famoso Puerto del Callao se extendía hasta la Isla de San Lorenzo, pero que debido a un castigo mandado por Dios, se ha reducido a lo que es.\\
	
	Dicen que los negros festejaban a un dios desconocido, danzando los bailes más inmorales, que causaban escrúpulos entre los chalacos que lo presenciaban.\\
	
	Quiso Dios poner fin a esta fiesta escandalosa de los negros, y como para borrar esta falta puso todo su vigor sobre las aguas tranquilas del océano, haciendo que crecieran enormemente las olas, y buscando terreno donde extenderse, taparon inmensas áreas del Puerto, trayendo el espanto y terror de los negros.\\
	
	Estos corrieron a salvar sus vidas, pero todo fue en vano. Nunca más el mar azotó como aquel día.\\
	
	Sin embargo, todos los años para Semana Santa el mar se embravece, como recordando que en tiempos antiguos estos terrenos no le pertenecían.
\end{quote}

\vspace{3mm}
\noindent\textit{Fuente: Relato recogido en el Puerto del Callao por Estela Westphalen Milano, alumna del cuarto año de media del Colegio Nacional “Miguel Grau” de Magdalena Nueva, Lima.}

\vspace{3mm}
\noindent\textit{Fuente:} Arguedas, J. M., \& Izquierdo Ríos, F. (1947). \textit{Mitos, leyendas y cuentos peruanos}. Ministerio de Educación Pública del Perú.

\newpage	
	
	\begin{thebibliography}{9}
		\addcontentsline{toc}{section}{Referencias}
		
		\bibitem{biblia}
		La Santa Biblia. (1995). \textit{Versión Latinoamericana}. Sociedad Bíblica Católica Internacional.
		
		\bibitem{vaticanoII}
		Concilio Vaticano II. (1965). \textit{Gaudium et Spes: Constitución pastoral sobre la Iglesia en el mundo actual}. Vaticano.
		
		\bibitem{dawson}
		Dawson, C. (1958). \textit{La religión y el origen de la cultura occidental}. Encuentro.
		
		\bibitem{arguedas}
		Arguedas, J. M., \& Izquierdo Ríos, F. (1947). \textit{Mitos, leyendas y cuentos peruanos}. Ministerio de Educación Pública del Perú, Ediciones de la Dirección de Educación Artística y Extensión.
		\bibitem{tomas} Tomás de Aquino. (2001). \textit{Suma teológica} (I–II, cuestión 18, artículo 4; texto latino y traducción). Biblioteca de Autores Cristianos (BAC).


		
	\end{thebibliography}
	
\end{document}