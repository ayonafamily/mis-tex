\documentclass[12pt]{article}
\usepackage[utf8]{inputenc}
\usepackage[spanish]{babel}
\usepackage[a4paper,margin=2.5cm]{geometry}
\usepackage{parskip}
\usepackage{hyperref}
\hypersetup{
    colorlinks=true,
    linkcolor=blue,
    urlcolor=blue,
    pdftitle={Yo también tuve una conversión – Del fundamentalismo a la compasión por Palestina},
    pdfauthor={Jorge L. Ayona Inglis}
}

\title{\textbf{Yo también tuve una conversión – Del fundamentalismo a la compasión por Palestina}}
\author{Jorge L. Ayona Inglis\\\href{mailto:jorge.ayona@estudiante.ucsm.edu.pe}{jorge.ayona@estudiante.ucsm.edu.pe}}
\date{\today}

\begin{document}

\maketitle

\section*{Introducción}

Nací en 1961. Fui evangélico, pastor incluso, y durante mucho tiempo viví bajo el influjo de una interpretación literal de la Biblia. Eso me hacía creer que Israel era siempre el lado justo. Que el pueblo palestino era solo enemigo. Que el relato bíblico moderno coincidía con la política actual, y que apoyar al Estado de Israel era deber cristiano.

Pero la historia, la vida, y sobre todo los rostros, me fueron enseñando otra cosa.

\section*{La conversión}

Hace unos años, tras una ofensiva militar en Gaza, comencé a cuestionar todo. Me pregunté: ¿qué pasa con los cristianos palestinos? ¿Qué sienten, qué viven? Descubrí testimonios silenciados, de católicos que vivían en la Franja de Gaza, en Cisjordania. No eran enemigos. Eran hermanos en la fe, sufriendo bajo bombas, bloqueos y muros.

Leí también la novela \textit{El Haj}, de León Uris. Por primera vez, entendí lo que significa ser narrado siempre como el villano. Vi la historia desde el otro lado. No para negar el Holocausto ni el dolor judío, sino para comprender que ningún pueblo tiene derecho a repetir la injusticia en nombre de su propio sufrimiento.

Hace algunos meses, en el funeral de una tía diplomática, conocí a la secretaria del embajador de Palestina. Me acerqué a ella, quizás el único en hacerlo. Le conté que yo también había cambiado. Que ahora veía. Que me dolía profundamente que su pueblo fuera siempre “el malo de la película”.

\section*{Una paradoja dolorosa}

Es terrible pensar que, en nombre de una identidad histórica, algunos hayan convertido en enemigos a quienes comparten con ellos sangre, lengua y memoria.

Que los descendientes de pueblos que fueron encerrados en guetos hoy levanten muros sobre pueblos semitas como ellos, es una herida abierta en la conciencia de la humanidad.

\section*{Cristianismo no es fanatismo sionista}

Hay quienes desde el poder político o religioso usan el nombre de Cristo para respaldar muros, masacres, ocupación. Lo hacen en nombre de una teología rota, descompuesta por el fanatismo y la ideología.

Yo fui pastor. Y sé lo que se predica. Pero también sé lo que Jesús hizo: se puso del lado del caído, no del imperio.

\section*{Una tierra “vacía”… que gritaba desde sus piedras}

Asistí a una conferencia en Lima, organizada por una agencia vinculada a Israel. Se hablaba de la tierra “deshabitada” de Palestina, de la legitimidad histórica de la ocupación. Me lo creí.

Pero luego leí. Pensé. Vi. Palestina nunca estuvo vacía. Vi que en ciudades como Zafet vivían judíos cabalistas desde hace generaciones. Pero también cristianos, musulmanes, campesinos que conocían cada piedra de su aldea.

\section*{No puede haber justicia sin verdad}

Apoyar a Israel no significa callar ante sus abusos.  
Apoyar al pueblo judío no exige deshumanizar al pueblo palestino.  
Y seguir a Cristo no autoriza nunca la complicidad con la violencia.

\section*{Epílogo – La conciencia ilustrada}

Hoy existen nuevos dogmas laicos. Uno es la supuesta supremacía moral del Estado de Israel, sustentada en derechos bíblicos más que históricos o jurídicos.

Incluso la Declaración Balfour fue una concesión política a Chaim Weizmann, por su ayuda al esfuerzo de guerra británico. Fue en Londres donde se decidió el destino de millones.

No impongo verdades. Apelo a la razón, la lógica y el sentido común.

\section*{Nota al lector}

Jesucristo dijo: ``Conoceréis la verdad, y la verdad os hará libres''. Buscar la verdad es buscarlo a Él. La Biblia no debe ser un libro que embote la conciencia, sino que libere.

Desde que comencé a leer la Escritura sin ideologías, encontré una libertad que me permite hacer preguntas. Esa búsqueda es lo que comparto aquí. Sólo pido al lector: un ejercicio de recta razón.

\section*{Lecturas recomendadas}

\begin{itemize}
    \item Edward Said, \textit{Orientalismo}.
    \item Ilan Pappé, \textit{La limpieza étnica de Palestina}.
    \item León Uris, \textit{El Hach}.
    \item Testimonios de cristianos palestinos (disponibles en páginas como \url{https://www.lpj.org}).
    \item Documental: \textit{The Gatekeepers} (Israel, 2012).
    \item Película: \textit{Munich} (dir. Steven Spielberg, 2005).
\end{itemize}

\end{document}
