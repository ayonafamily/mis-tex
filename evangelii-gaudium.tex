\documentclass[12pt]{article}

% PAQUETES
\usepackage[utf8]{inputenc}
\usepackage[T1]{fontenc}
\usepackage[spanish]{babel}
\usepackage[a4paper, margin=2.54cm]{geometry}
\usepackage{setspace}
\usepackage{parskip}
\usepackage{helvet}
\usepackage{hyperref}
\renewcommand{\familydefault}{\sfdefault}
\usepackage{csquotes}
\usepackage[style=apa, backend=biber, language=spanish]{biblatex}


%\addbibresource{bibliografia.bib}

% DATOS DEL DOCUMENTO
\title{Evangelii Gaudium E Identidad Cultural Y Religiosa Del Perú.}
\author{Jorge Ayona\\
	{\footnotesize \protect\href{https://orcid.org/0009-0006-6551-9681}{ORCID: 0009-0006-6551-9681}}\\
	\texttt{jorge.ayona@estudiante.ucsm.edu.pe}}

\date{29 de mayo del 2025}

\begin{document}
	\maketitle
	\newpage
	\onehalfspacing
	
\section*{Introducción}

En el presente trabajo, he considerado tanto el documento de Aparecida (2007) como la exhortación apostólica \textit{Evangelii Gaudium} reflejan los aportes decisivos del entonces arzobispo de Buenos Aires, Jorge Mario Bergoglio, después Papa Francisco, de feliz memoria. En ambos documentos se reconoce su compromiso con una Iglesia en salida, encarnada en los contextos reales de América Latina, y se expresa una opción preferencial por los pobres que no cae en ideologismos, sino que busca la reconciliación entre los pueblos, culturas y sectores sociales. Ambos documentos reflejan también un enfoque teológico y pastoral que busca encarnar el Evangelio en las culturas concretas de nuestros pueblos, desde una mirada integral, misionera y dialogante.

En Aparecida, el futuro Papa tuvo un rol decisivo como redactor del texto final. Allí se vislumbra ya su insistencia en una Iglesia en salida, comprometida con los más pobres, abierta al diálogo con todos y llamada a formar discípulos misioneros en medio del mundo. Esta línea se desarrollará con mayor madurez y fuerza en \textit{Evangelii Gaudium}, donde Francisco, ya como Sucesor de Pedro, presenta un programa eclesial que nace de su experiencia pastoral en América Latina, pero con resonancia universal.

Francisco, en coherencia con su formación ignaciana, vivió y promovió el principio de dar por cierta, en la medida de lo posible, la proposición del prójimo, tal como lo expresa San Ignacio de Loyola en los \textit{Ejercicios Espirituales}: 
\begin{displayquote}
	“se ha de presuponer que todo buen cristiano ha de ser más pronto a salvar la proposición del prójimo, que a condenarla [...]” (Ignacio de Loyola, 1548/2025, Anotación 22).
\end{displayquote}
Es decir, entenderlo desde su mejor intención. Esta actitud es clave para cualquier proceso de reconciliación, inculturación y diálogo en contextos tan diversos y complejos como el peruano.

Desde esta perspectiva, el presente trabajo busca analizar algunos núcleos fundamentales de \textit{Evangelii Gaudium} —la inculturación de la fe (nn. 68–70), la inclusión social de los pobres (nn. 186–216) y el diálogo social para la construcción de la paz (nn. 238–258)— a la luz de la realidad peruana y en diálogo con las intuiciones del Documento de Aparecida. En particular, se subrayan dos aportes clave del texto de Aparecida: el llamado a ser discípulos misioneros y la pastoral de élites, como respuestas pastorales frente a los desafíos actuales de evangelización, sin caer en reduccionismos ideológicos.

\section{La inculturación de la fe y la identidad cultural}

En los números 68 al 70 de \textit{Evangelii Gaudium}, el Papa Francisco insiste en que la fe cristiana no puede vivirse al margen de la cultura. Muy por el contrario, debe penetrar y transformarla desde dentro, respetando sus símbolos, su sensibilidad y su sabiduría ancestral. Como afirma el Papa citando a Juan Pablo II, “la fe que no se hace cultura es una fe no plenamente acogida, no totalmente pensada, no fielmente vivida” (EG 68). Esta afirmación resuena poderosamente en el contexto latinoamericano, y en particular en el Perú, donde conviven raíces indígenas, mestizas y occidentales que conforman una identidad compleja y viva.

Esta idea coincide con lo que desarrollé en un trabajo anterior acerca del historiador británico Christopher Dawson, quien sostiene que no existen “culturas cristianas” en sentido estricto, sino culturas que acogen los valores cristianos sin dejar de ser ellas mismas (Ayona, 2025a). Por tanto, la inculturación es un proceso de transformación interior, no de suplantación.

Este planteamiento resulta clave para comprender que la evangelización no pretende destruir lo propio de los pueblos, sino que, como recoge \textit{Evangelii Gaudium}, el anuncio cristiano no anula las culturas, sino que las fecunda desde dentro, respetando su propio dinamismo (cf. EG 69), dando frutos nuevos sin uniformidad ni colonialismo espiritual. La visión de Dawson y la enseñanza del Papa Francisco coinciden en que la gracia no suprime la cultura, sino que la supone y la transforma, como también lo expresa EG 115.

Este enfoque nos desafía a mirar con otros ojos la identidad nacional del Perú, una identidad profundamente marcada por la religiosidad popular, las celebraciones sincréticas y las expresiones de fe en la vida cotidiana de los pueblos andinos y amazónicos, las cuales no siempre han sido plenamente valoradas por la Iglesia institucional. La inculturación, entonces, no es solo un método pastoral, sino una exigencia teológica y misionera que afirma que Dios ya está obrando en el corazón de las culturas, y que la tarea de la Iglesia es reconocer, purificar y potenciar esa presencia.

En este sentido, Francisco nos exhorta a no caer en el reduccionismo de “importar modelos pastorales” ajenos a nuestras realidades locales. La misión exige creatividad, escucha y una profunda valoración de la sabiduría popular. Cuando la fe cristiana es verdaderamente inculturada, se convierte en fermento de dignidad, justicia y reconciliación dentro de las culturas, sin negar sus raíces ni su identidad.

Sin embargo, el proceso de inculturación también implica reconocer las debilidades y deformaciones presentes en las culturas populares, como el machismo, el alcoholismo, la violencia doméstica o las creencias supersticiosas que aún persisten en muchos contextos (Francisco, 2013, p. 69). Es responsabilidad de la Iglesia acompañar este proceso de purificación y maduración cultural para que la fe cristiana se traduzca en auténticos signos de justicia y fraternidad. Así, la inculturación no es solo un enriquecimiento cultural, sino una transformación profunda que permite superar las limitaciones sociales y espirituales de las comunidades.

Finalmente, el desafío de la inculturación también se refleja en la ruptura de la transmisión generacional de la fe, que ha afectado a muchos países latinoamericanos. El Papa Francisco advierte que la falta de espacios de diálogo familiar, la influencia de medios masivos de comunicación, el relativismo y el consumismo desenfrenado, junto con una pastoral insuficiente, han llevado al desencanto y la pérdida de identidad en muchos fieles (Francisco, 2013, pp. 69-70). Por tanto, la inculturación de la fe debe ir acompañada de una renovación pastoral que promueva una experiencia mística y comprometida con la realidad social, fortaleciendo así la identidad cristiana desde sus raíces culturales.

En este marco, el Perú presenta una riqueza particular: un mestizaje cultural que no debe entenderse como pérdida de lo indígena o lo europeo, sino como una oportunidad para la integración. Este mestizaje, que autores del “boom” latinoamericano han expresado a través de lo “real maravilloso”, puede constituir una vía de reconciliación con nosotros mismos. En lugar de vivir divididos, debemos reconocer que conviven diversos modos de vida, y que todos podemos aprender unos de otros. Esta es una justa exigencia para inculturar el Evangelio.

\begin{displayquote}
	“Una cultura popular evangelizada contiene valores de fe y de solidaridad que pueden provocar el desarrollo de una sociedad más justa y creyente, y posee una sabiduría peculiar que hay que saber reconocer con una mirada agradecida” (EG 68).
\end{displayquote}

Esta sabiduría popular, lejos de ser un residuo folclórico, es un lugar teológico donde el Espíritu Santo ya actúa, y donde la Iglesia debe aprender a escuchar y acompañar.

\section{La inclusión social de los pobres: una exigencia del Evangelio}

Del número 186 al 216, Francisco aborda la inclusión de los pobres como una dimensión esencial del anuncio del Evangelio. No se trata solo de una opción preferencial, sino de una exigencia teológica: 
\begin{displayquote}
	“No puede ser auténtica una evangelización que no proclame la centralidad de los pobres” (Francisco, 2013, §187).
\end{displayquote}

Desde esta óptica, también he sostenido que la evangelización no puede reducirse a un mero activismo social ni caer en ideologismos. Como señalé en un artículo de mi blog personal, es necesario partir de una experiencia viva del Evangelio que transforme tanto al evangelizador como a las estructuras sociales (Ayona, 2025b). La evangelización integral, entonces, no separa alma y cuerpo, ni fe y compromiso social.

El llamado a ser discípulos misioneros —como se expresa en Aparecida (CELAM, 2007, §14, §278)— implica precisamente esta presencia activa en las realidades más complejas, pero sin caer en la dialéctica del conflicto. La pobreza no debe ser instrumentalizada ideológicamente, sino evangelizada con ternura, buscando la inclusión real en las estructuras sociales, la promoción de la justicia y la dignidad humana.

\textit{Evangelii Gaudium} denuncia con fuerza la exclusión social como un pecado estructural que afecta a los pobres, señalando la cultura del descarte que prevalece en la sociedad actual (EG 53-59). En el Perú, esto es particularmente visible en la marginalización de pueblos indígenas, campesinos y sectores populares urbanos. La exclusión se manifiesta no solo en la pobreza material, sino en la exclusión cultural y social, lo que dificulta la participación plena en la sociedad y la Iglesia.

Francisco invita a la Iglesia a asumir una “pastoral de élites” que promueva una conversión de aquellos que tienen más poder, riqueza o influencia social, para que se comprometan con la justicia y el bien común. Esto es fundamental, pues el cambio social no puede depender solo de los pobres, sino también del compromiso ético y cristiano de quienes detentan el poder.

El desafío pastoral que enfrenta el Perú es grande, pero las enseñanzas de Aparecida y \textit{Evangelii Gaudium} nos ofrecen una guía para caminar hacia una Iglesia y una sociedad más inclusivas, reconciliadas y comprometidas con los valores del Evangelio.

\section{El diálogo social para la construcción de la paz}

Los números 238 al 258 de \textit{Evangelii Gaudium} abordan la importancia del diálogo social como camino para la construcción de la paz y la reconciliación. Esto es particularmente relevante en contextos como el peruano, marcado por décadas de conflicto interno y desigualdad estructural.

Francisco plantea que la paz social no es posible sin la justicia y sin un diálogo sincero entre todos los sectores de la sociedad, especialmente entre los más vulnerables y los que tienen responsabilidades públicas. El diálogo debe ser “un encuentro de personas que buscan el bien común, abiertos a escucharse mutuamente y a comprometerse” (EG 239).

La Iglesia está llamada a ser promotora de este diálogo, no solo a nivel comunitario, sino también en la esfera pública, acompañando procesos de reconciliación y superando las divisiones que generan odio y violencia. Esta misión implica promover la cultura del encuentro, como lo ha señalado con insistencia el Papa Francisco.

En el Perú, el reto es superar las heridas del pasado y las exclusiones presentes, construyendo espacios donde la diversidad cultural, social y política pueda expresarse y dialogar. El aporte de Aparecida en este sentido es fundamental: nos recuerda que la reconciliación es un proceso espiritual y social que requiere paciencia, verdad y perdón, pero también compromiso concreto con la justicia y la transformación estructural.

\section*{Conclusión}

La construcción de una identidad nacional reconciliada, desde la fe, no es tarea fácil. Implica una evangelización integral, una inculturación real y un diálogo social paciente. El Papa Francisco, heredero del espíritu de Aparecida y del discernimiento ignaciano, nos recuerda que ``la paz social no puede entenderse como la mera ausencia de violencia, sino como la justicia hecha cultura'' (Francisco, 2013, \S 218).

Por tanto, como discípulos misioneros llamados a construir una identidad nacional reconciliada, no podemos partir del conflicto o del prejuicio ideológico. El mismo Papa Francisco de feliz memoria, en coherencia con su formación ignaciana, vivió y promovió el principio de dar por cierta, en la medida de lo posible, la proposición del prójimo, tal como lo expresa San Ignacio de Loyola en los \emph{Ejercicios Espirituales}: ``se ha de presuponer que todo buen cristiano ha de ser más pronto a salvar la proposición del prójimo, que a condenarla [...]'' (Ignacio de Loyola, 1548/2025, Anotación 22). Esta disposición interior es indispensable para avanzar en una reconciliación auténticamente cristiana que escuche tanto a los pueblos originarios como a las clases educadas, sin excluir a nadie. Solo desde esta actitud evangélica será posible sanar heridas históricas como las de Bagua, evitar la manipulación ideológica, y caminar hacia una peruanidad integradora, mestiza y profundamente humana.

Así, el Perú ---con su diversidad cultural, sus heridas históricas y su potencial de integración--- está llamado a vivir el Evangelio desde su propia alma mestiza. El reto no es negar nuestra complejidad, sino abrazarla como lugar donde Dios sigue encarnándose.

La reconciliación nacional, en sociedades marcadas por la polarización, la corrupción y la desigualdad, exige una conversión profunda de todos los actores sociales, incluyendo a quienes ejercen el poder económico, político o cultural. En este sentido, el Documento de Aparecida subraya la urgencia de una pastoral de las élites, afirmando que ``es indispensable una pastoral que evangelice a los dirigentes en los diversos ámbitos de la vida social'' (Conferencia General del Episcopado Latinoamericano y del Caribe [CELAM], 2007, n. 512). Esta labor no responde a privilegios, sino a una exigencia del Evangelio: formar líderes que actúen con justicia, busquen el bien común y sanen las heridas del tejido social. Una auténtica reconciliación requiere que quienes toman decisiones clave estén también evangelizados, de modo que puedan contribuir con honestidad y compasión a la construcción de una sociedad más fraterna.


	
%	\printbibliography
	
\end{document}
