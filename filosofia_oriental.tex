\documentclass[12pt]{article}
\usepackage[utf8]{inputenc}
\usepackage[spanish]{babel}
\usepackage[T1]{fontenc}
\usepackage{geometry}
\geometry{a4paper, margin=2.5cm}
\usepackage{tabularx}
\usepackage[table]{xcolor}
\usepackage{titlesec}
\usepackage{lmodern}
\titleformat{\section}{\normalfont\Large\bfseries}{\thesection}{1em}{}

\title{Corrientes de la filosofía oriental y su utilidad pastoral}
\author{Jorge L. Ayona Inglis}
\date{}

\begin{document}

\maketitle

\section*{Visión comparada de corrientes orientales}

\rowcolors{2}{gray!10}{white}
\begin{tabularx}{\textwidth}{|>{\raggedright\arraybackslash}p{3cm}|>{\raggedright\arraybackslash}X|>{\raggedright\arraybackslash}X|>{\raggedright\arraybackslash}p{3cm}|}
\hline
\rowcolor{gray!30}
\textbf{Corriente / Autor} & \textbf{Idea central o texto clave} & \textbf{Posible aplicación pastoral o espiritual} & \textbf{Nivel de lectura} \\
\hline
\textbf{Confucianismo (Confucio)} & \emph{Los Analectas}: sabiduría práctica, ética relacional, respeto filial. & Fortalece valores como el respeto, la justicia, la moderación y la piedad familiar. Útil para formar conciencia moral. & Accesible \\
\hline
\textbf{Taoísmo (Lao-Tsé)} & \emph{Tao Te Ching}: el Tao como principio de equilibrio y sabiduría no forzada. & Invita a una espiritualidad del desapego, confianza, fluir con la voluntad divina. Compatible con la dirección espiritual contemplativa. & Medio \\
\hline
\textbf{Budismo (Buda)} & \emph{Dhammapada}: enseñanza sobre el sufrimiento, el desapego y la compasión. & Profundiza en la conciencia del ego, el sufrimiento, y la misericordia. Ayuda en la pastoral del duelo, ansiedad o adicciones. & Medio \\
\hline
\textbf{Vedanta / Upanishads} & Reflexión sobre el alma (Atman) y el Absoluto (Brahman). Unidad interior y trascendente. & Aporta intuiciones sobre la dimensión espiritual del ser humano. Útil para diálogos interreligiosos o mística comparada. & Avanzado \\
\hline
\textbf{Zen (Chan)} & Experiencia directa de la realidad, más allá de los conceptos. Silencio, meditación, Koans. & Refuerza la práctica del silencio interior y la presencia. Útil para retiros, contemplación o pedagogía del “aquí y ahora”. & Avanzado \\
\hline
\end{tabularx}

\vspace{1cm}
\section*{Observación final}
La filosofía oriental ofrece caminos de sabiduría que, integrados con discernimiento, pueden enriquecer la pastoral cristiana, especialmente en la atención interior, el desapego sano, la compasión y la conciencia.

\end{document}
