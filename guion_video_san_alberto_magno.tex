\documentclass[12pt]{article}
\usepackage[spanish]{babel}
\usepackage[utf8]{inputenc}
\usepackage{geometry}
\geometry{margin=2.5cm}
\usepackage{parskip}
\usepackage{graphicx}
\usepackage{setspace}
\usepackage{fontspec}
\setmainfont{Cormorant Garamond}
 % Cambia si usas otro sistema
\usepackage{titlesec}
\titleformat{\section}{\large\bfseries}{\thesection.}{0.5em}{}

\begin{document}
	
	\begin{center}
		\LARGE \textbf{San Alberto Magno: Mi santo patrón} \\
		\vspace{0.5em}
		\large 15 de noviembre
	\end{center}
	
	\vspace{1em}
	
	\onehalfspacing
	
	El 15 de noviembre no es una fecha cualquiera para mí. Ese día nací… y también se celebra la festividad de San Alberto Magno, mi santo patrón.
	
	¿Quién fue este gran hombre? San Alberto Magno vivió en el siglo XIII. Fue un dominico alemán, filósofo, teólogo, científico y obispo. Pero no era solo un sabio: fue conocido como \textit{Doctor Universalis}, porque abarcó prácticamente todos los saberes de su tiempo.
	
	Enseñó a otro gran santo: Tomás de Aquino. Pero antes de ser maestro, fue aprendiz apasionado. Estudió las ciencias naturales, la filosofía griega y, por supuesto, la teología cristiana. Su fe no estaba separada de la razón: veía en el conocimiento un camino hacia Dios.
	
	San Alberto no temía estudiar el mundo, porque sabía que todo lo creado habla del Creador. Por eso fue pionero en muchos campos: botánica, zoología, astronomía, química... ¡Siglos antes de que existiera la ciencia como la conocemos!
	
	Para mí, tener a San Alberto Magno como patrón es un honor y un desafío. Me recuerda que la fe y la razón no están en conflicto, y que el amor por la verdad —ya sea en un laboratorio, en un libro o en el silencio de la oración— siempre nos conduce a Dios.
	
	\vspace{1em}
	
	\begin{center}
		\textit{San Alberto Magno, ruega por nosotros.} \\
		\textit{Y gracias… por inspirar la búsqueda incansable del bien, la verdad y la belleza.}
	\end{center}
	
\end{document}
