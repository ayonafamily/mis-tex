\documentclass[12pt]{article}

% PAQUETES
\usepackage[utf8]{inputenc}
\usepackage[T1]{fontenc}
\usepackage[spanish]{babel}
\usepackage[a4paper, margin=2.54cm]{geometry}
\usepackage{setspace}
\usepackage{parskip}
\usepackage{helvet}
\usepackage{hyperref}
\renewcommand{\familydefault}{\sfdefault}
\usepackage{csquotes}
\usepackage[style=apa, backend=biber, language=spanish]{biblatex}


\addbibresource{bibliografia.bib}

% DATOS DEL DOCUMENTO
\title{El impacto de Internet en la Sociedad del Conocimiento: una vivencia personal}
\author{Jorge Ayona\\
	{\footnotesize \protect\href{https://orcid.org/0009-0006-6551-9681}{ORCID: 0009-0006-6551-9681}}\\
	\texttt{jorge.ayona@estudiante.ucsm.edu.pe}}

\date{25 de mayo del 2025}

\begin{document}
	\maketitle
	\newpage
	\onehalfspacing
	
	\section{Introducción}
	
	Desde el año 1998, empecé a explorar Internet en sus primeras etapas, fui testigo de un cambio revolucionario, sin estruendo, que transformó el acceso al conocimiento, las dinámicas sociales y las formas de comunicación. En aquel entonces, Mosaic, Netscape y Explorer eran los primeros navegadores. Usábamos herramientas como WordPerfect o Lotus 1-2-3, y las páginas web eran sumamente simples. Hoy, más de dos décadas después, reconozco que Internet no solo ha facilitado tareas, sino que ha reconfigurado el modo en que concebimos el conocimiento, las relaciones y la realidad cotidiana. Este ensayo propone una reflexión personal y crítica sobre las ventajas y desafíos que representa el uso de Internet en la llamada Sociedad del Conocimiento \parencite{castells2000}.
	
	\section{Ventajas Del Internet En La Sociedad Del Conocimiento}
	
	\subsection{Democratización del conocimiento}
	Una de las mayores contribuciones de Internet ha sido la democratización del acceso a la información. A finales de los noventa, recuerdo haber accedido desde mi oficina a varios sitios de universidades norteamericanas. En ellas encontraba archivos de clases de filosofía, todo en texto plano. Fue una experiencia asombrosa: poder consultar materiales de alto nivel sin haber salido de mi país. Años más tarde, en la universidad, accedí por primera vez a bases de datos académicas de manera formal para realizar trabajos de investigación. Esa facilidad de acceso cambió completamente la manera de estudiar y aprender.
	
	\subsection{Colaboración y productividad remota}
	En 2012, durante un curso universitario, trabajé en una presentación colaborativa usando Google Drive, que entonces tenía otro nombre. Lo hicimos un domingo en la noche y el martes ya lo estábamos presentando en clase. Mi compañera era de Maracaibo, pero estudiaba presencialmente aquí. Esa experiencia me reveló el potencial de la nube y del trabajo en equipo sin necesidad de reunirse físicamente \parencite{googleworkspace}.
	
	\subsection{Resiliencia Digital En Tiempos De Crisis}
	Durante la pandemia, Internet fue un salvavidas. Lo utilicé para acceder a información de salud, realizar compras, pagar servicios y llevar cursos en línea, que para entonces eran una novedad. También fue mi medio principal de contacto social: llamadas, videoconferencias y redes me permitieron mantener vínculos cuando salir de casa era impensable.
	
	\section{Desafíos del Internet en la Sociedad del Conocimiento}
	
	\subsection{Infobustión Y Sobrecarga Cognitiva}
	En un curso del año 2011, conocí un concepto llamado "infobustión" o "infobesidad", que describe la dificultad de procesar adecuadamente tanta información disponible en línea. En esta se vuelve difícil discernir lo verídico de lo falso, a menos que tengamos una base epistemológica sólida  \parencite{cornell2009}. Wikipedia era utilizada como fuente principal por muchos de mis compañeros de entonces, pero su fiabilidad fue motivo de cuestionamiento por parte de diversos autores \parencite{lim2009}.
	
	\subsection{Ciberdelincuencia, estafas y suplantación}
	Personalmente he sido víctima de estafas en línea y suplantación de identidad. Al respecto, conocí en línea una dama que intentó manipularme emocionalmente para obtener beneficios económicos, lo cual luego supe que era parte de un patrón conocido como "estafa romántica" \parencite{interpol2021}.
	
	\subsection{Contenido ilegal y peligros de la dark web}
	Por curiosidad, una vez ingresé a la dark web utilizando el navegador Tor. Fue impactante descubrir lo sencillo que resultaba toparse con plataformas que ofrecían contenidos ilegales \parencite{torproject}.
	
	\subsection{Extorsión y crimen organizado en redes sociales}
	Otro desafío preocupante es el uso de las redes sociales por parte del crimen organizado. En Perú, por ejemplo, se difundieron a través de plataformas digitales los crímenes cometidos en Pataz, mostrando ejecuciones y armas como si se tratara de advertencias públicas \parencite{bbc2023}. Grupos como ISIS llegaron a subir decapitaciones en redes sociales, en actos de violencia extrema \parencite{unodc2021}.
	
	\section{Conclusiones}
	
	Internet ha transformado nuestras vidas y la forma en que construimos y compartimos conocimiento. Es un universo sin fronteras físicas, un verdadero "cyberespacio", término acuñado por \parencite{gibson1984} para describir este mundo digital. Pero como todo espacio sin regulación suficiente, también plantea riesgos serios. La proliferación del cibercrimen ha obligado a que la mayoría de países cuenten con divisiones especializadas en delitos informáticos. Aprovechar Internet hoy exige no solo habilidades técnicas, sino también conciencia crítica, principios éticos y educación digital. Solo así podremos navegar este vasto océano de datos sin perdernos en él.
	
	\printbibliography
	
\end{document}
