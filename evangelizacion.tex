\section*{Providencia y arquetipos en la evangelización andina}

Desde una perspectiva junguiana, los arquetipos presentes en el inconsciente colectivo del Tahuantinsuyo —como la Madre, el Redentor o Viracocha— facilitaron la recepción simbólica del cristianismo. Esta apertura simbólica puede entenderse como un signo providencial: el Evangelio encontró resonancias profundas en símbolos ya vivos en la cultura andina.

\subsection*{Correspondencias simbólicas}
\begin{itemize}
	\item \textbf{Pachamama y la Virgen María}: Ambas son madres protectoras, ligadas a la vida y al pueblo. La Virgen fue comprendida como rostro amoroso y cercano de lo divino.
	\item \textbf{El Cristo sufriente y el Mártir andino}: El sacrificio redentor de Cristo fue asimilado en pueblos que conocían el sufrimiento como experiencia colectiva y religiosa.
	\item \textbf{Viracocha y el Dios Padre}: Dios creador, trascendente, que promete volver. Arquetipo del principio ordenador del cosmos.
\end{itemize}

\begin{quote}
	\emph{“Los símbolos religiosos no son inventos arbitrarios, sino expresiones necesarias de la psique, como los sueños.”} \\ — C. G. Jung, \emph{Psicología y religión}
\end{quote}

\textbf{Desde la fe}, esta conexión puede verse como semilla del Verbo ya presente en los pueblos (cf. San Justino Mártir). Dios, en su Providencia, no impuso el Evangelio por fuerza, sino que lo sembró en la tierra del alma andina.

\textbf{Cita bíblica:} “En él vivimos, nos movemos y existimos” (Hech 17, 28).

\textbf{Para meditar:}
\begin{itemize}
	\item ¿Qué símbolos católicos fueron más acogidos por el pueblo andino? ¿Por qué?
	\item ¿Podemos ver en la historia de la evangelización signos de continuidad simbólica más que ruptura?
	\item ¿Qué nos enseña esto sobre el respeto profundo por la psique colectiva de los pueblos?
\end{itemize}