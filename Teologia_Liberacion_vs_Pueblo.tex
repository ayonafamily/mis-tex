
\documentclass[12pt]{article}
\usepackage[utf8]{inputenc}
\usepackage[spanish]{babel}
\usepackage{geometry}
\usepackage{array}
\usepackage{lmodern}
\usepackage{booktabs}
\geometry{a4paper, margin=2.5cm}

\title{\textbf{Comparación entre la Teología de la Liberación y la Teología del Pueblo}}
\author{}
\date{}

\begin{document}

\maketitle

\section*{Tabla comparativa}

\renewcommand{\arraystretch}{1.5}
\begin{tabular}{>{\raggedright\arraybackslash}p{4cm} >{\raggedright\arraybackslash}p{6cm} >{\raggedright\arraybackslash}p{6cm}}
\toprule
\textbf{Aspecto} & \textbf{Teología de la Liberación} & \textbf{Teología del Pueblo} \\
\midrule
\textbf{Origen} & Décadas de 1960-70, post Concilio Vaticano II y Medellín & Décadas de 1960-70, contexto argentino, con Lucio Gera y Rafael Tello \\
\textbf{Contexto} & Pobreza estructural, represión política, dictaduras & Cultura popular, religiosidad del pueblo, espiritualidad encarnada \\
\textbf{Método} & Ver – Juzgar – Actuar, con análisis marxista de la realidad & Ver – Juzgar – Actuar, con lectura desde la cultura y la historia del pueblo \\
\textbf{Visión del pobre} & Oprimido por estructuras injustas, sujeto político & Sujeto creyente y portador de fe vivida \\
\textbf{Cristología} & Cristo como liberador de la opresión política y económica & Cristo presente en el caminar del pueblo \\
\textbf{Iglesia} & Agente de transformación social, crítica de estructuras & Iglesia madre que acompaña al pueblo desde su fe y cultura \\
\textbf{Relación con el Magisterio} & Tensiones con Roma, críticas por uso del marxismo & Mayor aceptación, influye en el Magisterio del Papa Francisco \\
\textbf{Representantes} & Gutiérrez, Boff, Sobrino & Gera, Tello, Methol Ferré, Bergoglio \\
\bottomrule
\end{tabular}

\end{document}
