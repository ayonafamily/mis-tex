\documentclass[12pt]{article}
\usepackage[utf8]{inputenc}
\usepackage[spanish]{babel}
\usepackage{lmodern}
\usepackage{parskip}
\usepackage{geometry}
\geometry{a4paper, margin=2.5cm}
\usepackage{titlesec}
\titleformat{\section}{\normalfont\Large\bfseries}{\thesection}{1em}{}

\title{Exploración Arquetípica Personal}
\author{Jorge Ayona}
\date{\today}

\begin{document}

\maketitle

\section*{Arquetipos que podrían resonar contigo}

\subsection*{ El Sabio / El Buscador}
\textbf{Motivación}: buscar la verdad, comprender el mundo, integrar conocimiento y espiritualidad.

\textbf{Manifestación en ti}: tu interés en Jung, Nietzsche, la teología, el crecimiento personal y el análisis profundo de la vida muestran una búsqueda continua por sentido y sabiduría.

\textbf{Sombra}: intelectualización excesiva, aislamiento, dificultad para conectar emocionalmente con lo simple o cotidiano.

\subsection*{ El Justo / El Mediador}
\textbf{Motivación}: vivir en coherencia con la justicia, el bien, la verdad moral.

\textbf{Manifestación en ti}: tu inclinación a tomar postura, incluso frente a figuras de autoridad injustas (como en el caso de la profesora), tu interés en el derecho, la ética, y el acompañamiento pastoral.

\textbf{Sombra}: rigidez, autoexigencia extrema, dificultad para aceptar la ambigüedad.

\subsection*{ El Cuidador / El Pastor}
\textbf{Motivación}: proteger, consolar, servir desde el corazón.

\textbf{Manifestación en ti}: tu papel como catequista, tu deseo de acompañar, tu sensibilidad hacia las heridas del alma (en otros y en ti mismo), tu experiencia con tu madre y tu hija.

\textbf{Sombra}: descuidarte a ti mismo, atraer relaciones donde das más de lo que recibes, confundir servicio con dependencia.

\subsection*{ El Místico / El Mediador Espiritual}
\textbf{Motivación}: unir cielo y tierra, lo visible con lo invisible.

\textbf{Manifestación en ti}: tu fascinación por Jung, Teresa de Ávila, Francisco de Sales, tu camino de fe profundamente reflexivo, tu interés por integrar lo psicológico y lo espiritual.

\textbf{Sombra}: evasión espiritual, autoengaño, dificultad para actuar con los pies en la tierra.

\section*{Preguntas para profundizar}
\begin{itemize}
  \item ¿Qué energía arquetípica aparece en mis sueños o decisiones clave?
  \item ¿Qué tipo de personaje suelo representar en los grupos?
  \item ¿De qué me han acusado (justa o injustamente) los demás?
  \item ¿Qué me inspira profundamente desde niño?
\end{itemize}

\end{document}
