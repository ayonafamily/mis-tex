
\documentclass[12pt]{article}
\usepackage[utf8]{inputenc}
\usepackage[spanish]{babel}
\usepackage{csquotes}
\usepackage[style=apa, backend=biber]{biblatex}
\addbibresource{bibliografia.bib}
\usepackage{geometry}
\geometry{a4paper, margin=1in}
\usepackage{setspace}
\onehalfspacing
\usepackage{titlesec}
\titleformat{\section}{\normalfont\Large\bfseries}{\thesection}{1em}{}

\title{La insostenibilidad de la doctrina protestante del \textit{Sola Scriptura}}
\author{}
\date{}

\begin{document}

\maketitle

\begin{abstract}
El presente trabajo analiza críticamente la doctrina protestante del \textit{Sola Scriptura} desde una perspectiva histórica, textual y teológica. Argumenta que el principio de que la Biblia es la única fuente de autoridad no puede sostenerse sin aceptar decisiones canónicas tomadas fuera del contexto apostólico y patrístico. Se examinan la historia del canon bíblico, el uso de la Septuaginta en el cristianismo primitivo y los fundamentos falaces que subyacen en la selección protestante de textos.
\end{abstract}

\section{Introducción}

La doctrina del \textit{Sola Scriptura}, uno de los pilares de la Reforma protestante, sostiene que la Biblia es la única regla de fe y práctica cristiana. Esta afirmación, sin embargo, presenta serios problemas históricos y teológicos, especialmente en cuanto a la formación del canon bíblico. En este trabajo se sostiene que el principio del \textit{Sola Scriptura} es insostenible, ya que depende de un canon delimitado tardíamente por autoridades rabínicas ajenas al cristianismo, excluyendo textos utilizados por los apóstoles y la Iglesia primitiva.

\section{El canon protestante y la exclusión de los deuterocanónicos}

El protestantismo adopta un canon del Antiguo Testamento que omite los llamados libros deuterocanónicos, basándose en la decisión de rabinos judíos del siglo II d.C. Esta decisión se consolidó en el sínodo de Jamnia, el cual excluyó libros escritos originalmente en griego, como los que contenía la Septuaginta.

\section{Uso de la Septuaginta en el cristianismo primitivo}

La Septuaginta fue la versión de la Biblia utilizada por los judíos de la diáspora y adoptada por los primeros cristianos. Más del 70\% de las citas del Antiguo Testamento en el Nuevo Testamento provienen de esta versión griega. Autores del Nuevo Testamento, como Mateo y Pablo, citaron textos que provienen directamente de la Septuaginta, incluyendo pasajes que no se encuentran en el texto masorético.

\section{Presencia de los deuterocanónicos en el Nuevo Testamento}

Diversos pasajes del Nuevo Testamento aluden o citan directamente libros deuterocanónicos. Por ejemplo, Hebreos 11:35 hace referencia a 2 Macabeos 7, mientras que Mateo 27:43 refleja Sabiduría 2:18. Estas referencias evidencian que los autores del Nuevo Testamento consideraban estos libros como parte de las Escrituras.

\section{La falacia del argumento basado en la lengua hebrea}

Uno de los argumentos utilizados por algunos protestantes para rechazar los deuterocanónicos es que no fueron escritos originalmente en hebreo. Sin embargo, esto es falso en varios casos, como el del libro de Sirácides, cuyos fragmentos en hebreo han sido hallados en Qumrán y la Genizá de El Cairo. Además, argumentar que el texto masorético es más fiable simplemente por estar en hebreo constituye una falacia del tipo \textit{non sequitur}, pues la fidelidad textual no depende únicamente del idioma.

\section{Conclusión}

La doctrina del \textit{Sola Scriptura} no se sostiene históricamente ni teológicamente. Se apoya en un canon definido por autoridades no cristianas después de la era apostólica y rechaza textos utilizados por la Iglesia primitiva. La selección protestante del canon bíblico se basa en argumentos falaces y omite aspectos fundamentales del desarrollo del cristianismo temprano.

\printbibliography

\end{document}
