\documentclass[12pt]{article}
\usepackage[utf8]{inputenc}
\usepackage[spanish]{babel}
\usepackage{geometry}
\usepackage{booktabs}
\usepackage{tabularx}
\usepackage{lmodern}
\usepackage{parskip}
\geometry{a4paper, margin=2.5cm}

\title{\textbf{Filosofía de un vistazo}}
\author{Jorge Ayona}
\date{\today}

\begin{document}

\maketitle

\section*{Filósofos Presocráticos}

\begin{tabularx}{\textwidth}{@{} l l l X @{}}
\toprule
\textbf{Filósofo} & \textbf{Siglo a.C.} & \textbf{Escuela o Región} & \textbf{Enseñanza Central} \\
\midrule
Tales de Mileto   & VII–VI  & Mileto             & El arjé es el \textbf{agua}. Fundador de la filosofía natural. \\
Anaximandro       & VI      & Mileto             & El arjé es el \textbf{ápeiron}, lo indeterminado. \\
Anaxímenes        & VI      & Mileto             & El arjé es el \textbf{aire}, base de todo por rarefacción y condensación. \\
Pitágoras         & VI      & Samos / Crotona    & El principio es el \textbf{número}. Armonía cósmica y transmigración del alma. \\
Jenófanes         & VI      & Colofón / Elea     & Crítica a la religión antropomórfica. Dios único, inmóvil e infinito. \\
Heráclito         & VI–V    & Éfeso              & Todo cambia: \emph{panta rhei}. El \textbf{fuego} es el arjé. \\
Parménides        & V       & Elea               & El ser es \textbf{uno}, eterno e inmutable. El cambio es ilusorio. \\
Zenón de Elea     & V       & Elea               & Paradojas del movimiento. Defensa lógica del ser único. \\
Empédocles        & V       & Agrigento (Sicilia) & Todo está compuesto por \textbf{cuatro elementos}: tierra, aire, fuego, agua. Amor y odio mueven el cosmos. \\
Anaxágoras        & V       & Clazómenas / Atenas & Introduce el \textbf{nous} (mente) como causa ordenadora. Todo está en todo. \\
Demócrito         & V–IV    & Abdera             & Todo está hecho de \textbf{átomos} y vacío. Fundador del atomismo con Leucipo. \\
Leucipo           & V       & Abdera             & Primer pensador en proponer el \textbf{atomismo}. Átomos en movimiento eterno. \\
\bottomrule
\end{tabularx}

\section*{Filosofía Griega}

\begin{tabularx}{\textwidth}{@{} l l l X @{}}
\toprule
\textbf{Filósofo} & \textbf{Siglo a.C.} & \textbf{Escuela o Corriente} & \textbf{Enseñanza Central} \\
\midrule
Sócrates           & V (470–399)       & Ética racional / Mayéutica   & La virtud es conocimiento. El alma debe ser cuidada. Preguntar lleva a la verdad. \\
Platón             & V–IV (427–347)    & Idealismo / Academia         & Mundo de las Ideas. El alma es inmortal. Justicia como armonía interior. \\
Aristóteles        & IV (384–322)      & Realismo / Liceo             & El ser como acto y potencia. Ética del término medio. Causalidad y lógica formal. \\
Diógenes de Sinope & IV (c. 412–323)   & Cinismo                      & Vida austera y autosuficiente. Crítica radical a la convención social. \\
Epicuro            & IV–III (341–270)  & Epicureísmo                  & El placer moderado como bien supremo. Ausencia de dolor (ataraxia). \\
Zenón de Citio     & IV–III (334–262)  & Estoicismo                   & Vivir conforme a la razón y la naturaleza. Aceptación del destino. \\
Pirron de Elis     & IV–III (c. 360–270) & Escepticismo                & La verdad no puede conocerse con certeza. Suspensión del juicio (epojé). \\
Plotino            & III d.C. (204–270) & Neoplatonismo               & Todo procede del Uno. El alma asciende a través de la contemplación. \\
\bottomrule
\end{tabularx}

\section*{Filosofía Judía y Árabe Medieval}

\begin{tabularx}{\textwidth}{@{} l l l X @{}}
\toprule
\textbf{Filósofo} & \textbf{Siglo} & \textbf{Tradición} & \textbf{Enseñanza Central} \\
\midrule
Filón de Alejandría & I a.C. – I d.C. & Judaísmo helenístico & Integra la filosofía griega (especialmente Platón y los estoicos) con la fe judía. Alegoría bíblica. \\
Saadia Gaón         & X              & Judaísmo             & Defiende la creación y la revelación frente al racionalismo. Reconciliación entre razón y fe. \\
Solomón Ibn Gabirol & XI             & Judaísmo / Neoplatonismo & El mundo emana de Dios. Todo tiene forma y materia espiritual. \\
Bahya ibn Paquda    & XI             & Judaísmo ético       & Énfasis en la vida espiritual interior. Obra: “Deberes del corazón”. \\
Judá Haleví          & XII            & Judaísmo / Apologética & La experiencia religiosa judía es superior a la filosofía griega. Defensa del judaísmo. \\
Maimónides (Rambam) & XII            & Judaísmo / Aristotelismo & Razón y fe son compatibles. Dios es incognoscible. Obra: “Guía de los perplejos”. \\
Al-Kindi             & IX             & Islam / Neoplatonismo & Primer filósofo árabe. Adaptación del pensamiento griego. El Uno y la creación. \\
Al-Farabi            & X              & Islam / Aristotelismo & Clasificación del conocimiento. Ciudad ideal. Intelecto como principio del alma. \\
Avicena (Ibn Sina)   & X–XI           & Islam / Aristotelismo & Existencia como necesidad. El alma es inmortal. Dios como causa necesaria. \\
Algazel (Al-Ghazali) & XI             & Islam / Misticismo   & Crítica al racionalismo. Defensa del sufismo. Obra: “La incoherencia de los filósofos”. \\
Averroes (Ibn Rushd) & XII            & Islam / Aristotelismo & Defensa racional de Aristóteles. Filosofía y religión no se contradicen. Comentarios a Aristóteles. \\
\bottomrule
\end{tabularx}

\section*{Filosofía Cristiana Medieval}

\begin{tabularx}{\textwidth}{@{} l l l X @{}}
\toprule
\textbf{Filósofo / Teólogo} & \textbf{Siglo} & \textbf{Corriente / Rol} & \textbf{Enseñanza Central} \\
\midrule
San Agustín de Hipona     & IV–V         & Neoplatonismo cristiano & El alma busca a Dios. El mal es ausencia de bien. Confesiones y Ciudad de Dios. \\
Boecio                    & V–VI         & Transmisor de filosofía & Tradujo y comentó a Aristóteles. Obra: “La consolación de la filosofía”. \\
Pseudo Dionisio Areopagita & V–VI        & Mística cristiana       & Teología negativa. Dios trasciende todo concepto. Jerarquía celeste. \\
Juan Escoto Eriúgena      & IX           & Neoplatonismo cristiano & Todo viene de Dios y vuelve a Él. Unidad de razón y fe. \\
San Anselmo de Canterbury & XI           & Escolástica inicial     & Argumento ontológico de la existencia de Dios. “Fe que busca entender”. \\
Pedro Abelardo            & XII          & Dialéctica escolástica  & Defensa de la razón en la teología. Ética de la intención. \\
San Buenaventura          & XIII         & Franciscano místico     & La razón iluminada por la fe. Integra Agustín con Aristóteles. \\
Santo Tomás de Aquino     & XIII         & Escolástica dominica    & Razón y fe se armonizan. Dios como acto puro. Teología sistemática (Suma Teológica). \\
Duns Scoto               & XIII–XIV     & Escolástica franciscana & Voluntad divina como fundamento del ser. Univocidad del ser. \\
Guillermo de Ockham      & XIV          & Nominalismo             & No hay universales reales. Principio de economía (Navaja de Ockham). \\
\bottomrule
\end{tabularx}

\section*{Filosofía Renacentista y Moderna}

\begin{tabularx}{\textwidth}{@{} l l l X @{}}
\toprule
\textbf{Filósofo} & \textbf{Siglo} & \textbf{Corriente / Enfoque} & \textbf{Enseñanza Central} \\
\midrule
Nicolás de Cusa         & XV           & Platonismo cristiano        & La verdad trasciende la razón. “Docta ignorancia”. El universo es infinito. \\
Marsilio Ficino         & XV           & Neoplatonismo renacentista  & Dignidad del alma. Tradujo a Platón. Unión entre alma y cosmos. \\
Giovanni Pico della Mirandola & XV     & Humanismo renacentista      & Libre albedrío y dignidad humana. Discurso sobre la dignidad del hombre. \\
Erasmo de Róterdam      & XV–XVI       & Humanismo cristiano         & Retorno al Evangelio puro. Crítica a la superstición y a la escolástica rígida. \\
Tomás Moro              & XVI          & Utopismo / Ética cristiana  & Crítica social y política. Ideal de una sociedad justa (Utopía). \\
Maquiavelo              & XVI          & Realismo político           & Separación entre moral y política. El fin justifica los medios. \\
Francis Bacon           & XVI–XVII     & Empirismo / Método científico & Saber es poder. Inducción científica. Crítica a los ídolos del pensamiento. \\
René Descartes          & XVII         & Racionalismo                & “Pienso, luego existo”. Método de duda. Dualismo mente-cuerpo. \\
Baruch Spinoza          & XVII         & Racionalismo / Panteísmo    & Dios y la naturaleza son una misma realidad. Ética geométrica. \\
John Locke              & XVII         & Empirismo                   & La mente es una tabula rasa. Origen empírico del conocimiento. \\
Gottfried Leibniz       & XVII–XVIII   & Racionalismo optimista      & Mundo ordenado por razones suficientes. Mónadas. El mejor de los mundos posibles. \\
George Berkeley         & XVIII        & Idealismo empírico          & Ser es ser percibido. Niega la existencia de la materia sin mente. \\
David Hume              & XVIII        & Empirismo escéptico         & Crítica a la causalidad. El yo es una colección de percepciones. \\
Immanuel Kant           & XVIII        & Idealismo trascendental     & El conocimiento surge de la experiencia estructurada por formas a priori. Revolución copernicana en filosofía. \\
\bottomrule
\end{tabularx}


\section*{Filosofía Contemporánea}

\begin{tabularx}{\textwidth}{@{} l l l X @{}}
\toprule
\textbf{Filósofo} & \textbf{Siglo} & \textbf{Corriente / Enfoque} & \textbf{Enseñanza Central} \\
\midrule
Georg W. F. Hegel     & XIX         & Idealismo absoluto      & La realidad es Espíritu en desarrollo. Dialéctica: tesis, antítesis, síntesis. \\
Arthur Schopenhauer   & XIX         & Pessimismo metafísico   & La voluntad ciega rige el mundo. El arte y la compasión son caminos de liberación. \\
Søren Kierkegaard     & XIX         & Existencialismo cristiano & El individuo ante Dios. Fe, angustia y decisión auténtica. \\
Karl Marx             & XIX         & Materialismo histórico   & La historia es lucha de clases. Crítica al capitalismo y religión como ideología. \\
Friedrich Nietzsche   & XIX         & Vitalismo / Nihilismo    & Dios ha muerto. Crítica a la moral tradicional. Voluntad de poder y superhombre. \\
Edmund Husserl        & XX          & Fenomenología            & Descripción rigurosa de la conciencia. Intencionalidad y suspensión del juicio (epoché). \\
Martin Heidegger      & XX          & Existencialismo hermenéutico & El ser humano como Dasein. Angustia ante la nada. Ser hacia la muerte. \\
Jean-Paul Sartre      & XX          & Existencialismo ateo     & La existencia precede a la esencia. Libertad radical y responsabilidad. \\
Simone de Beauvoir    & XX          & Feminismo existencialista & La mujer no nace, se hace. Crítica a la opresión estructural. \\
Ludwig Wittgenstein   & XX          & Filosofía del lenguaje   & Límite del lenguaje es el límite del mundo. Juegos de lenguaje. \\
Michel Foucault       & XX          & Postestructuralismo      & El saber y el poder están entrelazados. Genealogía de las instituciones. \\
Emmanuel Levinas      & XX          & Ética de la alteridad    & El rostro del otro me interpela éticamente. Filosofía como responsabilidad. \\
\bottomrule
\end{tabularx}
\section*{Filosofía Española}

\begin{tabularx}{\textwidth}{@{} l l l X @{}}
\toprule
\textbf{Filósofo / Pensador} & \textbf{Siglo} & \textbf{Corriente / Enfoque} & \textbf{Enseñanza Central} \\
\midrule
Séneca                    & I d.C.        & Estoicismo romano          & El sabio vive conforme a la razón. La virtud es suficiencia interior. \\
Isidoro de Sevilla        & VII           & Patrística hispana         & Síntesis entre saber clásico y cristianismo. Obra: “Etimologías”. \\
Ramón Llull               & XIII–XIV      & Mística y lógica cristiana & Uso del arte combinatoria para demostrar verdades de fe. Diálogo interreligioso. \\
Francisco Suárez          & XVI–XVII      & Escolástica jesuita        & Derecho natural y metafísica del ser. Precursor del derecho internacional. \\
Baltasar Gracián          & XVII          & Barroco / Moralismo        & Prudencia, disimulo y virtud como formas de sabiduría práctica. \\
Benito Jerónimo Feijoo    & XVIII         & Ilustración católica       & Defensa de la razón contra la superstición. Divulgación científica y crítica. \\
Miguel de Unamuno         & XIX–XX        & Existencialismo trágico    & Conflicto entre razón y fe. Hambre de inmortalidad. Filosofía del “sentimiento trágico de la vida”. \\
José Ortega y Gasset      & XX            & Raciovitalismo              & “Yo soy yo y mi circunstancia”. Crítica a la masificación. Necesidad de una élite culta. \\
Xavier Zubiri             & XX            & Inteligencia sentiente     & Conocimiento como experiencia radical del ser. Superación del dualismo. \\
María Zambrano            & XX            & Filosofía poética / mística & Razón poética. Unión entre filosofía, poesía y revelación interior. \\
\bottomrule
\end{tabularx}








\end{document}
