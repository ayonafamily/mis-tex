\documentclass[12pt]{article}
\usepackage[utf8]{inputenc}
\usepackage[spanish]{babel}
\usepackage{geometry}
\usepackage{setspace}
\usepackage{parskip}
\usepackage{csquotes}
\usepackage{fontspec}
\usepackage{titlesec}
\usepackage{hyperref}
\usepackage{biblatex}
\usepackage{graphicx}
\usepackage{titling}
\geometry{a4paper, margin=3cm}
\setlength{\parindent}{1.5em}
\setstretch{1.5}
\setmainfont{Times New Roman}

\titleformat{\section}{\large\bfseries}{\thesection.}{0.5em}{}

\begin{document}
%%%%%%%%%%%%%% Title page %%%%%%%%%%%%%%%%%%%%%%%%%%
\begin{titlepage}
	\begin{center}
		\includegraphics[width=0.3\textwidth]{logo-ucsm.png}\\[1cm]
		
		\textsc{\Large Universidad Católica de Santa María}\\[0.5cm]
		\textsc{Facultad de Ciencias Sociales}\\
		\textsc{Escuela Profesional de Teología}\\[2cm]
		
		\rule{\linewidth}{0.5mm} \\[0.4cm]
		{\huge \textbf{La imagen viva del Amor crucificado: \\[0.3cm] San Francisco de Asís como guía hacia Dios}}\\[0.4cm]
		\rule{\linewidth}{0.5mm} \\[1cm]
		
		\large{\textbf{Alumno:} Jorge L. Ayona Inglis\\
		\href{https://orcid.org/0009-0006-6551-9681}{\textbf{ORCID: 0009-0006-6551-9681}}\\[0cm]
		
		\textbf{Profesor:} Nicole Rosario Rodr\'iguez Segura\\[1cm]}
		
		Arequipa, \today %16 de julio de 2025
	\end{center}
\end{titlepage}
%%%%%%%%%%%%%%%%%%%%%%%%%%%%%%%%%%%%%%%%%%%%%%%%%%%%%
\clearpage

	
	\section*{Introducción}
	
		Aunque no conservo el registro exacto de la obra visual, la cual vi de ni\'no en alg\'un grabado o estampa, guardo en la memoria y el alma una representación poderosa que ha marcado mi reflexión espiritual: San Francisco de Asís, postrado en tierra, con el rostro inclinado hacia el suelo, los brazos extendidos en forma de cruz, y las manos sangrantes mostrando los estigmas. No se trata de una escena teatral, sino de una silenciosa proclamación de fe: el hombre reducido al polvo, configurado con el Crucificado, abrazando la cruz sin rebelión ni orgullo.
	
	Esta imagen no solo me habla del amor sin reservas que Francisco tuvo por Cristo, sino también del abajamiento radical que conduce a la unión con Dios. El suelo que besa con su rostro no es derrota, sino altar. Los brazos que extiende no imploran, se entregan. Las llagas que muestra no hieren, iluminan. Es Una imagen contemplada desde el corazón
	
	Como el salmista, Francisco parece decir: \textit{“Yo estoy postrado y humillado; que tu salvación, Señor, me levante”} (Sal 35,13)\cite{biblia}. Esta visión interior se vuelve así un ícono invisible, pero real, que me guía y me interpela.
	
	Las imágenes sacras, lejos de ser simples ornamentos devocionales, constituyen ventanas teológicas a través de las cuales el creyente contempla, se interroga y se deja transformar. La representación de San Francisco de Asís con los brazos extendidos en cruz, rostro en tierra y manos sangrantes por los estigmas, es una de esas imágenes que nos hablan sin palabras. En ella no solo se plasma la vida de un santo, sino un camino espiritual y existencial de retorno a Dios. Este ensayo propone una reflexión teológico-filosófica sobre cómo esta representación visual orienta al ser humano hacia la comunión con lo divino.
	
	
\begin{figure}[!h]
	\centering
	\includegraphics[width=0.7\linewidth]{san_francisco_oracion}
	\caption{San Francisco en oraci\'on}
	\label{fig:sanfranciscooracion}
\end{figure}

	
	\section{El cuerpo como icono: el lenguaje silencioso del gesto}
	
	El cuerpo de San Francisco, tendido en tierra, con los brazos abiertos, no expresa rebeldía, ni dramatismo. Es un cuerpo que ora. Un cuerpo que se ofrece. Como exhorta san Pablo: \textit{“Os exhorto, hermanos, por la misericordia de Dios, a que presentéis vuestros cuerpos como sacrificio vivo, santo y agradable a Dios”} (Rom 12,1) \cite{biblia}. Francisco, al adoptar la forma de cruz, se configura al Crucificado, no solo interiormente, sino exteriormente. Su gesto es una liturgia silenciosa, un sacramento viviente.
	
	San Ireneo afirma: \textit{“La gloria de Dios es el hombre viviente, y la vida del hombre es la visión de Dios”} (Adv. Haer. IV,20,7)\cite{ireneo}. En la imagen del santo, la gloria de Dios se refleja no en la grandeza humana, sino en su abajamiento radical. Francisco no busca exaltarse, sino abrazar su anonadamiento en Cristo (cf. Flp 2,5-11)\cite{biblia}.
	
	\section{Los estigmas: la pascua inscrita en la carne}
	
	La iconografía de Francisco con los estigmas no es una exaltación del dolor físico, sino una proclamación de amor configurado. \textit{“Estoy crucificado con Cristo; y ya no vivo yo, sino que Cristo vive en mí”} (Gál 2,20) \cite{biblia}. Al recibir las llagas de Cristo, Francisco no se convierte en un espectáculo, sino en un testigo. Las heridas no lo separan del mundo, lo hacen más humano. No lo elevan sobre los demás, lo hunden en la compasión.
	
	San Agustín escribe: \textit{“El amor no busca otra cosa que el ser amado: no desea otra recompensa que el amor mismo”} (In ep. Io. tract. IX,9) \cite{agustin}. La aceptación de los estigmas no es una prueba de poder, sino la señal de que Francisco ha amado hasta el extremo.
	
	\section{El rostro en tierra: camino de abajamiento}
	
	El detalle del rostro en tierra no es accesorio. Es, quizá, la clave de todo. En el Antiguo Testamento, los profetas y patriarcas caían rostro en tierra ante la manifestación divina (cf. Ex 3,6; Nm 16,22) \cite{biblia}. Quien ve a Dios se postra. Francisco se postra no ante un castigo, sino ante el misterio del amor. La tierra, símbolo de humildad y de origen, se convierte en su altar. Su actitud recuerda al salmista: \textit{“Yo, en cambio, estoy postrado y humillado; que tu salvación, Señor, me levante”} (Sal 35,13)\cite{biblia}.
	
	San Gregorio Magno escribió que \textit{“la verdadera humildad consiste en conocerse uno mismo tal como es, y someterse totalmente a Dios”} (Moralia in Job, XXIII) \cite{gregorio}. Francisco, al reconocerse nada, se deja llenar de Todo.
	
	\section{Una imagen que transforma}
	
	Contemplar esta imagen de San Francisco no es un ejercicio estético, sino un desafío existencial. Nos interpela: ¿A qué cruz me resisto? ¿Qué heridas me niego a asumir? ¿Dónde me falta humildad? La representación no es un fin en sí misma, sino un medio para tocar el Misterio. \textit{“El que quiera venir en pos de mí, que se niegue a sí mismo, cargue con su cruz cada día y me siga”} (Lc 9,23)\cite{biblia}. Esta imagen nos enseña que la cruz no es solo un sufrimiento a soportar, sino un amor a abrazar.
	
	\section*{Conclusión}
	
	La representación visual de San Francisco de Asís con los brazos en cruz, las manos estigmatizadas y el rostro en tierra, no solo evoca un hecho histórico o místico. Es una catequesis viviente sobre la unión con Cristo. Es la imagen de un hombre que ha hecho de su vida un eco del Evangelio, una extensión de la Pasión y una prueba de que Dios se deja encontrar en la pequeñez. Francisco nos muestra que, al abrazar la cruz sin rebelión ni orgullo, nos convertimos también nosotros en imagen del Amor que salva.
	
	\begin{thebibliography}{9}
		
		\bibitem{ireneo}
		San Ireneo de Lyon, \textit{Contra las herejías}, IV,20,7.
		
		\bibitem{agustin}
		San Agustín, \textit{Tratados sobre la primera carta de san Juan}, IX,9.
		
		\bibitem{gregorio}
		San Gregorio Magno, \textit{Moralia in Job}, Libro XXIII.
		
		\bibitem{biblia}
		Biblia de Jerusalén, Ed. Desclée de Brouwer, 2009.
		
	\end{thebibliography}
	
\end{document}
