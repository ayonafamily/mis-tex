\documentclass[12pt]{article}
\usepackage[utf8]{inputenc}
\usepackage[spanish]{babel}
\usepackage{geometry}
\geometry{a4paper, margin=2.5cm}
\usepackage{parskip}
\usepackage{lmodern}

\title{La sombra como camino pastoral}
\author{Jorge L. Ayona}
\date{\today}

\begin{document}

\maketitle

\section*{Cuando la sombra aparece}

El conflicto reciente con la profesora Abarca ha removido algo profundo en mí. Aunque el malestar inicial pareció tener causas externas, al meditarlo desde la luz de la psicología profunda y la espiritualidad, percibo que se trata de algo más: un encuentro con la sombra.

Según C.~G.~Jung, la sombra es esa parte de nosotros que no queremos ver, que reprimimos o rechazamos. En ocasiones, la proyectamos en los demás, especialmente en momentos de tensión o autoridad. Me doy cuenta de que, quizás, tanto ella como yo hemos proyectado nuestras sombras mutuas: ella, su necesidad de control; yo, mi miedo a afirmar mi libertad frente a figuras fuertes.

\section*{Del conflicto al discernimiento}

Lejos de buscar culpables, hoy contemplo este episodio como un espejo. Mi reacción emocional me mostró zonas heridas en mí: inseguridad, deseo de aprobación, temor a hablar con firmeza. Reconocerlas no me condena, sino que me libera. Aprendo que mi libertad no está en gritar o callar, sino en integrar lo que siento, discernir lo justo y actuar desde la verdad con caridad.

\section*{La pastoral y la sombra}

Esto me recuerda que el acompañamiento pastoral no puede prescindir del alma herida. Muchas veces las tensiones en espacios eclesiales, educativos o comunitarios, no son simples diferencias, sino choques de sombras no integradas. Sin crecimiento interior, el poder se vuelve violencia encubierta y la corrección, control disfrazado.

Una pastoral auténtica debe ayudar a los demás a reconciliarse con su sombra: no para justificar el pecado, sino para madurar en la virtud desde una humanidad redimida. Como decía San Francisco de Sales: ``la devoción verdadera no destruye nada, sino que lo ordena y lo eleva''. La sombra, en este sentido, puede ser el humus donde Dios cultiva la humildad, el discernimiento y la compasión.

\section*{Camino hacia la luz}

No huyo del conflicto. Ya no lo veo solo como injusticia, sino como una oportunidad para crecer. Si Dios permite esta situación, es porque aún hay algo que aprender y entregar. Le pido al Espíritu Santo que me dé sabiduría para no endurecerme, pero también firmeza para no callar cuando la verdad lo exige. Que el corazón de Cristo me enseñe a integrar mi sombra y a caminar hacia la luz, con verdad, caridad y libertad interior.

\end{document}
