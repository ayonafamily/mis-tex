\documentclass[]{article}
\usepackage{graphicx}
\usepackage[spanish, english]{babel}
%opening



\begin{document}
	
		\begin{titlepage}
			\centering
			{\scshape\Huge Recuperar el sentido de la vida en tiempos de subjetivismo moderno \par}
			\vspace{3cm}
			
			\vfill
			{\Large Autor: Jorge Ayona \par}
			
			%{\includegraphics[width=0.2\textwidth]{logo}\par}
			\vspace{1cm}
			{\bfseries\LARGE Universidad Católica Santa María De Arequipa \par}
			\vspace{1cm}
			{\scshape\Large Facultad de Teología \par}
			\vspace{3cm}
			
			\vfill
			{\Large Junio 2025 \par}
			\end{titlepage}

\selectlanguage{spanish} 
\begin{abstract}
	El presente artículo reflexiona sobre la desconexión del sentido trascendente en las sociedades modernas, en particular en aquellas de tradición occidental con altos niveles de prosperidad. A través de un análisis de La peste de Albert Camus y las propuestas éticas de Viktor Frankl y Alasdair MacIntyre, se argumenta que la pérdida del sentido de la vida y el auge del subjetivismo son fenómenos característicos de la modernidad. Se plantea la necesidad de reconectar con una ética de los valores que permita superar la fragmentación y recuperar un propósito compartido.
\end{abstract}
\selectlanguage{english}
\begin{abstract}
This article reflects on the disconnection from transcendence in modern societies, particularly in those with a Western tradition and high levels of prosperity. Through an analysis of Albert Camus's The Plague and the ethical proposals of Viktor Frankl and Alasdair MacIntyre, it is argued that the loss of life's meaning and the rise of subjectivism are characteristic phenomena of modernity. The need to reconnect with a value-based ethic that can overcome fragmentation and restore a shared purpose is emphasized.
\end{abstract}


\section{Introducción}
En las sociedades contemporáneas, caracterizadas por un alto nivel de prosperidad material y una desconexión progresiva de lo trascendente, se observa una tendencia hacia el subjetivismo y la pérdida del propósito compartido. Este fenómeno no es exclusivamente moderno, pero encuentra en la actualidad una expresión particular, donde la rutina, los pasatiempos vacíos y las relaciones humanas superficiales son percibidos como absurdos, tal como lo describe Albert Camus en La peste.
El objetivo de este artículo es explorar cómo esta desconexión puede ser comprendida y superada a través de un retorno a la ética de los valores, tomando como referencias fundamentales las perspectivas de Camus, Viktor Frankl y Alasdair MacIntyre.
\section{El absurdo en la obra de Albert Camus}
Camus, en La peste, describe la ciudad de Orán como un espacio donde la monotonía y la alienación predominan. Sus habitantes trabajan, se entretienen con pasatiempos triviales y mantienen relaciones humanas desprovistas de propósito, todo enmarcado en una rutina que resalta la ausencia de sentido trascendente. Esta visión resuena con el concepto de lo absurdo, entendido como la falta de correspondencia entre el anhelo humano de sentido y un universo indiferente.
Si bien Camus no propone una salida metafísica al absurdo, su obra invita a reflexionar sobre la naturaleza de las relaciones humanas y los valores que orientan nuestras acciones. Sin embargo, su enfoque, aunque poderoso, deja un vacío: ¿cómo abordar la búsqueda de sentido en un contexto donde la trascendencia parece haber sido descartada?
\section{Viktor Frankl y la búsqueda del propósito}
Frente a la visión de Camus, Viktor Frankl ofrece una perspectiva complementaria. En su obra El hombre en busca de sentido, Frankl sostiene que incluso en las circunstancias más adversas, el ser humano puede encontrar propósito a través de los valores, ya sea en el amor, el sufrimiento o la creación. A diferencia de Camus, Frankl introduce una dimensión trascendente que permite superar el vacío existencial.
La propuesta de Frankl subraya que la búsqueda de sentido no es meramente subjetiva, sino que se relaciona con valores objetivos que trascienden al individuo. Esta visión constituye un contrapunto esencial al subjetivismo moderno y al individualismo que caracteriza a las sociedades contemporáneas.
\section{Alasdair MacIntyre y la ética de los valores}
MacIntyre, en Tras la virtud, argumenta que la modernidad ha fragmentado la ética, reduciéndola a preferencias subjetivas desligadas de un marco narrativo común. Propone un retorno a una ética de los valores basada en prácticas virtuosas y en una narrativa compartida que permita a las personas encontrar un propósito colectivo.
La crítica de MacIntyre al individualismo moderno es especialmente relevante en un contexto donde el subjetivismo ha llevado a la pérdida de sentido compartido. Su propuesta enfatiza la necesidad de recuperar una ética orientada hacia el bien común, en contraste con la fragmentación actual.
\section{Subjetivismo moderno y trascendencia}
El subjetivismo moderno, caracterizado por el aislamiento y la desconexión de lo trascendente, refleja una crisis existencial que afecta tanto a nivel individual como social. Las sociedades que han alcanzado altos niveles de prosperidad material tienden a centrarse en el individuo, relegando los valores trascendentes al ámbito privado o descartándolos por completo.
Este fenómeno, que Camus describe literariamente y que Frankl y MacIntyre critican desde perspectivas éticas, plantea un desafío fundamental: ¿es posible superar esta desconexión y recuperar un propósito compartido en una era posmoderna?
\section{Conclusión}
La reflexión sobre el absurdo y la trascendencia en las sociedades modernas no es un ejercicio meramente académico, sino una necesidad urgente. La propuesta de Viktor Frankl de encontrar sentido en los valores y la ética de Alasdair MacIntyre basada en la narrativa compartida ofrecen herramientas valiosas para abordar el vacío existencial y el subjetivismo contemporáneo.
La obra de Camus, aunque situada en un contexto particular, continúa siendo relevante como un punto de partida para reflexionar sobre los desafíos éticos y existenciales de la modernidad. Superar el subjetivismo moderno implica no solo un cambio de perspectiva individual, sino también un esfuerzo colectivo por construir una sociedad orientada hacia valores trascendentes y un propósito compartido.
\end{document}
