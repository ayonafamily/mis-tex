\documentclass[12pt]{article}
\usepackage[utf8]{inputenc}
\usepackage[spanish]{babel}
\usepackage{setspace}
\usepackage{geometry}
\usepackage{parskip}
\usepackage{times}

\geometry{a4paper, margin=2.5cm}
\setlength{\parindent}{0pt}
\onehalfspacing

\title{\textbf{Trump, MAGA y la banalidad del mal:\\ Una lectura desde Hannah Arendt}}
\author{\textit{Jorge L. Ayona Inglis}}
\date{\today}

\begin{document}

\maketitle

\section*{Introducción}

Donald Trump llegó a la presidencia en 2016 con la promesa de ``hacer a América grande otra vez'' (\textit{Make America Great Again}), un lema que captó el malestar de millones de ciudadanos frustrados por las guerras interminables y el desprecio del \textit{establishment}. Uno de sus logros más citados por sus seguidores fue haber evitado iniciar nuevas guerras. Sin embargo, en su segundo mandato, Trump ha asumido una postura beligerante en política exterior: ataques preventivos a Irán, propuestas de expulsión masiva en Gaza y llamados a ``recuperar'' territorios estratégicos. ¿Cómo comprender este giro? ¿Es solo geopolítica, o hay algo más profundo? A la luz de \textit{Eichmann en Jerusalén}, de Hannah Arendt, podemos entender este fenómeno desde la noción de ``la banalidad del mal''.

\section*{La obediencia sin pensamiento}

Hannah Arendt, filósofa judía alemana, fue también víctima del avance del totalitarismo. Detenida por el régimen de Vichy en Francia y enviada al campo de internamiento de Gurs, logró escapar y refugiarse en Estados Unidos. Esta experiencia marcó profundamente su visión del mal y la política. 

En su obra \textit{Eichmann en Jerusalén}, Arendt describe a Adolf Eichmann ---organizador logístico del Holocausto--- no como un monstruo ideológico, sino como un burócrata obediente, incapaz de pensar por sí mismo. Su mal no residía en el odio, sino en la \textbf{irreflexión}: nunca se cuestionó si lo que hacía era bueno o malo. Solo cumplía con lo que le parecía ``normal'' en su contexto.

Hoy, vemos un fenómeno similar en sectores de la base MAGA. Personas comunes que, tras años de denunciar el ``imperialismo demócrata'', apoyan ahora bombardeos contra Irán o propuestas de expulsar a millones de palestinos de Gaza. No lo hacen necesariamente por odio, sino porque \textbf{repiten consignas, asumen una lógica binaria (nosotros vs. ellos)} y se identifican emocionalmente con un líder carismático. El pensamiento queda suspendido.

\section*{Lenguaje moral para actos inmorales}

Arendt también señala cómo el uso de clichés y lenguaje administrativo puede encubrir el horror. Eichmann hablaba de ``traslados'' o ``soluciones finales'', evitando así confrontar la realidad de los campos de exterminio. Hoy, el discurso político global emplea eufemismos similares: \textbf{``intervención humanitaria'', ``acción preventiva'', ``pacificación de Gaza''}. Estas expresiones permiten justificar lo injustificable, despojando de humanidad a las víctimas.

¿No es acaso inquietante que un movimiento que decía defender ``América primero'' ---con una supuesta actitud de no intervención--- acabe respaldando operaciones que implican muerte, desplazamiento y dominio territorial? ¿Qué parte de la conciencia moral se está desconectando?

\section*{La responsabilidad es individual}

Arendt deja claro que ningún sistema, líder o circunstancia histórica exime a una persona de su juicio moral. La responsabilidad es indelegable. En este sentido, apoyar sin cuestionar a Trump ---o a cualquier líder--- es ya una forma de renunciar al pensamiento. Lo que se necesita no es obediencia, sino \textbf{conciencia}.

\section*{Pensar para no obedecer: una advertencia desde Arendt}

Leer \textit{Eichmann en Jerusalén}, de Hannah Arendt, me ha producido una inquietud profunda. Me estremece ver cómo personas comunes, sin ser fanáticas ni monstruosas, colaboraron con el régimen nazi y con el Holocausto. No lo hicieron movidos por odio, sino por obediencia ciega: por seguir consignas sin reflexionar.

Esa obediencia ---esa ``banalidad del mal''--- no es cosa del pasado. Hoy, movimientos de la llamada ``nueva izquierda'' repiten frases sobre género o identidad sin razonamiento crítico, sin diálogo con la realidad. Y en la otra orilla, sectores de la derecha como el movimiento MAGA hacen lo mismo: consignas cerradas, culto al líder, violencia simbólica, rechazo al pensamiento disidente.

Entonces me pregunto: ¿cuál es la solución?  
La única respuesta honesta que encuentro es \textbf{educar para pensar}. Volver a las raíces de la filosofía, de la historia, de la lectura profunda. Formar jóvenes que no repitan lemas, sino que se atrevan a preguntarse: ``¿Esto es justo? ¿Es verdadero? ¿Es humano?''

Pensar ---dice Arendt--- no nos convierte en héroes, pero sí puede impedir que participemos en el mal. El pensamiento es resistencia. Y por eso, educar para el pensamiento no es un lujo: es una forma de salvación.

\section*{Conclusión}

Leer a Hannah Arendt no es solo estudiar el pasado, sino \textbf{mirarnos en el espejo del presente}. Su crítica a la banalidad del mal nos alerta contra la obediencia ciega, el lenguaje vacío y la incapacidad de pensar. La evolución del movimiento MAGA bajo Trump, de aislacionismo a belicismo moralizante, no se entiende sin analizar el modo en que las masas dejan de juzgar. Y eso ---precisamente--- es lo que permite que el mal vuelva a hacerse posible, no por odio, sino por \textbf{la ausencia de pensamiento}.

\end{document}
