\documentclass[12pt]{article}
\usepackage[utf8]{inputenc}
\usepackage[spanish]{babel}
\usepackage{geometry}
\usepackage{parskip}
\usepackage{hyperref}
\usepackage{setspace}
\geometry{a4paper, margin=2.5cm}
\title{La Pachamama y la Teología de la Liberación Reciclada:\\Una Crítica desde la Ortodoxia Católica}
\author{Jorge L. Ayona Inglis}
\date{\today}

\begin{document}

\maketitle
\onehalfspacing

\section*{Introducción}
La aparición del símbolo de la Pachamama en contextos eclesiales recientes, particularmente durante el Sínodo para la Amazonía (2019), ha suscitado fuertes reacciones dentro del catolicismo. Una de las críticas más consistentes sostiene que dicho símbolo no solo está fuera de lugar culturalmente (por ser un elemento andino en un contexto amazónico), sino que representa una herencia ideológica de la Teología de la Liberación, reciclada y adaptada a las nuevas corrientes ecologistas, lo que algunos denominan “marxismo cultural”. Este ensayo analiza dicha crítica desde el modelo argumentativo de Toulmin, aportando contexto histórico, teológico y doctrinal.

\section*{1. Afirmación principal (Claim)}
La figura de la Pachamama utilizada en ambientes eclesiales actuales representa una continuación ideológica de la Teología de la Liberación, ahora fusionada con discursos ecologistas y panteístas, lo cual constituye una forma de sincretismo doctrinal e infiltración ideológica dentro de la Iglesia.

\section*{2. Fundamentos históricos (Data)}
Durante los años 70 y 80, la Teología de la Liberación adoptó herramientas del análisis marxista para interpretar la realidad socioeconómica de América Latina. Algunos teólogos promovieron la incorporación de elementos culturales indígenas como signos de resistencia y dignificación popular. Tras la caída del Muro de Berlín y el descrédito del marxismo político, muchos de estos autores viraron hacia una “ecoteología” centrada en la tierra, los pueblos originarios y el respeto a la “Madre Tierra”. En los años 2000, esta línea de pensamiento se revitalizó en torno al discurso ambientalista, coincidiendo con el auge de gobiernos de izquierda indigenista en la región.

\section*{3. Garantía teológica (Warrant)}
Cuando una corriente teológica adopta ideas exógenas a la Revelación (como el materialismo histórico o el panteísmo ecológico) y las reviste de lenguaje cristiano, se produce una deformación de la fe: ya no se parte del dato revelado sino de una ideología secular, reconfigurando el mensaje del Evangelio.

\section*{4. Respaldo magisterial (Backing)}
Documentos como la \textit{Instrucción sobre algunos aspectos de la Teología de la Liberación} (\textit{Libertatis Nuntius}, 1984), redactado por el entonces Cardenal Ratzinger, advirtieron sobre el peligro de reducir el cristianismo a una praxis sociopolítica (n. 5–11). También señalaron que la inculturación debe respetar la integridad de la fe y no confundir elementos culturales con dogmas revelados (\textit{Redemptoris Missio}, n. 52). La asociación de la Pachamama con la Virgen María, como se insinuó en algunos sectores del Sínodo, contradice la iconografía y teología mariana tradicional de la Iglesia, generando confusión y escándalo (Catecismo de la Iglesia Católica, n. 971).

\section*{5. Calificador modal (Qualifier)}
Es razonable sostener que gran parte del discurso ambientalista en ciertos sectores eclesiales actuales está influido por esta evolución ideológica, aunque no toda preocupación por la ecología o las culturas originarias debe ser rechazada en bloque.

\section*{6. Refutaciones posibles y respuesta (Rebuttal)}
Quienes defienden la presencia de la Pachamama en actos eclesiales afirman que se trata de una expresión cultural respetuosa de las tradiciones ancestrales, enmarcada en la “ecología integral” propuesta por \textit{Laudato Si’} (nn. 11–13). Sin embargo, esto omite el hecho de que la Pachamama, en muchas comunidades, conserva una dimensión cultual y devocional incompatible con la fe cristiana. Además, su uso simbólico fuera de contexto (en un sínodo sobre la Amazonía cuando es una deidad andina) refuerza la idea de que se instrumentaliza con fines ideológicos, no teológicos.

\section*{Conclusión}
La utilización de la Pachamama como símbolo en actos eclesiales no solo es culturalmente inexacta, sino doctrinalmente peligrosa. Representa la supervivencia de una Teología de la Liberación reformulada al gusto de los nuevos tiempos: con rostro ecológico, pero con el mismo trasfondo de reinterpretación ideológica del cristianismo. La inculturación auténtica debe nacer del discernimiento teológico y del respeto por la fe revelada, no de la adaptación acrítica a las modas ideológicas del momento.

\begin{thebibliography}{9}

\bibitem{libertatis}
Congregación para la Doctrina de la Fe. 
\textit{Libertatis Nuntius: Instrucción sobre algunos aspectos de la Teología de la Liberación}. 
Vaticano, 1984. Disponible en: \url{https://www.vatican.va}

\bibitem{redemptoris}
Juan Pablo II. 
\textit{Redemptoris Missio: Carta encíclica sobre la misión de la Iglesia}. 
Vaticano, 1990.

\bibitem{laudato}
Papa Francisco. 
\textit{Laudato Si’: Encíclica sobre el cuidado de la casa común}. 
Vaticano, 2015.

\bibitem{catecismo}
Santa Sede. 
\textit{Catecismo de la Iglesia Católica}. 2ª ed. 
Ciudad del Vaticano, 1997.

\end{thebibliography}

\end{document}
