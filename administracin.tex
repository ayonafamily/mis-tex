\documentclass[a4paper,12pt]{article}
\usepackage[utf8]{inputenc}
\usepackage[spanish]{babel}
\usepackage{tabularx}
\usepackage{booktabs} % Para líneas horizontales más elegantes
\usepackage{caption}  % Para personalizar los títulos de tablas
\usepackage{geometry}
\geometry{a4paper, margin=2.5cm}
\usepackage{setspace}
\usepackage{titlesec}
\usepackage{lmodern}
\usepackage{fontspec}
\setmainfont{Liberation Sans} % libre y similar
\usepackage{parskip}
\usepackage{hyperref}
\hypersetup{
	colorlinks=true,
	linkcolor=black,
	urlcolor=blue
}

\titleformat{\section}{\large\bfseries}{\thesection.}{1em}{}
\setstretch{1.5}


\begin{document}
	
	\begin{titlepage}
		\centering
		\vspace*{3cm}
		
		{\LARGE\bfseries Notas de Estudio y Reflexión \par}
		\vspace{1cm}
		
		{\Large Jorge L. Ayona Inglis\par}
		
		\vfill
		
		\begin{flushright}
			\itshape
			“El mejor vino se sirve al final”\\
			(cf. Jn 2,10)
		\end{flushright}
		
		\vspace{2cm}
		
		{\large Julio de 2025\par}
		
	\end{titlepage}
	
	
	\section{Resumen: \textit{Principios de la administración científica} – F. W. Taylor}
	
	En esta obra, publicada en 1911, Frederick W. Taylor propone una nueva forma de organización del trabajo basada en el uso del método científico. Su objetivo era sustituir la administración empírica y artesanal por un enfoque racional y medible, con el fin de aumentar la eficiencia y productividad tanto del obrero como de la empresa.
	
	Taylor sostiene que para cada tarea existe ``una única mejor manera'' de ser realizada, la cual puede descubrirse mediante la observación, la medición y la experimentación. Introduce el estudio de tiempos y movimientos, la selección científica del trabajador, la capacitación específica, la cooperación estrecha entre dirección y operarios, y la distribución equitativa del trabajo intelectual y manual.
	
	El enfoque de Taylor privilegia la eficiencia y el control, aunque ha sido criticado por su visión mecanicista del trabajador. No obstante, sentó las bases para el desarrollo posterior de la ingeniería industrial y del análisis racional del trabajo.

\clearpage	
	
	
	\section{Principios de la administración científica – Frederick W. Taylor}
	
	Taylor propuso una administración basada en métodos científicos en lugar de la improvisación. Planteó que existe ``una única mejor manera'' de realizar cada tarea y que esta puede descubrirse mediante el estudio sistemático del trabajo.
	
	\begin{itemize}
		\item Sustituir el empirismo por la ciencia.
		\item Selección y entrenamiento científico de los trabajadores.
		\item Cooperación entre trabajadores y dirección.
		\item División equitativa del trabajo: la dirección piensa y planifica, el obrero ejecuta.
		\item Sistema de incentivos para la productividad.
	\end{itemize}
	
	\section{Taylor como precursor del análisis de procesos}
	
	Aunque el enfoque de Frederick W. Taylor ha sido tradicionalmente clasificado como parte de la administración científica, una lectura más profunda permite interpretarlo también como un pionero en el análisis de procesos.
	
	Taylor proponía observar detenidamente las tareas del trabajador, descomponerlas en partes mínimas, medir su duración y eliminar movimientos innecesarios. Esta forma de estudio sistemático, basada en el cronometraje y la secuencia lógica de actividades, anticipa lo que en décadas posteriores se consolidaría como análisis y mejora de procesos.
	
	\subsection*{Elementos que anticipan el enfoque moderno}
	\begin{itemize}
		\item \textbf{Descomposición de tareas:} Taylor analizaba el trabajo dividiéndolo en unidades mínimas para su comprensión y optimización.
		\item \textbf{Estudio de tiempos y movimientos:} Utilizaba observación directa y cronómetros para medir y ajustar cada acción.
		\item \textbf{Estandarización del método óptimo:} Tras identificar la mejor manera de realizar una tarea, esta debía replicarse por todos.
		\item \textbf{Secuencia lógica y eficiencia:} Reordenaba actividades para reducir el tiempo y esfuerzo, logrando fluidez en la ejecución.
	\end{itemize}
	
	Aunque su perspectiva era principalmente individual y mecanicista, su legado metodológico sentó las bases para herramientas modernas como los diagramas de flujo, el mapeo de procesos, el enfoque Lean y el ciclo de mejora continua (PDCA). En ese sentido, Taylor puede considerarse un antecedente directo de la ingeniería de procesos y de la gestión por procesos que predomina en las organizaciones actuales.
	
	
	\section*{Comparación: Enfoque de Taylor vs. Análisis de procesos moderno}
	
	\begin{table}[h!]
		\centering
		\caption*{\textbf{Cuadro comparativo entre el enfoque de Taylor y el análisis de procesos moderno}}
		\begin{tabularx}{\textwidth}{>{\bfseries}l X X}
			\toprule
			\textbf{Aspecto} & \textbf{Taylor (Administración científica)} & \textbf{Análisis de procesos moderno} \\
			\midrule
			Objeto de estudio & Tarea individual del operario & Proceso completo que atraviesa varias áreas \\
			\addlinespace
			Método & Observación directa y cronometraje & Modelado de procesos, BPMN, métricas, auditoría \\
			\addlinespace
			Objetivo & Eficiencia individual y ahorro de tiempo & Generar valor y eficiencia sistémica para el cliente \\
			\addlinespace
			Enfoque & Mecanicista, vertical, centrado en el obrero & Sistémico, transversal y centrado en el cliente interno o externo \\
			\addlinespace
			Repetición de tareas & Alta repetitividad, escasa variabilidad & Puede incluir variaciones, decisiones y mejoras continuas \\
			\addlinespace
			Herramientas & Cronómetro, hoja de tiempos, supervisión & Diagramas de flujo, mapas de procesos, indicadores de desempeño \\
			\addlinespace
			Nivel de intervención & Micro (puesto de trabajo) & Macro y meso (proceso y subproceso organizacional) \\
			\bottomrule
		\end{tabularx}
	\end{table}
	
	\section{Administración industrial y general – Henri Fayol}
	
	Fayol ofreció una visión integral de la administración como disciplina aplicable a cualquier organización. Propuso cinco funciones administrativas y catorce principios universales.
	
	\subsection*{Funciones administrativas}
	\begin{itemize}
		\item Prever (planificar)
		\item Organizar
		\item Mandar (dirigir)
		\item Coordinar
		\item Controlar
	\end{itemize}
	
	\subsection*{Principios destacados}
	División del trabajo, unidad de mando, autoridad y responsabilidad, centralización, orden, equidad, estabilidad del personal, iniciativa, espíritu de equipo, entre otros.
	
	\section{Teoría moderna: Ciclo PDCA – W. Edwards Deming}
	
	Deming introdujo el ciclo \textbf{PDCA} como núcleo de la mejora continua:
	
	\begin{enumerate}
		\item \textbf{Plan}: identificar el problema y establecer un plan.
		\item \textbf{Do}: ejecutar a pequeña escala.
		\item \textbf{Check}: verificar los resultados.
		\item \textbf{Act}: ajustar o estandarizar.
	\end{enumerate}
	
	Este enfoque plantea una gestión dinámica, participativa y centrada en el aprendizaje organizacional.
	
	\section{Evolución personal y reflexión}
	
	Cuando inicié mis estudios en administración en 1979, convivían múltiples escuelas —Taylor, Fayol, Koontz, O'Donnell, Reyes Ponce— generando una “selva teórica”. Muchas veces no sabíamos qué aplicar en la práctica.
	
	Al regresar en 2010 para completar mis estudios, encontré un panorama más ordenado gracias a la irrupción de la \textbf{teoría de la calidad total}, el enfoque sistémico y las herramientas de mejora continua. Fue una grata sorpresa: la administración ya no se limitaba a jerarquías o eficiencia mecanicista, sino que incorporaba la participación, la retroalimentación y el aprendizaje constante.
	
	Lo que me preocupa, sin embargo, es que muchas de estas competencias se han desplazado hacia la \textbf{ingeniería industrial}, que hoy lidera áreas como la mejora de procesos, la eficiencia operativa o la gestión de calidad. 
	
	No obstante, sigo convencido de que el administrador tiene un rol insustituible: visión estratégica, liderazgo humano, manejo del cambio y comprensión de la organización como una realidad social compleja. 
	
	Mi experiencia personal me permite integrar lo mejor del pensamiento clásico con las herramientas modernas. La administración, lejos de ser una ciencia muerta, es una disciplina viva, en permanente evolución.
	
	\clearpage
	
	\section*{Comparación entre Taylor, Fayol y Deming}
	
	\begin{table}[h!]
		\centering
		\caption*{\textbf{Cuadro comparativo: Taylor – Fayol – Deming}}
		\begin{tabularx}{\textwidth}{>{\bfseries}l X X X}
			\toprule
			\textbf{Aspecto} & \textbf{Taylor} & \textbf{Fayol} & \textbf{Deming} \\
			\midrule
			Enfoque principal & Eficiencia operativa mediante estudio científico del trabajo & Organización estructurada mediante principios generales de administración & Mejora continua de la calidad a través del ciclo PDCA \\
			\addlinespace
			Nivel de aplicación & Nivel operativo (obreros y tareas) & Nivel directivo y gerencial & Toda la organización (sistema completo) \\
			\addlinespace
			Objeto de estudio & Tarea individual & Funciones administrativas & Procesos organizacionales y sistemas de calidad \\
			\addlinespace
			Método & Estudio de tiempos y movimientos & Principios y funciones administrativas & Ciclo PDCA: Planificar, Hacer, Verificar, Actuar \\
			\addlinespace
			Meta principal & Aumentar la productividad individual & Lograr eficiencia organizacional estructurada & Alcanzar calidad total y mejora continua \\
			\addlinespace
			Relación con el trabajador & Visión mecánica y controladora & Jerárquica, pero con elementos humanistas & Participativa, centrada en el aprendizaje organizacional \\
			\addlinespace
			Cambio y adaptación & No considerados (método óptimo estático) & Limitada, enfocada en estructura estable & Cambio continuo como parte del proceso de gestión \\
			\bottomrule
		\end{tabularx}
	\end{table}
	
	\section*{De la Ciencia de la Administración al \textit{Big Data}: una continuidad}
	
	Durante mi formación universitaria en la década de 1970, particularmente desde 1979 en la Universidad de San Martín de Porres (USMP), tuve la oportunidad de ser formado dentro de la corriente conocida como \textbf{la Ciencia de la Administración}. El decano de ese entonces promovía esta escuela, que tenía como núcleo el uso de herramientas cuantitativas para la toma de decisiones. En consecuencia, llevé cursos como estadística, muestreo, investigación operativa, programación lineal, y teoría de decisiones.
	
	Esta corriente, también conocida como \textit{Management Science}, emergió después de la Segunda Guerra Mundial y aplicaba modelos matemáticos para resolver problemas administrativos complejos. Su premisa era clara: la administración no solo era arte o experiencia práctica, sino también una disciplina que podía ser formalizada y optimizada a través de métodos científicos.
	
	Lo notable es que, décadas después, observo cómo la administración moderna ha incorporado con fuerza el análisis de datos, los modelos predictivos, la inteligencia de negocios y el \textit{Big Data}, lo que confirma la vigencia de aquella formación que recibí. En lugar de haber quedado obsoleta, la escuela cuantitativa ha evolucionado y se ha convertido en la base de enfoques contemporáneos como la analítica empresarial, el análisis de procesos y la optimización de recursos a través de algoritmos.
	
	\textbf{Lo que en 1979 era investigación operativa, hoy se llama analítica avanzada. Lo que era muestreo, hoy es minería de datos.}
	
	Esta continuidad me confirma que la administración no es una moda, sino un campo en constante evolución, donde las bases científicas se renuevan, pero no desaparecen. En este sentido, valoro profundamente la visión que recibí, pues me permite comprender la lógica detrás de los modelos actuales y no depender exclusivamente de la tecnología o de software externo. La formación analítica me dio criterio, y ese criterio sigue siendo indispensable hoy.

\section*{Organización y Métodos como antesala a la lógica de sistemas}

Una de las áreas que más valoré durante mi formación y mi experiencia laboral temprana fue la de \textbf{Organización y Métodos (O\&M)}. Esta unidad se encargaba del análisis, simplificación y documentación de los procesos administrativos, así como de la elaboración del Manual de Organización y Funciones (MOF). Si bien parecía un trabajo netamente burocrático, en realidad implicaba una comprensión lógica y sistemática de las operaciones.

Con el tiempo, descubrí que esta forma de pensar era la misma que requería la programación: análisis funcional, diagramas de flujo, definición de pasos, eliminación de redundancias, optimización de secuencias. El paso de O\&M a los lenguajes de programación fue, para mí, natural.

\textbf{Lo que antes era una descripción de procedimientos, hoy se traduce en algoritmos. Lo que eran funciones administrativas, hoy son módulos de software.}

En retrospectiva, veo que O\&M fue una escuela rigurosa y silenciosa de pensamiento estructurado, que me preparó sin saberlo para el mundo digital y lógico de los sistemas informáticos. Esta conexión entre administración y tecnología es una muestra más de la riqueza y la versatilidad de la disciplina administrativa.

\begin{table}[h!]
	\centering
	\caption*{\textbf{Paralelismo entre Organización y Métodos y la lógica de programación}}
	\begin{tabularx}{\textwidth}{>{\bfseries}l X X}
		\toprule
		\textbf{Elemento} & \textbf{Organización y Métodos (O\&M)} & \textbf{Programación} \\
		\midrule
		Secuencia de pasos & Procedimiento administrativo & Algoritmo \\
		\addlinespace
		Reducción de redundancias & Simplificación de tareas repetitivas & Refactorización del código \\
		\addlinespace
		Estructura lógica & Diagrama de flujo de procesos & Diagrama de flujo de funciones \\
		\addlinespace
		Distribución de tareas & Manual de Organización y Funciones (MOF) & Definición de funciones o módulos \\
		\addlinespace
		Control de decisiones & Matriz de decisiones o rutas condicionales & Estructuras \texttt{if}, \texttt{switch}, \texttt{while} \\
		\addlinespace
		Optimización & Revisión de procesos y reducción de tiempos & Mejora de eficiencia algorítmica \\
		\bottomrule
	\end{tabularx}
\end{table}
	
\section*{Administración y vocación: una reconciliación tardía}

Al revisar estos apuntes, y repasar las bases históricas y conceptuales de la administración, no puedo evitar mirar atrás con cierta mezcla de claridad, nostalgia y sorpresa. Durante mis años de formación en la Universidad de San Martín de Porres, tuve contacto con aspectos técnicos y metódicos de la administración que resonaban profundamente conmigo: Organización y Métodos, análisis de procesos, estadísticas aplicadas, programación lineal, muestreo, investigación operativa. Aquello no era improvisación; era lógica, estructura, pensamiento.

Recuerdo con especial viveza aquel curso de procesamiento electrónico de datos que llevé a los 18 años. No solo obtuve el primer lugar superando incluso a ingenieros, sino que fui invitado por la empresa NCR a integrarme a su universidad corporativa. Había en mí una inclinación natural hacia lo sistemático, hacia el ordenamiento de la información, hacia la construcción de soluciones mediante la razón.

Y sin embargo, las circunstancias del país y del mercado laboral me condujeron por caminos que no correspondían con mi perfil. Me ofrecieron solo ventas. Y con el tiempo, eso hizo que desarrollara un rechazo profundo hacia la administración como carrera. No era para mí —creía—, pero con los años he comprendido que el problema no era la disciplina, sino el modo en que se la restringía y malinterpretaba.

Hoy puedo afirmar que la administración, cuando se vive desde su dimensión analítica, estratégica y estructural, sí era mi camino. En un entorno diferente, quizás habría sido un consultor de procesos, un analista de datos, un diseñador de sistemas de información. Pero incluso sin haber transitado formalmente por esa ruta, he reencontrado en la lectura y la reflexión una reconciliación con lo que verdaderamente me apasionaba.

Esta nota no es una queja tardía. Es más bien un reconocimiento agradecido. Porque, aunque los caminos laborales me hayan llevado por rutas distintas, la vocación se quedó dentro. Y ahora, al redescubrirla, me siento nuevamente en casa.

\section*{Epílogo: Sobrevivir a pesar del sistema – Testimonio de una vocación fiel}

Hoy puedo mirar hacia atrás y reconocer algo que durante muchos años me negué a ver: que no fracasé. Simplemente, fui empujado fuera de foco por un entorno que no supo, no quiso o no pudo dar lugar a quienes pensaban diferente. Crecí con una vocación analítica, con una profunda inclinación por la filosofía, la teología, la estructura, el orden, el pensamiento lógico. Pero en lugar de ser guiado, fui empujado a carreras que no había elegido, bajo criterios arbitrarios o familiares. Ingresé en primer lugar a la carrera de Cooperativismo —por obligación—, mientras otros, como mi propio primo, quedaron atrás. Él no soportó la humillación y se fue del país. Yo, en cambio, perseveré, aunque el sistema ya comenzaba a mostrarme sus grietas.

La universidad en esos años era caótica, improvisada, sin claridad en sus objetivos ni rutas. Nos decían que el bachillerato era automático, pero nadie sabía cómo. Perdimos casi un año con una profesora mayor que apenas duró unos meses. La confusión era la norma, no la excepción. Finalmente fui transferido a Administración, donde al menos los cursos se daban con cierta regularidad. Pero justo al llegar al final de la carrera, el apoyo económico de mis padres se retiró, y no sabíamos cuánto costaba realmente continuar. Todo era incertidumbre.

Luego ingresé al seminario evangélico, donde nuevamente destaqué: primero en conocimiento bíblico. Siempre fue así. Donde había estudio serio, yo sobresalía. Pero eso generaba envidias. Y, como en muchos entornos mediocres, ser competente se paga caro.

Pasaron los años y, sin grado académico, me encontré predicando a profesionales, dando seminarios sobre administración a gente que sí tenía títulos, pero muchas veces no tenía profundidad. Incluso una universidad quiso contratarme, pero me faltaba el grado. Me ofrecieron convalidaciones, pero a un costo que no podía cubrir.

El tiempo ha pasado. Hoy no tengo la fuerza física de antes. Mi cuerpo tiene sus límites, mis horarios de sueño son frágiles. Y sin embargo, mi mente no. Mi vocación tampoco. Y, sobre todo, mi conciencia ahora es más clara que nunca.

\textbf{No fracasé. Sobreviví.}

Sobreviví a un sistema que no supo qué hacer con alguien como yo. Sobreviví a la improvisación académica, al castigo de ser brillante sin respaldo económico, a la torpeza de quienes debieron guiarme. Sobreviví a las falsas promesas de la meritocracia. Y aún hoy, puedo decir que mi vocación está intacta.

Sé que pude haber sido un analista. Sé que aún puedo serlo. Y aunque mi historia no fue la del éxito convencional, fue la de una fidelidad profunda a lo que siempre supe que era: un pensador, un estructurador, un servidor del conocimiento. La administración no me falló. Fue el entorno el que no estuvo a la altura.

Ahora, con gratitud serena, escribo estas líneas no como reclamo, sino como reconocimiento. Como afirmación. Como acto de justicia silenciosa. Porque la vocación que sobrevive al olvido, al abandono, a la exclusión, es más fuerte que cualquier título. Y esa vocación, la mía, aún sigue viva.

	
	
\end{document}
