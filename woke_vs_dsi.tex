\documentclass[12pt]{article}
\usepackage[utf8]{inputenc}
\usepackage[spanish]{babel}
\usepackage{geometry}
\usepackage{parskip}
\usepackage{hyperref}
\usepackage{setspace}
\geometry{a4paper, margin=2.5cm}
\title{La Opción Preferencial por los Pobres:\\Entre la Caridad Evangélica y la Deriva Ideológica}
\author{Jorge L. Ayona Inglis}
\date{\today}

\begin{document}

\maketitle
\onehalfspacing

\noindent
La llamada “opción preferencial por los pobres” es una de las expresiones más significativas de la doctrina social de la Iglesia en el siglo XX. Inspirada por el Evangelio y arraigada en la tradición profética, esta opción expresa el compromiso de la Iglesia con quienes se encuentran en situaciones de vulnerabilidad, marginación o injusticia. Sin embargo, en ciertos sectores eclesiales y teológicos, esta noble intención ha sido objeto de reducciones ideológicas que desnaturalizan su sentido original, aproximándola a lógicas de confrontación propias de la llamada “discriminación positiva” promovida por el pensamiento \textit{woke} contemporáneo.

En su raíz auténtica, la opción preferencial por los pobres no es una exclusión de los demás, ni una forma de revancha social, sino una exigencia del amor cristiano que, a imitación de Cristo, se dirige primero a quienes más lo necesitan. El \textit{Compendio de la Doctrina Social de la Iglesia} la presenta como una manifestación concreta de la caridad y la justicia (nn. 182–184), en el marco del bien común y de la dignidad de toda persona humana \cite{compendio}.

El problema surge cuando esta opción es deformada por una visión ideologizada del pobre, no como sujeto de dignidad trascendente, sino como miembro de una clase social oprimida en lucha contra sus opresores. En ese punto, el pobre deja de ser prójimo para convertirse en emblema de una causa política. Esta deriva ha sido denunciada en documentos como la \textit{Libertatis Nuntius} (1984), que advirtió contra la absorción de la teología por parte de ideologías ajenas al Evangelio \cite{libertatis}.

La influencia de este sesgo ideológico puede rastrearse en sectores donde se promueve una pastoral que omite la evangelización de otros grupos humanos, como los profesionales, los empresarios o los gobernantes, a quienes se considera sospechosos por definición. Sin embargo, el \textit{Documento de Aparecida} (2007) marcó una corrección importante al incluir de manera expresa la “pastoral de las élites”, reconociendo que también estos sectores necesitan ser acompañados, formados y evangelizados \cite{aparecida}. La verdadera opción cristiana no excluye, sino que prioriza sin descalificar.

Además, la noción de “pobre” en el Evangelio no se limita a la pobreza material. Los \textit{anawim} del Antiguo Testamento y los pobres de espíritu de las Bienaventuranzas (cf. Mt 5,3) remiten a una actitud de humildad, dependencia de Dios y apertura a su Reino. Jesús acogió tanto a los excluidos como a los publicanos ricos y a los centuriones romanos, mostrando que la salvación está destinada a todos, y que la pobreza que él bendice es ante todo interior \cite{evangelio}.

En conclusión, la opción preferencial por los pobres, cuando se comprende desde la caridad cristiana, enriquece la misión evangelizadora de la Iglesia y purifica su testimonio. Pero cuando se reduce a una bandera ideológica, pierde su dimensión trascendente y se convierte en una herramienta de división. Frente a esta tentación, la Iglesia está llamada a mantener la fidelidad al Evangelio, evitando tanto el elitismo indiferente como el populismo ideológico. Solo así podrá anunciar a todos, pobres y ricos, la Buena Noticia de la salvación en Cristo.

\begin{thebibliography}{9}

\bibitem{compendio}
Pontificio Consejo Justicia y Paz. 
\textit{Compendio de la Doctrina Social de la Iglesia}. 
Libreria Editrice Vaticana, 2004.

\bibitem{libertatis}
Congregación para la Doctrina de la Fe. 
\textit{Libertatis Nuntius: Instrucción sobre algunos aspectos de la Teología de la Liberación}. 
Vaticano, 1984. Disponible en: \url{https://www.vatican.va}

\bibitem{aparecida}
Conferencia General del Episcopado Latinoamericano y del Caribe. 
\textit{Documento de Aparecida}. 
CELAM, 2007.

\bibitem{evangelio}
Biblia. 
\textit{Evangelio según San Mateo 5,3}. 
Versión oficial de la Sagrada Escritura. Conferencia Episcopal Española.

\end{thebibliography}

\end{document}
