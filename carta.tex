\documentclass{letter}
\usepackage[spanish]{babel}
\usepackage[utf8]{inputenc}
\usepackage{geometry}
\usepackage{graphicx}
\usepackage{hyperref}
\geometry{margin=2.5cm}

\begin{document}
	
	\begin{flushright}
		Lima, \today
	\end{flushright}
	
	\vspace{0.5cm}
	
	Estimada profesora\\
	\textbf{Ana Rosario Miaury Vilca}\\
	Presente.-
	
	\vspace{0.5cm}
	
	Reciba un cordial saludo.
	
	Le escribo para adjuntarle mi trabajo titulado \textit{“Símbolos y personajes representativos de tu ciudad: La estatua de Francisco Pizarro”}, correspondiente al Reto 3, Fase 3 del curso \textbf{Identidad y Peruanidad}.
	
	Le pido disculpas por haber excedido el límite sugerido de extensión. El motivo es que, al desarrollar el trabajo, me vi conducido a una investigación más profunda, tanto documental como interpretativa, que considero necesaria para dar cuenta del sentido histórico y simbólico del monumento elegido.
	
	Como alumno de Teología con interés especial en investigación histórica, he tomado este trabajo como un ejercicio serio de producción académica. De hecho, estoy en proceso de trabajar versiones publicables para contribuir al diálogo sobre identidad peruana y memoria colectiva.
	
	Gracias por su comprensión y por la oportunidad de explorar estos temas desde un enfoque integrador.
	
	\vspace{1cm}
	
	\includegraphics[width=0.2\linewidth]{firma2} % <- Solo esta línea es suficiente
	
	\vspace{0.1cm}
	\rule{5cm}{0.4pt} % Línea para firma a mano
	
	\noindent\textbf{Jorge L. Ayona Inglis} \\
	\href{https://orcid.org/0009-0006-6551-9681}{\texttt{ORCID: 0009-0006-6551-9681}}
	
\end{document}
