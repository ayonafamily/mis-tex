\documentclass[12pt,a4paper]{article}
\usepackage[utf8]{inputenc}
\usepackage[spanish]{babel}
\usepackage{amsmath,amssymb}
\usepackage{setspace}
\usepackage{geometry}
\usepackage{titlesec}
\usepackage{lipsum}
\usepackage{csquotes}

\geometry{margin=2.5cm}
\setstretch{1.3}
\titleformat{\section}{\Large\bfseries}{\thesection.}{0.5em}{}

\title{\textbf{Reflexiones sobre el conocimiento, la sombra y el ánimus}}
\author{Jorge L. Ayona Inglis}
\date{\today}

\begin{document}
	
	\maketitle
	
	\section{Sobre la crítica al conocimiento}
	
	A lo largo de mi vida, he notado que muchas personas han desvalorizado el saber con frases como \enquote{de nada vale saber si no saber qué hacer}, muchas veces expresadas como indirectas hacia mí. Esto me ha hecho reflexionar si se trata de una expresión de mi \textbf{sombra} —la parte inconsciente de mí mismo que no acepto del todo—, o si proviene de un conflicto de los demás con respecto al conocimiento.
	
	\subsection*{Reflexión}
	Es posible que ambas cosas estén presentes. Por un lado, puedo tener una herida que teme brillar por miedo al rechazo. Por otro, las personas que me dicen estas frases pueden estar proyectando su inseguridad, su sensación de inferioridad o incluso su trauma con respecto al saber.
	
	\textit{La verdadera humildad no consiste en apagar mi luz, sino en ponerla al servicio de los demás sin orgullo.}
	
	\section{Contenidos implantados y la sombra}
	
	Muchos de los pensamientos negativos o autocríticos que cargamos no provienen de nuestra esencia, sino que han sido \textbf{implantados} por nuestros padres, maestros o el entorno. Al no ser cuestionados, se alojan en la sombra y actúan desde el inconsciente.
	
	\subsection*{Proceso de integración}
	
	\begin{enumerate}
		\item \textbf{Observar sin juzgar:} Reconocer que ciertas voces internas no son nuestras.
		\item \textbf{Dialogar con ellas:} ¿Qué defienden? ¿Qué temen?
		\item \textbf{Elegir qué hacer:} ¿Deseo seguir obedeciendo esa voz o puedo vivir diferente?
		\item \textbf{Individuación:} Convertir ese conflicto en una oportunidad para crecer y afirmarme con libertad.
	\end{enumerate}
	
	\begin{quote}
		\textit{“No se ilumina imaginando figuras de luz, sino haciendo consciente la oscuridad.”} — Carl Gustav Jung
	\end{quote}
	
	\section{Manifestación del ánimus en mí}
	
	Según lo que he compartido de mi vida interior, mi relación con el conocimiento, con figuras femeninas y con el juicio interno, se pueden distinguir algunas formas en las que el \textbf{ánimus} se manifiesta:
	
	\subsection*{1. Como figura crítica interna}
	A veces se presenta como una voz que me exige actuar, producir o aplicar el conocimiento, en lugar de valorarlo por sí mismo. Esta voz puede ser la interiorización de figuras autoritarias del pasado.
	
	\subsection*{2. Como guía hacia el sentido}
	También se manifiesta como una energía que me mueve hacia el pensamiento trascendente: la teología, la filosofía, la búsqueda de sentido en la existencia y el sufrimiento.
	
	\subsection*{3. A través de figuras femeninas}
	Mi relación con algunas mujeres revela conflictos o ideales relacionados con el ánimus. A veces hay admiración, otras veces dolor o frustración. Esto sugiere que mi ánimus proyecta tanto guías como heridas en mis vínculos.
	
	\subsection*{Ideas asociadas a lo femenino}
	Mi visión del principio femenino está marcada por:
	
	\begin{itemize}
		\item Respeto y admiración por mujeres sabias o espirituales.
		\item Sensibilidad hacia el dolor femenino bajo estructuras opresivas.
		\item Heridas por experiencias de abandono, incomprensión o sacrificio emocional.
	\end{itemize}
	
	\section*{Conclusión}
	
	Estoy en un camino de \textbf{individuación}, reconociendo la sombra, el ánimus, y los mandatos heredados para poder vivir más libremente desde lo que soy. El conflicto entre lo que sé y lo que otros proyectan es parte del fuego alquímico que transforma mi alma.
	
	\begin{center}
		\textit{Mi saber no es una amenaza. Es un don. Y lo usaré con discernimiento y sin esconderme.}
	\end{center}
	
\end{document}
