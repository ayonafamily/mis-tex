\documentclass[12pt]{article}
\usepackage[utf8]{inputenc}
\usepackage[T1]{fontenc}
\usepackage[spanish]{babel}
\usepackage{csquotes}
\usepackage{setspace}
\usepackage{geometry}
\usepackage{helvet}
\renewcommand{\familydefault}{\sfdefault}

\usepackage[
backend=biber,
style=apa,
natbib=true
]{biblatex}

\addbibresource{referencias.bib}  % Archivo .bib

\geometry{margin=1in}
\setstretch{2}
\renewcommand{\rmdefault}{ptm}  % Times New Roman

\title{\uppercase{¿Qué Es Teología?}}
\author{Jorge Ayona \\ \texttt{jorge.ayona@estudiante.ucsm.edu.pe}}
\date{23 de mayo del 2025}

\begin{document}
	
	\maketitle
	
	La palabra teología viene del griego \textit{theós} (Dios) y \textit{lógos} (palabra, discurso, razón). Etimológicamente, significa “discurso sobre Dios”. La Real Academia Española define la teología como el “conjunto de conocimientos acerca de Dios y de sus atributos y perfecciones, que se deducen de la revelación y de la razón” \parencite[ s.v. teología]{rae2024}. Esta definición refleja bien la tensión esencial de este saber: ¿cómo puede el ser humano, limitado y finito, hablar con sentido sobre lo infinito y trascendente? Sin embargo, desde la antigüedad, la necesidad de pensar en lo divino ha estado presente en todas las culturas.
	
	En la filosofía griega, Aristóteles, por su parte, distingue en su \textit{Metafísica} (libro VI, 1) tres tipos de filosofía: natural, matemática y teológica. Esta última se refiere al estudio de la divinidad. En este sentido, la teología era ya considerada una forma de conocimiento superior, aunque aún filosófica y no como saber revelado \parencite{aristoteles2003}.
	
	Con el cristianismo, el término teología adquiere un nuevo matiz. Ya no se trata solo de pensar en Dios con la razón natural, sino de reflexionar sobre lo que Dios ha revelado de sí mismo. San Anselmo, en el siglo XI, expresó esto al decir, “la fe que busca entender”. La teología, así entendida, se convierte en un acto de fe que no renuncia a la razón, sino que la pone al servicio de la búsqueda de sentido \parencite{anselmo2005}.
	
	Al lado de esta teología revelada, los pensadores cristianos también desarrollaron una teología natural. Santo Tomás de Aquino, por ejemplo, en su \textit{Suma Teológica}, dedica buena parte del primer tratado a demostrar racionalmente la existencia de Dios, partiendo de la observación del mundo y del orden de las causas \parencite{aquinio2015}.
	
	Personalmente, acercarme a la teología ha sido como abrir una ventana hacia lo trascendente, a Dios. No la veo como un simple conocimiento, sino como un medio para iluminar mi propia vida. En un mundo lleno de ruido y superficialidad, la teología me invita a ir al fondo de las cosas, a buscar su sentido último. Quizás, emprender esta búsqueda sin tener todas las respuestas.
	
	\vspace{1cm}
	\printbibliography
	
\end{document}
