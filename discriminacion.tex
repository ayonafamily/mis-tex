\documentclass[12pt]{article}

% --------------------------
% Paquetes y configuración
% --------------------------

\usepackage[margin=2.54cm]{geometry} % Márgenes
\usepackage[spanish]{babel}          % Idioma español
\usepackage[utf8]{inputenc}         % Codificación UTF-8
\usepackage[T1]{fontenc}            % Codificación de salida
\usepackage{lmodern}                % Fuente Latin Modern
\usepackage{setspace}               % Para controlar el interlineado
\usepackage{titlesec}               % Personalizar títulos
\usepackage{fancyhdr}               % Encabezados personalizados

\setlength{\parindent}{0pt}         % Sin sangría
\setlength{\parskip}{1em}           % Espacio entre párrafos

% --------------------------
% Configuración del título
% --------------------------

\newcommand{\titulo}[1]{\begin{center}\textbf{\large #1}\end{center}}
\newcommand{\autor}[1]{\begin{center}#1\end{center}}
%\newcommand{\institucion}[1]{\begin{center}#1\end{center}}
%\newcommand{\curso}[1]{\begin{center}#1\end{center}}
\newcommand{\fecha}[1]{\begin{center}#1\end{center}}

% --------------------------
% Inicio del documento
% --------------------------

\begin{document}

\titulo{La dignidad envejece, pero no se apaga}
\autor{Jorge L. Ayona Inglis}
%\institucion{Universidad Católica de Santa María}
%\curso{Taller de Arte – Nicole Rosario Rodríguez Segura}
\fecha{\today}

En medio de un mundo que exalta la juventud, la novedad y la inmediatez, el Salmo 71 (70 en la numeración de la Biblia Vulgata) se levanta como un testimonio silencioso pero poderoso de una realidad silenciada: la experiencia del adulto mayor. Este salmo es una súplica de quien ha recorrido una larga vida confiando en Dios, pero que ahora, ante la fragilidad de la vejez, se siente vulnerable y clama: “No me desampares ahora que estoy viejo y tengo canas” (Sal 71,18).

El anciano del salmo no habla desde la resignación ni la autocompasión, sino desde una conciencia profunda de que su vida, con todas sus experiencias, fracasos, aprendizajes y encuentros con Dios, tiene todavía una misión por cumplir: “Hasta que anuncie tu poder a la posteridad”. Esta frase es clave. La vejez no es un cierre inútil ni una etapa estéril. Es, por el contrario, un tiempo de transmisión, de legado, de siembra en los corazones jóvenes.

Sin embargo, nuestra sociedad actual, centrada en lo productivo, ha instaurado lo que el Papa Francisco llama la “cultura del descarte”, en la que las personas mayores son marginadas como si ya no tuvieran nada valioso que ofrecer. Se privilegia lo nuevo y se considera lo antiguo —y con ello, a quienes han vivido más— como obsoleto. Esto ocurre en los espacios laborales, en las decisiones familiares e incluso en la vida pública. Se olvida que la sabiduría, a diferencia de la información, nace de la vida vivida y reflexionada.

Desde mi propia vivencia como adulto mayor, puedo decir que las palabras del salmista me interpelan hondamente. A veces he sentido esa forma de desprecio sutil —o no tan sutil— que implica ser empujado, ignorado o incluso insultado por ser mayor. El cuerpo duele más, el paso se hace más lento, y la sensación de indefensión se vuelve más frecuente. Pero en lugar de encerrarme en el resentimiento, encuentro en este salmo una invitación a descubrir que mi vida aún tiene una misión: compartir lo que sé, lo que he vivido, lo que he aprendido con Dios y en la vida.

El Papa Francisco ha dicho que “la vejez no es una enfermedad, sino un privilegio”, y ha recordado que los ancianos tienen la responsabilidad profética de transmitir la memoria del pueblo. Así como el profeta Elías, que en un momento deseó morir en el desierto, fue reanimado por Dios para convertirse en guía de una nueva generación, también nosotros, los mayores, estamos llamados a acompañar, enseñar, y sembrar con esperanza, incluso cuando el mundo no lo reconozca.

La discriminación hacia los ancianos no siempre es visible ni violenta. A veces se expresa en el olvido, en el aislamiento, en la falta de escucha. Pero una sociedad que no valora a sus mayores es una sociedad que se desconecta de sus raíces y de su alma. Por eso elegí este pasaje bíblico: porque nos obliga a mirar donde normalmente no miramos. Y porque nos recuerda que en la vejez también hay belleza, dignidad, y una misión que nadie puede arrebatar: la de seguir siendo testigos del amor y la fidelidad de Dios a lo largo de toda una vida.
\section*{Referencias}
\addcontentsline{toc}{section}{Referencias}

Conferencia Episcopal Española. (1995). \emph{Sagrada Biblia. Versión oficial de la Conferencia Episcopal Española}. Biblioteca de Autores Cristianos.

Francisco. (2013, 24 de noviembre). \emph{Evangelii Gaudium: Exhortación apostólica sobre el anuncio del Evangelio en el mundo actual}. Vaticano. \url{https://www.vatican.va/content/francesco/es/apost_exhortations/documents/papa-francesco_esortazione-ap_20131124_evangelii-gaudium.html} 

Francisco. (2022, 15 de junio). \emph{Catequesis sobre la vejez: La alianza entre las generaciones}. Audiencia general. \url{https://www.vatican.va/content/francesco/es/audiences/2022/documents/20220615-udienza-generale.html} 

Pontificio Consejo para la Familia. (1999). \emph{El anciano en la familia: Carta a las familias sobre la pastoral de las personas mayores}. Libreria Editrice Vaticana.

\end{document}
