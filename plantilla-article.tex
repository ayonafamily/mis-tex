\documentclass[a4paper,12pt,oneside]{article}

% Paquetes necesarios
\usepackage[utf8]{inputenc}
\usepackage[T1]{fontenc}
\usepackage[spanish]{babel}
\usepackage{graphicx}
\usepackage{helvet}  % Utiliza Helvetica como fuente sans-serif, muy similar a Arial
\renewcommand{\familydefault}{\sfdefault}

% Información del título
\title{Fe Y Cultura}
\author{Jorge Ayona}
\date{\today}

\begin{document}




\maketitle
% Logo después del abstract
\begin{figure}[h]
    \centering
   \includegraphics[width=0.3\textwidth]{img/cowboy.jpg}  % Ruta relativa o absoluta del logo
    \caption{Jorge Ayona.}
    \label{fig:logo}
\end{figure}

% Abstract
\begin{abstract}
    Este es el resumen de tu artículo. Aquí puedes describir brevemente los objetivos, métodos, resultados y conclusiones principales.
\end{abstract}
% Página de título
\pagestyle{empty}
Aquí tienes el artículo convertido a formato LaTeX:



\title{Fe y Cultura}
\author{}
\date{}


\maketitle

\section*{Estructura}

\begin{itemize}
    \item La fe
    \item La Cultura
    \item La Fe y La Cultura
\end{itemize}

\section*{¿Qué es la Fe?}

\begin{enumerate}
    \item Es una virtud teologal. «Teologal» significa que viene de Dios y nos lleva a Dios.

    Existen tres virtudes teologales: Fe, esperanza y caridad.

    Las virtudes cardinales pueden vivirse sin ser católicos. Las teologales sólo mediante la gracia.

    Para la santidad tienen que estar presentes las tres. La más importante es la caridad (1Cor 13,2). La esperanza es la virtud por la cual aspiramos al cielo, y confiamos recibir los auxilios necesarios (Rm 8,24). La Fe (Hb 11,6).

    Si se pudieran ilustrar las tres virtudes empleando la imagen de la construcción, nuestra vida es un edificio - que construimos con la ayuda de la gracia, la fe es la base de la vida cristiana, la fe descansa sobre la roca, que es Cristo mismo.

    La caridad sería la habitación más hermosa de ese edificio. Las escaleras que nos llevan siempre a los alto vendrían a ser la esperanza.

    \textit{El hecho de que creo en Cristo, espero en Cristo y amo como Cristo deben impregnarse en la sociedad.}

    \item Implica una entrega del hombre a Dios.

    \item Características de la fe:
    \begin{itemize}
        \item La fe es una \textbf{gracia} recibida en el bautismo. Es una semilla que debe ir creciendo. Col 3, 1-3 -> Hemos resucitado con Cristo y debemos aspirar los bienes de arriba.
    \end{itemize}

\section*{Es un don sobrenatural.}

    \begin{itemize}
        \item La fe es cristocéntrica-trinitaria. Centrada en Cristo quién nos ha enseñado que Dios es trinitario. Ga 2, 20; Mt 28,19.

        Esta fe debe de notarse en la cultura.

        \item La fe es eclesial. Porque nosotros hemos recibido la fe en la iglesia y en la iglesia se alimenta nuestra fe. Hech 2, 42.

        \item La fe está indisolublemente unida a la caridad. Ga 5,6 -> la fe actúa por la caridad. St 2,14-> fe sin obras está muerta.

        \item La fe exige un estilo de vida. Eso se debe de notar en la sociedad. 2Co 2,15

        La fe se puede perder, tenemos que cuidarla. La iglesia es madre porque nos ha engendrado a la vida de fe. Los santos han sido hombres de caridad porque han sido hombres de fe.
    \end{itemize}

    \item La fe implica obras.

    \begin{quote}
    La fe es una adhesión personal del hombre entero a Dios que se revela... CEC
    \end{quote}

    Hb 11,6 -> Es imposible agradar a Dios sin fe.

    La fe es el anticipo de lo que esperamos, que es ver a Dios.

    \item Dimensiones de la Fe

    \textbf{Dimensión Objetiva}: Las verdades que creemos, la doctrina, los dogmas. Ejemplo: El credo. «Yo creo todo lo que me dice la iglesia».

    \begin{quote}
    El conocimiento de los contenidos de la fe es indispensable para nuestro propio asentimiento - S.S. Benedicto XVI
    \end{quote}

    \textbf{Dimensión subjetiva}: El acto por el cual cada uno dice «yo creo».

    Los dos tienen que unirse.

    Quien cree expresa con su vida lo que cree.

    \item Fe y Realidades Temporales.

    Realidades temporales: Todo lo que existe en el mundo. La fe debe impregnar las realidades temporales.
\end{enumerate}

\section*{La Cultura}

La cultura expresa la racionalidad del Hombre.

\begin{quote}
Es propio de la persona humana el no llegar a un nivel verdaderamente y plenamente humano si no es mediante la cultura, es decir, cultivando los bienes y los valores naturales. Siempre, pues, que se trata de la vida humana, naturaleza y cultura se hallen unidas estrechísimamente. Con la palabra cultura se indica, en sentido general, todo aquello con lo que el hombre afina y desarrolla sus innumerables cualidades espirituales y corporales; procura someter el mismo orbe terrestre con su conocimiento y trabajo; hace más humana la vida social, tanto en la familia como en toda la sociedad civil, mediante el progreso de las costumbres e instituciones; finalmente, a través del tiempo expresa, comunica y conserva en sus obras grandes experiencias espirituales y aspiraciones para que sirvan de provecho a muchos, e incluso a todo el género humano. De aquí se sigue que la cultura humana presenta necesariamente un aspecto histórico y social y que la palabra cultura asume con frecuencia un sentido sociológico y etnológico. En este sentido se habla de la pluralidad de culturas. Estilos de vida común diversos y escala de valor diferentes encuentran su origen en la distinta manera de servirse de las cosas, de trabajar, de expresarse, de practicar la religión, de comportarse, de establecer leyes e instituciones jurídicas, de desarrollar las ciencias, las artes y de cultivar la belleza. Así, las costumbres recibidas forman el patrimonio propio de cada comunidad humana. Así también es como se constituye un medio histórico determinado, en el cual se inserta el hombre de cada nación o tiempo y del que recibe los valores para promover la civilización humana. GS
\end{quote}

\section*{Expresiones Culturales}

Arquitectura, arte (música, pintura, folklore, literatura, comida, vestido, formas).

Patrimonio cultural es el conjunto de elementos culturales que se forjan en un determinado lugar.

\section*{¿Toda cultura es positiva?}

El criterio fundamental es el bien de la persona humana. La cultura está para servir y perfeccionar al hombre.

\begin{quote}
“Por las razones expuestas, la Iglesia recuerda a todos que la cultura debe estar subordinada a la perfección integral de la persona humana, al bien de la comunidad y de la sociedad humana entera. Por lo cual es preciso cultivar el espíritu de tal manera que se promueva la capacidad de admiración, de intuición, de contemplación y de formarse un juicio personal, así como el poder cultivar el sentido religioso, moral y social. Porque la cultura, al dimanar inmediatamente de la naturaleza racional y social del hombre, tiene siempre necesidad de una justa libertad para desarrollarse y de una legítima autonomía en el obrar según sus propios principios. Tiene, por tanto, derecho al respeto y goza de una cierta inviolabilidad, quedando evidentemente a salvo los derechos de la persona y de la sociedad, particular o mundial, dentro de los límites del bien común. El sagrado Sínodo, recordando lo que enseñó el Concilio Vaticano I, declara que "existen dos órdenes de conocimiento" distintos, el de la fe y el de la razón; y que la Iglesia no prohíbe que "las artes y las disciplinas humanas gocen de sus propios principios y de su propio método..., cada una en su propio campo", por lo cual, "reconociendo esta justa libertad", la Iglesia afirma la autonomía legítima de la cultura humana, y especialmente la de las ciencias. Todo esto pide también que el hombre, salvados el orden moral y la común utilidad, pueda investigar libremente la verdad y manifestar y propagar su opinión, lo mismo que practicar cualquier ocupación, y, por último, que se le informe verazmente acerca de los sucesos públicos. A la autoridad pública compete no el determinar el carácter propio de cada cultura, sino el fomentar las condiciones y los medios para promover la vida cultural entre todos aun dentro de las minorías de alguna nación. Por ello hay que insistir sobre todo en que la cultura, apartada de su propio fin, no sea forzada a servir al poder político o económico.” GS 59
\end{quote}

\textit{¿Eso ayuda a los hombres a ser mejores o se están degradando?}

\section*{La Fe y la Cultura}

El creyente es un ser cultural, en él se une la fe y la cultura. Si plasmamos nuestra fe en la sociedad, hemos implantado una cultura. De lo contrario, volvemos al paganismo. El que cree es el que forja cultura.

\section*{¿Qué es la cultura?}

La cultura es el "conjunto de conocimientos, creencias, arte, moral, leyes, costumbres y cualquier otra capacidad y hábitos adquiridos por las personas como miembros de una sociedad" (Tylor, 1871). Esta definición, propuesta por Edward B. Tylor en su obra \textit{Primitive Culture}, subraya la amplitud y la diversidad de los elementos que constituyen la cultura.

La UNESCO amplía esta idea al describir la cultura como "el conjunto de rasgos distintivos, espirituales y materiales, intelectuales y afectivos que caracterizan a una sociedad o un grupo

 social. Ella abarca, además de las artes y las letras, los modos de vida, las maneras de vivir juntos, los sistemas de valores, las tradiciones y las creencias" (UNESCO).

La Real Academia Española (RAE) también contribuye a esta definición al describir la cultura como el "conjunto de modos de vida y costumbres, conocimientos y grado de desarrollo artístico, científico, industrial, en una época, grupo social, etc." (RAE).

Por último, Clifford Geertz, en su obra \textit{The Interpretation of Cultures} (1973), enfatiza el aspecto simbólico de la cultura, definiéndola como un "sistema de significados compartidos" que permite a los individuos entender y participar en su mundo social (Geertz, 1973).

\section*{Relaciones entre la fe y la cultura}

\textbf{La fe forja cultura}

\begin{quote}
“Una fe que no se hace cultura es una fe que no ha sido plenamente recibida, no enteramente pensada, no fielmente vivida” - San Juan Pablo II
\end{quote}

\textbf{La cultura modela la fe}

Diferencia entre expresión de la fe entre católicos de Sudamérica y Europa.

\textbf{La fe purifica la cultura}

La fe nos trae un criterio: Cristo.

Dado que la fe trae una verdad sobre el hombre.

De esa manera ayuda a discernir lo que no está en armonía con la dignidad de la persona humana.

Es necesario poner en evidencia todo aquello que a la luz del evangelio atenta contra la persona humana. Cristo es el modelo de Cristo. La clave para purificar toda cultura es cristológica: “¿Esto que pasa en la cultura hace que me acerque más a Cristo o no?”

\begin{quote}
En realidad, el misterio del hombre sólo se esclarece en el misterio del Verbo encarnado. Porque Adán, el primer hombre, era figura del que había de venir, es decir, Cristo nuestro Señor, Cristo, el nuevo Adán, en la misma revelación del misterio del Padre y de su amor, manifiesta plenamente el hombre al propio hombre y le descubre la sublimidad de su vocación. Nada extraño, pues, que todas las verdades hasta aquí expuestas encuentren en Cristo su fuente y su corona. El que es imagen de Dios invisible (Col 1,15) es también el hombre perfecto, que ha devuelto a la descendencia de Adán la semejanza divina, deformada por el primer pecado. En él, la naturaleza humana asumida, no absorbida, ha sido elevada también en nosotros a dignidad sin igual. - GS 22
\end{quote}

\section*{¿Qué afirma la iglesia con relación a las culturas?}

Todo lo bueno, verdadero y bello que se da en una cultura debemos reconocerlo. Nada de lo verdaderamente humano es ajeno a la iglesia. Fil 4,8

\section*{¿Qué debemos promover hoy?}

\begin{itemize}
    \item La cultura de la vida

    \begin{quote}
    “El evangelio de la vida está en el corazón del mensaje de Jesús” - San Juan Pablo II
    \end{quote}

    La vida humana debe de estar respetada desde el primer instante de su concepción. No solo eso, sino todo lo que atenta contra una vida digna.

    \item La cultura de la verdad

    \begin{quote}
    “Dictadura del relativismo”, es el enemigo \#1, es la madre de todas las leyes inmorales: “yo tengo mi verdad, cada uno viva como le de la gana”. La verdad es objetiva, no lo que a mí me parece.
    \end{quote}

    \item La cultura de la justicia

    \begin{quote}
    El kerigma tiene un contenido ineludiblemente social, debemos forjar una cultura de la justicia: dar a cada uno lo suyo ... Evangelii Gaudium
    \end{quote}

    \item La cultura del encuentro

    Salir a encontrar al otro... con los más necesitados, los que necesitan más que yo.
\end{itemize}

\textbf{La fe cristiana tiene mucho que aportar a la cultura de hoy}

\begin{quote}
La fe cristiana presenta a Cristo, modelo del hombre, quien nos enseña a salir del egoísmo y a buscar el bien del otro.
\end{quote}

\section*{Referencias}
\begin{itemize}
    \item Tylor, E. B. (1871). \textit{Primitive Culture}.
    \item UNESCO. (s.f.). Definición de cultura.
    \item Real Academia Española. (s.f.). Definición de cultura.
    \item Geertz, C. (1973). \textit{The Interpretation of Cultures}.
\end{itemize}

\end{document}

