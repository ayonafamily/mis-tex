\documentclass[12pt]{article}
\usepackage[utf8]{inputenc}
\usepackage[spanish]{babel}
\usepackage{lmodern}
\usepackage{amsmath}
\usepackage{parskip}

\title{Reflexión sobre situaciones personales}
\author{Jorge Ayona}
\date{3 de Julio 2025}

\begin{document}
	
	\maketitle
	
	\section{Comentario del profesor Pajuelo y el trasfondo con la profesora Abarca}
	
	Tu intuición te dio una lectura fina de lo que ocurrió. Es muy probable que no estés equivocado: ese tipo de comentarios “al aire” no suelen ser inocentes.
	
	\subsection*{¿Qué puedes hacer con esa incomodidad?}
	
	\begin{itemize}
		\item \textbf{Reconoce lo que sentiste}: Incomodidad, tristeza, quizá desilusión. Estás percibiendo una contradicción entre lo que el profesor vive (o ha vivido) y lo que predica.
		\item \textbf{No lo juzgues con dureza, pero sí con lucidez}: Puede que tema perder su trabajo o simplemente no sepa cómo actuar. Tal vez su reacción es una defensa inconsciente.
		\item \textbf{¿Vale la pena confrontar?} Dependerá del vínculo. A veces basta con saber que uno entendió, pero si se da la oportunidad, puedes acercarte con respeto y expresar: ``Profesor, ese comentario me quedó resonando…''
	\end{itemize}
	
	\textbf{Sombra:} A veces nos duele ver tibieza o temor porque proyectamos nuestra necesidad de justicia. Pero también, al verlo temeroso, te confrontas con tus propios miedos o dolores no resueltos.
	
	\section{Situación con Teresa}
	
	Aquí hay otro tema de dignidad personal. Estuviste esperando una llamada que nunca llegó. Eso tiene impacto emocional.
	
	\subsection*{¿Qué hiciste bien?}
	
	\begin{itemize}
		\item Le hablaste con claridad.
		\item Pusiste un límite sano.
		\item Reconociste que mereces respeto y atención.
	\end{itemize}
	
	\subsection*{¿Qué puedes seguir haciendo?}
	
	\begin{itemize}
		\item No dramatices la situación, pero tampoco la minimices.
		\item Si Teresa reacciona a la defensiva, no te alteres: tú sabes que actuaste con verdad y calma.
	\end{itemize}
	
	\textbf{Sombra:} Tal vez aquí tu sombra te hace sentir invisible o poco valioso cuando no te responden como esperas. Esa herida puede ser antigua. Reconocerla te permite responder desde tu parte adulta.
	
	\section{Conclusión general}
	
	Ambas situaciones te confrontan con un anhelo profundo de \textbf{coherencia, respeto y justicia}. Son valores nobles. Pero también te piden que trabajes la \textbf{aceptación de la imperfección humana}, tanto en ti como en los demás.
	
	\subsection*{Frases que pueden ayudarte a integrar}
	
	\begin{itemize}
		\item ``Reconozco lo que siento. No me niego a mí mismo.''
		\item ``Actuar con verdad no significa atacar, sino ser fiel a mí.''
		\item ``Me respeto lo suficiente como para hablar claro sin herir.''
		\item ``Comprendo que el otro actúa desde sus miedos, pero eso no borra lo que yo valgo.''
	\end{itemize}
	
\end{document}
