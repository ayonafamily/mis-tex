\documentclass[12pt]{article}
\usepackage[utf8]{inputenc}
\usepackage[spanish]{babel}
\usepackage{geometry}
\usepackage{tikz}
\usetikzlibrary{arrows.meta, positioning}

\geometry{a4paper, margin=2.5cm}

\tikzset{
  box/.style={
    rectangle, draw=black, rounded corners, 
    minimum width=4.5cm, minimum height=1.5cm, 
    text centered, font=\small
  },
  arrow/.style={
    thick, ->, >=stealth, shorten >=2pt, shorten <=2pt
  }
}

\title{Mapa conceptual: Freud – Jung – San Pablo – Mi Camino Interior}
\author{Jorge Ayona}
\date{\today}

\begin{document}
\maketitle

\begin{center}
\begin{tikzpicture}

% Nodos con coordenadas fijas
\node[box, fill=red!20] (freud) at (0,0) {Freud\\Inconsciente = pulsiones reprimidas};
\node[box, fill=yellow!30] (jung) at (0,-3) {Jung\\Sombra = parte negada del alma};
\node[box, fill=green!25] (pablo) at (6,0) {San Pablo\\Hombre viejo / carne\\vs. Hombre nuevo en Cristo};
\node[box, fill=blue!20] (yo) at (0,-6) {Mi camino\\Integrar mi sombra, sanar,\\vivir en plenitud espiritual};

% Flechas
\draw[arrow] (freud) -- (jung) node[midway, left] {Trasciende};
\draw[arrow] (jung) -- (yo) node[midway, left] {Trabajo interior};
\draw[arrow] (pablo) -- (yo) node[midway, right] {Gracia y redención};
\draw[arrow] (freud) -- (pablo) node[midway, above] {Visiones opuestas};

\end{tikzpicture}
\end{center}

\vspace{1em}

\section*{Interpretación del mapa}

\begin{itemize}
  \item \textbf{Freud} representa una visión materialista y reduccionista del inconsciente.
  \item \textbf{Jung} introduce el lenguaje de los símbolos, los arquetipos y la dimensión espiritual.
  \item \textbf{San Pablo} ofrece una visión del alma redimida por gracia: del hombre viejo al nuevo en Cristo.
  \item \textbf{Mi camino}: integrar la sombra con ayuda de la gracia, el símbolo y la conciencia plena.
\end{itemize}

\end{document}
